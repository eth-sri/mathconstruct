\section{bmo}
\subsection{bmo-shortlist-2008-n1}
\subsubsection{Variation}
\textbf{Actual Problem}\\
Show that for $a=12$, there exists a natural number that has the number $a$ (the sequence of digits that constitute $a$) at its beginning, and which decreases $a$ times when $a$ is moved from its beginning to its end (any number of zeros that appear in the beginning of the number obtained in this way are to be removed). Find an example.

Output the answer as an integer inside of $\boxed{...}$. For example $\boxed{123}$.

\textbf{Revised Problem}\\
Demonstrate that, for $a=12$, there exists a natural number starting with the digits of $a$ such that when $a$ is moved from the start to the end of this number, the number diminishes by a factor of $a$ (any leading zeros in the resulting number should be discarded). Provide an example of such a number.

Present your answer as an integer enclosed in a box, such as $\boxed{123}$.

\subsubsection{Variation}
\textbf{Actual Problem}\\
Show that for $a=19$, there exists a natural number that has the number $a$ (the sequence of digits that constitute $a$) at its beginning, and which decreases $a$ times when $a$ is moved from its beginning to its end (any number of zeros that appear in the beginning of the number obtained in this way are to be removed). Find an example.

Output the answer as an integer inside of $\boxed{...}$. For example $\boxed{123}$.

\textbf{Revised Problem}\\
Demonstrate that for the value \( a = 19 \), there is a natural number which begins with the digits of \( a \) and becomes \( a \) times smaller when the digits of \( a \) are transferred from the start to the end of the number (removing any leading zeros in the resulting number). Provide an example of such a number.

Express the solution as an integer within \(\boxed{...}\). For instance, \(\boxed{123}\).

\subsubsection{Variation}
\textbf{Actual Problem}\\
Show that for $a=8$, there exists a natural number that has the number $a$ (the sequence of digits that constitute $a$) at its beginning, and which decreases $a$ times when $a$ is moved from its beginning to its end (any number of zeros that appear in the beginning of the number obtained in this way are to be removed). Find an example.

Output the answer as an integer inside of $\boxed{...}$. For example $\boxed{123}$.

\textbf{Revised Problem}\\
Demonstrate that given $a=8$, there is a natural number which starts with the digits of $a$, and when these digits are moved to the end of the number, the new number is $a$ times smaller than the original. Find such an example.

Present the solution as an integer enclosed in $\boxed{...}$. For instance, $\boxed{123}$.

\subsubsection{Variation}
\textbf{Actual Problem}\\
Show that for $a=9$, there exists a natural number that has the number $a$ (the sequence of digits that constitute $a$) at its beginning, and which decreases $a$ times when $a$ is moved from its beginning to its end (any number of zeros that appear in the beginning of the number obtained in this way are to be removed). Find an example.

Output the answer as an integer inside of $\boxed{...}$. For example $\boxed{123}$.

\textbf{Revised Problem}\\
Demonstrate that when $a=9$, there is a natural number beginning with the digit sequence $a$ that becomes $a$ times smaller once $a$ is transferred from the start to the end of the number (removing any leading zeros that might appear in the result). Provide an example to illustrate this.

Provide the answer formatted as an integer within $\boxed{...}$. For example, $\boxed{123}$.

\subsection{bmo-shortlist-2008-n5}
\subsubsection{Variation}
\textbf{Actual Problem}\\
Let $(a_n)$ be a sequence with $a_1=0$ and $a_{n+1}=2+a_n$ for odd $n$ and $a_{n+1}=2a_n$ for even $n$. There are infinitely many values of $n$ such that for $p=31$
$$ b=\frac{2^{2p}-1}{3} \mid a_n. $$ Give $N=23$ examples of such $n$.

Output the answer as a comma separated list inside of $\boxed{...}$. For example $\boxed{1, 2, 3}$.

\textbf{Revised Problem}\\
Consider a sequence \( (a_n) \) where \( a_1 = 0 \) and for any integer \( n \), the next term is defined as follows: \( a_{n+1} = 2 + a_n \) when \( n \) is odd, and \( a_{n+1} = 2a_n \) when \( n \) is even. There are infinitely many integers \( n \) such that for \( p = 31 \), the expression
\[ b = \frac{2^{2p} - 1}{3} \]
divides \( a_n \). Provide exactly \( N = 23 \) examples of such integers \( n \).

Present your answer as a list of integers separated by commas within a box, like this: \(\boxed{1, 2, 3}\).

\subsubsection{Variation}
\textbf{Actual Problem}\\
Let $(a_n)$ be a sequence with $a_1=0$ and $a_{n+1}=2+a_n$ for odd $n$ and $a_{n+1}=2a_n$ for even $n$. There are infinitely many values of $n$ such that for $p=269$
$$ b=\frac{2^{2p}-1}{3} \mid a_n. $$ Give $N=34$ examples of such $n$.

Output the answer as a comma separated list inside of $\boxed{...}$. For example $\boxed{1, 2, 3}$.

\textbf{Revised Problem}\\
Consider the sequence $(a_n)$ where $a_1 = 0$ and for each integer $n$, the following rules apply: if $n$ is odd, then $a_{n+1} = 2 + a_n$; if $n$ is even, then $a_{n+1} = 2a_n$. Determine infinitely many values of $n$ for which $b = \frac{2^{2p} - 1}{3}$ divides $a_n$, where $p = 269$. Provide a list containing 34 such values of $n$.

Present the solution as a list of numbers separated by commas enclosed in $\boxed{...}$. For instance, $\boxed{1, 2, 3}$.

\subsubsection{Variation}
\textbf{Actual Problem}\\
Let $(a_n)$ be a sequence with $a_1=0$ and $a_{n+1}=2+a_n$ for odd $n$ and $a_{n+1}=2a_n$ for even $n$. There are infinitely many values of $n$ such that for $p=197$
$$ b=\frac{2^{2p}-1}{3} \mid a_n. $$ Give $N=18$ examples of such $n$.

Output the answer as a comma separated list inside of $\boxed{...}$. For example $\boxed{1, 2, 3}$.

\textbf{Revised Problem}\\
Consider the sequence \((a_n)\) with initial term \(a_1 = 0\). The sequence is defined such that if \(n\) is odd, then \(a_{n+1} = 2 + a_n\), and if \(n\) is even, then \(a_{n+1} = 2a_n\). Determine the values of \(n\) such that for \(p = 197\),
\[ b = \frac{2^{2p} - 1}{3} \mid a_n. \]
Provide 18 examples of such \(n\).

Present the answer as a comma-separated list inside of \(\boxed{...}\). For instance, \(\boxed{1, 2, 3}\).

\subsubsection{Variation}
\textbf{Actual Problem}\\
Let $(a_n)$ be a sequence with $a_1=0$ and $a_{n+1}=2+a_n$ for odd $n$ and $a_{n+1}=2a_n$ for even $n$. There are infinitely many values of $n$ such that for $p=43$
$$ b=\frac{2^{2p}-1}{3} \mid a_n. $$ Give $N=13$ examples of such $n$.

Output the answer as a comma separated list inside of $\boxed{...}$. For example $\boxed{1, 2, 3}$.

\textbf{Revised Problem}\\
Consider a sequence \((a_n)\) where the initial term is \( a_1 = 0 \). The rule for progression is: if \( n \) is odd, then \( a_{n+1} = 2 + a_n \); if \( n \) is even, then \( a_{n+1} = 2a_n \). There are infinitely many integers \( n \) such that, for \( p = 43\), the expression 
$$ b = \frac{2^{2p} - 1}{3} $$ divides \( a_n \). Provide \( N = 13 \) instances of such \( n \).

Present your answer as a comma-separated series enclosed within \(\boxed{...}\). For example, \(\boxed{1, 2, 3}\).

\subsection{bmo-shortlist-2014-c1}
\subsubsection{Variation}
\textbf{Actual Problem}\\
The International Mathematical Olympiad is being organized in Japan, where a folklore belief is that the number $4$ brings bad luck. The opening ceremony takes place at the Grand Theatre where each row has the capacity of $N=56$ seats. Give an example for how $m = 30$ contestants can be seated in a single row with the restriction that no two of them are $4$ seats apart (so that bad luck during the competition is avoided).

Output the sequence of the state of the seats as a comma-separated list in $\boxed{...}$, where a '1' signifies an occupied seat, and '0' signifies an empty one. Ex: $\boxed{[1,1,0,0,1,0,1]}$

\textbf{Revised Problem}\\
The International Mathematical Olympiad is taking place in Japan, a country where the number $4$ is considered to be unlucky. The event's opening ceremony is held at the Grand Theatre, which has rows of $N=56$ seats each. Your task is to provide an example of how $m = 30$ participants can be seated in a single row, ensuring that no two participants are positioned exactly 4 seats away from each other, in order to avoid any potential misfortune during the competition.

Present the seating arrangement as a comma-separated list in $\boxed{...}$, where '1' denotes a seat that is occupied, and '0' denotes an unoccupied seat. Example: $\boxed{[1,1,0,0,1,0,1]}$

\subsubsection{Variation}
\textbf{Actual Problem}\\
The International Mathematical Olympiad is being organized in Japan, where a folklore belief is that the number $4$ brings bad luck. The opening ceremony takes place at the Grand Theatre where each row has the capacity of $N=128$ seats. Give an example for how $m = 65$ contestants can be seated in a single row with the restriction that no two of them are $4$ seats apart (so that bad luck during the competition is avoided).

Output the sequence of the state of the seats as a comma-separated list in $\boxed{...}$, where a '1' signifies an occupied seat, and '0' signifies an empty one. Ex: $\boxed{[1,1,0,0,1,0,1]}$

\textbf{Revised Problem}\\
The International Mathematical Olympiad, hosted in Japan, observes a local superstition that the number 4 is considered unlucky. During the opening event at the Grand Theatre, each row is equipped with $N=128$ seats. Illustrate a seating arrangement for $m = 65$ competitors in one row, ensuring that no pair of them is separated by exactly 4 seats, thus avoiding any potential misfortune during the event.

Present the seating configuration as a comma-separated list within $\boxed{...}$, where '1' represents an occupied seat and '0' represents an empty seat. For example: $\boxed{[1,1,0,0,1,0,1]}$.

\subsubsection{Variation}
\textbf{Actual Problem}\\
The International Mathematical Olympiad is being organized in Japan, where a folklore belief is that the number $4$ brings bad luck. The opening ceremony takes place at the Grand Theatre where each row has the capacity of $N=64$ seats. Give an example for how $m = 34$ contestants can be seated in a single row with the restriction that no two of them are $4$ seats apart (so that bad luck during the competition is avoided).

Output the sequence of the state of the seats as a comma-separated list in $\boxed{...}$, where a '1' signifies an occupied seat, and '0' signifies an empty one. Ex: $\boxed{[1,1,0,0,1,0,1]}$

\textbf{Revised Problem}\\
In Japan, where it is traditionally believed that the number $4$ is associated with misfortune, the International Mathematical Olympiad is being held. The opening event occurs at the Grand Theatre, where each row consists of $N=64$ seats. Provide an example of how $m = 34$ participants can be arranged in a single row, ensuring that no two participants are seated exactly $4$ seats apart, thereby avoiding the supposed bad luck.

Present the arrangement of seats as a comma-separated sequence within $\boxed{...}$, where '1' indicates a seat is occupied and '0' indicates it is empty. For example: $\boxed{[1,1,0,0,1,0,1]}$.

\subsubsection{Variation}
\textbf{Actual Problem}\\
The International Mathematical Olympiad is being organized in Japan, where a folklore belief is that the number $4$ brings bad luck. The opening ceremony takes place at the Grand Theatre where each row has the capacity of $N=44$ seats. Give an example for how $m = 24$ contestants can be seated in a single row with the restriction that no two of them are $4$ seats apart (so that bad luck during the competition is avoided).

Output the sequence of the state of the seats as a comma-separated list in $\boxed{...}$, where a '1' signifies an occupied seat, and '0' signifies an empty one. Ex: $\boxed{[1,1,0,0,1,0,1]}$

\textbf{Revised Problem}\\
In Japan, where a superstition exists regarding the number $4$, the International Mathematical Olympiad is being held, and the seating arrangement for the opening ceremony must adhere to this belief. There are $N=44$ seats in a row at the Grand Theatre, and $m = 24$ contestants need to be seated such that no two contestants are exactly 4 seats apart. Design a seating arrangement for one row in which this condition is met, ensuring good fortune during the competition.

Present the seating arrangement as a sequence of seat states in a comma-separated list enclosed in $\boxed{...}$, where '1' represents a seat occupied by a contestant, and '0' represents an empty seat. For example: $\boxed{[1,1,0,0,1,0,1]}$.

\subsection{bmo-shortlist-2015-n7}
\subsubsection{Variation}
\textbf{Actual Problem}\\
A positive integer $m$ shall be called an anagram of positive $n$ if every digit $a$ appears as many times in the decimal representation of $m$ as it appears in the decimal representation of $n$. Find $N=5$ different combinations of $4$ different positive integers such that each of the four is the anagram of the sum of the other $3$.

Output the sequences as comma-separated tuples inside of \boxed, e.g. \boxed{(1, 2, 3, 4), (7, 10, 11, 12), (2, 3, 4, 9)}.

\textbf{Revised Problem}\\
Consider a positive integer $m$ to be an anagram of another positive integer $n$ if every digit in the decimal representation of $n$ appears the same number of times in the decimal representation of $m$. Identify $N=5$ unique sets, each containing $4$ distinct positive integers, where each integer in the set is an anagram of the sum of the other three integers.

Present the sets as comma-separated tuples enclosed within \boxed, such as \boxed{(1, 2, 3, 4), (7, 10, 11, 12), (2, 3, 4, 9)}.

\subsubsection{Variation}
\textbf{Actual Problem}\\
A positive integer $m$ shall be called an anagram of positive $n$ if every digit $a$ appears as many times in the decimal representation of $m$ as it appears in the decimal representation of $n$. Find $N=8$ different combinations of $4$ different positive integers such that each of the four is the anagram of the sum of the other $3$.

Output the sequences as comma-separated tuples inside of \boxed, e.g. \boxed{(1, 2, 3, 4), (7, 10, 11, 12), (2, 3, 4, 9)}.

\textbf{Revised Problem}\\
A positive integer $m$ is defined as an anagram of another positive integer $n$ if the count of each digit $a$ in the decimal form of $m$ matches the count of $a$ in the decimal form of $n$. Determine $N=8$ distinct sets of $4$ unique positive integers such that every integer in each set is an anagram of the sum of the other three integers.

Present the sets as comma-separated tuples encapsulated within \boxed, for example, \boxed{(5, 6, 7, 8), (11, 14, 25, 28), (3, 6, 9, 12)}.

\subsubsection{Variation}
\textbf{Actual Problem}\\
A positive integer $m$ shall be called an anagram of positive $n$ if every digit $a$ appears as many times in the decimal representation of $m$ as it appears in the decimal representation of $n$. Find $N=3$ different combinations of $4$ different positive integers such that each of the four is the anagram of the sum of the other $3$.

Output the sequences as comma-separated tuples inside of \boxed, e.g. \boxed{(1, 2, 3, 4), (7, 10, 11, 12), (2, 3, 4, 9)}.

\textbf{Revised Problem}\\
A positive integer $x$ is termed an anagram of another positive integer $y$ if the decimal representation of $x$ consists of exactly the same digits as that of $y$, with the same frequency for each digit. Identify $N=3$ distinct sets of $4$ positive integers such that for each set, each integer is an anagram of the total sum of the other three integers in the set.

Present the sets as comma-separated tuples enclosed within \boxed, for example, \boxed{(1, 2, 3, 4), (7, 10, 11, 12), (2, 3, 4, 9)}.

\subsubsection{Variation}
\textbf{Actual Problem}\\
A positive integer $m$ shall be called an anagram of positive $n$ if every digit $a$ appears as many times in the decimal representation of $m$ as it appears in the decimal representation of $n$. Find $N=6$ different combinations of $4$ different positive integers such that each of the four is the anagram of the sum of the other $3$.

Output the sequences as comma-separated tuples inside of \boxed, e.g. \boxed{(1, 2, 3, 4), (7, 10, 11, 12), (2, 3, 4, 9)}.

\textbf{Revised Problem}\\
A positive integer $m$ is defined as an anagram of a positive integer $n$ if the digits that appear in $m$ are exactly the same in both type and frequency as those in $n$. Your task is to identify $N=6$ unique groups of $4$ distinct positive integers such that every integer in the group is an anagram of the total of the other three integers.

Present the groups as comma-separated tuples enclosed in \boxed, for example, \boxed{(1, 2, 3, 4), (7, 10, 11, 12), (2, 3, 4, 9)}.

\subsection{bmo-shortlist-2018-c1}
\subsubsection{Variation}
\textbf{Actual Problem}\\
$11$ tennis players take part in a championship. Before starting the championship, a commission puts the players in a row depending on how good they think the players are. During the championship, every player plays with every other player exactly once, and each match has a winner. A match is called surprising if the winner was rated lower by the commission. At the end of the tournament, players are arranged in a line based on the number of victories they have achieved. In the event of a tie, the commission's initial order is used to decide which player will be higher.
It turns out that the final order is exactly the same as the commission's initial order. 

Give an example of a tournament that exhibits 40 surprising matches.

Output the match results as a matrix where the entry at indices $i,j$ is 1 if player $i$ won against player $j$. Output the answer between \verb|\begin{array}{...}| and \verb|\end{array}| inside of $\boxed{...}$. For example, $\boxed{\begin{array}{ccc}0 & 1 & 1 \\ 0 & 0 & 0 \\ 0 & 1 & 0\end{array}}$.

\textbf{Revised Problem}\\
A championship features 11 tennis players. Prior to the tournament, a commission ranks the players in a line according to their perceived skill levels. Each player competes against every other player exactly once, with a winner determined in each match. A match is termed surprising if the player ranked lower by the commission wins. After the tournament, the players are ordered by the number of matches they won. In the case of a tie, the commission's initial ranking resolves the order.

The final ranking of the players aligns perfectly with the commission's initial ranking. Construct a scenario for this tournament where there are exactly 40 surprising matches.

Present the match outcomes using a matrix format where the element at position $i,j$ is 1 if player $i$ defeats player $j$. Arrange your response between \verb|\begin{array}{...}| and \verb|\end{array}| wrapped within $\boxed{...}$. For instance, $\boxed{\begin{array}{ccc}0 & 1 & 1 \\ 0 & 0 & 0 \\ 0 & 1 & 0\end{array}}$.

\subsubsection{Variation}
\textbf{Actual Problem}\\
$15$ tennis players take part in a championship. Before starting the championship, a commission puts the players in a row depending on how good they think the players are. During the championship, every player plays with every other player exactly once, and each match has a winner. A match is called surprising if the winner was rated lower by the commission. At the end of the tournament, players are arranged in a line based on the number of victories they have achieved. In the event of a tie, the commission's initial order is used to decide which player will be higher.
It turns out that the final order is exactly the same as the commission's initial order. 

Give an example of a tournament that exhibits 77 surprising matches.

Output the match results as a matrix where the entry at indices $i,j$ is 1 if player $i$ won against player $j$. Output the answer between \verb|\begin{array}{...}| and \verb|\end{array}| inside of $\boxed{...}$. For example, $\boxed{\begin{array}{ccc}0 & 1 & 1 \\ 0 & 0 & 0 \\ 0 & 1 & 0\end{array}}$.

\textbf{Revised Problem}\\
Fifteen tennis players participate in a championship. Prior to the tournament, a committee ranks the players in a sequence based on their perceived skill levels. Each player competes against every other player exactly once, with each match resulting in one winner. A match is deemed surprising if the victor was deemed inferior by the committee's initial ranking. At the conclusion of the tournament, players are ordered according to their total number of wins. If two players have the same number of victories, the committee's original order is used to break the tie. It turns out that the final ranking of the players is identical to the committee's initial ranking.

Provide an example of a tournament scenario that results in 77 surprising matches.

Present the match outcomes as a matrix where the entry at position $i,j$ is 1 if player $i$ defeated player $j$. Your answer should be enclosed within \verb|\begin{array}{...}| and \verb|\end{array}| inside of $\boxed{...}$. For instance, $\boxed{\begin{array}{ccc}0 & 1 & 1 \\ 0 & 0 & 0 \\ 0 & 1 & 0\end{array}}$.

\subsubsection{Variation}
\textbf{Actual Problem}\\
$9$ tennis players take part in a championship. Before starting the championship, a commission puts the players in a row depending on how good they think the players are. During the championship, every player plays with every other player exactly once, and each match has a winner. A match is called surprising if the winner was rated lower by the commission. At the end of the tournament, players are arranged in a line based on the number of victories they have achieved. In the event of a tie, the commission's initial order is used to decide which player will be higher.
It turns out that the final order is exactly the same as the commission's initial order. 

Give an example of a tournament that exhibits 26 surprising matches.

Output the match results as a matrix where the entry at indices $i,j$ is 1 if player $i$ won against player $j$. Output the answer between \verb|\begin{array}{...}| and \verb|\end{array}| inside of $\boxed{...}$. For example, $\boxed{\begin{array}{ccc}0 & 1 & 1 \\ 0 & 0 & 0 \\ 0 & 1 & 0\end{array}}$.

\textbf{Revised Problem}\\
In a tennis championship, there are $9$ participants. Initially, a panel ranks the players from best to worst in a single line based on their perceived abilities. As the championship progresses, every participant competes against all other players exactly once, resulting in definitive winners for each match. A match is deemed surprising if the player ranked lower by the panel emerges victorious. After all matches are played, players are realigned in order based on their total wins. For any ties in victories, the initial panel ranking is used to resolve positions. It is observed that the final order of players matches the initial ranking by the panel exactly.

Provide an example of tournament results that feature exactly 26 surprising matches.

Present the match outcomes in a matrix format, where the cell at position $i,j$ contains 1 if player $i$ defeated player $j$. Present the matrix within \verb|\begin{array}{...}| and \verb|\end{array}|, enclosed by $\boxed{...}$. For instance, $\boxed{\begin{array}{ccc}0 & 1 & 1 \\ 0 & 0 & 0 \\ 0 & 1 & 0\end{array}}$.

\subsubsection{Variation}
\textbf{Actual Problem}\\
$17$ tennis players take part in a championship. Before starting the championship, a commission puts the players in a row depending on how good they think the players are. During the championship, every player plays with every other player exactly once, and each match has a winner. A match is called surprising if the winner was rated lower by the commission. At the end of the tournament, players are arranged in a line based on the number of victories they have achieved. In the event of a tie, the commission's initial order is used to decide which player will be higher.
It turns out that the final order is exactly the same as the commission's initial order. 

Give an example of a tournament that exhibits 100 surprising matches.

Output the match results as a matrix where the entry at indices $i,j$ is 1 if player $i$ won against player $j$. Output the answer between \verb|\begin{array}{...}| and \verb|\end{array}| inside of $\boxed{...}$. For example, $\boxed{\begin{array}{ccc}0 & 1 & 1 \\ 0 & 0 & 0 \\ 0 & 1 & 0\end{array}}$.

\textbf{Revised Problem}\\
In a tennis championship, 17 players compete against each other. Before the tournament begins, a committee ranks these players from best to worst. Each player plays one match against every other player, with each match resulting in a winner. If a lower-ranked player beats a higher-ranked one, the match is deemed surprising. At the tournament's conclusion, players are ranked based on their total match wins. If two players have the same number of wins, the committee's initial ranking determines their final position. Surprisingly, the tournament's final ranking matches the committee's original ranking exactly.

Construct a tournament with exactly 100 surprising matches.

Present the match outcomes in a matrix format, where the entry at row $i$ and column $j$ is 1 if player $i$ defeats player $j$. Enclose your response inside \verb|\begin{array}{...}| and \verb|\end{array}| within $\boxed{...}$. For example, $\boxed{\begin{array}{ccc}0 & 1 & 1 \\ 0 & 0 & 0 \\ 0 & 1 & 0\end{array}}$.

\subsection{bmo-shortlist-2019-c1}
\subsubsection{Variation}
\textbf{Actual Problem}\\
$N=100$ couples are invited to a traditional Moldovan dance. The $2N=200$ people stand in a line, and then in a $\textit{step}$, many (not necessarily adjacent) swap positions. Find an ordering where there is a person that is not dancing next to their partner after $N-2=98$ steps, regardless of the steps performed.

Output the order as a comma-separated sequence of numbers from 1 to the number of couples, where each number appears exactly twice, inside a \boxed. Consider each 2 duplicate numbers to represent a given couple. For example: \boxed{( 1,2,3,3,1,2)}

\textbf{Revised Problem}\\
There are $N = 100$ pairs invited to a traditional Moldovan dance event. All $2N = 200$ participants are initially arranged in a single line. During each $\textit{step}$, multiple people (not necessarily next to each other) can exchange positions. You need to determine a lineup such that after $N-2 = 98$ steps, at least one pair remains not standing adjacent to each other, regardless of the swaps made.

Present the sequence as a comma-separated list of numbers ranging from 1 to the total number of pairs, where each number appears exactly twice, enclosed in a \boxed. Each pair is represented by two identical numbers. For instance: \boxed{( 1,2,3,3,1,2)}

\subsubsection{Variation}
\textbf{Actual Problem}\\
$N=54$ couples are invited to a traditional Moldovan dance. The $2N=108$ people stand in a line, and then in a $\textit{step}$, many (not necessarily adjacent) swap positions. Find an ordering where there is a person that is not dancing next to their partner after $N-2=52$ steps, regardless of the steps performed.

Output the order as a comma-separated sequence of numbers from 1 to the number of couples, where each number appears exactly twice, inside a \boxed. Consider each 2 duplicate numbers to represent a given couple. For example: \boxed{( 1,2,3,3,1,2)}

\textbf{Revised Problem}\\
There are $N=54$ pairs of individuals invited to participate in a traditional Moldovan dance. A total of $2N=108$ people line up in a sequence, and during each $\textit{step}$, any number of them (not necessarily next to each other) can exchange their positions. Determine a sequence in which, even after performing $N-2=52$ steps, there is at least one individual not standing next to their partner, regardless of the swaps executed.

Present the sequence as a list of numbers from 1 to the number of pairs, with each number appearing exactly twice, separated by commas, and enclosed within a \boxed. Each pair of identical numbers represents a couple. For example: \boxed{(1,2,3,3,1,2)}

\subsubsection{Variation}
\textbf{Actual Problem}\\
$N=22$ couples are invited to a traditional Moldovan dance. The $2N=44$ people stand in a line, and then in a $\textit{step}$, many (not necessarily adjacent) swap positions. Find an ordering where there is a person that is not dancing next to their partner after $N-2=20$ steps, regardless of the steps performed.

Output the order as a comma-separated sequence of numbers from 1 to the number of couples, where each number appears exactly twice, inside a \boxed. Consider each 2 duplicate numbers to represent a given couple. For example: \boxed{( 1,2,3,3,1,2)}

\textbf{Revised Problem}\\
In a traditional Moldovan dance, $N=22$ couples, making $2N=44$ individuals, assemble in a line. During each $\textit{step}$, any number of these individuals, not necessarily next to each other, can swap places. Determine a sequence such that after $N-2=20$ steps, there remains at least one individual who is not positioned adjacent to their partner, irrespective of how the steps are executed.

Present the sequence as a list of numbers from 1 to the number of couples, where each number is repeated exactly twice, formatted inside a \boxed. Each pair of identical numbers represents a couple. For example: \boxed{(1, 2, 3, 3, 1, 2)}

\subsubsection{Variation}
\textbf{Actual Problem}\\
$N=12$ couples are invited to a traditional Moldovan dance. The $2N=24$ people stand in a line, and then in a $\textit{step}$, many (not necessarily adjacent) swap positions. Find an ordering where there is a person that is not dancing next to their partner after $N-2=10$ steps, regardless of the steps performed.

Output the order as a comma-separated sequence of numbers from 1 to the number of couples, where each number appears exactly twice, inside a \boxed. Consider each 2 duplicate numbers to represent a given couple. For example: \boxed{( 1,2,3,3,1,2)}

\textbf{Revised Problem}\\
At a traditional Moldovan dance event, there are $N=12$ couples invited. This means $2N=24$ individuals need to arrange themselves in a line. During this dance, they can undergo a series of $\textit{steps}$ where multiple pairs of individuals can swap their positions. Your task is to find a sequence in which, after $N-2=10$ steps, there is at least one person who is not standing next to their partner, regardless of how the swapping steps are executed.

Provide the final arrangement as a sequence of numbers from 1 to the total number of couples, with each number appearing exactly twice, separated by commas within a \boxed. Each pair of identical numbers represents a couple. Example: \boxed{(1,2,3,3,1,2)}.

\subsection{bmo-shortlist-2019-c2}
\subsubsection{Variation}
\textbf{Actual Problem}\\
Give a $5\times 5$ array consisting of pairwise distinct natural numbers from $1$ to $25$ that contains a $2\times 2$ subarray of numbers whose sum is no more than $45$.

Output the answer between \verb|\begin{array}{...}| and \verb|\end{array}| inside of $\boxed{...}$. For example, $\boxed{\begin{array}{ccc}1 & 2 & 3 \\ 4 & 5 & 6 \\ 7 & 8 & 9\end{array}}$.

\textbf{Revised Problem}\\
Design a $5\times 5$ grid filled with unique natural numbers ranging between $1$ and $25$. Within this grid, ensure that there exists a $2\times 2$ block of numbers where the total sum does not exceed $45$.

Present the solution using \verb|\begin{array}{...}| and \verb|\end{array}|, encapsulated within $\boxed{...}$. For instance, $\boxed{\begin{array}{ccc}1 & 2 & 3 \\ 4 & 5 & 6 \\ 7 & 8 & 9\end{array}}$.

\subsubsection{Variation}
\textbf{Actual Problem}\\
Give a $5\times 5$ array consisting of pairwise distinct natural numbers from $1$ to $25$ that contains a $2\times 2$ subarray of numbers whose sum is no more than $51$.

Output the answer between \verb|\begin{array}{...}| and \verb|\end{array}| inside of $\boxed{...}$. For example, $\boxed{\begin{array}{ccc}1 & 2 & 3 \\ 4 & 5 & 6 \\ 7 & 8 & 9\end{array}}$.

\textbf{Revised Problem}\\
Construct a $5\times 5$ matrix using all the natural numbers from $1$ to $25$, each appearing exactly once, such that there exists at least one $2\times 2$ sub-matrix within it that has a total sum of no more than 51.

Present your solution enclosed by \verb|\begin{array}{...}| and \verb|\end{array}| within $\boxed{...}$. For instance, $\boxed{\begin{array}{ccc}1 & 2 & 3 \\ 4 & 5 & 6 \\ 7 & 8 & 9\end{array}}$.

\subsubsection{Variation}
\textbf{Actual Problem}\\
Give a $5\times 5$ array consisting of pairwise distinct natural numbers from $1$ to $25$ that contains a $2\times 2$ subarray of numbers whose sum is no more than $47$.

Output the answer between \verb|\begin{array}{...}| and \verb|\end{array}| inside of $\boxed{...}$. For example, $\boxed{\begin{array}{ccc}1 & 2 & 3 \\ 4 & 5 & 6 \\ 7 & 8 & 9\end{array}}$.

\textbf{Revised Problem}\\
Construct a $5\times 5$ grid using distinct natural numbers ranging from $1$ to $25$. Ensure that the grid includes at least one $2\times 2$ section where the total sum of its numbers is at most $47$.

Present the solution within the syntax \verb|\begin{array}{...}| and \verb|\end{array}| enclosed by $\boxed{...}$. For instance, $\boxed{\begin{array}{ccc}1 & 2 & 3 \\ 4 & 5 & 6 \\ 7 & 8 & 9\end{array}}$.

\subsubsection{Variation}
\textbf{Actual Problem}\\
Give a $5\times 5$ array consisting of pairwise distinct natural numbers from $1$ to $25$ that contains a $2\times 2$ subarray of numbers whose sum is no more than $48$.

Output the answer between \verb|\begin{array}{...}| and \verb|\end{array}| inside of $\boxed{...}$. For example, $\boxed{\begin{array}{ccc}1 & 2 & 3 \\ 4 & 5 & 6 \\ 7 & 8 & 9\end{array}}$.

\textbf{Revised Problem}\\
Construct a $5\times 5$ matrix filled with unique natural numbers ranging from 1 to 25 such that at least one $2\times 2$ section of this matrix has a sum that does not exceed 48.

Present your solution enclosed within \verb|\begin{array}{...}| and \verb|\end{array}| inside $\boxed{...}$. For instance, $\boxed{\begin{array}{ccc}1 & 2 & 3 \\ 4 & 5 & 6 \\ 7 & 8 & 9\end{array}}$.

\section{bulgarian}
\subsection{bulgarian-ifym-2015-p7-d4-8th}
\subsubsection{Variation}
\textbf{Actual Problem}\\
Show four integers $a, b, c, d$ each of which has an absolute value greater than $10000001$ such that $\frac{1}{a} + \frac{1}{b} + \frac{1}{c} + \frac{1}{d} = \frac{1}{abcd}$.

Output the answer as a comma separated list inside of $\boxed{...}$. For example $\boxed{1, 2, 3}$.

\textbf{Revised Problem}\\
Identify four integers $p, q, r, s$ such that each has an absolute value exceeding $10000001$, and they satisfy the equation $\frac{1}{p} + \frac{1}{q} + \frac{1}{r} + \frac{1}{s} = \frac{1}{pqrs}$.

Present your solution as a comma-separated sequence enclosed in $\boxed{...}$. For instance, use $\boxed{1, 2, 3}$.

\subsubsection{Variation}
\textbf{Actual Problem}\\
Show four integers $a, b, c, d$ each of which has an absolute value greater than $131072$ such that $\frac{1}{a} + \frac{1}{b} + \frac{1}{c} + \frac{1}{d} = \frac{1}{abcd}$.

Output the answer as a comma separated list inside of $\boxed{...}$. For example $\boxed{1, 2, 3}$.

\textbf{Revised Problem}\\
Identify four integers $x, y, z, w$ such that the absolute value of each integer exceeds $131072$, and they satisfy the equation $\frac{1}{x} + \frac{1}{y} + \frac{1}{z} + \frac{1}{w} = \frac{1}{xyzw}$.

Present the answer as a sequence of numbers separated by commas within $\boxed{...}$. For example, $\boxed{1, 2, 3}$.

\subsubsection{Variation}
\textbf{Actual Problem}\\
Show four integers $a, b, c, d$ each of which has an absolute value greater than $512$ such that $\frac{1}{a} + \frac{1}{b} + \frac{1}{c} + \frac{1}{d} = \frac{1}{abcd}$.

Output the answer as a comma separated list inside of $\boxed{...}$. For example $\boxed{1, 2, 3}$.

\textbf{Revised Problem}\\
Identify four integers \( a, b, c, \) and \( d \) where the absolute value of each integer exceeds 512, and they satisfy the equation \(\frac{1}{a} + \frac{1}{b} + \frac{1}{c} + \frac{1}{d} = \frac{1}{abcd}\).

Present your solution as a list of integers separated by commas within a box. For example, \(\boxed{1, 2, 3}\).

\subsubsection{Variation}
\textbf{Actual Problem}\\
Show four integers $a, b, c, d$ each of which has an absolute value greater than $64$ such that $\frac{1}{a} + \frac{1}{b} + \frac{1}{c} + \frac{1}{d} = \frac{1}{abcd}$.

Output the answer as a comma separated list inside of $\boxed{...}$. For example $\boxed{1, 2, 3}$.

\textbf{Revised Problem}\\
Identify four integers \(a, b, c, d\) such that the absolute value of each integer exceeds 64, and they satisfy the equation \(\frac{1}{a} + \frac{1}{b} + \frac{1}{c} + \frac{1}{d} = \frac{1}{abcd}\).

Write the solution as a comma-separated list enclosed within \(\boxed{...}\). For instance, \(\boxed{1, 2, 3}\).

\subsection{bulgarian-ifym-2022-d1-p6-8th}
\subsubsection{Variation}
\textbf{Actual Problem}\\
Show that there exists a sequence of $20$ natural numbers with the following property: each of the numbers in the sequence is divisible by $343$ and each number after the first is obtained from the previous one by crossing out some of its non-zero digits.

Output the answer as a comma separated list inside of $\boxed{...}$. For example $\boxed{1, 2, 3}$.

\textbf{Revised Problem}\\
Demonstrate the existence of a sequence consisting of 20 natural numbers that satisfies the following conditions: each number in the sequence is a multiple of 343, and each subsequent number in the sequence is formed by deleting some non-zero digits from the preceding number.

Present the solution as a comma-separated list enclosed within $\boxed{...}$. For instance, $\boxed{1, 2, 3}$.

\subsubsection{Variation}
\textbf{Actual Problem}\\
Show that there exists a sequence of $34$ natural numbers with the following property: each of the numbers in the sequence is divisible by $6321$ and each number after the first is obtained from the previous one by crossing out some of its non-zero digits.

Output the answer as a comma separated list inside of $\boxed{...}$. For example $\boxed{1, 2, 3}$.

\textbf{Revised Problem}\\
Demonstrate the existence of a sequence comprising 34 natural numbers such that each number in the sequence is a multiple of 6321, and each number (except the first) is formed by deleting some non-zero digits from the preceding number.

Present the solution as a list separated by commas and enclosed within $\boxed{...}$. For instance, $\boxed{1, 2, 3}$.

\subsubsection{Variation}
\textbf{Actual Problem}\\
Show that there exists a sequence of $28$ natural numbers with the following property: each of the numbers in the sequence is divisible by $2211$ and each number after the first is obtained from the previous one by crossing out some of its non-zero digits.

Output the answer as a comma separated list inside of $\boxed{...}$. For example $\boxed{1, 2, 3}$.

\textbf{Revised Problem}\\
Demonstrate the existence of a sequence composed of $28$ natural numbers that satisfies the following condition: each number in the sequence is a multiple of $2211$, and each subsequent number in the sequence is formed by removing one or more non-zero digits from its predecessor.

Present your solution as a list of numbers separated by commas and enclosed in a $\boxed{...}$. For instance, $\boxed{1, 2, 3}$.

\subsubsection{Variation}
\textbf{Actual Problem}\\
Show that there exists a sequence of $12$ natural numbers with the following property: each of the numbers in the sequence is divisible by $936$ and each number after the first is obtained from the previous one by crossing out some of its non-zero digits.

Output the answer as a comma separated list inside of $\boxed{...}$. For example $\boxed{1, 2, 3}$.

\textbf{Revised Problem}\\
Demonstrate that it is possible to construct a sequence consisting of 12 natural numbers such that each number in the sequence is divisible by 936, and each subsequent number is derived from the preceding one by eliminating one or more of its non-zero digits.

Present the solution as a list separated by commas within a $\boxed{...}$ structure. For example, $\boxed{1, 2, 3}$.

\subsection{bulgarian-mo-r2-2021-8-4}
\subsubsection{Variation}
\textbf{Actual Problem}\\
We denote $$ A_{n} = 6561^{n} - 6 \cdot 729^{n} - 4 \cdot 81^{n} + 2 \cdot 3^{2 n + 3} - 45. $$ Show by example that there exists an $n>0$ such that $2^{30}$ divides $A_{n}$

Output the answer as an integer inside of $\boxed{...}$. For example $\boxed{123}$.

\textbf{Revised Problem}\\
Consider the expression \( A_{n} = 6561^{n} - 6 \times 729^{n} - 4 \times 81^{n} + 2 \times 3^{2n + 3} - 45 \). Demonstrate with an example that there is a positive integer \( n \) such that \( A_n \) is divisible by \( 2^{30} \).

Express the solution as a number enclosed within a box, like this: \(\boxed{123}\).

\subsubsection{Variation}
\textbf{Actual Problem}\\
We denote $$ A_{n} = 6561^{n} - 6 \cdot 729^{n} - 4 \cdot 81^{n} + 2 \cdot 3^{2 n + 3} - 45. $$ Show by example that there exists an $n>0$ such that $2^{54}$ divides $A_{n}$

Output the answer as an integer inside of $\boxed{...}$. For example $\boxed{123}$.

\textbf{Revised Problem}\\
Consider the expression \( B_n = 6561^n - 6 \times 729^n - 4 \times 81^n + 2 \times 3^{2n+3} - 45 \). Demonstrate that there is at least one positive integer \( n \) for which \( 2^{54} \) is a factor of \( B_n \).

Present the solution as an integer enclosed in \(\boxed{...}\). For instance, \(\boxed{123}\).

\subsubsection{Variation}
\textbf{Actual Problem}\\
We denote $$ A_{n} = 6561^{n} - 6 \cdot 729^{n} - 4 \cdot 81^{n} + 2 \cdot 3^{2 n + 3} - 45. $$ Show by example that there exists an $n>0$ such that $2^{38}$ divides $A_{n}$

Output the answer as an integer inside of $\boxed{...}$. For example $\boxed{123}$.

\textbf{Revised Problem}\\
Define the sequence \( B_n = 6561^n - 6 \times 729^n - 4 \times 81^n + 2 \times 3^{2n+3} - 45 \). Demonstrate that there is a positive integer \( n \) such that \( B_n \) is divisible by \( 2^{38} \).

Present the answer as an integer enclosed in \(\boxed{...}\). For example, \(\boxed{123}\).

\subsubsection{Variation}
\textbf{Actual Problem}\\
We denote $$ A_{n} = 6561^{n} - 6 \cdot 729^{n} - 4 \cdot 81^{n} + 2 \cdot 3^{2 n + 3} - 45. $$ Show by example that there exists an $n>0$ such that $2^{33}$ divides $A_{n}$

Output the answer as an integer inside of $\boxed{...}$. For example $\boxed{123}$.

\textbf{Revised Problem}\\
Consider the sequence defined by \( B_n = 6561^n - 6 \times 729^n - 4 \times 81^n + 2 \times 3^{2n + 3} - 45 \). Demonstrate with an example that there is an integer \( n > 0 \) such that \( 2^{33} \) is a divisor of \( B_n \).

Present the solution as an integer enclosed within a box, such as \(\boxed{123}\).

\subsection{bulgarian-pms-10-4}
\subsubsection{Variation}
\textbf{Actual Problem}\\
We will call a graph $G$ a divisibility graph if a distinct natural number can be written in each of its vertices such that its edges correspond to all pairs $(u, v)$ for which either $\frac{u}{v}$ or $\frac{v}{u}$ is an integer. Exhibit a divisibility graph with exactly $15$ vertices and $43$ edges.

Output the answer as a pair of lists inside $\boxed{...}$. The first list should contain the number assigned to each vertex. The second list should contain the edges, represented as pairs of 1-based indices, formatted as lists of length 2. Format the final output as $\boxed{[vertices, edges]}$. For example, $\boxed{[[1,2,3], [[1, 2], [1, 3]]]}$.

\textbf{Revised Problem}\\
Consider a graph $G$ which we define as a divisibility graph if it is possible to assign a unique natural number to each vertex, such that for any two vertices connected by an edge, the number at one vertex divides the number at the other. Find a divisibility graph that consists of 15 vertices and exactly 43 edges.

Present your solution as a pair of lists enclosed within $\boxed{...}$. The first list should detail the numbers assigned to each vertex. The second list should describe the edges, with each edge represented as a pair of 1-based indices, formatted as lists of length 2. The final solution should be in the format $\boxed{[vertices, edges]}$. For instance, $\boxed{[[1,2,3], [[1, 2], [1, 3]]]}$.

\subsubsection{Variation}
\textbf{Actual Problem}\\
We will call a graph $G$ a divisibility graph if a distinct natural number can be written in each of its vertices such that its edges correspond to all pairs $(u, v)$ for which either $\frac{u}{v}$ or $\frac{v}{u}$ is an integer. Exhibit a divisibility graph with exactly $16$ vertices and $83$ edges.

Output the answer as a pair of lists inside $\boxed{...}$. The first list should contain the number assigned to each vertex. The second list should contain the edges, represented as pairs of 1-based indices, formatted as lists of length 2. Format the final output as $\boxed{[vertices, edges]}$. For example, $\boxed{[[1,2,3], [[1, 2], [1, 3]]]}$.

\textbf{Revised Problem}\\
Consider a graph $G$ that qualifies as a divisibility graph if you can assign a unique natural number to each vertex. The edges of the graph connect vertices if and only if for the corresponding pair of numbers $(u, v)$, either $u$ divides $v$ or $v$ divides $u$. Demonstrate a divisibility graph having precisely $16$ vertices and $83$ edges.

Present your solution as a pair of lists enclosed in $\boxed{...}$. The first list should represent the number assigned to each vertex. The second list should consist of the edges, depicted as pairs of indices (1-based), each formatted as a list of two elements. The final output should look like $\boxed{[vertices, edges]}$. For instance, $\boxed{[[1,2,3], [[1, 2], [1, 3]]]}$.

\subsubsection{Variation}
\textbf{Actual Problem}\\
We will call a graph $G$ a divisibility graph if a distinct natural number can be written in each of its vertices such that its edges correspond to all pairs $(u, v)$ for which either $\frac{u}{v}$ or $\frac{v}{u}$ is an integer. Exhibit a divisibility graph with exactly $12$ vertices and $66$ edges.

Output the answer as a pair of lists inside $\boxed{...}$. The first list should contain the number assigned to each vertex. The second list should contain the edges, represented as pairs of 1-based indices, formatted as lists of length 2. Format the final output as $\boxed{[vertices, edges]}$. For example, $\boxed{[[1,2,3], [[1, 2], [1, 3]]]}$.

\textbf{Revised Problem}\\
Consider a graph $G$ to be a divisibility graph if we can assign a unique natural number to each vertex such that an edge exists between every pair of vertices $(u, v)$ whenever either $\frac{u}{v}$ or $\frac{v}{u}$ results in an integer. Construct a divisibility graph that consists of exactly 12 vertices and 66 edges.

Present the solution as a pair of lists encapsulated within $\boxed{...}$. The first list should specify the natural number allocated to each vertex. The second list should detail the edges, depicted as pairs of indices (starting from 1), formatted as lists containing two elements each. The final presentation should look like $\boxed{[vertices, edges]}$. For instance, $\boxed{[[1,2,3], [[1, 2], [1, 3]]]}$.

\subsubsection{Variation}
\textbf{Actual Problem}\\
We will call a graph $G$ a divisibility graph if a distinct natural number can be written in each of its vertices such that its edges correspond to all pairs $(u, v)$ for which either $\frac{u}{v}$ or $\frac{v}{u}$ is an integer. Exhibit a divisibility graph with exactly $10$ vertices and $32$ edges.

Output the answer as a pair of lists inside $\boxed{...}$. The first list should contain the number assigned to each vertex. The second list should contain the edges, represented as pairs of 1-based indices, formatted as lists of length 2. Format the final output as $\boxed{[vertices, edges]}$. For example, $\boxed{[[1,2,3], [[1, 2], [1, 3]]]}$.

\textbf{Revised Problem}\\
A graph $G$ is termed as a divisibility graph if we can assign a unique natural number to each of its vertices such that there is an edge between two vertices $(u, v)$ if either $\frac{u}{v}$ or $\frac{v}{u}$ results in an integer. Construct a divisibility graph that consists of 10 vertices and contains 32 edges.

Present the solution as a pair of lists enclosed in $\boxed{...}$. The first list should detail the natural numbers assigned to each vertex. The second list should describe the edges, represented as pairs of indices starting from 1, formatted as lists of size 2. Structure the final output as $\boxed{[vertices, edges]}$. For example, $\boxed{[[1,2,3], [[1, 2], [1, 3]]]}$.

\subsection{bulgarian-pms-2008-8-3}
\subsubsection{Variation}
\textbf{Actual Problem}\\
For every natural number $n > 1$, let $R_{n}$ denote the largest possible remainder from the division of an $n$-digit natural number by the sum of its digits. There are infinitely many natural numbers $n$ for which $R_{n} = 9 n - 2$. Exhibit $5$ such $n$, and the corresponding $n$-digit number for which this is satisfied.

Output a list of tuples, containing the numbers $n$ and the corresponding $n$-digit number for which $R_{n} = 9 n - 2$. Output the sequences as a list of comma-separated tuples inside of \boxed, e.g. \boxed{(2, 19), (3, 123)}.

\textbf{Revised Problem}\\
For every natural number \( n > 1 \), define \( R_{n} \) as the highest remainder possible when an \( n \)-digit natural number is divided by the sum of its digits. There are an infinite number of natural numbers \( n \) for which \( R_{n} = 9n - 2 \). Identify 5 such \( n \), along with the \( n \)-digit number where this condition holds.

Provide an answer as a list of tuples, each containing a number \( n \) and the matching \( n \)-digit number where \( R_{n} = 9n - 2 \). Format your response in a list of comma-separated tuples within \boxed, such as \boxed{(2, 19), (3, 123)}.

\subsubsection{Variation}
\textbf{Actual Problem}\\
For every natural number $n > 1$, let $R_{n}$ denote the largest possible remainder from the division of an $n$-digit natural number by the sum of its digits. There are infinitely many natural numbers $n$ for which $R_{n} = 9 n - 2$. Exhibit $6$ such $n$, and the corresponding $n$-digit number for which this is satisfied.

Output a list of tuples, containing the numbers $n$ and the corresponding $n$-digit number for which $R_{n} = 9 n - 2$. Output the sequences as a list of comma-separated tuples inside of \boxed, e.g. \boxed{(2, 19), (3, 123)}.

\textbf{Revised Problem}\\
Consider each natural number $n > 1$. Define $R_{n}$ as the greatest remainder possible when an $n$-digit natural number is divided by the sum of its digits. There exist infinitely many values of $n$ where $R_{n}$ equals $9n - 2$. Identify 6 such values of $n$, along with a corresponding $n$-digit number that satisfies this condition.

Present your answers as a collection of tuples, where each tuple consists of a number $n$ and an associated $n$-digit number such that $R_{n} = 9n - 2$. Format the output as a list of tuples enclosed in \boxed, for example, \boxed{(2, 19), (3, 123)}.

\subsubsection{Variation}
\textbf{Actual Problem}\\
For every natural number $n > 1$, let $R_{n}$ denote the largest possible remainder from the division of an $n$-digit natural number by the sum of its digits. There are infinitely many natural numbers $n$ for which $R_{n} = 9 n - 2$. Exhibit $4$ such $n$, and the corresponding $n$-digit number for which this is satisfied.

Output a list of tuples, containing the numbers $n$ and the corresponding $n$-digit number for which $R_{n} = 9 n - 2$. Output the sequences as a list of comma-separated tuples inside of \boxed, e.g. \boxed{(2, 19), (3, 123)}.

\textbf{Revised Problem}\\
Consider all natural numbers \( n > 1 \). Define \( R_n \) as the maximum remainder obtained from dividing an \( n \)-digit natural number by the sum of its digits. There exist infinitely many natural numbers \( n \) such that \( R_n = 9n - 2 \). Identify four such values of \( n \), and provide the corresponding \( n \)-digit natural number that achieves this remainder.

Present your solution as a list of tuples, where each tuple consists of a number \( n \) and the corresponding \( n \)-digit number satisfying \( R_n = 9n - 2 \). Format your answer as a series of comma-separated tuples enclosed within \boxed, for instance, \boxed{(2, 19), (3, 123)}.

\subsubsection{Variation}
\textbf{Actual Problem}\\
For every natural number $n > 1$, let $R_{n}$ denote the largest possible remainder from the division of an $n$-digit natural number by the sum of its digits. There are infinitely many natural numbers $n$ for which $R_{n} = 9 n - 2$. Exhibit $3$ such $n$, and the corresponding $n$-digit number for which this is satisfied.

Output a list of tuples, containing the numbers $n$ and the corresponding $n$-digit number for which $R_{n} = 9 n - 2$. Output the sequences as a list of comma-separated tuples inside of \boxed, e.g. \boxed{(2, 19), (3, 123)}.

\textbf{Revised Problem}\\
For any natural number \( n > 1 \), define \( R_{n} \) as the greatest remainder achievable when dividing an \( n \)-digit natural number by the sum of its digits. There are infinitely many values of \( n \) such that \( R_{n} = 9n - 2 \). Identify three distinct values of \( n \) and provide the corresponding \( n \)-digit number for each, which satisfies this condition.

Display a list of tuples, where each tuple contains a number \( n \) and the corresponding \( n \)-digit number such that \( R_{n} = 9n - 2 \). Present the list as comma-separated tuples enclosed in \boxed, for example, \boxed{(2, 19), (3, 123)}.

\subsection{bulgarian-pms-2021-10-3}
\subsubsection{Variation}
\textbf{Actual Problem}\\
Show that the numbers $0,1, \ldots, 15^{2}-1$ can be arranged in a table with $15$ rows and $15$ columns such that each quotient and each remainder obtained by dividing these numbers by $15$ appears exactly once in each row and column.

Output the answer between \verb|\begin{array}{...}| and \verb|\end{array}| inside of $\boxed{...}$. For example, $\boxed{\begin{array}{ccc}1 & 2 & 3 \\ 4 & 5 & 6 \\ 7 & 8 & 9\end{array}}$.

\textbf{Revised Problem}\\
Demonstrate that the sequence of numbers from \(0\) to \(15^2 - 1\) can be placed into a 15x15 grid such that each number in the form \(n = 15q + r\) appears uniquely per row and column, where \(q\) is the quotient and \(r\) is the remainder when divided by 15.

Present your solution within the format \verb|\begin{array}{...}| and \verb|\end{array}| encapsulated in $\boxed{...}$. For illustration, $\boxed{\begin{array}{ccc}1 & 2 & 3 \\ 4 & 5 & 6 \\ 7 & 8 & 9\end{array}}$.

\subsubsection{Variation}
\textbf{Actual Problem}\\
Show that the numbers $0,1, \ldots, 13^{2}-1$ can be arranged in a table with $13$ rows and $13$ columns such that each quotient and each remainder obtained by dividing these numbers by $13$ appears exactly once in each row and column.

Output the answer between \verb|\begin{array}{...}| and \verb|\end{array}| inside of $\boxed{...}$. For example, $\boxed{\begin{array}{ccc}1 & 2 & 3 \\ 4 & 5 & 6 \\ 7 & 8 & 9\end{array}}$.

\textbf{Revised Problem}\\
Demonstrate that it is possible to organize the numbers from $0$ to $13^2 - 1$ into a $13 \times 13$ grid such that, in every row and every column, each quotient and each remainder obtained from dividing these numbers by $13$ appear exactly once.

Present your solution within \verb|\begin{array}{...}| and \verb|\end{array}| wrapped inside of $\boxed{...}$. For instance, $\boxed{\begin{array}{ccc}1 & 2 & 3 \\ 4 & 5 & 6 \\ 7 & 8 & 9\end{array}}$.

\subsubsection{Variation}
\textbf{Actual Problem}\\
Show that the numbers $0,1, \ldots, 9^{2}-1$ can be arranged in a table with $9$ rows and $9$ columns such that each quotient and each remainder obtained by dividing these numbers by $9$ appears exactly once in each row and column.

Output the answer between \verb|\begin{array}{...}| and \verb|\end{array}| inside of $\boxed{...}$. For example, $\boxed{\begin{array}{ccc}1 & 2 & 3 \\ 4 & 5 & 6 \\ 7 & 8 & 9\end{array}}$.

\textbf{Revised Problem}\\
Prove that it is possible to organize the numbers from $0$ to $80$ in a $9 \times 9$ grid such that each row and column contains each possible quotient and remainder when these numbers are divided by $9$ exactly once.

Present your solution enclosed between \verb|\begin{array}{...}| and \verb|\end{array}| within a $\boxed{...}$ structure. For instance, $\boxed{\begin{array}{ccc}1 & 2 & 3 \\ 4 & 5 & 6 \\ 7 & 8 & 9\end{array}}$.

\subsubsection{Variation}
\textbf{Actual Problem}\\
Show that the numbers $0,1, \ldots, 7^{2}-1$ can be arranged in a table with $7$ rows and $7$ columns such that each quotient and each remainder obtained by dividing these numbers by $7$ appears exactly once in each row and column.

Output the answer between \verb|\begin{array}{...}| and \verb|\end{array}| inside of $\boxed{...}$. For example, $\boxed{\begin{array}{ccc}1 & 2 & 3 \\ 4 & 5 & 6 \\ 7 & 8 & 9\end{array}}$.

\textbf{Revised Problem}\\
Demonstrate that the integers $0, 1, \ldots, 48$ can be organized into a $7 \times 7$ grid such that within each row and each column, every quotient and remainder from dividing these numbers by $7$ appears exactly once.

Present your solution enclosed within \verb|\begin{array}{...}| and \verb|\end{array}| inside of $\boxed{...}$. For instance, $\boxed{\begin{array}{ccc}1 & 2 & 3 \\ 4 & 5 & 6 \\ 7 & 8 & 9\end{array}}$.

\subsection{bulgarian-pms-2022-10-p3}
\subsubsection{Variation}
\textbf{Actual Problem}\\
The natural numbers $p, q$ are such that for every real number $x$

$$
(x+1)^{p}(x-3)^{q}=x^{n}+a_{1} x^{n-1}+a_{2} x^{n-2}+\cdots+a_{n-1} x+a_{n},
$$

where $n=p+q$ and $a_{1}, \ldots, a_{n}$ are real numbers. There are infinitely many pairs $(p, q)$ for which $a_{1}=a_{2}$. Exhibit $20$ such pairs.

Output the sequences as a list of comma-separated tuples inside of \boxed, e.g. \boxed{[(2, 19), (3, 123)]}.

\textbf{Revised Problem}\\
The natural numbers $p, q$ are such that for every real number $x$

$$
(x+1)^{p}(x-3)^{q}=x^{n}+a_{1} x^{n-1}+a_{2} x^{n-2}+\cdots+a_{n-1} x+a_{n},
$$

where $n=p+q$ and $a_{1}, \ldots, a_{n}$ are real numbers. There are infinitely many pairs $(p, q)$ for which $a_{1}=a_{2}$. Exhibit $20$ such pairs.

Provide the list of pairs in the format of comma-separated tuples inside \boxed, such as \boxed{[(2, 19), (3, 123)]}.

\subsubsection{Variation}
\textbf{Actual Problem}\\
The natural numbers $p, q$ are such that for every real number $x$

$$
(x+1)^{p}(x-3)^{q}=x^{n}+a_{1} x^{n-1}+a_{2} x^{n-2}+\cdots+a_{n-1} x+a_{n},
$$

where $n=p+q$ and $a_{1}, \ldots, a_{n}$ are real numbers. There are infinitely many pairs $(p, q)$ for which $a_{1}=a_{2}$. Exhibit $59$ such pairs.

Output the sequences as a list of comma-separated tuples inside of \boxed, e.g. \boxed{[(2, 19), (3, 123)]}.

\textbf{Revised Problem}\\
The natural numbers $p, q$ are such that for every real number $x$

$$
(x+1)^{p}(x-3)^{q}=x^{n}+a_{1} x^{n-1}+a_{2} x^{n-2}+\cdots+a_{n-1} x+a_{n},
$$

where $n=p+q$ and $a_{1}, \ldots, a_{n}$ are real numbers. There are infinitely many pairs $(p, q)$ for which $a_{1}=a_{2}$. Exhibit $59$ such pairs.

Present the pairs as a list of tuples, each separated by a comma and enclosed in \boxed, for example, \boxed{[(2, 19), (3, 123)]}.

\subsubsection{Variation}
\textbf{Actual Problem}\\
The natural numbers $p, q$ are such that for every real number $x$

$$
(x+1)^{p}(x-3)^{q}=x^{n}+a_{1} x^{n-1}+a_{2} x^{n-2}+\cdots+a_{n-1} x+a_{n},
$$

where $n=p+q$ and $a_{1}, \ldots, a_{n}$ are real numbers. There are infinitely many pairs $(p, q)$ for which $a_{1}=a_{2}$. Exhibit $27$ such pairs.

Output the sequences as a list of comma-separated tuples inside of \boxed, e.g. \boxed{[(2, 19), (3, 123)]}.

\textbf{Revised Problem}\\
The natural numbers $p, q$ are such that for every real number $x$

$$
(x+1)^{p}(x-3)^{q}=x^{n}+a_{1} x^{n-1}+a_{2} x^{n-2}+\cdots+a_{n-1} x+a_{n},
$$

where $n=p+q$ and $a_{1}, \ldots, a_{n}$ are real numbers. There are infinitely many pairs $(p, q)$ for which $a_{1}=a_{2}$. Exhibit $27$ such pairs.

Present the list of pairs as a sequence of tuples separated by commas within a \boxed, for example, \boxed{[(2, 19), (3, 123)]}.

\subsubsection{Variation}
\textbf{Actual Problem}\\
The natural numbers $p, q$ are such that for every real number $x$

$$
(x+1)^{p}(x-3)^{q}=x^{n}+a_{1} x^{n-1}+a_{2} x^{n-2}+\cdots+a_{n-1} x+a_{n},
$$

where $n=p+q$ and $a_{1}, \ldots, a_{n}$ are real numbers. There are infinitely many pairs $(p, q)$ for which $a_{1}=a_{2}$. Exhibit $17$ such pairs.

Output the sequences as a list of comma-separated tuples inside of \boxed, e.g. \boxed{[(2, 19), (3, 123)]}.

\textbf{Revised Problem}\\
The natural numbers $p, q$ are such that for every real number $x$

$$
(x+1)^{p}(x-3)^{q}=x^{n}+a_{1} x^{n-1}+a_{2} x^{n-2}+\cdots+a_{n-1} x+a_{n},
$$

where $n=p+q$ and $a_{1}, \ldots, a_{n}$ are real numbers. There are infinitely many pairs $(p, q)$ for which $a_{1}=a_{2}$. Exhibit $17$ such pairs.

Present the sequences as a list of tuples separated by commas enclosed in \boxed, for example, \boxed{[(2, 19), (3, 123)]}.

\section{bxmo}
\subsection{bxmo-2011-1}
\subsubsection{Variation}
\textbf{Actual Problem}\\
An ordered pair of integers $(m, n)$ with $1 < m < n$ is said to be a Benelux couple if the following two conditions hold: $m$ has the same prime divisors as $n$, and $m + 1$ has the same prime divisors as $n + 1$.

Find 3 Benelux couples $(m, n)$ with $m \leqslant 14$ such that for any two pairs $(m_1, n_1), (m_2, n_2)$ with $n_1 > n_2$ in your list, $\frac{n_1}{n_2} \geqslant 4$.

Output the answer as a comma separated list of lists inside of \boxed{...}. The first integer in a list should be equal to $m$, the second to $n$. For instance, \boxed{(2,5),(3,6)}.

\textbf{Revised Problem}\\
Consider a pair of integers $(m, n)$ where $1 < m < n$. Such pairs are termed "Benelux couples" if they satisfy two conditions: the integers $m$ and $n$ have the same prime divisors, and similarly, the integers $m + 1$ and $n + 1$ share the same prime divisors.

Identify 3 Benelux couples $(m, n)$ such that $m \leq 14$ and for any two pairs $(m_1, n_1)$ and $(m_2, n_2)$, if $n_1 > n_2$, then the ratio $\frac{n_1}{n_2} \geq 4$.

Present the solution as a list of lists separated by commas, enclosed within \boxed{...}. Each list should start with $m$ and be followed by $n$. For example, \boxed{(2,5),(3,6)}.

\subsubsection{Variation}
\textbf{Actual Problem}\\
An ordered pair of integers $(m, n)$ with $1 < m < n$ is said to be a Benelux couple if the following two conditions hold: $m$ has the same prime divisors as $n$, and $m + 7$ has the same prime divisors as $n + 49$.

Find 8 Benelux couples $(m, n)$ with $m \leqslant 2034$ such that for any two pairs $(m_1, n_1), (m_2, n_2)$ with $n_1 > n_2$ in your list, $\frac{n_1}{n_2} \geqslant 4$.

Output the answer as a comma separated list of lists inside of \boxed{...}. The first integer in a list should be equal to $m$, the second to $n$. For instance, \boxed{(2,5),(3,6)}.

\textbf{Revised Problem}\\
A pair of integers $(m, n)$ with $1 < m < n$ is defined as a Benelux couple if it satisfies the following criteria: both $m$ and $n$ share identical prime divisors, and similarly, $m + 7$ and $n + 49$ have the same prime divisors.

Identify 8 Benelux couples $(m, n)$ where $m \leq 2034$, ensuring that for any two pairs $(m_1, n_1)$ and $(m_2, n_2)$ in your solution, if $n_1 > n_2$, then it must be that $\frac{n_1}{n_2} \geq 4$.

Present the solution as a list of comma-separated pairs enclosed within \boxed{...}. Each pair should list $m$ as the first element and $n$ as the second element. For example, \boxed{(2,5),(3,6)}.

\subsubsection{Variation}
\textbf{Actual Problem}\\
An ordered pair of integers $(m, n)$ with $1 < m < n$ is said to be a Benelux couple if the following two conditions hold: $m$ has the same prime divisors as $n$, and $m + 3$ has the same prime divisors as $n + 9$.

Find 3 Benelux couples $(m, n)$ with $m \leqslant 26$ such that for any two pairs $(m_1, n_1), (m_2, n_2)$ with $n_1 > n_2$ in your list, $\frac{n_1}{n_2} \geqslant 4$.

Output the answer as a comma separated list of lists inside of \boxed{...}. The first integer in a list should be equal to $m$, the second to $n$. For instance, \boxed{(2,5),(3,6)}.

\textbf{Revised Problem}\\
Consider a pair of integers $(m, n)$ where $1 < m < n$. Such a pair is termed a Benelux couple if it satisfies the following two criteria: both $m$ and $n$ share identical prime factors, and $m + 3$ shares the same prime factors with $n + 9$.

Identify 3 Benelux couples $(m, n)$ where $m \leq 26$. Additionally, ensure that for any two pairs $(m_1, n_1)$ and $(m_2, n_2)$, if $n_1$ is greater than $n_2$, then the condition $\frac{n_1}{n_2} \geq 4$ holds.

Present your solution as a comma-separated list of lists enclosed within \boxed{...}. Each list should contain the first integer as $m$ and the second integer as $n$. For example, \boxed{(2,5),(3,6)}.

\subsubsection{Variation}
\textbf{Actual Problem}\\
An ordered pair of integers $(m, n)$ with $1 < m < n$ is said to be a Benelux couple if the following two conditions hold: $m$ has the same prime divisors as $n$, and $m + 4$ has the same prime divisors as $n + 16$.

Find 10 Benelux couples $(m, n)$ with $m \leqslant 8184$ such that for any two pairs $(m_1, n_1), (m_2, n_2)$ with $n_1 > n_2$ in your list, $\frac{n_1}{n_2} \geqslant 4$.

Output the answer as a comma separated list of lists inside of \boxed{...}. The first integer in a list should be equal to $m$, the second to $n$. For instance, \boxed{(2,5),(3,6)}.

\textbf{Revised Problem}\\
Consider a pair of integers $(m, n)$, where $1 < m < n$, and call it a Benelux couple if it satisfies these criteria: both $m$ and $n$ share identical sets of prime factors, and $m + 4$ shares the same prime factors as $n + 16$. 

Identify 10 distinct Benelux couples $(m, n)$ with $m \leqslant 8184$ that are ordered such that for any two pairs $(m_1, n_1)$ and $(m_2, n_2)$ in your collection, if $n_1 > n_2$, then it holds that $\frac{n_1}{n_2} \geqslant 4$.

Present the solution as a comma-separated list of lists enclosed in \boxed{...}. Each list should contain two integers, the first being $m$ and the second being $n$. For example, \boxed{(2,5),(3,6)}.

\subsection{bxmo-2015-4}
\subsubsection{Variation}
\textbf{Actual Problem}\\
An arithmetic progression is a set of the form ${a, a+d, . . . , a+kd}$, where $a, d, k$ are positive integers and $k > 2$. Thus an arithmetic progression has at least three elements and the successive elements have difference $d$, called the common difference of the arithmetic progression.

Let $n = 15$ be a positive integer. For each partition of the set $\{1, 2, 3, ..., 3n\}$ into arithmetic progressions, we consider the sum $S$ of the respective common differences of these arithmetic progressions. The maximum value $S$ can attain is $n^2$. Partition the set into sets of arithmetic progressions such that $S$ attains this maximum.

Output the answer as a comma separated list of lists inside of \boxed{...}. For example \boxed{(1,2,3),(4,5,6),(7,8,9)}.

\textbf{Revised Problem}\\
An arithmetic progression is a sequence of numbers of the form ${a, a+d, \ldots, a+kd}$, where $a$, $d$, and $k$ are positive integers, and $k$ must be greater than 2. This means an arithmetic progression must have at least three terms, with each consecutive pair of terms differing by $d$, known as the common difference.

Consider $n = 15$ as a positive integer. For every way of dividing the set $\{1, 2, 3, \ldots, 3n\}$ into arithmetic progressions, calculate the total $S$ of the common differences of these arithmetic progressions. The highest possible value for $S$ is $n^2$. Arrange the set into arithmetic progressions so that $S$ reaches this maximum.

Present your answer as a sequence of lists separated by commas, enclosed within \boxed{...}. For example \boxed{(1,2,3),(4,5,6),(7,8,9)}.

\subsubsection{Variation}
\textbf{Actual Problem}\\
An arithmetic progression is a set of the form ${a, a+d, . . . , a+kd}$, where $a, d, k$ are positive integers and $k > 2$. Thus an arithmetic progression has at least three elements and the successive elements have difference $d$, called the common difference of the arithmetic progression.

Let $n = 17$ be a positive integer. For each partition of the set $\{1, 2, 3, ..., 3n\}$ into arithmetic progressions, we consider the sum $S$ of the respective common differences of these arithmetic progressions. The maximum value $S$ can attain is $n^2$. Partition the set into sets of arithmetic progressions such that $S$ attains this maximum.

Output the answer as a comma separated list of lists inside of \boxed{...}. For example \boxed{(1,2,3),(4,5,6),(7,8,9)}.

\textbf{Revised Problem}\\
An arithmetic progression consists of numbers in the form \( a, a+d, \ldots, a+kd \) where \( a, d, k \) are all positive integers and \( k \geq 2 \). This means there are at least three elements in the sequence, each differing by \( d \), which is known as the sequence's common difference.

Given \( n = 17 \), a positive integer, consider the set \(\{1, 2, 3, \ldots, 51\}\). We aim to partition this set into multiple arithmetic progressions. For each partition, compute the total \( S \), which is the sum of the common differences from each arithmetic progression. The objective is to maximize \( S \) such that it reaches the value \( n^2 \). Determine how to organize these progressions to achieve this maximum sum.

Present the solution as a list of tuples separated by commas, all enclosed within \boxed{...}. For example, \boxed{(1,2,3),(4,5,6),(7,8,9)}.

\subsubsection{Variation}
\textbf{Actual Problem}\\
An arithmetic progression is a set of the form ${a, a+d, . . . , a+kd}$, where $a, d, k$ are positive integers and $k > 2$. Thus an arithmetic progression has at least three elements and the successive elements have difference $d$, called the common difference of the arithmetic progression.

Let $n = 9$ be a positive integer. For each partition of the set $\{1, 2, 3, ..., 3n\}$ into arithmetic progressions, we consider the sum $S$ of the respective common differences of these arithmetic progressions. The maximum value $S$ can attain is $n^2$. Partition the set into sets of arithmetic progressions such that $S$ attains this maximum.

Output the answer as a comma separated list of lists inside of \boxed{...}. For example \boxed{(1,2,3),(4,5,6),(7,8,9)}.

\textbf{Revised Problem}\\
An arithmetic sequence is described as a series of numbers like \({a, a+d, \ldots, a+kd}\), where \(a\), \(d\), and \(k\) are positive integers and \(k > 2\). This means an arithmetic sequence consists of at least three numbers, with \(d\) being the constant difference between consecutive terms, known as the common difference.

Consider a positive integer \(n = 9\). The task is to divide the set \(\{1, 2, 3, \ldots, 3n\}\) into arithmetic sequences. For each division into these sequences, calculate the sum \(S\) of their common differences. The goal is to achieve the highest possible value for \(S\), which is \(n^2\). Organize the set into arithmetic sequences to reach this maximum sum.

Present the solution as a list of lists, separated by commas, enclosed within \boxed{...}. For instance, \boxed{(1,2,3),(4,5,6),(7,8,9)}.

\subsubsection{Variation}
\textbf{Actual Problem}\\
An arithmetic progression is a set of the form ${a, a+d, . . . , a+kd}$, where $a, d, k$ are positive integers and $k > 2$. Thus an arithmetic progression has at least three elements and the successive elements have difference $d$, called the common difference of the arithmetic progression.

Let $n = 6$ be a positive integer. For each partition of the set $\{1, 2, 3, ..., 3n\}$ into arithmetic progressions, we consider the sum $S$ of the respective common differences of these arithmetic progressions. The maximum value $S$ can attain is $n^2$. Partition the set into sets of arithmetic progressions such that $S$ attains this maximum.

Output the answer as a comma separated list of lists inside of \boxed{...}. For example \boxed{(1,2,3),(4,5,6),(7,8,9)}.

\textbf{Revised Problem}\\
An arithmetic progression consists of numbers of the form ${a, a+d, \ldots, a+kd}$, where $a$, $d$, and $k$ are positive integers with $k > 2$. This implies each arithmetic progression has at least three terms, with $d$ being the common difference.

Consider the integer $n = 6$. For every division of the set $\{1, 2, 3, \ldots, 18\}$ into arithmetic progressions, calculate the sum $S$ of the common differences of these progressions. The goal is to achieve the maximum possible value of $S$, which is $n^2$. Arrange the set into arithmetic progressions such that this maximum value of $S$ is reached.

Present your solution as a list of lists, separated by commas, and enclosed in \boxed{...}. For instance, \boxed{(1,2,3),(4,5,6),(7,8,9)}.

\subsection{bxmo-2019-2}
\subsubsection{Variation}
\textbf{Actual Problem}\\
Pawns and rooks are placed on a $9 \times 9$ chessboard, with at most one piece on each of the $9^2$ squares. A rook can see another rook if they are in the same row or column and all squares between them are empty. 

The maximum number of pawns $p$ for which $p$ pawns and $p + 9$ rooks can be placed on the chessboard in such a way that no two rooks can see each other is $4^2$. 

Find a configuration that places $4^2$ pawns and $4^2 + 9$ rooks on the board, such that no two rooks can see each other.

Output the answer between \verb|\begin{array}{...}| and \verb|\end{array}| inside of $\boxed{...}$. For example, $\boxed{\begin{array}{ccc}1 & 2 & 3 \\ 4 & 5 & 6 \\ 7 & 8 & 9\end{array}}$.
If a square contains a rook, write 2. If a square contains a pawn, write 1. Otherwise, write 0.

\textbf{Revised Problem}\\
On a $9 \times 9$ chessboard, you can place pawns and rooks with only one piece per square. A rook has the ability to "see" another rook if they are aligned in the same row or column with no intervening pieces. 

Determine the greatest number of pawns \( p \) such that \( p \) pawns and \( p + 9 \) rooks can be arranged on the board where no pair of rooks can see each other, which is calculated to be \( 4^2 \).

Provide a specific arrangement of \( 4^2 \) pawns and \( 4^2 + 9 \) rooks on the board, ensuring that no two rooks can see one another.

Present the solution using the format \verb|\begin{array}{...}| and \verb|\end{array}| encapsulated within $\boxed{...}$. For instance, $\boxed{\begin{array}{ccc}1 & 2 & 3 \\ 4 & 5 & 6 \\ 7 & 8 & 9\end{array}}$.
Use the number 2 for a square containing a rook, 1 for a square with a pawn, and 0 for an empty square.

\subsubsection{Variation}
\textbf{Actual Problem}\\
Pawns and rooks are placed on a $17 \times 17$ chessboard, with at most one piece on each of the $17^2$ squares. A rook can see another rook if they are in the same row or column and all squares between them are empty. 

The maximum number of pawns $p$ for which $p$ pawns and $p + 17$ rooks can be placed on the chessboard in such a way that no two rooks can see each other is $8^2$. 

Find a configuration that places $8^2$ pawns and $8^2 + 17$ rooks on the board, such that no two rooks can see each other.

Output the answer between \verb|\begin{array}{...}| and \verb|\end{array}| inside of $\boxed{...}$. For example, $\boxed{\begin{array}{ccc}1 & 2 & 3 \\ 4 & 5 & 6 \\ 7 & 8 & 9\end{array}}$.
If a square contains a rook, write 2. If a square contains a pawn, write 1. Otherwise, write 0.

\textbf{Revised Problem}\\
You have a $17 \times 17$ chessboard, with pawns and rooks positioned such that each square can contain at most one piece. A rook can observe another rook if they share the same row or column and all the squares in between are unoccupied.

Determine the greatest number of pawns $p$ such that $p$ pawns and $p + 17$ rooks can be arranged on the board, ensuring that no two rooks have a clear line of sight to each other. The maximum value of $p$ is $8^2$.

Construct a scenario where you place $8^2$ pawns and $8^2 + 17$ rooks on the chessboard, with the condition that no two rooks are able to observe one another.

Present the solution using the format \verb|\begin{array}{...}| and \verb|\end{array}| within $\boxed{...}$. For instance, $\boxed{\begin{array}{ccc}1 & 2 & 3 \\ 4 & 5 & 6 \\ 7 & 8 & 9\end{array}}$. Use 2 to denote a square with a rook, 1 for a square with a pawn, and 0 for empty squares.

\subsubsection{Variation}
\textbf{Actual Problem}\\
Pawns and rooks are placed on a $5 \times 5$ chessboard, with at most one piece on each of the $5^2$ squares. A rook can see another rook if they are in the same row or column and all squares between them are empty. 

The maximum number of pawns $p$ for which $p$ pawns and $p + 5$ rooks can be placed on the chessboard in such a way that no two rooks can see each other is $2^2$. 

Find a configuration that places $2^2$ pawns and $2^2 + 5$ rooks on the board, such that no two rooks can see each other.

Output the answer between \verb|\begin{array}{...}| and \verb|\end{array}| inside of $\boxed{...}$. For example, $\boxed{\begin{array}{ccc}1 & 2 & 3 \\ 4 & 5 & 6 \\ 7 & 8 & 9\end{array}}$.
If a square contains a rook, write 2. If a square contains a pawn, write 1. Otherwise, write 0.

\textbf{Revised Problem}\\
On a $5 \times 5$ chessboard, pawns and rooks are positioned such that each square holds no more than one piece. A rook can "see" another if they share the same row or column without any piece between them. Determine the highest number of pawns $p$ such that $p$ pawns and $p + 5$ rooks are placed on the board without any two rooks seeing each other. It is given that the maximum value for $p$ is $2^2$.

Identify an arrangement that positions $2^2$ pawns and $2^2 + 5$ rooks on the board so that no two rooks have line of sight with each other.

Provide your solution using the format \verb|\begin{array}{...}| and \verb|\end{array}| within $\boxed{...}$. For instance, $\boxed{\begin{array}{ccc}1 & 2 & 3 \\ 4 & 5 & 6 \\ 7 & 8 & 9\end{array}}$. Use 2 to denote a rook, 1 for a pawn, and 0 for an empty square.

\subsubsection{Variation}
\textbf{Actual Problem}\\
Pawns and rooks are placed on a $11 \times 11$ chessboard, with at most one piece on each of the $11^2$ squares. A rook can see another rook if they are in the same row or column and all squares between them are empty. 

The maximum number of pawns $p$ for which $p$ pawns and $p + 11$ rooks can be placed on the chessboard in such a way that no two rooks can see each other is $5^2$. 

Find a configuration that places $5^2$ pawns and $5^2 + 11$ rooks on the board, such that no two rooks can see each other.

Output the answer between \verb|\begin{array}{...}| and \verb|\end{array}| inside of $\boxed{...}$. For example, $\boxed{\begin{array}{ccc}1 & 2 & 3 \\ 4 & 5 & 6 \\ 7 & 8 & 9\end{array}}$.
If a square contains a rook, write 2. If a square contains a pawn, write 1. Otherwise, write 0.

\textbf{Revised Problem}\\
Consider an $11 \times 11$ chessboard where pawns and rooks are placed such that each square can hold at most one piece. A rook is able to see another rook if they are positioned in the same row or column with no pawns or other pieces obstructing the line of sight between them.

Determine the highest number of pawns, $p$, such that you can place $p$ pawns and $p + 11$ rooks on the board, ensuring that no two rooks can see each other. This value of $p$ is given as $5^2$.

Provide an arrangement for placing $5^2$ pawns and $5^2 + 11$ rooks on the board so that no two rooks have a direct line of sight to each other.

Present your solution using the format \verb|\begin{array}{...}| and \verb|\end{array}| enclosed within $\boxed{...}$. For example, $\boxed{\begin{array}{ccc}1 & 2 & 3 \\ 4 & 5 & 6 \\ 7 & 8 & 9\end{array}}$.
Use 2 to denote a square with a rook, 1 for a square with a pawn, and 0 for an empty square.

\subsection{bxmo-2020-4}
\subsubsection{Variation}
\textbf{Actual Problem}\\
A divisor $d$ of a positive integer $n$ is said to be a close divisor of $n$ if $\sqrt{n} < d < 2\sqrt{n}$

Find a positive integer with exactly 60 close divisors.

Output the answer as an integer inside of $\boxed{...}$. For example $\boxed{123}$.
Your answer can contain mathematical operations using valid LaTeX notation.

\textbf{Revised Problem}\\
A divisor \( d \) of a positive number \( n \) is known as a close divisor of \( n \) if it satisfies the condition \( \sqrt{n} < d < 2\sqrt{n} \).

Identify a positive integer that possesses exactly 60 close divisors.

Present your solution as an integer enclosed within \(\boxed{...}\). As an illustration, \(\boxed{123}\).
You are allowed to use mathematical operations in valid LaTeX notation for your answer.

\subsubsection{Variation}
\textbf{Actual Problem}\\
A divisor $d$ of a positive integer $n$ is said to be a close divisor of $n$ if $\sqrt{n} < d < 2\sqrt{n}$

Find a positive integer with exactly 108 close divisors.

Output the answer as an integer inside of $\boxed{...}$. For example $\boxed{123}$.
Your answer can contain mathematical operations using valid LaTeX notation.

\textbf{Revised Problem}\\
Identify a positive integer that possesses exactly 108 divisors that lie within the range $\sqrt{n} < d < 2\sqrt{n}$, where $d$ is a divisor of the integer $n$.

Provide the solution as an integer enclosed within $\boxed{...}$. For example, express your answer as $\boxed{123}$. Your response may include mathematical expressions using valid LaTeX notation.

\subsubsection{Variation}
\textbf{Actual Problem}\\
A divisor $d$ of a positive integer $n$ is said to be a close divisor of $n$ if $\sqrt{n} < d < 2\sqrt{n}$

Find a positive integer with exactly 76 close divisors.

Output the answer as an integer inside of $\boxed{...}$. For example $\boxed{123}$.
Your answer can contain mathematical operations using valid LaTeX notation.

\textbf{Revised Problem}\\
A positive integer $n$ has a divisor $d$ defined as a close divisor if it fulfills the condition $\sqrt{n} < d < 2\sqrt{n}$. Determine a positive integer that possesses precisely 76 close divisors.

Present the solution as an integer enclosed within $\boxed{...}$. For instance, you should write $\boxed{123}$. Your response may include mathematical expressions using appropriate LaTeX syntax.

\subsubsection{Variation}
\textbf{Actual Problem}\\
A divisor $d$ of a positive integer $n$ is said to be a close divisor of $n$ if $\sqrt{n} < d < 2\sqrt{n}$

Find a positive integer with exactly 66 close divisors.

Output the answer as an integer inside of $\boxed{...}$. For example $\boxed{123}$.
Your answer can contain mathematical operations using valid LaTeX notation.

\textbf{Revised Problem}\\
Identify a positive integer that possesses precisely 66 divisors, each of which lies strictly between the square root of the integer and twice the square root of the integer.

Present the solution as an integer enclosed within $\boxed{...}$. For instance, $\boxed{456}$. You may include mathematical expressions using LaTeX notation if necessary.

\subsection{bxmo-2021-2}
\subsubsection{Variation}
\textbf{Actual Problem}\\
Pebbles are placed on a $7 \times 7$ board in such a way that each square contains at most one pebble. The pebble set of a square of the board is the collection of all pebbles which are in the same row or column as this square.

The least number of pebbles that can be placed on the board in such a way that no two squares have the same pebble set is 10. Give a configuration that achieves this minimum.

Output the answer between \verb|\begin{array}{...}| and \verb|\end{array}| inside of $\boxed{...}$. For example, $\boxed{\begin{array}{ccc}1 & 2 & 3 \\ 4 & 5 & 6 \\ 7 & 8 & 9\end{array}}$.
If a square contains a pebble, write 1, otherwise write 0.

\textbf{Revised Problem}\\
Pebbles are arranged on a \(7 \times 7\) grid so that each cell holds at most one pebble. The pebble collection of a cell is comprised of all pebbles that are in the same row or column as that cell.

Determine the smallest number of pebbles that can be placed on the board such that no two cells share the same pebble collection. This minimum is 10. Provide a configuration that demonstrates this minimum placement.

Present the solution enclosed within \verb|\begin{array}{...}|

\subsubsection{Variation}
\textbf{Actual Problem}\\
Pebbles are placed on a $13 \times 13$ board in such a way that each square contains at most one pebble. The pebble set of a square of the board is the collection of all pebbles which are in the same row or column as this square.

The least number of pebbles that can be placed on the board in such a way that no two squares have the same pebble set is 19. Give a configuration that achieves this minimum.

Output the answer between \verb|\begin{array}{...}| and \verb|\end{array}| inside of $\boxed{...}$. For example, $\boxed{\begin{array}{ccc}1 & 2 & 3 \\ 4 & 5 & 6 \\ 7 & 8 & 9\end{array}}$.
If a square contains a pebble, write 1, otherwise write 0.

\textbf{Revised Problem}\\
Consider a $13 \times 13$ grid where pebbles are distributed such that each cell holds at most one pebble. For any given cell, its pebble set comprises all pebbles located in the same row or column as that cell.

Determine a configuration for placing the minimal number of pebbles, precisely 19, on the grid so that no two cells share the identical pebble set. Provide an example of such a configuration.

Display your solution using the format \verb|\begin{array}{...}| and \verb|\end{array}| within a $\boxed{...}$ structure. For instance, $\boxed{\begin{array}{ccc}1 & 2 & 3 \\ 4 & 5 & 6 \\ 7 & 8 & 9\end{array}}$. In the grid, indicate the presence of a pebble with a 1, and its absence with a 0.

\subsubsection{Variation}
\textbf{Actual Problem}\\
Pebbles are placed on a $9 \times 9$ board in such a way that each square contains at most one pebble. The pebble set of a square of the board is the collection of all pebbles which are in the same row or column as this square.

The least number of pebbles that can be placed on the board in such a way that no two squares have the same pebble set is 13. Give a configuration that achieves this minimum.

Output the answer between \verb|\begin{array}{...}| and \verb|\end{array}| inside of $\boxed{...}$. For example, $\boxed{\begin{array}{ccc}1 & 2 & 3 \\ 4 & 5 & 6 \\ 7 & 8 & 9\end{array}}$.
If a square contains a pebble, write 1, otherwise write 0.

\textbf{Revised Problem}\\
Consider a $9 \times 9$ grid where pebbles are placed such that each cell holds a maximum of one pebble. Define the pebble set for a cell as the set of pebbles located in the same row or column as that cell. Determine a configuration of pebbles such that the pebble set is unique for every cell on the grid. The minimum number of pebbles required to achieve this is 13. Provide an arrangement that meets this criterion.

Present the solution within \verb|\begin{array}{...}| and \verb|\end{array}| enclosed in $\boxed{...}$. For instance, $\boxed{\begin{array}{ccc}1 & 2 & 3 \\ 4 & 5 & 6 \\ 7 & 8 & 9\end{array}}$. Enter '1' for a cell containing a pebble and '0' otherwise.

\subsubsection{Variation}
\textbf{Actual Problem}\\
Pebbles are placed on a $11 \times 11$ board in such a way that each square contains at most one pebble. The pebble set of a square of the board is the collection of all pebbles which are in the same row or column as this square.

The least number of pebbles that can be placed on the board in such a way that no two squares have the same pebble set is 16. Give a configuration that achieves this minimum.

Output the answer between \verb|\begin{array}{...}| and \verb|\end{array}| inside of $\boxed{...}$. For example, $\boxed{\begin{array}{ccc}1 & 2 & 3 \\ 4 & 5 & 6 \\ 7 & 8 & 9\end{array}}$.
If a square contains a pebble, write 1, otherwise write 0.

\textbf{Revised Problem}\\
Consider a board with dimensions of $11 \times 11$, where each square can hold no more than one pebble. Define the pebble set of a square as the set of all pebbles located in the same row or column of that square. The challenge is to determine the smallest number of pebbles required such that no two squares on the board share an identical pebble set. This minimum number is known to be 16. Provide a configuration of pebbles that achieves this minimum requirement.

Present the solution by placing it between \verb|\begin{array}{...}| and \verb|\end{array}| within a $\boxed{...}$. For instance, $\boxed{\begin{array}{ccc}1 & 2 & 3 \\ 4 & 5 & 6 \\ 7 & 8 & 9\end{array}}$. Use 1 to indicate the presence of a pebble in a square, and 0 if it is empty.

\section{croatian}
\subsection{croatian-2012-4}
\subsubsection{Variation}
\textbf{Actual Problem}\\
Given $d = 1046$, find a natural number $n$ divisible by $d$, such that in its decimal representation, one can delete some digit different from $0$, so that the resulting number is also divisible by $d$.

Output the answer as an integer inside of $\boxed{...}$. For example $\boxed{123}$.

\textbf{Revised Problem}\\
Consider $d = 1046$. Determine a natural number $n$ that is a multiple of $d$ such that if you remove one of its non-zero digits, the resulting number remains a multiple of $d$.

Present the solution as a single integer enclosed within $\boxed{...}$. For instance, $\boxed{123}$.

\subsubsection{Variation}
\textbf{Actual Problem}\\
Given $d = 442001$, find a natural number $n$ divisible by $d$, such that in its decimal representation, one can delete some digit different from $0$, so that the resulting number is also divisible by $d$.

Output the answer as an integer inside of $\boxed{...}$. For example $\boxed{123}$.

\textbf{Revised Problem}\\
Let $d = 442001$. Determine a natural number $n$ which is a multiple of $d$ and has the property that eliminating a non-zero digit from its decimal form still leaves a number that is divisible by $d$.

Provide the solution as an integer enclosed in $\boxed{...}$. For instance, $\boxed{123}$.

\subsubsection{Variation}
\textbf{Actual Problem}\\
Given $d = 597853$, find a natural number $n$ divisible by $d$, such that in its decimal representation, one can delete some digit different from $0$, so that the resulting number is also divisible by $d$.

Output the answer as an integer inside of $\boxed{...}$. For example $\boxed{123}$.

\textbf{Revised Problem}\\
Determine a natural number \( n \) that is a multiple of \( 597853 \) and has the property that removing one of its non-zero digits results in another number that is also a multiple of \( 597853 \).

Present your solution encapsulated in a box, such as \(\boxed{123}\).

\subsubsection{Variation}
\textbf{Actual Problem}\\
Given $d = 994869$, find a natural number $n$ divisible by $d$, such that in its decimal representation, one can delete some digit different from $0$, so that the resulting number is also divisible by $d$.

Output the answer as an integer inside of $\boxed{...}$. For example $\boxed{123}$.

\textbf{Revised Problem}\\
Given \( d = 994869 \), identify a natural number \( n \) which is divisible by \( d \) such that removing any single digit (not equal to zero) from \( n \) yields another number that remains divisible by \( d \).

Present the solution as an integer enclosed in \(\boxed{...}\). For instance, \(\boxed{123}\).

\subsection{croatian-2013-4}
\subsubsection{Variation}
\textbf{Actual Problem}\\
Find $10$ different positive integers $n$, all with different numbers of digits, that have more than two different prime divisors and for which $2^n - 8$ is divisible by $n$.

Output the answer as a comma separated list inside of $\boxed{...}$. For example $\boxed{1, 2, 3}$.

\textbf{Revised Problem}\\
Identify 10 distinct positive integers, each with a unique number of digits, that possess more than two unique prime factors and satisfy the condition that \( 2^n - 8 \) is divisible by \( n \).

Present your solution as a list of numbers separated by commas, enclosed within \(\boxed{...}\). For example \(\boxed{1, 2, 3}\).

\subsubsection{Variation}
\textbf{Actual Problem}\\
Find $13$ different positive integers $n$, all with different numbers of digits, that have more than two different prime divisors and for which $2^n - 8$ is divisible by $n$.

Output the answer as a comma separated list inside of $\boxed{...}$. For example $\boxed{1, 2, 3}$.

\textbf{Revised Problem}\\
Identify 13 unique positive integers \( n \), each possessing a different number of digits, such that each \( n \) contains more than two distinct prime factors and divides the expression \( 2^n - 8 \).

Present your solution as a list of numbers, separated by commas, enclosed in a box. For instance, express it as \(\boxed{1, 2, 3}\).

\subsubsection{Variation}
\textbf{Actual Problem}\\
Find $9$ different positive integers $n$, all with different numbers of digits, that have more than two different prime divisors and for which $2^n - 8$ is divisible by $n$.

Output the answer as a comma separated list inside of $\boxed{...}$. For example $\boxed{1, 2, 3}$.

\textbf{Revised Problem}\\
Identify $9$ distinct positive integers $n$, each having a unique digit count, such that each integer has more than two different prime factors and also divides $2^n - 8$.

Present your solution as a list of numbers separated by commas, enclosed within $\boxed{...}$. For example, $\boxed{1, 2, 3}$.

\subsubsection{Variation}
\textbf{Actual Problem}\\
Find $7$ different positive integers $n$, all with different numbers of digits, that have more than two different prime divisors and for which $2^n - 8$ is divisible by $n$.

Output the answer as a comma separated list inside of $\boxed{...}$. For example $\boxed{1, 2, 3}$.

\textbf{Revised Problem}\\
Identify seven distinct positive integers, each with a unique number of digits, such that each integer has more than two different prime factors and divides the expression $2^n - 8$.

Present your answer as a comma-separated list enclosed in $\boxed{...}$. As an example: $\boxed{1, 2, 3}$.

\subsection{croatian-2014-2}
\subsubsection{Variation}
\textbf{Actual Problem}\\
Let $M = 4$ and $N = 9$. Your task is to color all sides and diagonals of a regular $N$-gon using one of exactly $M$ colors so that there do not exist three different vertices of the $N$-gon
that determine a triangle whose sides are colored with exactly two colors.
Note that one can prove that for $M = 4$, this is the largest $N$ for which this is possible.


You should output your answer as a ${N} \times {N}$ matrix, where the entry in row $i$ and column $j$ is an integer between $1$ and $M$ representing the color of the side/diagonal connecting vertex $i$ and vertex $j$.
 
Output the answer between \verb|\begin{array}{...}| and \verb|\end{array}| inside of $\boxed{...}$. 

For example, $\boxed{\begin{array}{ccc}1 & 1 & 3 \\ 2 & 1 & 1 \\ 2 & 2 & 3\end{array}}$.

\textbf{Revised Problem}\\
Consider a regular polygon with $N = 9$ sides. You have $M = 4$ different colors at your disposal. Your objective is to color each edge and diagonal of the polygon such that it's impossible to find any set of three vertices that forms a triangle with sides colored using only two of these colors. It is known that with $M = 4$ colors, this is the maximum value of $N$ for which such a coloring is achievable.

Present your solution as a matrix with dimensions ${N} \times {N}$, where each entry in row $i$ and column $j$ is an integer from 1 to $M$, indicating the color used for the connection between vertex $i$ and vertex $j$.

Surround your solution with \verb|\begin{array}{...}| and \verb|\end{array}| within a $\boxed{...}$ environment. 

For instance, $\boxed{\begin{array}{ccc}1 & 1 & 3 \\ 2 & 1 & 1 \\ 2 & 2 & 3\end{array}}$.

\subsubsection{Variation}
\textbf{Actual Problem}\\
Let $M = 3$ and $N = 4$. Your task is to color all sides and diagonals of a regular $N$-gon using one of exactly $M$ colors so that there do not exist three different vertices of the $N$-gon
that determine a triangle whose sides are colored with exactly two colors.
Note that one can prove that for $M = 3$, this is the largest $N$ for which this is possible.


You should output your answer as a ${N} \times {N}$ matrix, where the entry in row $i$ and column $j$ is an integer between $1$ and $M$ representing the color of the side/diagonal connecting vertex $i$ and vertex $j$.
 
Output the answer between \verb|\begin{array}{...}| and \verb|\end{array}| inside of $\boxed{...}$. 

For example, $\boxed{\begin{array}{ccc}1 & 1 & 3 \\ 2 & 1 & 1 \\ 2 & 2 & 3\end{array}}$.

\textbf{Revised Problem}\\
Consider a regular polygon with $4$ sides (a quadrilateral). You need to color each of its sides and diagonals using exactly $3$ different colors. Ensure that no three vertices form a triangle where the edges are colored using only two distinct colors. It is known that when using 3 colors, a quadrilateral is the largest polygon where this condition can be achieved.

Your solution should be presented as a $4 \times 4$ matrix where each element in the matrix is an integer from $1$ to $3$, representing the color assigned to the segment connecting vertex $i$ and vertex $j$.

Present your solution within \verb|\begin{array}{...}| and \verb|\end{array}| enclosed by $\boxed{...}$. 

For instance, $\boxed{\begin{array}{ccc}1 & 1 & 3 \\ 2 & 1 & 1 \\ 2 & 2 & 3\end{array}}$.

\subsubsection{Variation}
\textbf{Actual Problem}\\
Let $M = 6$ and $N = 25$. Your task is to color all sides and diagonals of a regular $N$-gon using one of exactly $M$ colors so that there do not exist three different vertices of the $N$-gon
that determine a triangle whose sides are colored with exactly two colors.
Note that one can prove that for $M = 6$, this is the largest $N$ for which this is possible.


You should output your answer as a ${N} \times {N}$ matrix, where the entry in row $i$ and column $j$ is an integer between $1$ and $M$ representing the color of the side/diagonal connecting vertex $i$ and vertex $j$.
 
Output the answer between \verb|\begin{array}{...}| and \verb|\end{array}| inside of $\boxed{...}$. 

For example, $\boxed{\begin{array}{ccc}1 & 1 & 3 \\ 2 & 1 & 1 \\ 2 & 2 & 3\end{array}}$.

\textbf{Revised Problem}\\
Consider a regular polygon with $25$ sides. You are required to assign one of $6$ distinct colors to each side and diagonal of this polygon such that no set of three vertices forms a triangle where the sides are colored using only two different colors. It can be demonstrated that using $6$ colors is the maximum number for which this condition can be satisfied when $N = 25$.

Represent your solution as a $25 \times 25$ matrix, where each element $(i, j)$ of the matrix is an integer ranging from $1$ to $6$, indicating the color of the segment connecting vertices $i$ and $j$.

Your solution should be enclosed within \verb|\begin{array}{...}| and \verb|\end{array}|, encapsulated by $\boxed{...}$.

For instance, a sample output could look like $\boxed{\begin{array}{ccc}1 & 1 & 3 \\ 2 & 1 & 1 \\ 2 & 2 & 3\end{array}}$.

\subsection{croatian-2015-4}
\subsubsection{Variation}
\textbf{Actual Problem}\\
Given $n = 712998$, find non-negative integers $a$ and $b$ such that the number $4a^2 + 9b^2 - 1$ is divisible by $n$.

Present your answer as \boxed{a, b}, e.g. \boxed{1, 2}.

\textbf{Revised Problem}\\
Determine non-negative integers \( a \) and \( b \) such that the expression \( 4a^2 + 9b^2 - 1 \) is a multiple of \( n = 712998 \).

Express your answer in the format \boxed{a, b}, for example, \boxed{1, 2}.

\subsubsection{Variation}
\textbf{Actual Problem}\\
Given $n = 919362$, find non-negative integers $a$ and $b$ such that the number $4a^2 + 9b^2 - 1$ is divisible by $n$.

Present your answer as \boxed{a, b}, e.g. \boxed{1, 2}.

\textbf{Revised Problem}\\
Given \( n = 919362 \), identify non-negative integers \( a \) and \( b \) such that the expression \( 4a^2 + 9b^2 - 1 \) is divisible by \( n \).

Provide your solution in the form \boxed{a, b}, for example, \boxed{1, 2}.

\subsubsection{Variation}
\textbf{Actual Problem}\\
Given $n = 599586$, find non-negative integers $a$ and $b$ such that the number $4a^2 + 9b^2 - 1$ is divisible by $n$.

Present your answer as \boxed{a, b}, e.g. \boxed{1, 2}.

\textbf{Revised Problem}\\
Determine non-negative integer values $a$ and $b$ such that the expression $4a^2 + 9b^2$ is congruent to 1 modulo $n = 599586$.

Write your solution in the form \boxed{a, b}, for example, \boxed{1, 2}.

\subsubsection{Variation}
\textbf{Actual Problem}\\
Given $n = 810534$, find non-negative integers $a$ and $b$ such that the number $4a^2 + 9b^2 - 1$ is divisible by $n$.

Present your answer as \boxed{a, b}, e.g. \boxed{1, 2}.

\textbf{Revised Problem}\\
Calculate non-negative integers \(a\) and \(b\) such that the expression \(4a^2 + 9b^2 - 1\) is a multiple of \(n = 810534\).

Display your solution in the format \boxed{a, b}, for example \boxed{1, 2}.

\subsection{croatian-2017-2}
\subsubsection{Variation}
\textbf{Actual Problem}\\
An unusual bishop is a chess piece that can be turned towards any of the four diagonally adjacent squares and can attack any square that is either in front of it or left or right of it (similar as the usual bishop with the exception that it cannot see behind itself).
Two fields of the chessboard are diagonally adjacent if they share exactly one common vertex. Your task is to place at least $20$ unusual bishops on the $8 \times 8$ chessboard so that no bishop can attack another.

The chessboard is represented as a matrix in LaTeX. 
Present your solution by replacing some empty squares of the chessboard (denoted with "o") with the letters corresponding to the directions of the bishops (A - bishop turned top left, B - bishop turned top right, C - bishop turned bottom right, D - bishop turned bottom left):

$$\begin{array}{cccccccc}
o & o & o & o & o & o & o & o \\
o & o & o & o & o & o & o & o \\
o & o & o & o & o & o & o & o \\
o & o & o & o & o & o & o & o \\
o & o & o & o & o & o & o & o \\
o & o & o & o & o & o & o & o \\
o & o & o & o & o & o & o & o \\
o & o & o & o & o & o & o & o \\
\end{array}$$

Place the whole resulting array inside of $\boxed{...}$.


\textbf{Revised Problem}\\
Consider a distinctive bishop that is a chess piece capable of facing any of the four diagonally neighboring squares, with the ability to attack squares directly ahead, or to the left or right (similar to a typical bishop, except it cannot attack behind). Two squares on a chessboard are said to be diagonally adjacent if they share a single common vertex. You need to arrange at least $20$ of these distinctive bishops on an $8 \times 8$ chessboard such that no bishop is positioned to attack another.

The layout of the chessboard is depicted as a matrix in LaTeX format. Display your solution by replacing some of the empty squares (denoted by "o") with letters indicating the bishops' orientations (A - facing top left, B - facing top right, C - facing bottom right, D - facing bottom left):

$$\begin{array}{cccccccc}
o & o & o & o & o & o & o & o \\
o & o & o & o & o & o & o & o \\
o & o & o & o & o & o & o & o \\
o & o & o & o & o & o & o & o \\
o & o & o & o & o & o & o & o \\
o & o & o & o & o & o & o & o \\
o & o & o & o & o & o & o & o \\
o & o & o & o & o & o & o & o \\
\end{array}$$

Enclose the entire resulting array within $\boxed{...}$.

\subsubsection{Variation}
\textbf{Actual Problem}\\
An unusual bishop is a chess piece that can be turned towards any of the four diagonally adjacent squares and can attack any square that is either in front of it or left or right of it (similar as the usual bishop with the exception that it cannot see behind itself).
Two fields of the chessboard are diagonally adjacent if they share exactly one common vertex. Your task is to place at least $19$ unusual bishops on the $8 \times 8$ chessboard so that no bishop can attack another.

The chessboard is represented as a matrix in LaTeX. 
Present your solution by replacing some empty squares of the chessboard (denoted with "o") with the letters corresponding to the directions of the bishops (A - bishop turned top left, B - bishop turned top right, C - bishop turned bottom right, D - bishop turned bottom left):

$$\begin{array}{cccccccc}
o & o & o & o & o & o & o & o \\
o & o & o & o & o & o & o & o \\
o & o & o & o & o & o & o & o \\
o & o & o & o & o & o & o & o \\
o & o & o & o & o & o & o & o \\
o & o & o & o & o & o & o & o \\
o & o & o & o & o & o & o & o \\
o & o & o & o & o & o & o & o \\
\end{array}$$

Place the whole resulting array inside of $\boxed{...}$.


\textbf{Revised Problem}\\
An unconventional bishop is a chess piece that can be oriented towards any of the four diagonal directions and can attack any square directly ahead, to the left, or to the right of its orientation (similar to a standard bishop except it cannot attack behind itself). Two squares on the chessboard are diagonally adjacent if they share precisely one common corner. Your objective is to arrange at least 19 unconventional bishops on the $8 \times 8$ chessboard such that no two bishops can attack each other.

The chessboard should be depicted as a matrix in LaTeX. Display your solution by substituting some empty squares on the chessboard (represented as "o") with the letters indicating the orientations of the bishops (A - bishop facing top left, B - bishop facing top right, C - bishop facing bottom right, D - bishop facing bottom left):

$$\begin{array}{cccccccc}
o & o & o & o & o & o & o & o \\
o & o & o & o & o & o & o & o \\
o & o & o & o & o & o & o & o \\
o & o & o & o & o & o & o & o \\
o & o & o & o & o & o & o & o \\
o & o & o & o & o & o & o & o \\
o & o & o & o & o & o & o & o \\
o & o & o & o & o & o & o & o \\
\end{array}$$

Enclose the entire resulting array within $\boxed{...}$.

\subsubsection{Variation}
\textbf{Actual Problem}\\
An unusual bishop is a chess piece that can be turned towards any of the four diagonally adjacent squares and can attack any square that is either in front of it or left or right of it (similar as the usual bishop with the exception that it cannot see behind itself).
Two fields of the chessboard are diagonally adjacent if they share exactly one common vertex. Your task is to place at least $17$ unusual bishops on the $8 \times 8$ chessboard so that no bishop can attack another.

The chessboard is represented as a matrix in LaTeX. 
Present your solution by replacing some empty squares of the chessboard (denoted with "o") with the letters corresponding to the directions of the bishops (A - bishop turned top left, B - bishop turned top right, C - bishop turned bottom right, D - bishop turned bottom left):

$$\begin{array}{cccccccc}
o & o & o & o & o & o & o & o \\
o & o & o & o & o & o & o & o \\
o & o & o & o & o & o & o & o \\
o & o & o & o & o & o & o & o \\
o & o & o & o & o & o & o & o \\
o & o & o & o & o & o & o & o \\
o & o & o & o & o & o & o & o \\
o & o & o & o & o & o & o & o \\
\end{array}$$

Place the whole resulting array inside of $\boxed{...}$.


\textbf{Revised Problem}\\
A unique chess piece, termed the unusual bishop, can face one of four diagonal directions. It can attack squares that are directly in front of it, as well as to its left and right, similar to a traditional bishop, except it cannot attack squares behind it. Two squares on the chessboard are considered diagonally adjacent if they share only one vertex. The challenge is to position at least 17 unusual bishops on an 8x8 chessboard such that no bishop is able to attack any other.

Represent the chessboard as a matrix using LaTeX. Show your arrangement by substituting some empty squares of the chessboard (indicated by "o") with letters indicating the bishops' orientation (A - facing top left, B - facing top right, C - facing bottom right, D - facing bottom left):

$$\begin{array}{cccccccc}
o & o & o & o & o & o & o & o \\
o & o & o & o & o & o & o & o \\
o & o & o & o & o & o & o & o \\
o & o & o & o & o & o & o & o \\
o & o & o & o & o & o & o & o \\
o & o & o & o & o & o & o & o \\
o & o & o & o & o & o & o & o \\
o & o & o & o & o & o & o & o \\
\end{array}$$

Enclose the entire matrix with $\boxed{...}$.

\subsubsection{Variation}
\textbf{Actual Problem}\\
An unusual bishop is a chess piece that can be turned towards any of the four diagonally adjacent squares and can attack any square that is either in front of it or left or right of it (similar as the usual bishop with the exception that it cannot see behind itself).
Two fields of the chessboard are diagonally adjacent if they share exactly one common vertex. Your task is to place at least $16$ unusual bishops on the $8 \times 8$ chessboard so that no bishop can attack another.

The chessboard is represented as a matrix in LaTeX. 
Present your solution by replacing some empty squares of the chessboard (denoted with "o") with the letters corresponding to the directions of the bishops (A - bishop turned top left, B - bishop turned top right, C - bishop turned bottom right, D - bishop turned bottom left):

$$\begin{array}{cccccccc}
o & o & o & o & o & o & o & o \\
o & o & o & o & o & o & o & o \\
o & o & o & o & o & o & o & o \\
o & o & o & o & o & o & o & o \\
o & o & o & o & o & o & o & o \\
o & o & o & o & o & o & o & o \\
o & o & o & o & o & o & o & o \\
o & o & o & o & o & o & o & o \\
\end{array}$$

Place the whole resulting array inside of $\boxed{...}$.


\textbf{Revised Problem}\\
An unique bishop is a chess piece that can be oriented towards one of the four diagonally adjacent squares and is able to attack any square that lies directly ahead of it or to its immediate diagonal left or right (similar to a regular bishop but without the ability to attack behind itself). Two squares on a chessboard are considered diagonally adjacent if they share exactly one vertex. Your objective is to position at least $16$ unique bishops on an $8 \times 8$ chessboard such that no bishop is able to attack another.

The chessboard should be depicted as a matrix in LaTeX.
Display your arrangement by substituting some empty squares on the chessboard (represented by "o") with the letters that denote the directions the bishops are facing (A - bishop facing top left, B - bishop facing top right, C - bishop facing bottom right, D - bishop facing bottom left):

$$\begin{array}{cccccccc}
o & o & o & o & o & o & o & o \\
o & o & o & o & o & o & o & o \\
o & o & o & o & o & o & o & o \\
o & o & o & o & o & o & o & o \\
o & o & o & o & o & o & o & o \\
o & o & o & o & o & o & o & o \\
o & o & o & o & o & o & o & o \\
o & o & o & o & o & o & o & o \\
\end{array}$$

Enclose the entire resulting array within $\boxed{...}$.

\subsection{croatian-2018-4}
\subsubsection{Variation}
\textbf{Actual Problem}\\
Given $n = 6$, find positive integers $a_1, a_2, \ldots, a_n$ such that $\frac{a_j+a_i}{a_j-a_i}$ is a positive integer for all $1 \le i < j \le n$.

Output the answer as a comma separated list inside of $\boxed{...}$. For example $\boxed{1, 2, 3}$.

\textbf{Revised Problem}\\
With $n = 6$, identify positive integers $b_1, b_2, \ldots, b_n$ so that the value of $\frac{b_j+b_i}{b_j-b_i}$ is a positive integer for every pair of indices $1 \le i < j \le n$.

Present your solution as a list of numbers separated by commas, enclosed within $\boxed{...}$. For instance, $\boxed{1, 2, 3}$.

\subsubsection{Variation}
\textbf{Actual Problem}\\
Given $n = 7$, find positive integers $a_1, a_2, \ldots, a_n$ such that $\frac{a_j+a_i}{a_j-a_i}$ is a positive integer for all $1 \le i < j \le n$.

Output the answer as a comma separated list inside of $\boxed{...}$. For example $\boxed{1, 2, 3}$.

\textbf{Revised Problem}\\
Consider $n = 7$. Determine positive integers $b_1, b_2, \ldots, b_n$ such that for every $1 \le k < m \le n$, the expression $\frac{b_m+b_k}{b_m-b_k}$ is a positive integer.

Present your solution as a list of numbers separated by commas within the format $\boxed{...}$. For example $\boxed{4, 5, 6}$.

\subsubsection{Variation}
\textbf{Actual Problem}\\
Given $n = 8$, find positive integers $a_1, a_2, \ldots, a_n$ such that $\frac{a_j+a_i}{a_j-a_i}$ is a positive integer for all $1 \le i < j \le n$.

Output the answer as a comma separated list inside of $\boxed{...}$. For example $\boxed{1, 2, 3}$.

\textbf{Revised Problem}\\
Let $n = 8$. Determine positive integers $b_1, b_2, \ldots, b_n$ such that for every pair of indices $1 \le k < m \le n$, the expression $\frac{b_m+b_k}{b_m-b_k}$ yields a positive integer.

Present the solution as a list separated by commas inside $\boxed{...}$. For instance, $\boxed{1, 2, 3}$.

\subsection{croatian-2020-4}
\subsubsection{Variation}
\textbf{Actual Problem}\\
Find $n = 8$ different tuples of positive integers $(a, b)$ such that sum of reciprocals of all positive divisors is the same for both $a$ and $b$.
Additionally, both numbers in the $i$-th tuple of your solution should have at least $i$ digits (for $i = 1, 2, \ldots, n$).


Output the answer as a comma separated list inside of $\boxed{...}$. For example $\boxed{(1, 2), (3, 4), (5, 6)}$.

\textbf{Revised Problem}\\
Identify $n = 8$ distinct pairs of positive integers $(a, b)$ such that for each pair, the sum of the reciprocals of all divisors of $a$ is equal to the sum of the reciprocals of all divisors of $b$. Moreover, for the $i$-th pair in your solution, both numbers must have at least $i$ digits (for $i = 1, 2, \ldots, n$).

Present your answer as a list of tuples separated by commas enclosed within $\boxed{...}$. For instance, $\boxed{(1, 2), (3, 4), (5, 6)}$.

\subsubsection{Variation}
\textbf{Actual Problem}\\
Find $n = 14$ different tuples of positive integers $(a, b)$ such that sum of reciprocals of all positive divisors is the same for both $a$ and $b$.
Additionally, both numbers in the $i$-th tuple of your solution should have at least $i$ digits (for $i = 1, 2, \ldots, n$).


Output the answer as a comma separated list inside of $\boxed{...}$. For example $\boxed{(1, 2), (3, 4), (5, 6)}$.

\textbf{Revised Problem}\\
Identify $n = 14$ unique pairs of positive integers $(a, b)$ such that the sum of the reciprocals of all divisors is equal for both $a$ and $b$. Furthermore, for the $i$-th pair in your solution, each number must have at least $i$ digits (where $i = 1, 2, \ldots, n$).

Present your answer as a list of pairs separated by commas within $\boxed{...}$. For instance, $\boxed{(1, 2), (3, 4), (5, 6)}$.

\subsubsection{Variation}
\textbf{Actual Problem}\\
Find $n = 10$ different tuples of positive integers $(a, b)$ such that sum of reciprocals of all positive divisors is the same for both $a$ and $b$.
Additionally, both numbers in the $i$-th tuple of your solution should have at least $i$ digits (for $i = 1, 2, \ldots, n$).


Output the answer as a comma separated list inside of $\boxed{...}$. For example $\boxed{(1, 2), (3, 4), (5, 6)}$.

\textbf{Revised Problem}\\
Identify $n = 10$ unique pairs of positive integers $(a, b)$ such that the harmonic mean of the divisors of $a$ equals that of $b$. Furthermore, for the $i$-th pair in your solution, both numbers must have no fewer than $i$ digits (for $i = 1, 2, \ldots, n$).

Present your answer as a comma-separated list enclosed in $\boxed{...}$. For instance, $\boxed{(1, 2), (3, 4), (5, 6)}$.

\subsubsection{Variation}
\textbf{Actual Problem}\\
Find $n = 11$ different tuples of positive integers $(a, b)$ such that sum of reciprocals of all positive divisors is the same for both $a$ and $b$.
Additionally, both numbers in the $i$-th tuple of your solution should have at least $i$ digits (for $i = 1, 2, \ldots, n$).


Output the answer as a comma separated list inside of $\boxed{...}$. For example $\boxed{(1, 2), (3, 4), (5, 6)}$.

\textbf{Revised Problem}\\
Identify $n = 11$ distinct pairs of positive integers $(a, b)$ such that the sum of the reciprocals of all positive divisors of $a$ equals that of $b$. Additionally, for the $i$-th pair in your solution, both integers must have at least $i$ digits, where $i$ ranges from 1 to $n$.

Present your answer as a list of pairs, separated by commas, enclosed within $\boxed{...}$. For instance, $\boxed{(1, 2), (3, 4), (5, 6)}$.

\subsection{croatian-2022-1}
\subsubsection{Variation}
\textbf{Actual Problem}\\
Find a sequence of positive real numbers $(a_1, a_1, \ldots, a_{20})$ such that the following equation holds for at least
9 indices $i \in \{ 1,2,\ldots,20 \}$: $$ a_i a_{i+3} = a_{i}a_{i+1} + a_{i+1}a_{i+2} + a_{i+2}a_{i+3}$$ where we assume that
$a_{j+20} = a_j$ for $j \in \{1,2,3\}$.
Note that one can prove that the given equation cannot hold for more than $9$ indices.


Output the answer as a comma separated list inside of $\boxed{...}$. For example $\boxed{1, 2, 3}$.

\textbf{Revised Problem}\\
Determine a sequence of positive real numbers $(a_1, a_2, \ldots, a_{20})$ such that the equation is satisfied for at least 9 indices $i \in \{ 1, 2, \ldots, 20 \}$: $$ a_i a_{i+3} = a_{i}a_{i+1} + a_{i+1}a_{i+2} + a_{i+2}a_{i+3}$$ where it is understood that $a_{j+20} = a_j$ for $j \in \{1, 2, 3\}$.
It is important to note that it can be shown that this equation cannot be satisfied for more than 9 indices.

Present the solution as a sequence separated by commas within a $\boxed{...}$. For instance, $\boxed{1, 2, 3}$.

\subsubsection{Variation}
\textbf{Actual Problem}\\
Find a sequence of positive real numbers $(a_1, a_1, \ldots, a_{28})$ such that the following equation holds for at least
13 indices $i \in \{ 1,2,\ldots,28 \}$: $$ a_i a_{i+3} = a_{i}a_{i+1} + a_{i+1}a_{i+2} + a_{i+2}a_{i+3}$$ where we assume that
$a_{j+28} = a_j$ for $j \in \{1,2,3\}$.
Note that one can prove that the given equation cannot hold for more than $13$ indices.


Output the answer as a comma separated list inside of $\boxed{...}$. For example $\boxed{1, 2, 3}$.

\textbf{Revised Problem}\\
Identify a sequence of positive real numbers $(b_1, b_2, \ldots, b_{28})$ such that for no fewer than 13 indices $j \in \{1, 2, \ldots, 28\}$, the equation $$ b_j b_{j+3} = b_j b_{j+1} + b_{j+1} b_{j+2} + b_{j+2} b_{j+3} $$ is satisfied, where we assume that $b_{k+28} = b_k$ for $k \in \{1, 2, 3\}$. It is known that it is impossible for the equation to be satisfied for more than 13 indices.

Present the solution as a list separated by commas within $\boxed{...}$. For instance, $\boxed{1, 2, 3}$.

\subsubsection{Variation}
\textbf{Actual Problem}\\
Find a sequence of positive real numbers $(a_1, a_1, \ldots, a_{16})$ such that the following equation holds for at least
7 indices $i \in \{ 1,2,\ldots,16 \}$: $$ a_i a_{i+3} = a_{i}a_{i+1} + a_{i+1}a_{i+2} + a_{i+2}a_{i+3}$$ where we assume that
$a_{j+16} = a_j$ for $j \in \{1,2,3\}$.
Note that one can prove that the given equation cannot hold for more than $7$ indices.


Output the answer as a comma separated list inside of $\boxed{...}$. For example $\boxed{1, 2, 3}$.

\textbf{Revised Problem}\\
Determine a sequence of positive real numbers $(a_1, a_2, \ldots, a_{16})$ such that for at least 7 indices $i$ in the set $\{1, 2, \ldots, 16\}$, the equation $$ a_i a_{i+3} = a_{i}a_{i+1} + a_{i+1}a_{i+2} + a_{i+2}a_{i+3} $$ is satisfied. The sequence is periodic, meaning $a_{j+16} = a_j$ for $j = 1, 2, 3$. It can be shown that it is impossible for this equation to be satisfied for more than 7 indices.

Present your answer as a comma-separated list within $\boxed{...}$. For instance, write $\boxed{1, 2, 3}$.

\subsubsection{Variation}
\textbf{Actual Problem}\\
Find a sequence of positive real numbers $(a_1, a_1, \ldots, a_{22})$ such that the following equation holds for at least
10 indices $i \in \{ 1,2,\ldots,22 \}$: $$ a_i a_{i+3} = a_{i}a_{i+1} + a_{i+1}a_{i+2} + a_{i+2}a_{i+3}$$ where we assume that
$a_{j+22} = a_j$ for $j \in \{1,2,3\}$.
Note that one can prove that the given equation cannot hold for more than $10$ indices.


Output the answer as a comma separated list inside of $\boxed{...}$. For example $\boxed{1, 2, 3}$.

\textbf{Revised Problem}\\
Identify a sequence of positive real numbers $(a_1, a_2, \ldots, a_{22})$ such that the following condition is met for 10 or more indices $i \in \{ 1, 2, \ldots, 22 \}$: $$ a_i a_{i+3} = a_i a_{i+1} + a_{i+1} a_{i+2} + a_{i+2} a_{i+3}$$ where it is assumed that $a_{j+22} = a_j$ for $j \in \{1, 2, 3\}$. It is known that the equation cannot be satisfied for more than 10 indices.

Present the solution as a list of numbers separated by commas, enclosed in $\boxed{...}$. For instance, $\boxed{1, 2, 3}$.

\subsection{croatian-2023-5}
\subsubsection{Variation}
\textbf{Actual Problem}\\
Let $k = 4$ and $l = 5$. Your task is to dissect the $10 \times 10$ square into $14$ rectangles whose sides are parallel to the sides of the square
and lengths of the sides are integers, such that: (i) a line going through one row of the square intersects at least $k$ rectangles, (ii) a line going through one column of the square
intersects at least $l$ rectangles.

Output the rectangles by replacing x with an integer, where cells corresponding to the same integer denote the same rectangle.

$$\begin{array}{cccccccccc}
x & x & x & x & x & x & x & x & x & x \\
x & x & x & x & x & x & x & x & x & x \\
x & x & x & x & x & x & x & x & x & x \\
x & x & x & x & x & x & x & x & x & x \\
x & x & x & x & x & x & x & x & x & x \\
x & x & x & x & x & x & x & x & x & x \\
x & x & x & x & x & x & x & x & x & x \\
x & x & x & x & x & x & x & x & x & x \\
x & x & x & x & x & x & x & x & x & x \\
x & x & x & x & x & x & x & x & x & x \\
\end{array}$$

For example, the following grid consists of 4 rectangles denoted by the integers 0-3:
$$
\begin{array}{ccc}
0 & 0 & 2 \\
0 & 0 & 2 \\
1 & 3 & 3 \\
\end{array}
$$

Put the answer inside of $\boxed{...}$.


\textbf{Revised Problem}\\
Consider a 10 by 10 grid. Divide this grid into 14 rectangles with sides aligned parallel to the grid's sides, and ensure that the lengths of the rectangles' sides are whole numbers. The divisions must satisfy the following conditions: (i) any straight line that traverses one horizontal row of the grid should pass through at least 4 different rectangles, and (ii) any straight line that traverses one vertical column of the grid should intersect at least 5 distinct rectangles.

Present the rectangles by filling the grid with integers, where each integer uniquely identifies a rectangle, and all cells belonging to the same rectangle have the same integer.

$$\begin{array}{cccccccccc}
x & x & x & x & x & x & x & x & x & x \\
x & x & x & x & x & x & x & x & x & x \\
x & x & x & x & x & x & x & x & x & x \\
x & x & x & x & x & x & x & x & x & x \\
x & x & x & x & x & x & x & x & x & x \\
x & x & x & x & x & x & x & x & x & x \\
x & x & x & x & x & x & x & x & x & x \\
x & x & x & x & x & x & x & x & x & x \\
x & x & x & x & x & x & x & x & x & x \\
x & x & x & x & x & x & x & x & x & x \\
\end{array}$$

As an illustration, the grid below is divided into 4 rectangles, each represented by the numbers 0 to 3:
$$
\begin{array}{ccc}
0 & 0 & 2 \\
0 & 0 & 2 \\
1 & 3 & 3 \\
\end{array}
$$

Include the solution within $\boxed{...}$.

\subsubsection{Variation}
\textbf{Actual Problem}\\
Let $k = 7$ and $l = 7$. Your task is to dissect the $12 \times 12$ square into $24$ rectangles whose sides are parallel to the sides of the square
and lengths of the sides are integers, such that: (i) a line going through one row of the square intersects at least $k$ rectangles, (ii) a line going through one column of the square
intersects at least $l$ rectangles.

Output the rectangles by replacing x with an integer, where cells corresponding to the same integer denote the same rectangle.

$$\begin{array}{cccccccccccc}
x & x & x & x & x & x & x & x & x & x & x & x \\
x & x & x & x & x & x & x & x & x & x & x & x \\
x & x & x & x & x & x & x & x & x & x & x & x \\
x & x & x & x & x & x & x & x & x & x & x & x \\
x & x & x & x & x & x & x & x & x & x & x & x \\
x & x & x & x & x & x & x & x & x & x & x & x \\
x & x & x & x & x & x & x & x & x & x & x & x \\
x & x & x & x & x & x & x & x & x & x & x & x \\
x & x & x & x & x & x & x & x & x & x & x & x \\
x & x & x & x & x & x & x & x & x & x & x & x \\
x & x & x & x & x & x & x & x & x & x & x & x \\
x & x & x & x & x & x & x & x & x & x & x & x \\
\end{array}$$

For example, the following grid consists of 4 rectangles denoted by the integers 0-3:
$$
\begin{array}{ccc}
0 & 0 & 2 \\
0 & 0 & 2 \\
1 & 3 & 3 \\
\end{array}
$$

Put the answer inside of $\boxed{...}$.


\textbf{Revised Problem}\\
Consider a $12 \times 12$ square grid. You need to partition this grid into 24 rectangles, with each rectangle's sides aligned parallel to the grid's edges and having integer-length sides. The conditions to satisfy are: (i) any line drawn horizontally across the grid must intersect at least 7 different rectangles, and (ii) any line drawn vertically must also intersect at least 7 different rectangles.

Express the rectangles by substituting x with different integers, where each integer represents a unique rectangle, and cells with the same integer belong to the same rectangle.

$$\begin{array}{cccccccccccc}
x & x & x & x & x & x & x & x & x & x & x & x \\
x & x & x & x & x & x & x & x & x & x & x & x \\
x & x & x & x & x & x & x & x & x & x & x & x \\
x & x & x & x & x & x & x & x & x & x & x & x \\
x & x & x & x & x & x & x & x & x & x & x & x \\
x & x & x & x & x & x & x & x & x & x & x & x \\
x & x & x & x & x & x & x & x & x & x & x & x \\
x & x & x & x & x & x & x & x & x & x & x & x \\
x & x & x & x & x & x & x & x & x & x & x & x \\
x & x & x & x & x & x & x & x & x & x & x & x \\
x & x & x & x & x & x & x & x & x & x & x & x \\
x & x & x & x & x & x & x & x & x & x & x & x \\
\end{array}$$

For illustration, consider this grid consisting of 4 rectangles represented by the integers 0 through 3:
$$
\begin{array}{ccc}
0 & 0 & 2 \\
0 & 0 & 2 \\
1 & 3 & 3 \\
\end{array}
$$

Place your solution within $\boxed{...}$.

\subsubsection{Variation}
\textbf{Actual Problem}\\
Let $k = 5$ and $l = 4$. Your task is to dissect the $10 \times 10$ square into $14$ rectangles whose sides are parallel to the sides of the square
and lengths of the sides are integers, such that: (i) a line going through one row of the square intersects at least $k$ rectangles, (ii) a line going through one column of the square
intersects at least $l$ rectangles.

Output the rectangles by replacing x with an integer, where cells corresponding to the same integer denote the same rectangle.

$$\begin{array}{cccccccccc}
x & x & x & x & x & x & x & x & x & x \\
x & x & x & x & x & x & x & x & x & x \\
x & x & x & x & x & x & x & x & x & x \\
x & x & x & x & x & x & x & x & x & x \\
x & x & x & x & x & x & x & x & x & x \\
x & x & x & x & x & x & x & x & x & x \\
x & x & x & x & x & x & x & x & x & x \\
x & x & x & x & x & x & x & x & x & x \\
x & x & x & x & x & x & x & x & x & x \\
x & x & x & x & x & x & x & x & x & x \\
\end{array}$$

For example, the following grid consists of 4 rectangles denoted by the integers 0-3:
$$
\begin{array}{ccc}
0 & 0 & 2 \\
0 & 0 & 2 \\
1 & 3 & 3 \\
\end{array}
$$

Put the answer inside of $\boxed{...}$.


\textbf{Revised Problem}\\
Consider a $10 \times 10$ grid that you need to divide into $14$ rectangles. Each rectangle must have sides that are aligned with the grid lines, and the lengths of these sides must be integers. Ensure that: (i) every horizontal line passing through a row intersects at least $5$ different rectangles, and (ii) every vertical line passing through a column intersects at least $4$ different rectangles.

To represent the rectangles, substitute x with unique integers such that all cells with the same integer represent the same rectangle.

$$\begin{array}{cccccccccc}
x & x & x & x & x & x & x & x & x & x \\
x & x & x & x & x & x & x & x & x & x \\
x & x & x & x & x & x & x & x & x & x \\
x & x & x & x & x & x & x & x & x & x \\
x & x & x & x & x & x & x & x & x & x \\
x & x & x & x & x & x & x & x & x & x \\
x & x & x & x & x & x & x & x & x & x \\
x & x & x & x & x & x & x & x & x & x \\
x & x & x & x & x & x & x & x & x & x \\
x & x & x & x & x & x & x & x & x & x \\
\end{array}$$

For illustration, the grid below contains 4 rectangles represented by the integers 0-3:
$$
\begin{array}{ccc}
0 & 0 & 2 \\
0 & 0 & 2 \\
1 & 3 & 3 \\
\end{array}
$$

Provide your answer enclosed in $\boxed{...}$.

\subsubsection{Variation}
\textbf{Actual Problem}\\
Let $k = 4$ and $l = 4$. Your task is to dissect the $6 \times 6$ square into $12$ rectangles whose sides are parallel to the sides of the square
and lengths of the sides are integers, such that: (i) a line going through one row of the square intersects at least $k$ rectangles, (ii) a line going through one column of the square
intersects at least $l$ rectangles.

Output the rectangles by replacing x with an integer, where cells corresponding to the same integer denote the same rectangle.

$$\begin{array}{cccccc}
x & x & x & x & x & x \\
x & x & x & x & x & x \\
x & x & x & x & x & x \\
x & x & x & x & x & x \\
x & x & x & x & x & x \\
x & x & x & x & x & x \\
\end{array}$$

For example, the following grid consists of 4 rectangles denoted by the integers 0-3:
$$
\begin{array}{ccc}
0 & 0 & 2 \\
0 & 0 & 2 \\
1 & 3 & 3 \\
\end{array}
$$

Put the answer inside of $\boxed{...}$.


\textbf{Revised Problem}\\
Consider a $6 \times 6$ grid that you need to partition into $12$ rectangles. Each rectangle's sides must be parallel to the grid's sides, and both the length and the width of each rectangle must be whole numbers. The partitioning must satisfy the following criteria: (i) any line drawn horizontally across the grid must pass through at least $4$ distinct rectangles, and (ii) any line drawn vertically across the grid must also pass through at least $4$ distinct rectangles.

Represent the rectangles by assigning each a unique integer. Within the grid, cells marked with the same integer belong to the same rectangle.

$$\begin{array}{cccccc}
x & x & x & x & x & x \\
x & x & x & x & x & x \\
x & x & x & x & x & x \\
x & x & x & x & x & x \\
x & x & x & x & x & x \\
x & x & x & x & x & x \\
\end{array}$$

For example, the grid below is divided into 4 rectangles, represented by the integers 0-3:
$$
\begin{array}{ccc}
0 & 0 & 2 \\
0 & 0 & 2 \\
1 & 3 & 3 \\
\end{array}
$$

Provide your solution in the format $\boxed{...}$.

\section{dutch}
\subsection{dutch-2010-4}
\subsubsection{Variation}
\textbf{Actual Problem}\\
For $m = 1000$, determine $10$ pairs $(x,y)$ of rational numbers with $0 < x < 1$ and $0 < y < 1$ for which $x+my$ and $mx+y$ are both integers. An example for $m = 3$ is $(x, y) = \left(\frac{3}{8},\frac{7}{8}\right)$ since $x+3y=\frac{3}{8} + \frac{21}{8} = \frac{24}{8} = 3$ and $3x+y=\frac{9}{8} + \frac{7}{8} = \frac{16}{8} = 2$.

$(x,y)$ and $(y,x)$ are not considered to be the same pair.

Give your solution as a list of fraction pairs within a single $\boxed{}$ environment. For instance, $\boxed{(\frac{1}{2},\frac{3}{4}),(\frac{1}{3},\frac{2}{4})}$

\textbf{Revised Problem}\\
For $m = 1000$, identify $10$ distinct pairs of rational numbers $(x, y)$ such that $0 < x < 1$, $0 < y < 1$, and both expressions $x + my$ and $mx + y$ result in integer values. For example, when $m = 3$, the pair $(x, y) = \left(\frac{3}{8}, \frac{7}{8}\right)$ satisfies the conditions because $x + 3y = \frac{3}{8} + \frac{21}{8} = \frac{24}{8} = 3$ and $3x + y = \frac{9}{8} + \frac{7}{8} = \frac{16}{8} = 2$. Note that pairs such as $(x, y)$ and $(y, x)$ are considered different.

Present your answer as a sequence of fraction pairs within a single $\boxed{}$ container. For example, $\boxed{(\frac{1}{2}, \frac{3}{4}), (\frac{1}{3}, \frac{2}{4})}$.

\subsubsection{Variation}
\textbf{Actual Problem}\\
For $m = 28670$, determine $22$ pairs $(x,y)$ of rational numbers with $0 < x < 1$ and $0 < y < 1$ for which $x+my$ and $mx+y$ are both integers. An example for $m = 3$ is $(x, y) = \left(\frac{3}{8},\frac{7}{8}\right)$ since $x+3y=\frac{3}{8} + \frac{21}{8} = \frac{24}{8} = 3$ and $3x+y=\frac{9}{8} + \frac{7}{8} = \frac{16}{8} = 2$.

$(x,y)$ and $(y,x)$ are not considered to be the same pair.

Give your solution as a list of fraction pairs within a single $\boxed{}$ environment. For instance, $\boxed{(\frac{1}{2},\frac{3}{4}),(\frac{1}{3},\frac{2}{4})}$

\textbf{Revised Problem}\\
Given \(m = 28670\), identify 22 pairs of rational numbers \((x, y)\) where \(0 < x < 1\) and \(0 < y < 1\) such that both \(x + my\) and \(mx + y\) are integers. For instance, for \(m = 3\), one such pair is \((x, y) = \left(\frac{3}{8}, \frac{7}{8}\right)\) because \(x + 3y = \frac{3}{8} + \frac{21}{8} = \frac{24}{8} = 3\) and \(3x + y = \frac{9}{8} + \frac{7}{8} = \frac{16}{8} = 2\).

Please note that \((x, y)\) and \((y, x)\) are treated as distinct pairs.

Present your answer as a list of rational number pairs enclosed within a single \(\boxed{}\) environment. For example, \(\boxed{(\frac{1}{2}, \frac{3}{4}), (\frac{1}{3}, \frac{2}{4})}\)

\subsubsection{Variation}
\textbf{Actual Problem}\\
For $m = 5402$, determine $28$ pairs $(x,y)$ of rational numbers with $0 < x < 1$ and $0 < y < 1$ for which $x+my$ and $mx+y$ are both integers. An example for $m = 3$ is $(x, y) = \left(\frac{3}{8},\frac{7}{8}\right)$ since $x+3y=\frac{3}{8} + \frac{21}{8} = \frac{24}{8} = 3$ and $3x+y=\frac{9}{8} + \frac{7}{8} = \frac{16}{8} = 2$.

$(x,y)$ and $(y,x)$ are not considered to be the same pair.

Give your solution as a list of fraction pairs within a single $\boxed{}$ environment. For instance, $\boxed{(\frac{1}{2},\frac{3}{4}),(\frac{1}{3},\frac{2}{4})}$

\textbf{Revised Problem}\\
For \( m = 5402 \), identify 28 distinct pairs \((x, y)\) of rational numbers where \( 0 < x < 1 \) and \( 0 < y < 1 \) such that both \( x + my \) and \( mx + y \) are integers. For instance, when \( m = 3 \), a valid pair is \((x, y) = \left(\frac{3}{8}, \frac{7}{8}\right)\) because \( x + 3y = \frac{3}{8} + \frac{21}{8} = \frac{24}{8} = 3 \) and \( 3x + y = \frac{9}{8} + \frac{7}{8} = \frac{16}{8} = 2 \).

Pairs \((x, y)\) and \((y, x)\) are counted as different.

Present your answer as a sequence of fraction pairs inside a single \(\boxed{}\) structure. For example, \(\boxed{(\frac{1}{2},\frac{3}{4}),(\frac{1}{3},\frac{2}{4})}\)

\subsubsection{Variation}
\textbf{Actual Problem}\\
For $m = 29282$, determine $11$ pairs $(x,y)$ of rational numbers with $0 < x < 1$ and $0 < y < 1$ for which $x+my$ and $mx+y$ are both integers. An example for $m = 3$ is $(x, y) = \left(\frac{3}{8},\frac{7}{8}\right)$ since $x+3y=\frac{3}{8} + \frac{21}{8} = \frac{24}{8} = 3$ and $3x+y=\frac{9}{8} + \frac{7}{8} = \frac{16}{8} = 2$.

$(x,y)$ and $(y,x)$ are not considered to be the same pair.

Give your solution as a list of fraction pairs within a single $\boxed{}$ environment. For instance, $\boxed{(\frac{1}{2},\frac{3}{4}),(\frac{1}{3},\frac{2}{4})}$

\textbf{Revised Problem}\\
For \( m = 29282 \), identify 11 pairs \((x, y)\) of rational numbers such that both \(0 < x < 1\) and \(0 < y < 1\), where \(x + my\) and \(mx + y\) result in integer values. As an illustration, consider \( m = 3 \): one such pair is \((x, y) = \left(\frac{3}{8}, \frac{7}{8}\right)\) because \(x + 3y = \frac{3}{8} + \frac{21}{8} = \frac{24}{8} = 3\) and \(3x + y = \frac{9}{8} + \frac{7}{8} = \frac{16}{8} = 2\).

Note that the pair \((x, y)\) is distinct from \((y, x)\).

Present your answer as a sequence of fraction pairs enclosed in a single \(\boxed{}\) environment. For example, \(\boxed{(\frac{1}{2}, \frac{3}{4}), (\frac{1}{3}, \frac{2}{4})}\)

\subsection{dutch-2012-2}
\subsubsection{Variation}
\textbf{Actual Problem}\\
We number the columns of an $n \times n$ board from 1 to $n$. In each cell, we place a number. This is done in such a way that each row precisely contains the numbers 1 to n (in some order), and also each column contains the numbers 1 to $n$ (in some order). Next, each cell that contains a number greater than the cell’s column number, is coloured blue. In the figure below you can see an example for the case $n = 3$.

$$\begin{array}{ccc}
3 & 1 & 2 \\
1 & 2 & 3 \\
2 & 3 & 1
\end{array}$$

The blue cells are:
$$\begin{array}{ccc}
1 & 0 & 0 \\
0 & 0 & 0 \\
1 & 1 & 0
\end{array}$$

Suppose that $n = 5$. The numbers can be placed such that each row contains the same number of blue cells. Give an example of such a placement.

Output the answer between \verb|\begin{array}{...}| and \verb|\end{array}| inside of $\boxed{...}$. For example, $\boxed{\begin{array}{ccc}1 & 2 & 3 \\ 4 & 5 & 6 \\ 7 & 8 & 9\end{array}}$.

\textbf{Revised Problem}\\
Consider a square board of size $n \times n$ where columns are labeled from 1 to $n$. Each cell on this board is filled with a number so that each row and each column contains all numbers from 1 to $n$, but potentially in any sequence. A cell is painted blue if the number in it is greater than the column it is in. Below is an illustration for $n = 3$.

$$\begin{array}{ccc}
3 & 1 & 2 \\
1 & 2 & 3 \\
2 & 3 & 1
\end{array}$$

The cells colored blue are represented by:
$$\begin{array}{ccc}
1 & 0 & 0 \\
0 & 0 & 0 \\
1 & 1 & 0
\end{array}$$

If $n = 5$, arrange the numbers such that each row has an identical count of blue cells. Provide an example of such an arrangement.

Present your solution within the format \verb|\begin{array}{...}| and \verb|\end{array}| enclosed by $\boxed{...}$. For instance, $\boxed{\begin{array}{ccc}1 & 2 & 3 \\ 4 & 5 & 6 \\ 7 & 8 & 9\end{array}}$.

\subsubsection{Variation}
\textbf{Actual Problem}\\
We number the columns of an $n \times n$ board from 1 to $n$. In each cell, we place a number. This is done in such a way that each row precisely contains the numbers 1 to n (in some order), and also each column contains the numbers 1 to $n$ (in some order). Next, each cell that contains a number greater than the cell’s column number, is coloured blue. In the figure below you can see an example for the case $n = 3$.

$$\begin{array}{ccc}
3 & 1 & 2 \\
1 & 2 & 3 \\
2 & 3 & 1
\end{array}$$

The blue cells are:
$$\begin{array}{ccc}
1 & 0 & 0 \\
0 & 0 & 0 \\
1 & 1 & 0
\end{array}$$

Suppose that $n = 11$. The numbers can be placed such that each row contains the same number of blue cells. Give an example of such a placement.

Output the answer between \verb|\begin{array}{...}| and \verb|\end{array}| inside of $\boxed{...}$. For example, $\boxed{\begin{array}{ccc}1 & 2 & 3 \\ 4 & 5 & 6 \\ 7 & 8 & 9\end{array}}$.

\textbf{Revised Problem}\\
Consider an \( n \times n \) grid where columns are labeled from 1 to \( n \). Each cell in the grid contains a number such that each row has all the numbers from 1 to \( n \) (in any order), and each column also has all the numbers from 1 to \( n \) (in any order). Cells become colored blue if the number inside a cell exceeds the column number of that cell. Below is an example for \( n = 3 \):

$$\begin{array}{ccc}
3 & 1 & 2 \\
1 & 2 & 3 \\
2 & 3 & 1
\end{array}$$

The corresponding blue cells are:
$$\begin{array}{ccc}
1 & 0 & 0 \\
0 & 0 & 0 \\
1 & 1 & 0
\end{array}$$

For \( n = 11 \), arrange the numbers such that every row contains an equal number of blue cells. Provide an example of such a configuration.

Present the solution using the format \verb|\begin{array}{...}| and \verb|\end{array}| within $\boxed{...}$. For instance, $\boxed{\begin{array}{ccc}1 & 2 & 3 \\ 4 & 5 & 6 \\ 7 & 8 & 9\end{array}}$.

\subsubsection{Variation}
\textbf{Actual Problem}\\
We number the columns of an $n \times n$ board from 1 to $n$. In each cell, we place a number. This is done in such a way that each row precisely contains the numbers 1 to n (in some order), and also each column contains the numbers 1 to $n$ (in some order). Next, each cell that contains a number greater than the cell’s column number, is coloured blue. In the figure below you can see an example for the case $n = 3$.

$$\begin{array}{ccc}
3 & 1 & 2 \\
1 & 2 & 3 \\
2 & 3 & 1
\end{array}$$

The blue cells are:
$$\begin{array}{ccc}
1 & 0 & 0 \\
0 & 0 & 0 \\
1 & 1 & 0
\end{array}$$

Suppose that $n = 7$. The numbers can be placed such that each row contains the same number of blue cells. Give an example of such a placement.

Output the answer between \verb|\begin{array}{...}| and \verb|\end{array}| inside of $\boxed{...}$. For example, $\boxed{\begin{array}{ccc}1 & 2 & 3 \\ 4 & 5 & 6 \\ 7 & 8 & 9\end{array}}$.

\textbf{Revised Problem}\\
Consider an \(n \times n\) board where columns are numbered from 1 to \(n\). Each cell on the board is filled with a number such that every row includes each number from 1 to \(n\) exactly once, and every column also contains each number from 1 to \(n\) exactly once. A cell is colored blue if the number in it is greater than the column number of that cell. Below is an illustration for \(n = 3\):

$$\begin{array}{ccc}
3 & 1 & 2 \\
1 & 2 & 3 \\
2 & 3 & 1
\end{array}$$

In this example, the blue cells are:

$$\begin{array}{ccc}
1 & 0 & 0 \\
0 & 0 & 0 \\
1 & 1 & 0
\end{array}$$

Now, assume \(n = 7\). Arrange the numbers on the board such that each row has an equal number of blue cells. Provide an example of such an arrangement.

Present your solution within \verb|\begin{array}{...}| and \verb|\end{array}| enclosed in a box using the format $\boxed{...}$. For instance, $\boxed{\begin{array}{ccc}1 & 2 & 3 \\ 4 & 5 & 6 \\ 7 & 8 & 9\end{array}}$.

\subsubsection{Variation}
\textbf{Actual Problem}\\
We number the columns of an $n \times n$ board from 1 to $n$. In each cell, we place a number. This is done in such a way that each row precisely contains the numbers 1 to n (in some order), and also each column contains the numbers 1 to $n$ (in some order). Next, each cell that contains a number greater than the cell’s column number, is coloured blue. In the figure below you can see an example for the case $n = 3$.

$$\begin{array}{ccc}
3 & 1 & 2 \\
1 & 2 & 3 \\
2 & 3 & 1
\end{array}$$

The blue cells are:
$$\begin{array}{ccc}
1 & 0 & 0 \\
0 & 0 & 0 \\
1 & 1 & 0
\end{array}$$

Suppose that $n = 13$. The numbers can be placed such that each row contains the same number of blue cells. Give an example of such a placement.

Output the answer between \verb|\begin{array}{...}| and \verb|\end{array}| inside of $\boxed{...}$. For example, $\boxed{\begin{array}{ccc}1 & 2 & 3 \\ 4 & 5 & 6 \\ 7 & 8 & 9\end{array}}$.

\textbf{Revised Problem}\\
Consider an \( n \times n \) grid where each column is labeled from 1 to \( n \). Populate each cell with a number such that every row contains all the numbers from 1 to \( n \) in some sequence, and every column also includes all numbers from 1 to \( n \) in some sequence. Cells in which the number exceeds the column number are shaded blue. For example, when \( n = 3 \), the configuration looks like this:

$$\begin{array}{ccc}
3 & 1 & 2 \\
1 & 2 & 3 \\
2 & 3 & 1
\end{array}$$

The blue cells are represented as:
$$\begin{array}{ccc}
1 & 0 & 0 \\
0 & 0 & 0 \\
1 & 1 & 0
\end{array}$$

For \( n = 13 \), arrange the numbers so that each row has an identical count of blue cells. Provide such an arrangement as an example.

Present your solution using the format \verb|\begin{array}{...}| and \verb|\end{array}| enclosed within $\boxed{...}$. As an illustration, $\boxed{\begin{array}{ccc}1 & 2 & 3 \\ 4 & 5 & 6 \\ 7 & 8 & 9\end{array}}$.

\subsection{dutch-2014-3}
\subsubsection{Variation}
\textbf{Actual Problem}\\
At a volleyball tournament, each team plays exactly once against each other team. Each game has a winning team, which gets 1 point. The losing team gets 0 points. Draws do not occur. In the final ranking, only one team turns out to have the least number of points (so there is no shared last place). Moreover, each team, except for the team having the least number of points, lost exactly one game against a team that got less points in the final ranking.

Show, by providing an example, that the number of teams could be equal to 7.

Output the answer between \verb|\begin{array}{...}| and \verb|\end{array}| inside of $\boxed{...}$. For example, $\boxed{\begin{array}{ccc}1 & 2 & 3 \\ 4 & 5 & 6 \\ 7 & 8 & 9\end{array}}$.
The element in the $i$-th row and $j$-th column should be 1 if team $i$ won against team $j$, and 0 otherwise. The diagonal should be 0.

\textbf{Revised Problem}\\
In a volleyball tournament, each team competes exactly once against every other team. Each match results in a winner, who earns 1 point, while the loser receives 0 points. Ties are not possible. Ultimately, one team has the lowest number of points, ensuring no ties for last place. Furthermore, every team, apart from the one with the fewest points, has lost precisely one match against a team that has fewer total points by the end of the tournament.

Demonstrate, by constructing an example, that it is possible to have 7 teams with these conditions.

Present your solution between \verb|\begin{array}{...}| and \verb|\end{array}| enclosed within $\boxed{...}$. For instance, $\boxed{\begin{array}{ccc}1 & 2 & 3 \\ 4 & 5 & 6 \\ 7 & 8 & 9\end{array}}$. The entry in the $i$-th row and $j$-th column should be 1 if team $i$ defeated team $j$, and 0 otherwise. The diagonal entries should be 0.

\subsubsection{Variation}
\textbf{Actual Problem}\\
At a volleyball tournament, each team plays exactly once against each other team. Each game has a winning team, which gets 1 point. The losing team gets 0 points. Draws do not occur. In the final ranking, only one team turns out to have the least number of points (so there is no shared last place). Moreover, each team, except for the team having the least number of points, lost exactly one game against a team that got less points in the final ranking.

Show, by providing an example, that the number of teams could be equal to 10.

Output the answer between \verb|\begin{array}{...}| and \verb|\end{array}| inside of $\boxed{...}$. For example, $\boxed{\begin{array}{ccc}1 & 2 & 3 \\ 4 & 5 & 6 \\ 7 & 8 & 9\end{array}}$.
The element in the $i$-th row and $j$-th column should be 1 if team $i$ won against team $j$, and 0 otherwise. The diagonal should be 0.

\textbf{Revised Problem}\\
In a volleyball championship, each team competes exactly once against every other team. Each match has a winner awarded 1 point, while the losing team earns 0 points. Matches cannot end in a draw. In the tournament's final standings, only one team has the fewest points, ensuring a clear last place. Additionally, each team, except for the one with the fewest points, has lost one match to a team with a lower final ranking.

Demonstrate with an example that the tournament can consist of 10 teams.

Present your solution using \verb|\begin{array}{...}| and \verb|\end{array}| inside of $\boxed{...}$. For instance, $\boxed{\begin{array}{ccc}1 & 2 & 3 \\ 4 & 5 & 6 \\ 7 & 8 & 9\end{array}}$. Use 1 in the $i$-th row and $j$-th column if team $i$ defeated team $j$, and 0 otherwise. Ensure the diagonal entries are 0.

\subsubsection{Variation}
\textbf{Actual Problem}\\
At a volleyball tournament, each team plays exactly once against each other team. Each game has a winning team, which gets 1 point. The losing team gets 0 points. Draws do not occur. In the final ranking, only one team turns out to have the least number of points (so there is no shared last place). Moreover, each team, except for the team having the least number of points, lost exactly one game against a team that got less points in the final ranking.

Show, by providing an example, that the number of teams could be equal to 8.

Output the answer between \verb|\begin{array}{...}| and \verb|\end{array}| inside of $\boxed{...}$. For example, $\boxed{\begin{array}{ccc}1 & 2 & 3 \\ 4 & 5 & 6 \\ 7 & 8 & 9\end{array}}$.
The element in the $i$-th row and $j$-th column should be 1 if team $i$ won against team $j$, and 0 otherwise. The diagonal should be 0.

\textbf{Revised Problem}\\
Consider a volleyball tournament where every team competes against every other team exactly once. In each match, the victorious team earns 1 point, while the defeated team receives no points. Ties do not occur in this tournament. At the end of the tournament, a single team finishes with the fewest points, ensuring no tie for the last position. Additionally, each team, except the one with the lowest score, loses precisely one match to a team that has accumulated fewer points in the final standings.

Demonstrate, with an example, that it is possible for there to be 8 teams in this tournament.

Present your solution enclosed within \verb|\begin{array}{...}| and \verb|\end{array}| inside of $\boxed{...}$. For instance, $\boxed{\begin{array}{ccc}1 & 2 & 3 \\ 4 & 5 & 6 \\ 7 & 8 & 9\end{array}}$. Each element in the $i$-th row and $j$-th column should be 1 if team $i$ defeated team $j$, and 0 if not. The diagonal should contain 0s.

\subsubsection{Variation}
\textbf{Actual Problem}\\
At a volleyball tournament, each team plays exactly once against each other team. Each game has a winning team, which gets 1 point. The losing team gets 0 points. Draws do not occur. In the final ranking, only one team turns out to have the least number of points (so there is no shared last place). Moreover, each team, except for the team having the least number of points, lost exactly one game against a team that got less points in the final ranking.

Show, by providing an example, that the number of teams could be equal to 11.

Output the answer between \verb|\begin{array}{...}| and \verb|\end{array}| inside of $\boxed{...}$. For example, $\boxed{\begin{array}{ccc}1 & 2 & 3 \\ 4 & 5 & 6 \\ 7 & 8 & 9\end{array}}$.
The element in the $i$-th row and $j$-th column should be 1 if team $i$ won against team $j$, and 0 otherwise. The diagonal should be 0.

\textbf{Revised Problem}\\
In a volleyball championship, every team competes against each of the other teams exactly once. For each match, the winning team earns 1 point, while the loser earns 0 points. There are no ties in any match. In the final standings, only one team finishes with the fewest points, meaning there is a unique last-ranked team. Additionally, every team, except the one with the fewest points, has lost exactly one match to a team with a lower point total in the final standings.

Demonstrate with an example that it is possible for the number of teams to be 11.

Present the solution within \verb|\begin{array}{...}| and \verb|\end{array}| wrapped inside $\boxed{...}$. For instance, $\boxed{\begin{array}{ccc}1 & 2 & 3 \\ 4 & 5 & 6 \\ 7 & 8 & 9\end{array}}$.
The entry in the $i$-th row and $j$-th column should be 1 if team $i$ defeated team $j$, and 0 if not. Ensure the diagonal entries are 0.

\subsection{dutch-2018-1}
\subsubsection{Variation}
\textbf{Actual Problem}\\
We call a positive integer a shuffle number if the following hold:

\begin{enumerate}
\item All digits are nonzero.
\item The number is divisible by 11.
\item The number is divisible by 12. If you put the digits in any other order, you again have a number that is divisible by 12.
\end{enumerate}
Give 50 distinct 10-digit shuffle numbers.

Output the answer as a comma separated list inside of $\boxed{...}$. For example $\boxed{1, 2, 3}$.

\textbf{Revised Problem}\\
Consider a positive integer referred to as a shuffle number if it satisfies the following conditions:

1. Each digit in the number is greater than zero.
2. The integer is a multiple of 11.
3. The integer is a multiple of 12, and rearranging its digits in any sequence still results in a number that is a multiple of 12.

Find 50 unique 10-digit shuffle numbers.

Present your answer as a list of numbers separated by commas within $\boxed{...}$. For instance, $\boxed{1, 2, 3}$.

\subsubsection{Variation}
\textbf{Actual Problem}\\
We call a positive integer a shuffle number if the following hold:

\begin{enumerate}
\item All digits are nonzero.
\item The number is divisible by 11.
\item The number is divisible by 12. If you put the digits in any other order, you again have a number that is divisible by 12.
\end{enumerate}
Give 31 distinct 17-digit shuffle numbers.

Output the answer as a comma separated list inside of $\boxed{...}$. For example $\boxed{1, 2, 3}$.

\textbf{Revised Problem}\\
A positive integer is termed a shuffle number if it satisfies the following criteria:

\begin{enumerate}
\item Every digit in the number is greater than zero.
\item The integer is a multiple of 11.
\item The integer is a multiple of 12, and any permutation of its digits also results in a number that is divisible by 12.
\end{enumerate}
Provide 31 unique shuffle numbers, each consisting of 17 digits.

Present the solution as a list of numbers separated by commas and enclosed within $\boxed{...}$. For instance, $\boxed{1, 2, 3}$.

\subsubsection{Variation}
\textbf{Actual Problem}\\
We call a positive integer a shuffle number if the following hold:

\begin{enumerate}
\item All digits are nonzero.
\item The number is divisible by 11.
\item The number is divisible by 12. If you put the digits in any other order, you again have a number that is divisible by 12.
\end{enumerate}
Give 41 distinct 13-digit shuffle numbers.

Output the answer as a comma separated list inside of $\boxed{...}$. For example $\boxed{1, 2, 3}$.

\textbf{Revised Problem}\\
A positive integer is termed a shuffle number if it satisfies the following criteria:

\begin{enumerate}
\item Each digit is greater than zero.
\item The integer is a multiple of 11.
\item The integer is a multiple of 12, and any permutation of its digits also results in a number that is a multiple of 12.
\end{enumerate}
Identify 41 unique shuffle numbers that each have 13 digits.

Present your solution as a list of numbers separated by commas within a $\boxed{...}$ format. For instance, $\boxed{1, 2, 3}$.

\subsubsection{Variation}
\textbf{Actual Problem}\\
We call a positive integer a shuffle number if the following hold:

\begin{enumerate}
\item All digits are nonzero.
\item The number is divisible by 11.
\item The number is divisible by 12. If you put the digits in any other order, you again have a number that is divisible by 12.
\end{enumerate}
Give 10 distinct 11-digit shuffle numbers.

Output the answer as a comma separated list inside of $\boxed{...}$. For example $\boxed{1, 2, 3}$.

\textbf{Revised Problem}\\
A positive integer is called a shuffle number if it satisfies the following conditions:

\begin{enumerate}
\item Each digit is a nonzero integer.
\item The integer is a multiple of 11.
\item The integer is a multiple of 12, and rearranging its digits in any order still results in a number that is divisible by 12.
\end{enumerate}
List 10 different 11-digit shuffle numbers.

Present your answer as a series of numbers separated by commas, enclosed within $\boxed{...}$. For instance, $\boxed{1, 2, 3}$.

\subsection{dutch-2018-2}
\subsubsection{Variation}
\textbf{Actual Problem}\\
The numbers 1 to 15 are each coloured blue or red. Determine 4 possible colourings that satisfy the following rules:

\begin{enumerate}
\item The number 15 is red.
\item If numbers $x$ and $y$ have different colours and $x + y \leqslant 15$, then $x + y$ is blue.
\item If numbers $x$ and $y$ have different colours and $x \cdot y \leqslant 15$, then $x + y$ is red.
\end{enumerate}

Output a list of comma separated lists inside of $\boxed{...}$. Each list indicates the numbers that are coloured red for that option, all other numbers are coloured blue. For example, $\boxed{(1,2,3),(1,5,7)}$ indicates two solutions where in the first only the numbers 1, 2, and 3 are coloured red. 

\textbf{Revised Problem}\\
Each of the numbers from 1 to 15 is colored either red or blue. Identify four different arrangements of these colorings that adhere to the following conditions:

1. The number 15 must always be colored red.
2. Whenever two numbers $x$ and $y$ have different colors and $x + y \leq 15$, the sum $x + y$ must be colored blue.
3. Whenever two numbers $x$ and $y$ have different colors and $x \cdot y \leq 15$, the sum $x + y$ must be colored red.

Provide your answer as a set of lists separated by commas, enclosed within $\boxed{...}$. Each list represents a grouping of numbers that are colored red for that particular solution, with all other numbers being blue. For instance, $\boxed{(1,2,3),(1,5,7)}$ signifies two solutions where in the first list, the numbers 1, 2, and 3 are colored red.

\subsubsection{Variation}
\textbf{Actual Problem}\\
The numbers 1 to 32 are each coloured blue or red. Determine 6 possible colourings that satisfy the following rules:

\begin{enumerate}
\item The number 32 is red.
\item If numbers $x$ and $y$ have different colours and $x + y \leqslant 32$, then $x + y$ is blue.
\item If numbers $x$ and $y$ have different colours and $x \cdot y \leqslant 32$, then $x + y$ is red.
\end{enumerate}

Output a list of comma separated lists inside of $\boxed{...}$. Each list indicates the numbers that are coloured red for that option, all other numbers are coloured blue. For example, $\boxed{(1,2,3),(1,5,7)}$ indicates two solutions where in the first only the numbers 1, 2, and 3 are coloured red. 

\textbf{Revised Problem}\\
The integers from 1 through 32 are each assigned one of two colors: blue or red. Identify 6 feasible color arrangements that adhere to the following criteria:

1. The integer 32 must be assigned the color red.
2. If two numbers, $x$ and $y$, have distinct colors and their sum is at most 32, then their sum must be assigned the color blue.
3. If two numbers, $x$ and $y$, have distinct colors and their product does not exceed 32, then their sum is required to be red.

Present your solutions as a series of comma-separated lists enclosed within $\boxed{...}$. Each list represents the set of numbers that are colored red for a particular solution, with all remaining numbers colored blue. For instance, $\boxed{(1,2,3),(1,5,7)}$ demonstrates two solutions where, in the first solution, only numbers 1, 2, and 3 are red.

\subsubsection{Variation}
\textbf{Actual Problem}\\
The numbers 1 to 24 are each coloured blue or red. Determine 5 possible colourings that satisfy the following rules:

\begin{enumerate}
\item The number 24 is red.
\item If numbers $x$ and $y$ have different colours and $x + y \leqslant 24$, then $x + y$ is blue.
\item If numbers $x$ and $y$ have different colours and $x \cdot y \leqslant 24$, then $x + y$ is red.
\end{enumerate}

Output a list of comma separated lists inside of $\boxed{...}$. Each list indicates the numbers that are coloured red for that option, all other numbers are coloured blue. For example, $\boxed{(1,2,3),(1,5,7)}$ indicates two solutions where in the first only the numbers 1, 2, and 3 are coloured red. 

\textbf{Revised Problem}\\
The integers from 1 to 24 are each painted either blue or red. Identify 5 different configurations that adhere to the following conditions:

\begin{enumerate}
\item The integer 24 is painted red.
\item If integers $x$ and $y$ are painted in different colors and $x + y \leqslant 24$, then $x + y$ must be painted blue.
\item If integers $x$ and $y$ are painted in different colors and $x \cdot y \leqslant 24$, then $x + y$ must be painted red.
\end{enumerate}

Provide a list of comma-separated lists enclosed in $\boxed{...}$. Each list represents the integers that are painted red for that specific configuration, with all other integers painted blue. For instance, $\boxed{(1,2,3),(1,5,7)}$ shows two solutions where in the first configuration, integers 1, 2, and 3 are red.

\subsubsection{Variation}
\textbf{Actual Problem}\\
The numbers 1 to 21 are each coloured blue or red. Determine 4 possible colourings that satisfy the following rules:

\begin{enumerate}
\item The number 21 is red.
\item If numbers $x$ and $y$ have different colours and $x + y \leqslant 21$, then $x + y$ is blue.
\item If numbers $x$ and $y$ have different colours and $x \cdot y \leqslant 21$, then $x + y$ is red.
\end{enumerate}

Output a list of comma separated lists inside of $\boxed{...}$. Each list indicates the numbers that are coloured red for that option, all other numbers are coloured blue. For example, $\boxed{(1,2,3),(1,5,7)}$ indicates two solutions where in the first only the numbers 1, 2, and 3 are coloured red. 

\textbf{Revised Problem}\\
Consider the numbers from 1 to 21, each painted either blue or red. Identify four distinct ways to paint these numbers such that the following conditions hold:

\begin{enumerate}
\item The number 21 is painted red.
\item If two numbers $x$ and $y$ are painted in different colors and $x + y \leqslant 21$, then the sum $x + y$ must be painted blue.
\item If two numbers $x$ and $y$ are painted in different colors and $x \cdot y \leqslant 21$, then the sum $x + y$ must be painted red.
\end{enumerate}

Provide a list of comma-separated tuples enclosed in $\boxed{...}$. Each tuple represents the numbers that are painted red in a particular scenario, with all remaining numbers being painted blue. For example, $\boxed{(1,2,3),(1,5,7)}$ indicates two scenarios where in the first scenario, only numbers 1, 2, and 3 are red.

\subsection{dutch-2019-2}
\subsubsection{Variation}
\textbf{Actual Problem}\\
There are $n$ guests at a party. Any two guests are either friends or not friends. Every guest is friends with exactly 4 of the other guests. Whenever a guest is not friends with two other guests, those two other guests cannot be friends with each other either.

Find $3$ configurations of friendships that satisfy these conditions. Each of the configurations should have a different value of $n$.

Output each configuration between \verb|\begin{array}{...}| and \verb|\end{array}|. All configurations should appear inside of a single $\boxed{...}$. For example, $\boxed{\begin{array}{ccc}0 & 1 & 0 \\ 1 & 0 & 0 \\ 0 & 0 & 0\end{array},\begin{array}{cc}0 & 1 \\ 1 & 0\end{array}}$.
The element in the $i$-th row and $j$-th column should be 1 if guest $i$ is friends with guest $j$, and 0 otherwise. Elements on the diagonal should be 0.

\textbf{Revised Problem}\\
At a social gathering, there are $n$ attendees. Each pair of attendees is either acquaintances or not. Every attendee has exactly 4 acquaintances among the other attendees. Furthermore, if two attendees are not acquainted with each other, they also cannot share any mutual acquaintances.

Identify $3$ distinct patterns of acquaintance that meet these criteria, each with a different number of attendees, $n$.

Present each pattern within \verb|\begin{array}{...}| and \verb|\end{array}|. Place all patterns together inside a single $\boxed{...}$. For instance, $\boxed{\begin{array}{ccc}0 & 1 & 0 \\ 1 & 0 & 0 \\ 0 & 0 & 0\end{array},\begin{array}{cc}0 & 1 \\ 1 & 0\end{array}}$. The entry at the $i$-th row and $j$-th column should be 1 if attendee $i$ is acquainted with attendee $j$, and 0 if not. Diagonal entries should always be 0.

\subsubsection{Variation}
\textbf{Actual Problem}\\
There are $n$ guests at a party. Any two guests are either friends or not friends. Every guest is friends with exactly 7 of the other guests. Whenever a guest is not friends with two other guests, those two other guests cannot be friends with each other either.

Find $2$ configurations of friendships that satisfy these conditions. Each of the configurations should have a different value of $n$.

Output each configuration between \verb|\begin{array}{...}| and \verb|\end{array}|. All configurations should appear inside of a single $\boxed{...}$. For example, $\boxed{\begin{array}{ccc}0 & 1 & 0 \\ 1 & 0 & 0 \\ 0 & 0 & 0\end{array},\begin{array}{cc}0 & 1 \\ 1 & 0\end{array}}$.
The element in the $i$-th row and $j$-th column should be 1 if guest $i$ is friends with guest $j$, and 0 otherwise. Elements on the diagonal should be 0.

\textbf{Revised Problem}\\
At a social gathering, there are $n$ attendees. Each pair of attendees is classified as either friends or not friends. Every attendee is friends with precisely 7 other attendees. Additionally, if two attendees are not friends with a third attendee, they must also not be friends with each other.

Identify two unique friendship arrangements that adhere to these criteria, with each arrangement having a distinct number of attendees, $n$.

Present each arrangement using the \verb|\begin{array}{...}| and \verb|\end{array}| environment. All arrangements should be enclosed within a single $\boxed{...}$. For instance, $\boxed{\begin{array}{ccc}0 & 1 & 0 \\ 1 & 0 & 0 \\ 0 & 0 & 0\end{array},\begin{array}{cc}0 & 1 \\ 1 & 0\end{array}}$.
The entry in the $i$-th row and $j$-th column should be 1 if attendee $i$ is friends with attendee $j$, and 0 if they are not. Diagonal elements should be 0.

\subsubsection{Variation}
\textbf{Actual Problem}\\
There are $n$ guests at a party. Any two guests are either friends or not friends. Every guest is friends with exactly 5 of the other guests. Whenever a guest is not friends with two other guests, those two other guests cannot be friends with each other either.

Find $2$ configurations of friendships that satisfy these conditions. Each of the configurations should have a different value of $n$.

Output each configuration between \verb|\begin{array}{...}| and \verb|\end{array}|. All configurations should appear inside of a single $\boxed{...}$. For example, $\boxed{\begin{array}{ccc}0 & 1 & 0 \\ 1 & 0 & 0 \\ 0 & 0 & 0\end{array},\begin{array}{cc}0 & 1 \\ 1 & 0\end{array}}$.
The element in the $i$-th row and $j$-th column should be 1 if guest $i$ is friends with guest $j$, and 0 otherwise. Elements on the diagonal should be 0.

\textbf{Revised Problem}\\
At a gathering with $n$ attendees, each pair of attendees is either friends or not. Each attendee has precisely 5 friends among the other attendees. If any two attendees are both not friends with a given attendee, they must also not be friends with each other.

Determine two distinct friendship patterns that adhere to these criteria. Each pattern should correspond to a different number of attendees, $n$.

Present each friendship pattern using \verb|\begin{array}{...}| and \verb|\end{array}|. Place all patterns within a single $\boxed{...}$. For instance, $\boxed{\begin{array}{ccc}0 & 1 & 0 \\ 1 & 0 & 0 \\ 0 & 0 & 0\end{array},\begin{array}{cc}0 & 1 \\ 1 & 0\end{array}}$.
The entry at the $i$-th row and $j$-th column should be 1 if attendee $i$ is friends with attendee $j$, and 0 otherwise. The diagonal entries must be 0.

\subsubsection{Variation}
\textbf{Actual Problem}\\
There are $n$ guests at a party. Any two guests are either friends or not friends. Every guest is friends with exactly 8 of the other guests. Whenever a guest is not friends with two other guests, those two other guests cannot be friends with each other either.

Find $3$ configurations of friendships that satisfy these conditions. Each of the configurations should have a different value of $n$.

Output each configuration between \verb|\begin{array}{...}| and \verb|\end{array}|. All configurations should appear inside of a single $\boxed{...}$. For example, $\boxed{\begin{array}{ccc}0 & 1 & 0 \\ 1 & 0 & 0 \\ 0 & 0 & 0\end{array},\begin{array}{cc}0 & 1 \\ 1 & 0\end{array}}$.
The element in the $i$-th row and $j$-th column should be 1 if guest $i$ is friends with guest $j$, and 0 otherwise. Elements on the diagonal should be 0.

\textbf{Revised Problem}\\
Imagine a party with $n$ attendees. Between any two attendees, they are either friends or they are not. Each attendee has precisely 8 friends among the other attendees. Additionally, if an attendee is not friends with two others, then those two others are also not friends with each other.

Identify 3 distinct configurations of friendships that adhere to these conditions. Each configuration should employ a different value of $n$.

Present each configuration using \verb|\begin{array}{...}| and \verb|\end{array}|. Ensure all configurations are enclosed within a single $\boxed{...}$. For instance, $\boxed{\begin{array}{ccc}0 & 1 & 0 \\ 1 & 0 & 0 \\ 0 & 0 & 0\end{array},\begin{array}{cc}0 & 1 \\ 1 & 0\end{array}}$.
Within the matrix, the entry in the $i$-th row and $j$-th column should be 1 if attendee $i$ is friends with attendee $j$, and 0 otherwise. Diagonal entries should all be 0.

\subsection{dutch-2021-3}
\subsubsection{Variation}
\textbf{Actual Problem}\\
A frog jumps around on the grid points in the plane, from one grid point to another. The frog starts at the point (0, 0). Then it makes, successively, a jump of one step horizontally, a jump of 2 steps vertically, a jump of 3 steps horizontally, a jump of 4 steps vertically, et cetera. For $n = 32$, find a sequence of jumps that brings the frog back to the point $(0, 0)$ after $n$ jumps.

Output the answer as comma separated list inside of \boxed{...}. For example \boxed{(0,0),(1,0),(1,2),..,(0,0)}. Each point should be in the form $(x, y)$ and indicates the next step in the frog's journey, beginning and ending with (0,0).

\textbf{Revised Problem}\\
A frog hops between grid points on a plane, starting from the origin (0, 0). It first makes a horizontal jump of 1 unit, followed by a vertical jump of 2 units, then a horizontal jump of 3 units, a vertical jump of 4 units, and continues this pattern. For $n = 32$, determine a sequence of jumps that returns the frog to the point (0, 0) after completing $n$ jumps.

Provide the solution as a comma-separated list enclosed within \boxed{...}. For instance, \boxed{(0,0),(1,0),(1,2),..,(0,0)}. Ensure each point is formatted as $(x, y)$ and represents the frog's consecutive positions from the start to the endpoint (0,0).

\subsubsection{Variation}
\textbf{Actual Problem}\\
A frog jumps around on the grid points in the plane, from one grid point to another. The frog starts at the point (0, 0). Then it makes, successively, a jump of one step horizontally, a jump of 2 steps vertically, a jump of 3 steps horizontally, a jump of 4 steps vertically, et cetera. For $n = 55$, find a sequence of jumps that brings the frog back to the point $(0, 0)$ after $n$ jumps.

Output the answer as comma separated list inside of \boxed{...}. For example \boxed{(0,0),(1,0),(1,2),..,(0,0)}. Each point should be in the form $(x, y)$ and indicates the next step in the frog's journey, beginning and ending with (0,0).

\textbf{Revised Problem}\\
A frog is hopping on a grid in the plane, moving from one grid point to another. It begins its journey at the origin, (0, 0). The frog then performs a series of jumps: the first jump is 1 unit horizontally, the second jump is 2 units vertically, the third jump is 3 units horizontally, the fourth jump is 4 units vertically, and this pattern continues. Determine a sequence of 55 jumps that returns the frog to the origin, (0, 0), at the end of these jumps.

Present the solution as a list of coordinates enclosed in \boxed{...}. For instance, \boxed{(0,0),(1,0),(1,2),..,(0,0)}. Each coordinate should be in the format $(x, y)$ representing each position in the frog's path, starting and concluding with (0,0).

\subsubsection{Variation}
\textbf{Actual Problem}\\
A frog jumps around on the grid points in the plane, from one grid point to another. The frog starts at the point (0, 0). Then it makes, successively, a jump of one step horizontally, a jump of 2 steps vertically, a jump of 3 steps horizontally, a jump of 4 steps vertically, et cetera. For $n = 39$, find a sequence of jumps that brings the frog back to the point $(0, 0)$ after $n$ jumps.

Output the answer as comma separated list inside of \boxed{...}. For example \boxed{(0,0),(1,0),(1,2),..,(0,0)}. Each point should be in the form $(x, y)$ and indicates the next step in the frog's journey, beginning and ending with (0,0).

\textbf{Revised Problem}\\
A frog is hopping on a grid in a plane, starting its journey at the origin point (0, 0). It follows a specific jumping pattern: the first jump is one unit horizontally, the second jump is two units vertically, the third jump is three units horizontally, the fourth jump is four units vertically, and so on. For $n = 39$, find a sequence of hops that allows the frog to return to its starting position, (0, 0), after completing $n$ hops.

Present the solution as a sequence of coordinates in a comma-separated list enclosed within \boxed{...}. For instance, \boxed{(0,0),(1,0),(1,2),..,(0,0)}. Each coordinate should be formatted as $(x, y)$, representing each step in the frog's path, starting and concluding at (0,0).

\subsubsection{Variation}
\textbf{Actual Problem}\\
A frog jumps around on the grid points in the plane, from one grid point to another. The frog starts at the point (0, 0). Then it makes, successively, a jump of one step horizontally, a jump of 2 steps vertically, a jump of 3 steps horizontally, a jump of 4 steps vertically, et cetera. For $n = 40$, find a sequence of jumps that brings the frog back to the point $(0, 0)$ after $n$ jumps.

Output the answer as comma separated list inside of \boxed{...}. For example \boxed{(0,0),(1,0),(1,2),..,(0,0)}. Each point should be in the form $(x, y)$ and indicates the next step in the frog's journey, beginning and ending with (0,0).

\textbf{Revised Problem}\\
A frog is hopping on the grid points in a plane, moving from one grid point to another. It begins its journey at the point (0, 0). The frog then makes a series of moves: 1 step in a horizontal direction, followed by 2 steps in a vertical direction, then 3 steps horizontally, 4 steps vertically, and so on. Determine a sequence of 40 such moves that results in the frog returning to its starting position of (0, 0).

Present the solution as a comma-separated list enclosed in \boxed{...}. For instance, \boxed{(0,0),(1,0),(1,2),..,(0,0)}. Each coordinate should be formatted as $(x, y)$ and represent the subsequent position of the frog, starting and finishing at (0,0).

\subsection{dutch-2024-2}
\subsubsection{Variation}
\textbf{Actual Problem}\\
Mila stands on an infinitely large board divided into squares and starts moving. An $n$-jump is a movement in which Mila moves one square left, right, up, or down and then $n$ squares in a direction perpendicular to that. Below is an example where Mila starts in the (0,0) square and first does a 1-jump, followed by a 2-jump and then a 3-jump, 4-jump and 5-jump:

(0,0),(1,1),(2,-1),(5,0),(9,1),(4,0)

Suppose Mila first does a 1-jump, then a 2-jump, then a 3-jump, a 4-jump, and so on. Finally, she does a 36-jump. Find a sequence of jumps that brings Mila back to the (0,0) square after these $36$ jumps.

Output the answer as a comma separated list inside of \boxed{...}. For example \boxed{(0,0),(1,0),(1,2),..,(0,0)}. Each point should be in the form $(x, y)$ and indicates the next step in the frog's journey, beginning and ending with (0,0).

\textbf{Revised Problem}\\
Mila is on an unbounded grid composed of square tiles and begins her journey from an initial position. A jump of magnitude $n$ involves moving one square in any cardinal direction (north, south, east, or west), followed by moving $n$ squares in a direction that is perpendicular to the initial movement. For instance, consider Mila starting at coordinates (0,0) and executing a series of jumps: a 1-jump, a 2-jump, a 3-jump, a 4-jump, followed by a 5-jump:

(0,0),(1,1),(2,-1),(5,0),(9,1),(4,0)

Imagine Mila begins with a 1-jump, proceeds to a 2-jump, then a 3-jump, continues this pattern up to a 36-jump. Determine the sequence of jumps that results in Mila returning to her origin point (0,0) after completing all 36 jumps.

Present the solution as a list of coordinates enclosed within \boxed{...}. For example \boxed{(0,0),(1,0),(1,2),..,(0,0)}. Each coordinate should be formatted as $(x, y)$, representing each step in Mila's path, with both the starting and ending points being (0,0).

\subsubsection{Variation}
\textbf{Actual Problem}\\
Mila stands on an infinitely large board divided into squares and starts moving. An $n$-jump is a movement in which Mila moves one square left, right, up, or down and then $n$ squares in a direction perpendicular to that. Below is an example where Mila starts in the (0,0) square and first does a 1-jump, followed by a 2-jump and then a 3-jump, 4-jump and 5-jump:

(0,0),(1,1),(2,-1),(5,0),(9,1),(4,0)

Suppose Mila first does a 1-jump, then a 2-jump, then a 3-jump, a 4-jump, and so on. Finally, she does a 53-jump. Find a sequence of jumps that brings Mila back to the (0,0) square after these $53$ jumps.

Output the answer as a comma separated list inside of \boxed{...}. For example \boxed{(0,0),(1,0),(1,2),..,(0,0)}. Each point should be in the form $(x, y)$ and indicates the next step in the frog's journey, beginning and ending with (0,0).

\textbf{Revised Problem}\\
Mila is positioned on a limitless grid composed of square tiles and begins her journey by moving across them. An $n$-jump consists of stepping one tile either north, south, east, or west, followed by $n$ tiles in a direction that is perpendicular to the initial direction. Consider the following scenario where Mila initiates her movement from the tile at coordinates (0,0). She performs a 1-jump, then a 2-jump, subsequently a 3-jump, a 4-jump, and continues this pattern up to a 5-jump:

(0,0),(1,1),(2,-1),(5,0),(9,1),(4,0)

Assume Mila executes her jumps in the sequence of a 1-jump, followed by a 2-jump, next a 3-jump, then a 4-jump, continuing in this manner up to a 53-jump. Determine a sequence of such jumps that enables Mila to return to her initial position at (0,0) following these 53 movements.

Express the solution as a list of coordinates, separated by commas, enclosed within \boxed{...}. For instance, \boxed{(0,0),(1,0),(1,2),..,(0,0)}. Each coordinate pair should be formatted as $(x, y)$ and should specify Mila's subsequent position at each step of her journey, starting and concluding at (0,0).

\subsubsection{Variation}
\textbf{Actual Problem}\\
Mila stands on an infinitely large board divided into squares and starts moving. An $n$-jump is a movement in which Mila moves one square left, right, up, or down and then $n$ squares in a direction perpendicular to that. Below is an example where Mila starts in the (0,0) square and first does a 1-jump, followed by a 2-jump and then a 3-jump, 4-jump and 5-jump:

(0,0),(1,1),(2,-1),(5,0),(9,1),(4,0)

Suppose Mila first does a 1-jump, then a 2-jump, then a 3-jump, a 4-jump, and so on. Finally, she does a 37-jump. Find a sequence of jumps that brings Mila back to the (0,0) square after these $37$ jumps.

Output the answer as a comma separated list inside of \boxed{...}. For example \boxed{(0,0),(1,0),(1,2),..,(0,0)}. Each point should be in the form $(x, y)$ and indicates the next step in the frog's journey, beginning and ending with (0,0).

\textbf{Revised Problem}\\
Mila is positioned on an endless grid divided into squares and begins a sequence of movements. An $n$-jump involves Mila moving one square in one of the four cardinal directions (left, right, up, or down), followed by moving $n$ squares in a direction perpendicular to the first. Below is an illustration where Mila starts at position (0,0) and executes a sequence of jumps: initially a 1-jump, then a 2-jump, followed by a 3-jump, then a 4-jump, and finally a 5-jump:

(0,0),(1,1),(2,-1),(5,0),(9,1),(4,0)

Assume Mila performs a sequence of jumps starting with a 1-jump, then a 2-jump, a 3-jump, a 4-jump, continuing up to a 37-jump. Determine a sequence of these jumps that returns Mila to the (0,0) position after completing all 37 jumps.

Present the solution as a comma-separated list enclosed within \boxed{...}. For instance, \boxed{(0,0),(1,0),(1,2),..,(0,0)}. Each position should be formatted as $(x, y)$ and represent Mila’s position after each jump, beginning and concluding with (0,0).

\subsubsection{Variation}
\textbf{Actual Problem}\\
Mila stands on an infinitely large board divided into squares and starts moving. An $n$-jump is a movement in which Mila moves one square left, right, up, or down and then $n$ squares in a direction perpendicular to that. Below is an example where Mila starts in the (0,0) square and first does a 1-jump, followed by a 2-jump and then a 3-jump, 4-jump and 5-jump:

(0,0),(1,1),(2,-1),(5,0),(9,1),(4,0)

Suppose Mila first does a 1-jump, then a 2-jump, then a 3-jump, a 4-jump, and so on. Finally, she does a 28-jump. Find a sequence of jumps that brings Mila back to the (0,0) square after these $28$ jumps.

Output the answer as a comma separated list inside of \boxed{...}. For example \boxed{(0,0),(1,0),(1,2),..,(0,0)}. Each point should be in the form $(x, y)$ and indicates the next step in the frog's journey, beginning and ending with (0,0).

\textbf{Revised Problem}\\
Mila is positioned on an infinite grid of squares, each labeled with Cartesian coordinates. She starts her movement from the square at coordinates (0,0). A jump labeled as an $n$-jump involves Mila moving one square in either the horizontal or vertical direction, followed by $n$ squares in the direction perpendicular to the first move. An example sequence of Mila's movement starting at (0,0) could be: performing a 1-jump, then a 2-jump, followed by a 3-jump, continuing with a 4-jump, and finally a 5-jump, resulting in the following path:

(0,0), (1,1), (2,-1), (5,0), (9,1), (4,0)

Mila begins by making a 1-jump, followed by a 2-jump, then a 3-jump, proceeding with a 4-jump, and so forth, ultimately completing a 28-jump. Determine a valid sequence of these jumps that will return Mila to the starting coordinate (0,0) after a total of 28 jumps.

Present your solution as a series of coordinates enclosed in \boxed{...}. For instance, \boxed{(0,0),(1,0),(1,2),..,(0,0)}. Each pair of coordinates should be formatted as $(x, y)$ and should represent a point in Mila's journey, starting and concluding with (0,0).

\section{emc}
\subsection{emc-2016-1}
\subsubsection{Variation}
\textbf{Actual Problem}\\
Find a sequence $a_1, ..., a_{10}$ of positive integers such that every sum

$$a_r + a_{r+1} + ... + a_{s-1} + a_s$$

(with $1 \leqslant r \leqslant s \leqslant 10$) is a composite number, but

$\text{gcd}(a_i, a_{i+1}) = 1$ for all $i = 1,2,...,9$ and $\text{gcd}(a_i, a_{i+2}) = 1$ for all $i = 1,2,...,8$.

Output the answer as a comma separated list of integers inside of $\boxed{...}$. For example $\boxed{1,2,3,4}$.

\textbf{Revised Problem}\\
Determine a sequence of ten positive integers $b_1, b_2, ..., b_{10}$ such that every possible sum of the form

$$b_r + b_{r+1} + ... + b_{s-1} + b_s$$

(where $1 \leq r \leq s \leq 10$) results in a composite number. Additionally, ensure that $\text{gcd}(b_i, b_{i+1}) = 1$ for each $i = 1, 2, \ldots, 9$ and $\text{gcd}(b_i, b_{i+2}) = 1$ for each $i = 1, 2, \ldots, 8$.

Present your solution as a sequence of integers separated by commas, enclosed in $\boxed{...}$. For instance, $\boxed{1,2,3,4}$.

\subsubsection{Variation}
\textbf{Actual Problem}\\
Find a sequence $a_1, ..., a_{14}$ of positive integers such that every sum

$$a_r + a_{r+1} + ... + a_{s-1} + a_s$$

(with $1 \leqslant r \leqslant s \leqslant 14$) is a composite number, but

$\text{gcd}(a_i, a_{i+1}) = 1$ for all $i = 1,2,...,13$ and $\text{gcd}(a_i, a_{i+2}) = 1$ for all $i = 1,2,...,12$.

Output the answer as a comma separated list of integers inside of $\boxed{...}$. For example $\boxed{1,2,3,4}$.

\textbf{Revised Problem}\\
Identify a sequence of positive integers $b_1, b_2, \ldots, b_{14}$ such that each possible sum

$$b_t + b_{t+1} + \cdots + b_{u-1} + b_u$$

(for $1 \le t \le u \le 14$) results in a composite number. Additionally, ensure that $\text{gcd}(b_j, b_{j+1}) = 1$ for each $j = 1, 2, \ldots, 13$ and $\text{gcd}(b_j, b_{j+2}) = 1$ for each $j = 1, 2, \ldots, 12$.

Present your solution as a list of integers separated by commas, enclosed in $\boxed{...}$. For instance, $\boxed{5,6,7,8}$.

\subsubsection{Variation}
\textbf{Actual Problem}\\
Find a sequence $a_1, ..., a_{8}$ of positive integers such that every sum

$$a_r + a_{r+1} + ... + a_{s-1} + a_s$$

(with $1 \leqslant r \leqslant s \leqslant 8$) is a composite number, but

$\text{gcd}(a_i, a_{i+1}) = 1$ for all $i = 1,2,...,7$ and $\text{gcd}(a_i, a_{i+2}) = 1$ for all $i = 1,2,...,6$.

Output the answer as a comma separated list of integers inside of $\boxed{...}$. For example $\boxed{1,2,3,4}$.

\textbf{Revised Problem}\\
Identify a sequence of positive integers $b_1, b_2, \ldots, b_8$ such that the sum of any consecutive subsequence 

$$b_r + b_{r+1} + \cdots + b_s$$ 

(where $1 \leq r \leq s \leq 8$) results in a composite number. Furthermore, ensure that $\text{gcd}(b_i, b_{i+1}) = 1$ for every $i = 1, 2, \ldots, 7$ and $\text{gcd}(b_i, b_{i+2}) = 1$ for $i = 1, 2, \ldots, 6$.

Present your solution as a list of integers separated by commas, enclosed within $\boxed{...}$. For instance, $\boxed{1,2,3,4}$.

\subsubsection{Variation}
\textbf{Actual Problem}\\
Find a sequence $a_1, ..., a_{11}$ of positive integers such that every sum

$$a_r + a_{r+1} + ... + a_{s-1} + a_s$$

(with $1 \leqslant r \leqslant s \leqslant 11$) is a composite number, but

$\text{gcd}(a_i, a_{i+1}) = 1$ for all $i = 1,2,...,10$ and $\text{gcd}(a_i, a_{i+2}) = 1$ for all $i = 1,2,...,9$.

Output the answer as a comma separated list of integers inside of $\boxed{...}$. For example $\boxed{1,2,3,4}$.

\textbf{Revised Problem}\\
Identify a sequence $b_1, b_2, ..., b_{11}$ of positive integers such that every sum

$$b_x + b_{x+1} + ... + b_{y-1} + b_y$$

for $1 \leq x \leq y \leq 11$ results in a composite number, and also

$\text{gcd}(b_j, b_{j+1}) = 1$ for all $j = 1,2,...,10$ and $\text{gcd}(b_j, b_{j+2}) = 1$ for all $j = 1,2,...,9$.

Present the sequence as a comma-separated list of integers inside $\boxed{...}$. For instance $\boxed{1,2,3,4}$.

\subsection{emc-2016-3}
\subsubsection{Variation}
\textbf{Actual Problem}\\
For all positive integers $n$ there exist $n$ distinct, positive rational numbers with sum of their squares equal to $n$. Find $20$ distinct rationals that achieve this for $n = 20$.

Output the answer as a comma separated list of fractions inside of $\boxed{...}$. For example $\boxed{\frac{1}{2}, \frac{2}{3}, \frac{1}{1}}$.

\textbf{Revised Problem}\\
For every positive integer $n$, it is possible to find $n$ unique, positive rational numbers such that the total sum of their squares is $n$. Determine a set of $20$ different rational numbers that fulfill this requirement for $n = 20$.

Present the solution as a list of fractions, separated by commas, enclosed within $\boxed{...}$. For instance, $\boxed{\frac{1}{2}, \frac{2}{3}, \frac{1}{1}}$.

\subsubsection{Variation}
\textbf{Actual Problem}\\
For all positive integers $n$ there exist $n$ distinct, positive rational numbers with sum of their squares equal to $n$. Find $34$ distinct rationals that achieve this for $n = 34$.

Output the answer as a comma separated list of fractions inside of $\boxed{...}$. For example $\boxed{\frac{1}{2}, \frac{2}{3}, \frac{1}{1}}$.

\textbf{Revised Problem}\\
Given any positive integer $n$, it is possible to identify $n$ distinct positive rational numbers such that the sum of their squares is equal to $n$. Determine 34 unique rational numbers that satisfy these conditions for $n = 34$.

Present the solution as a list of fractions separated by commas, enclosed within a box, like so: $\boxed{\frac{1}{2}, \frac{2}{3}, \frac{1}{1}}$.

\subsubsection{Variation}
\textbf{Actual Problem}\\
For all positive integers $n$ there exist $n$ distinct, positive rational numbers with sum of their squares equal to $n$. Find $18$ distinct rationals that achieve this for $n = 18$.

Output the answer as a comma separated list of fractions inside of $\boxed{...}$. For example $\boxed{\frac{1}{2}, \frac{2}{3}, \frac{1}{1}}$.

\textbf{Revised Problem}\\
For every positive integer $n$, it is possible to find $n$ distinct positive rational numbers such that the sum of their squares is equal to $n$. Identify 18 distinct rational numbers that satisfy this condition when $n = 18$.

Present the solution as a list of fractions separated by commas, enclosed within $\boxed{...}$. For instance, format your answer as $\boxed{\frac{1}{2}, \frac{2}{3}, \frac{1}{1}}$.

\subsubsection{Variation}
\textbf{Actual Problem}\\
For all positive integers $n$ there exist $n$ distinct, positive rational numbers with sum of their squares equal to $n$. Find $13$ distinct rationals that achieve this for $n = 13$.

Output the answer as a comma separated list of fractions inside of $\boxed{...}$. For example $\boxed{\frac{1}{2}, \frac{2}{3}, \frac{1}{1}}$.

\textbf{Revised Problem}\\
For every positive integer \( n \), it is possible to identify \( n \) unique positive rational numbers whose squared sums equal \( n \). Determine 13 unique rational numbers that satisfy this condition for \( n = 13 \).

Present your solution as a list of fractions separated by commas, encapsulated within $\boxed{...}$. For instance, $\boxed{\frac{1}{2}, \frac{2}{3}, \frac{1}{1}}$.

\subsection{emc-2021-1}
\subsubsection{Variation}
\textbf{Actual Problem}\\
Alice drew a regular $20$-gon in the plane. Bob then labelled each vertex of the $20$-gon with a real number, in such a way that the labels of consecutive vertices differ by at most $1$. Then, for every pair of non-consecutive vertices whose labels differ by at most $1$, Alice drew a diagonal connecting them. Let $d$ be the number of diagonals Alice drew. The least possible value that $d$ can obtain is $17$. Find a configuration of labels that achieves this.

Output the answer as a comma separated list inside of $\boxed{...}$. For example $\boxed{1, 2, 3}$.
The $i$-th element of the list should be the label of the $i$-th vertex.

\textbf{Revised Problem}\\
Alice sketched a regular polygon with 20 sides in a plane. Bob assigned a real number to each vertex, ensuring that the difference between numbers on consecutive vertices is no greater than 1. Alice then connected vertices with a diagonal if they were not consecutive and if the difference between their numbers was at most 1. Let $d$ represent the total number of diagonals Alice drew. Determine a labeling of the vertices such that the smallest possible value of $d$ is 17.

Present your solution as a sequence of numbers enclosed in $\boxed{...}$ and separated by commas. For instance, $\boxed{1, 2, 3}$. The $i$-th number should correspond to the label of the $i$-th vertex.

\subsubsection{Variation}
\textbf{Actual Problem}\\
Alice drew a regular $34$-gon in the plane. Bob then labelled each vertex of the $34$-gon with a real number, in such a way that the labels of consecutive vertices differ by at most $1$. Then, for every pair of non-consecutive vertices whose labels differ by at most $1$, Alice drew a diagonal connecting them. Let $d$ be the number of diagonals Alice drew. The least possible value that $d$ can obtain is $31$. Find a configuration of labels that achieves this.

Output the answer as a comma separated list inside of $\boxed{...}$. For example $\boxed{1, 2, 3}$.
The $i$-th element of the list should be the label of the $i$-th vertex.

\textbf{Revised Problem}\\
Alice has sketched a regular polygon with 34 sides on a flat surface. Bob assigns each vertex of this 34-sided polygon a real number label, with the restriction that the labels on adjacent vertices can differ by no more than 1. Alice then connects pairs of non-adjacent vertices with a diagonal if their assigned labels differ by no more than 1. Suppose $d$ represents the total number of such diagonals that Alice can draw. Determine a configuration of labels that leads to the minimum possible value of $d$, which is 31.

Present your answer as a sequence of numbers separated by commas within $\boxed{...}$. For instance, $\boxed{1, 2, 3}$. The label for the $i$-th vertex should correspond to the $i$-th element in the sequence.

\subsubsection{Variation}
\textbf{Actual Problem}\\
Alice drew a regular $18$-gon in the plane. Bob then labelled each vertex of the $18$-gon with a real number, in such a way that the labels of consecutive vertices differ by at most $1$. Then, for every pair of non-consecutive vertices whose labels differ by at most $1$, Alice drew a diagonal connecting them. Let $d$ be the number of diagonals Alice drew. The least possible value that $d$ can obtain is $15$. Find a configuration of labels that achieves this.

Output the answer as a comma separated list inside of $\boxed{...}$. For example $\boxed{1, 2, 3}$.
The $i$-th element of the list should be the label of the $i$-th vertex.

\textbf{Revised Problem}\\
Alice sketched a regular $18$-gon on a flat surface. Bob then assigned a real number to each vertex, ensuring that the difference between labels of adjacent vertices is no more than $1$. Alice then drew a diagonal between every pair of non-adjacent vertices with labels differing by at most $1$. Let $d$ represent the count of diagonals drawn by Alice. The minimum possible value for $d$ is $15$. Determine a labeling of the vertices that results in this minimum.

Present your answer as a sequence of numbers separated by commas within $\boxed{...}$. For instance, $\boxed{1, 2, 3}$. The $i$-th number in the sequence should correspond to the label of the $i$-th vertex.

\subsubsection{Variation}
\textbf{Actual Problem}\\
Alice drew a regular $13$-gon in the plane. Bob then labelled each vertex of the $13$-gon with a real number, in such a way that the labels of consecutive vertices differ by at most $1$. Then, for every pair of non-consecutive vertices whose labels differ by at most $1$, Alice drew a diagonal connecting them. Let $d$ be the number of diagonals Alice drew. The least possible value that $d$ can obtain is $10$. Find a configuration of labels that achieves this.

Output the answer as a comma separated list inside of $\boxed{...}$. For example $\boxed{1, 2, 3}$.
The $i$-th element of the list should be the label of the $i$-th vertex.

\textbf{Revised Problem}\\
Alice has drawn a regular $13$-gon on a flat surface. Bob proceeds to assign a real number to each vertex of the $13$-gon, ensuring that the numbers on adjacent vertices differ by no more than $1$. Subsequently, Alice connects each pair of vertices, that are not adjacent, with a diagonal if the difference between their assigned numbers is at most $1$. Define $d$ as the total count of diagonals Alice creates. Determine a labeling scheme that results in $d$ being minimized to the value $10$. Describe this configuration of vertex labels.

Present the solution as a list of numbers separated by commas enclosed within $\boxed{...}$. For instance, $\boxed{1, 2, 3}$.
The number in the $i$-th position of the list corresponds to the label of the $i$-th vertex.

\subsection{emc-2022-2}
\subsubsection{Variation}
\textbf{Actual Problem}\\
We say a positive integer $n$ is lovely if there exists a positive integer $k$ and (not necessarily distinct) positive integers $d_1, d_2, ..., d_k$ such that $n = d_1 d_2 ... d_k$ and

$$d_i^2 | n + d_i$$

for all $i \in \{1, 2, ..., k\}$. Find $10$ distinct lovely numbers and output them together with $d_1, d_2, \ldots, d_k$.

Output the sequences as comma-separated tuples inside of \boxed, e.g. \boxed{(3, 3, 1), (4, 2, 2)}. The first integer in a tuple is the lovely number and the rest are the divisors $d_1, ..., d_k$.

\textbf{Revised Problem}\\
A positive integer $n$ is called lovely if there exists a positive integer $k$ along with positive integers $d_1, d_2, ..., d_k$ (which are not necessarily different) such that $n = d_1 d_2 ... d_k$ and 

$$d_i^2 \text{ divides } n + d_i$$ 

for every $i$ from $1$ to $k$. Your task is to determine $10$ distinct lovely numbers and display them along with the integers $d_1, d_2, \ldots, d_k$ that satisfy the conditions.

Present the sequences as comma-separated tuples within \boxed, for instance, \boxed{(3, 3, 1), (4, 2, 2)}. The first number in each tuple should be the lovely number, followed by the divisors $d_1, d_2, ..., d_k$.

\subsubsection{Variation}
\textbf{Actual Problem}\\
We say a positive integer $n$ is lovely if there exists a positive integer $k$ and (not necessarily distinct) positive integers $d_1, d_2, ..., d_k$ such that $n = d_1 d_2 ... d_k$ and

$$d_i^2 | n + d_i$$

for all $i \in \{1, 2, ..., k\}$. Find $11$ distinct lovely numbers and output them together with $d_1, d_2, \ldots, d_k$.

Output the sequences as comma-separated tuples inside of \boxed, e.g. \boxed{(3, 3, 1), (4, 2, 2)}. The first integer in a tuple is the lovely number and the rest are the divisors $d_1, ..., d_k$.

\textbf{Revised Problem}\\
A positive integer $n$ is defined as lovely if there exists some positive integer $k$ along with positive integers $d_1, d_2, ..., d_k$ such that $n = d_1 d_2 ... d_k$ and

$$d_i^2 \text{ divides } n + d_i$$

for every $i$ in the set $\{1, 2, ..., k\}$. Your task is to identify $11$ distinct lovely numbers and present them alongside their corresponding integers $d_1, d_2, \ldots, d_k$.

Present the sequences in the form of comma-separated tuples enclosed in \boxed, for example, \boxed{(3, 3, 1), (4, 2, 2)}. Each tuple should start with the lovely number, followed by the divisors $d_1, ..., d_k$.

\subsubsection{Variation}
\textbf{Actual Problem}\\
We say a positive integer $n$ is lovely if there exists a positive integer $k$ and (not necessarily distinct) positive integers $d_1, d_2, ..., d_k$ such that $n = d_1 d_2 ... d_k$ and

$$d_i^2 | n + d_i$$

for all $i \in \{1, 2, ..., k\}$. Find $7$ distinct lovely numbers and output them together with $d_1, d_2, \ldots, d_k$.

Output the sequences as comma-separated tuples inside of \boxed, e.g. \boxed{(3, 3, 1), (4, 2, 2)}. The first integer in a tuple is the lovely number and the rest are the divisors $d_1, ..., d_k$.

\textbf{Revised Problem}\\
A positive integer \( n \) is termed "lovely" if there exists a positive integer \( k \) and positive integers \( d_1, d_2, \ldots, d_k \) (which may not all be different) such that \( n = d_1 \cdot d_2 \cdot \ldots \cdot d_k \) and the following condition holds:

$$d_i^2 \mid n + d_i$$

for every \( i \) in the set \(\{1, 2, \ldots, k\}\). Identify 7 unique lovely numbers and provide them along with their corresponding sequences of \( d_1, d_2, \ldots, d_k \).

Present the results as sequences formatted as comma-separated tuples within \boxed, for instance, \boxed{(3, 3, 1), (4, 2, 2)}. Within each tuple, the initial number is the lovely number followed by its divisors \( d_1, \ldots, d_k \).

\subsubsection{Variation}
\textbf{Actual Problem}\\
We say a positive integer $n$ is lovely if there exists a positive integer $k$ and (not necessarily distinct) positive integers $d_1, d_2, ..., d_k$ such that $n = d_1 d_2 ... d_k$ and

$$d_i^2 | n + d_i$$

for all $i \in \{1, 2, ..., k\}$. Find $5$ distinct lovely numbers and output them together with $d_1, d_2, \ldots, d_k$.

Output the sequences as comma-separated tuples inside of \boxed, e.g. \boxed{(3, 3, 1), (4, 2, 2)}. The first integer in a tuple is the lovely number and the rest are the divisors $d_1, ..., d_k$.

\textbf{Revised Problem}\\
A positive integer \( n \) is termed "lovely" if there exists a positive integer \( k \) and positive integers \( d_1, d_2, \ldots, d_k \) (which do not need to be distinct) such that \( n = d_1 d_2 \cdots d_k \) and the condition

$$d_i^2 \text{ divides } (n + d_i)$$

holds true for every \( i \) from 1 to \( k \). Your task is to identify 5 unique lovely numbers and list them along with their corresponding divisors \( d_1, d_2, \ldots, d_k \).

Present the results as comma-separated tuples enclosed in \boxed, for instance, \boxed{(3, 3, 1), (4, 2, 2)}. Each tuple should start with the lovely number followed by its divisors \( d_1, \ldots, d_k \).

\subsection{emc-2023-2}
\subsubsection{Variation}
\textbf{Actual Problem}\\
Let $n = 10$ be an integer. There are $n$ points in the plane, no three of them collinear. Each day, Tom erases one of the points, until there are three points left. On the $i$-th, for $1 \leqslant i \leqslant n -3$, before erasing that day's point, Tom writes down the positive integer $v(i)$ such that the convex hull of the points at that moment has $v(i)$ vertices. Finally, he writes down $v(n-2)=3$. The greatest possible value of the expression

$$|v(1) - v(2)| + |v(2) - v(3)| + ... + |v(n-3) - v(n-2)|$$

is $2n-8$. Find a set of $n$ points for which this value is achieved.

Output the sequences as comma-separated tuples inside of \boxed, e.g. \boxed{(1, 2), (1, 2, 3), (1, 1)}. Tom removes the $i$-the point in the list on the $i$-th day

\textbf{Revised Problem}\\
Consider an integer $n = 10$. We have $n$ distinct points in a plane, with no three points being collinear. Each day, Tom eliminates one of these points, continuing until only three points are left. On day $i$, where $1 \leqslant i \leqslant n - 3$, before removing that day's point, Tom records the number $v(i)$, representing the vertex count of the convex hull formed by the remaining points. At the end, he notes $v(n-2) = 3$. Determine the maximum possible value of the expression:

$$|v(1) - v(2)| + |v(2) - v(3)| + \ldots + |v(n-3) - v(n-2)|,$$

which is known to be $2n-8$. Identify a configuration of these $n$ points that allows this maximum value to be achieved.

Present the sequences as comma-separated tuples enclosed in \boxed, for example, \boxed{(1, 2), (1, 2, 3), (1, 1)}. The point removed on the $i$-th day corresponds to the $i$-th point in the list.

\subsubsection{Variation}
\textbf{Actual Problem}\\
Let $n = 32$ be an integer. There are $n$ points in the plane, no three of them collinear. Each day, Tom erases one of the points, until there are three points left. On the $i$-th, for $1 \leqslant i \leqslant n -3$, before erasing that day's point, Tom writes down the positive integer $v(i)$ such that the convex hull of the points at that moment has $v(i)$ vertices. Finally, he writes down $v(n-2)=3$. The greatest possible value of the expression

$$|v(1) - v(2)| + |v(2) - v(3)| + ... + |v(n-3) - v(n-2)|$$

is $2n-8$. Find a set of $n$ points for which this value is achieved.

Output the sequences as comma-separated tuples inside of \boxed, e.g. \boxed{(1, 2), (1, 2, 3), (1, 1)}. Tom removes the $i$-the point in the list on the $i$-th day

\textbf{Revised Problem}\\
Consider $n = 32$, which is an integer representing the number of points positioned in a plane such that no three are in a straight line. Each day, Tom removes one point, continuing until only three points remain. On the $i$-th day, where $1 \leqslant i \leqslant n - 3$, before removing a point, Tom notes down the positive integer $v(i)$, which signifies the number of vertices in the convex hull of the remaining points at that time. Eventually, he records $v(n-2)=3$. Determine the maximum possible value of the expression

$$|v(1) - v(2)| + |v(2) - v(3)| + \cdots + |v(n-3) - v(n-2)|$$

which turns out to be $2n-8$. Identify a configuration of $n$ points that results in achieving this value.

Express the sequences as comma-separated tuples enclosed in \boxed, such as \boxed{(1, 2), (1, 2, 3), (1, 1)}. Tom removes the $i$-th point from the list on the $i$-th day.

\subsubsection{Variation}
\textbf{Actual Problem}\\
Let $n = 9$ be an integer. There are $n$ points in the plane, no three of them collinear. Each day, Tom erases one of the points, until there are three points left. On the $i$-th, for $1 \leqslant i \leqslant n -3$, before erasing that day's point, Tom writes down the positive integer $v(i)$ such that the convex hull of the points at that moment has $v(i)$ vertices. Finally, he writes down $v(n-2)=3$. The greatest possible value of the expression

$$|v(1) - v(2)| + |v(2) - v(3)| + ... + |v(n-3) - v(n-2)|$$

is $2n-8$. Find a set of $n$ points for which this value is achieved.

Output the sequences as comma-separated tuples inside of \boxed, e.g. \boxed{(1, 2), (1, 2, 3), (1, 1)}. Tom removes the $i$-the point in the list on the $i$-th day

\textbf{Revised Problem}\\
Consider an integer $n = 9$. In the plane, there are $n$ points arranged such that no trio of them lies on the same straight line. Each day, Tom removes a point from the collection, continuing until only three points remain. On the $i$-th day, where $1 \leq i \leq n - 3$, just before erasing the point for that day, Tom records the number $v(i)$, representing the number of vertices that form the convex hull of the remaining points. At the end, he notes $v(n-2)=3$. Determine the maximum possible value of the expression

$$|v(1) - v(2)| + |v(2) - v(3)| + \ldots + |v(n-3) - v(n-2)|,$$

which is given to be $2n-8$. Identify a configuration of $n$ points that achieves this maximum value.

Present the sequences as comma-separated tuples within \boxed, such as \boxed{(1, 2), (1, 2, 3), (1, 1)}. Tom removes the $i$-th point from the list on the $i$-th day.

\subsubsection{Variation}
\textbf{Actual Problem}\\
Let $n = 35$ be an integer. There are $n$ points in the plane, no three of them collinear. Each day, Tom erases one of the points, until there are three points left. On the $i$-th, for $1 \leqslant i \leqslant n -3$, before erasing that day's point, Tom writes down the positive integer $v(i)$ such that the convex hull of the points at that moment has $v(i)$ vertices. Finally, he writes down $v(n-2)=3$. The greatest possible value of the expression

$$|v(1) - v(2)| + |v(2) - v(3)| + ... + |v(n-3) - v(n-2)|$$

is $2n-8$. Find a set of $n$ points for which this value is achieved.

Output the sequences as comma-separated tuples inside of \boxed, e.g. \boxed{(1, 2), (1, 2, 3), (1, 1)}. Tom removes the $i$-the point in the list on the $i$-th day

\textbf{Revised Problem}\\
Consider an integer $n = 35$. Imagine there are $n$ points positioned in the plane such that no three points are collinear. Every day, Tom removes one of these points until only three remain. On the $i$-th day, for $1 \leqslant i \leqslant n - 3$, before removing a point, Tom records a positive integer $v(i)$ that represents the number of vertices in the convex hull formed by the remaining points. He eventually notes that $v(n-2)=3$. The task is to determine the maximum value of the following expression:

$$|v(1) - v(2)| + |v(2) - v(3)| + \cdots + |v(n-3) - v(n-2)|$$

This maximum value is known to be $2n-8$. Identify a configuration of $n$ points that allows this value to be achieved.

Present the sequences as comma-separated tuples enclosed in \boxed, for example, \boxed{(1, 2), (1, 2, 3), (1, 1)}. On the $i$-th day, Tom removes the point corresponding to the $i$-th entry from the list.

\section{imc}
\subsection{imc-2012-2}
\subsubsection{Variation}
\textbf{Actual Problem}\\
Find an $7 \times 7$ matrix of rank at most 3 which has zeros along the main diagonal and strictly positive real numbers off the main diagonal.

Output the answer between \verb|\begin{array}{...}| and \verb|\end{array}| inside of $\boxed{...}$. For example, $\boxed{\begin{array}{ccc}1 & 2 & 3 \\ 4 & 5 & 6 \\ 7 & 8 & 9\end{array}}$.

\textbf{Revised Problem}\\
Construct a $7 \times 7$ matrix with a rank not exceeding 3, ensuring that all diagonal elements are zero and all non-diagonal elements are strictly positive real numbers.

Present the solution within \verb|\begin{array}{...}| and \verb|\end{array}|, and encapsulate it in $\boxed{...}$. As an illustration, use the format $\boxed{\begin{array}{ccc}1 & 2 & 3 \\ 4 & 5 & 6 \\ 7 & 8 & 9\end{array}}$.

\subsubsection{Variation}
\textbf{Actual Problem}\\
Find an $19 \times 19$ matrix of rank at most 3 which has zeros along the main diagonal and strictly positive real numbers off the main diagonal.

Output the answer between \verb|\begin{array}{...}| and \verb|\end{array}| inside of $\boxed{...}$. For example, $\boxed{\begin{array}{ccc}1 & 2 & 3 \\ 4 & 5 & 6 \\ 7 & 8 & 9\end{array}}$.

\textbf{Revised Problem}\\
Construct a $19 \times 19$ matrix with a rank no greater than 3, where all elements on the primary diagonal are zero, and all elements not on the diagonal are strictly positive real numbers.

Present your solution within \verb|\begin{array}{...}| and \verb|\end{array}| encapsulated in $\boxed{...}$. For instance, $\boxed{\begin{array}{ccc}1 & 2 & 3 \\ 4 & 5 & 6 \\ 7 & 8 & 9\end{array}}$.

\subsubsection{Variation}
\textbf{Actual Problem}\\
Find an $8 \times 8$ matrix of rank at most 3 which has zeros along the main diagonal and strictly positive real numbers off the main diagonal.

Output the answer between \verb|\begin{array}{...}| and \verb|\end{array}| inside of $\boxed{...}$. For example, $\boxed{\begin{array}{ccc}1 & 2 & 3 \\ 4 & 5 & 6 \\ 7 & 8 & 9\end{array}}$.

\textbf{Revised Problem}\\
Construct an $8 \times 8$ matrix with a rank no greater than 3. Ensure that all the diagonal elements are zero, and that every element not on the diagonal is a strictly positive real number.

Present the matrix using \verb|\begin{array}{...}| and \verb|\end{array}| enclosed within $\boxed{...}$. For instance, $\boxed{\begin{array}{ccc}1 & 2 & 3 \\ 4 & 5 & 6 \\ 7 & 8 & 9\end{array}}$.

\subsubsection{Variation}
\textbf{Actual Problem}\\
Find an $9 \times 9$ matrix of rank at most 3 which has zeros along the main diagonal and strictly positive real numbers off the main diagonal.

Output the answer between \verb|\begin{array}{...}| and \verb|\end{array}| inside of $\boxed{...}$. For example, $\boxed{\begin{array}{ccc}1 & 2 & 3 \\ 4 & 5 & 6 \\ 7 & 8 & 9\end{array}}$.

\textbf{Revised Problem}\\
Determine a $9 \times 9$ matrix with a rank not exceeding 3, where all the diagonal elements are zero and all off-diagonal elements are positive real numbers.

Present your solution enclosed within \verb|\begin{array}{...}| and \verb|\end{array}|, encapsulated by $\boxed{...}$. For instance, $\boxed{\begin{array}{ccc}1 & 2 & 3 \\ 4 & 5 & 6 \\ 7 & 8 & 9\end{array}}$.

\subsection{imc-2013-3}
\subsubsection{Variation}
\textbf{Actual Problem}\\
There are $2n$ students in a school ($n \in N, n \geq 2$). Each week $n$ students go on a trip.
Given $n = 10$, your task is to organize 6 trips such that every two students go together on at least one trip. Present your solution 
in the form of a 6 x $2n$ matrix $A$ such that $A_{ij} = 1$ if student $j$ goes on trip $i$ and $A_{ij} = 0$ otherwise.

Output the answer between \verb|\begin{array}{...}| and \verb|\end{array}| inside of $\boxed{...}$. For example, $\boxed{\begin{array}{ccc}1 & 2 & 3 \\ 4 & 5 & 6 \\ 7 & 8 & 9\end{array}}$.

\textbf{Revised Problem}\\
A school has $2n$ students, where $n$ is a natural number and $n \geq 2$. Each week, a group of $n$ students is taken on an excursion. With $n$ set to 10, determine how to arrange 6 such excursions so that every pair of students is on the same excursion at least once. Represent your arrangement as a 6 x $2n$ matrix $A$ where $A_{ij} = 1$ if student $j$ is included in excursion $i$, and $A_{ij} = 0$ otherwise.

Provide the solution formatted between \verb|\begin{array}{...}| and \verb|\end{array}| enclosed in $\boxed{...}$. For instance, $\boxed{\begin{array}{ccc}1 & 2 & 3 \\ 4 & 5 & 6 \\ 7 & 8 & 9\end{array}}$.

\subsubsection{Variation}
\textbf{Actual Problem}\\
There are $2n$ students in a school ($n \in N, n \geq 2$). Each week $n$ students go on a trip.
Given $n = 14$, your task is to organize 6 trips such that every two students go together on at least one trip. Present your solution 
in the form of a 6 x $2n$ matrix $A$ such that $A_{ij} = 1$ if student $j$ goes on trip $i$ and $A_{ij} = 0$ otherwise.

Output the answer between \verb|\begin{array}{...}| and \verb|\end{array}| inside of $\boxed{...}$. For example, $\boxed{\begin{array}{ccc}1 & 2 & 3 \\ 4 & 5 & 6 \\ 7 & 8 & 9\end{array}}$.

\textbf{Revised Problem}\\
A school has $2n$ students, where $n$ is a natural number and $n \geq 2$. Each week, a group of $n$ students participates in a trip. For $n = 14$, devise a plan for 6 weekly trips ensuring that each pair of students attends at least one trip together. Your solution should be represented as a 6 x $2n$ matrix $A$. In this matrix, $A_{ij} = 1$ indicates that student $j$ is attending trip $i$, while $A_{ij} = 0$ means they are not.

Present your solution using the format \verb|\begin{array}{...}| and \verb|\end{array}| within $\boxed{...}$. For instance, $\boxed{\begin{array}{ccc}1 & 2 & 3 \\ 4 & 5 & 6 \\ 7 & 8 & 9\end{array}}$.

\subsubsection{Variation}
\textbf{Actual Problem}\\
There are $2n$ students in a school ($n \in N, n \geq 2$). Each week $n$ students go on a trip.
Given $n = 8$, your task is to organize 6 trips such that every two students go together on at least one trip. Present your solution 
in the form of a 6 x $2n$ matrix $A$ such that $A_{ij} = 1$ if student $j$ goes on trip $i$ and $A_{ij} = 0$ otherwise.

Output the answer between \verb|\begin{array}{...}| and \verb|\end{array}| inside of $\boxed{...}$. For example, $\boxed{\begin{array}{ccc}1 & 2 & 3 \\ 4 & 5 & 6 \\ 7 & 8 & 9\end{array}}$.

\textbf{Revised Problem}\\
There are $2n$ students at a school, where $n$ is a natural number and $n \geq 2$. Each week, $n$ of these students participate in a trip. When $n = 8$, your task is to schedule 6 trips so that each pair of students travels together on at least one trip. Represent your solution as a 6 x $2n$ matrix $A$, with $A_{ij} = 1$ indicating student $j$ attends trip $i$, and $A_{ij} = 0$ otherwise.

Enclose the output between \verb|\begin{array}{...}| and \verb|\end{array}| within $\boxed{...}$. For instance, $\boxed{\begin{array}{ccc}1 & 2 & 3 \\ 4 & 5 & 6 \\ 7 & 8 & 9\end{array}}$.

\subsubsection{Variation}
\textbf{Actual Problem}\\
There are $2n$ students in a school ($n \in N, n \geq 2$). Each week $n$ students go on a trip.
Given $n = 11$, your task is to organize 6 trips such that every two students go together on at least one trip. Present your solution 
in the form of a 6 x $2n$ matrix $A$ such that $A_{ij} = 1$ if student $j$ goes on trip $i$ and $A_{ij} = 0$ otherwise.

Output the answer between \verb|\begin{array}{...}| and \verb|\end{array}| inside of $\boxed{...}$. For example, $\boxed{\begin{array}{ccc}1 & 2 & 3 \\ 4 & 5 & 6 \\ 7 & 8 & 9\end{array}}$.

\textbf{Revised Problem}\\
In a school, there are $2n$ students, where $n$ is a natural number and $n \geq 2$. Each week, $n$ students are selected to go on an excursion. For the case where $n = 11$, your task is to plan 6 excursions in such a way that every pair of students participates together on at least one excursion. Represent your solution as a 6 x $2n$ matrix $B$ where $B_{ij} = 1$ indicates student $j$ is attending excursion $i$, and $B_{ij} = 0$ otherwise.

Present the solution using \verb|\begin{array}{...}| and \verb|\end{array}| encapsulated within $\boxed{...}$. For instance, $\boxed{\begin{array}{ccc}1 & 2 & 3 \\ 4 & 5 & 6 \\ 7 & 8 & 9\end{array}}$.

\subsection{imc-2018-6}
\subsubsection{Variation}
\textbf{Actual Problem}\\
Find $6$ x $10$ real matrix such that all of its columns are non-zero vectors and for every pair of indices $i, j$ such that $|i - j| > 1$, the vectors at columns $i$ and $j$ are orthogonal.

Output the answer between \verb|\begin{array}{...}| and \verb|\end{array}| inside of $\boxed{...}$. For example, $\boxed{\begin{array}{ccc}1 & 2 & 3 \\ 4 & 5 & 6 \\ 7 & 8 & 9\end{array}}$.

\textbf{Revised Problem}\\
Determine a 6 by 10 matrix composed of real numbers such that each column is a non-zero vector. Additionally, for any two columns indexed by $i$ and $j$, if the difference between $i$ and $j$ is greater than 1, the vectors corresponding to these columns must be orthogonal.

Present your solution within \verb|\begin{array}{...}| and \verb|\end{array}|, encapsulated in $\boxed{...}$. For instance, $\boxed{\begin{array}{ccc}1 & 2 & 3 \\ 4 & 5 & 6 \\ 7 & 8 & 9\end{array}}$.

\subsubsection{Variation}
\textbf{Actual Problem}\\
Find $10$ x $14$ real matrix such that all of its columns are non-zero vectors and for every pair of indices $i, j$ such that $|i - j| > 1$, the vectors at columns $i$ and $j$ are orthogonal.

Output the answer between \verb|\begin{array}{...}| and \verb|\end{array}| inside of $\boxed{...}$. For example, $\boxed{\begin{array}{ccc}1 & 2 & 3 \\ 4 & 5 & 6 \\ 7 & 8 & 9\end{array}}$.

\textbf{Revised Problem}\\
Construct a $10 \times 14$ matrix composed of real numbers such that each column is a non-zero vector. Additionally, ensure that for any two columns labeled $i$ and $j$, the vectors are orthogonal if the difference between $i$ and $j$ is greater than 1.

Present your solution encapsulated within \verb|\begin{array}{...}| and \verb|\end{array}| inside $\boxed{...}$. For instance, $\boxed{\begin{array}{ccc}1 & 2 & 3 \\ 4 & 5 & 6 \\ 7 & 8 & 9\end{array}}$.

\subsubsection{Variation}
\textbf{Actual Problem}\\
Find $9$ x $10$ real matrix such that all of its columns are non-zero vectors and for every pair of indices $i, j$ such that $|i - j| > 1$, the vectors at columns $i$ and $j$ are orthogonal.

Output the answer between \verb|\begin{array}{...}| and \verb|\end{array}| inside of $\boxed{...}$. For example, $\boxed{\begin{array}{ccc}1 & 2 & 3 \\ 4 & 5 & 6 \\ 7 & 8 & 9\end{array}}$.

\textbf{Revised Problem}\\
Construct a $9 \times 10$ real matrix such that every column is a vector with at least one non-zero element, and for any pair of column indices $i$ and $j$ where $|i - j| > 1$, the vectors corresponding to columns $i$ and $j$ are orthogonal.

Present your solution enclosed within \verb|\begin{array}{...}| and \verb|\end{array}| tags, and place this inside $\boxed{...}$. For instance, $\boxed{\begin{array}{ccc}1 & 2 & 3 \\ 4 & 5 & 6 \\ 7 & 8 & 9\end{array}}$.

\subsubsection{Variation}
\textbf{Actual Problem}\\
Find $4$ x $8$ real matrix such that all of its columns are non-zero vectors and for every pair of indices $i, j$ such that $|i - j| > 1$, the vectors at columns $i$ and $j$ are orthogonal.

Output the answer between \verb|\begin{array}{...}| and \verb|\end{array}| inside of $\boxed{...}$. For example, $\boxed{\begin{array}{ccc}1 & 2 & 3 \\ 4 & 5 & 6 \\ 7 & 8 & 9\end{array}}$.

\textbf{Revised Problem}\\
Determine a $4 \times 8$ matrix with real entries such that none of its columns are zero vectors, and for any pair of indices $i$ and $j$, the vectors in columns $i$ and $j$ are orthogonal if the absolute difference $|i - j|$ is greater than 1.

Present your solution enclosed within \verb|\begin{array}{...}| and \verb|\end{array}| and encapsulated by $\boxed{...}$. For instance, $\boxed{\begin{array}{ccc}1 & 2 & 3 \\ 4 & 5 & 6 \\ 7 & 8 & 9\end{array}}$.

\subsection{imc-2019-9}
\subsubsection{Variation}
\textbf{Actual Problem}\\
Find $4\times4$ real invertible matrices $A$ and $B$ such that $AB - BA = B^2A$.

Output the matrices inside of \boxed{...} in the following format:
\boxed{
    \begin{pmatrix}
        ...
    \end{pmatrix},
    \begin{pmatrix}
        ...
    \end{pmatrix}
}
where the first matrix is $A$ and the second matrix is $B$.


\textbf{Revised Problem}\\
Identify two invertible matrices $A$ and $B$, each of size $4 \times 4$ with real number entries, such that they fulfill the equation $AB - BA = B^2A$.

Present your solution by enclosing the matrices within \boxed{...}, formatted as follows:
\boxed{
    \begin{pmatrix}
        ...
    \end{pmatrix},
    \begin{pmatrix}
        ...
    \end{pmatrix}
}
where the first matrix corresponds to $A$ and the second to $B$.

\subsubsection{Variation}
\textbf{Actual Problem}\\
Find $18\times18$ real invertible matrices $A$ and $B$ such that $AB - BA = B^2A$.

Output the matrices inside of \boxed{...} in the following format:
\boxed{
    \begin{pmatrix}
        ...
    \end{pmatrix},
    \begin{pmatrix}
        ...
    \end{pmatrix}
}
where the first matrix is $A$ and the second matrix is $B$.


\textbf{Revised Problem}\\
Identify two real invertible matrices $A$ and $B$, each of size $18 \times 18$, that satisfy the equation $AB - BA = B^2A$.

Present the matrices within \boxed{...} in this manner:
\boxed{
    \begin{pmatrix}
        ...
    \end{pmatrix},
    \begin{pmatrix}
        ...
    \end{pmatrix}
}
where the first matrix represents $A$ and the second matrix represents $B$.

\subsubsection{Variation}
\textbf{Actual Problem}\\
Find $8\times8$ real invertible matrices $A$ and $B$ such that $AB - BA = B^2A$.

Output the matrices inside of \boxed{...} in the following format:
\boxed{
    \begin{pmatrix}
        ...
    \end{pmatrix},
    \begin{pmatrix}
        ...
    \end{pmatrix}
}
where the first matrix is $A$ and the second matrix is $B$.


\textbf{Revised Problem}\\
Determine two invertible matrices $A$ and $B$, each of size $8 \times 8$ with real entries, that satisfy the equation $AB - BA = B^2A$.

Express the matrices within \boxed{...} using the format:
\boxed{
    \begin{pmatrix}
        ...
    \end{pmatrix},
    \begin{pmatrix}
        ...
    \end{pmatrix}
}
where the first matrix represents $A$ and the second represents $B$.

\subsubsection{Variation}
\textbf{Actual Problem}\\
Find $14\times14$ real invertible matrices $A$ and $B$ such that $AB - BA = B^2A$.

Output the matrices inside of \boxed{...} in the following format:
\boxed{
    \begin{pmatrix}
        ...
    \end{pmatrix},
    \begin{pmatrix}
        ...
    \end{pmatrix}
}
where the first matrix is $A$ and the second matrix is $B$.


\textbf{Revised Problem}\\
Identify two invertible matrices, \( A \) and \( B \), both of dimensions \( 14 \times 14 \), such that they satisfy the equation \( AB - BA = B^2A \).

Present the solution by displaying the matrices within \boxed{...} using the format below:
\boxed{
    \begin{pmatrix}
        ...
    \end{pmatrix},
    \begin{pmatrix}
        ...
    \end{pmatrix}
}
where \( A \) is the first matrix and \( B \) is the second matrix.

\section{imo}
\subsection{imo-shortlist-2000-c4}
\subsubsection{Variation}
\textbf{Actual Problem}\\
Given $n=10$ and $k=5$, place $20$ pawns on an $n \times n$ board such that no row or column contains $k$ adjacent unoccupied squares.
One can show that this is the smallest number of pawns that can be placed on the board while satisfying the property.
Note that $n/2 < k \leq 2n/3$.

Output the answer as $n \times n$ matrix consisting of 0s (no pawn) and 1s (pawn) formatted using \verb|\begin{array}{...}| and \verb|\end{array}| inside of $\boxed{...}$. For example, $\boxed{\begin{array}{ccc}0 & 0 & 1 \\ 1 & 0 & 0 \\ 1 & 0 & 0\end{array}}$.

\textbf{Revised Problem}\\
Consider an \( n \times n \) chessboard where \( n = 10 \) and \( k = 5 \). You need to position 20 pawns on the board such that no row or column has \( k \) consecutive empty squares. It can be proven that this arrangement uses the fewest pawns possible to meet the given criteria.
Remember, the value of \( k \) satisfies \( n/2 < k \leq 2n/3 \).

Present your solution as an \( n \times n \) grid made up of 0s (indicating an empty square) and 1s (indicating a square with a pawn), formatted using \verb|\begin{array}{...}| and \verb|\end{array}| enclosed within $\boxed{...}$. For instance, a 3x3 grid would be written as $\boxed{\begin{array}{ccc}0 & 0 & 1 \\ 1 & 0 & 0 \\ 1 & 0 & 0\end{array}}$.

\subsubsection{Variation}
\textbf{Actual Problem}\\
Given $n=21$ and $k=14$, place $28$ pawns on an $n \times n$ board such that no row or column contains $k$ adjacent unoccupied squares.
One can show that this is the smallest number of pawns that can be placed on the board while satisfying the property.
Note that $n/2 < k \leq 2n/3$.

Output the answer as $n \times n$ matrix consisting of 0s (no pawn) and 1s (pawn) formatted using \verb|\begin{array}{...}| and \verb|\end{array}| inside of $\boxed{...}$. For example, $\boxed{\begin{array}{ccc}0 & 0 & 1 \\ 1 & 0 & 0 \\ 1 & 0 & 0\end{array}}$.

\textbf{Revised Problem}\\
Consider an \( n \times n \) chessboard where \( n = 21 \) and \( k = 14 \). Your task is to arrange 28 pawns on the board in such a way that no row or column has \( k \) or more consecutive empty squares. It can be proven that this arrangement uses the least number of pawns needed to satisfy this condition. Remember, \( n/2 < k \leq 2n/3 \).

Present your solution as an \( n \times n \) grid using 0s to represent empty squares and 1s to denote squares occupied by pawns. Format your answer within $\boxed{...}$ using LaTeX array syntax, such as \verb|\begin{array}{...}| and \verb|\end{array}|. For instance, $\boxed{\begin{array}{ccc}0 & 0 & 1 \\ 1 & 0 & 0 \\ 1 & 0 & 0\end{array}}$.

\subsubsection{Variation}
\textbf{Actual Problem}\\
Given $n=13$ and $k=7$, place $24$ pawns on an $n \times n$ board such that no row or column contains $k$ adjacent unoccupied squares.
One can show that this is the smallest number of pawns that can be placed on the board while satisfying the property.
Note that $n/2 < k \leq 2n/3$.

Output the answer as $n \times n$ matrix consisting of 0s (no pawn) and 1s (pawn) formatted using \verb|\begin{array}{...}| and \verb|\end{array}| inside of $\boxed{...}$. For example, $\boxed{\begin{array}{ccc}0 & 0 & 1 \\ 1 & 0 & 0 \\ 1 & 0 & 0\end{array}}$.

\textbf{Revised Problem}\\
You have a board of size \(13 \times 13\) and you need to position 24 pawns on it. Arrange the pawns so that no row or column has exactly 7 adjacent squares without a pawn. It has been proven that this arrangement uses the fewest pawns while meeting the condition. Keep in mind that the value of \( k \) falls between \( n/2 \) and \( 2n/3 \).

Present your solution as a \(13 \times 13\) matrix using 0s and 1s, where 0 indicates an empty square and 1 represents a square with a pawn. Format this using \verb|\begin{array}{...}| and \verb|\end{array}| within $\boxed{...}$, for example, $\boxed{\begin{array}{ccc}0 & 0 & 1 \\ 1 & 0 & 0 \\ 1 & 0 & 0\end{array}}$.

\subsubsection{Variation}
\textbf{Actual Problem}\\
Given $n=10$ and $k=6$, place $16$ pawns on an $n \times n$ board such that no row or column contains $k$ adjacent unoccupied squares.
One can show that this is the smallest number of pawns that can be placed on the board while satisfying the property.
Note that $n/2 < k \leq 2n/3$.

Output the answer as $n \times n$ matrix consisting of 0s (no pawn) and 1s (pawn) formatted using \verb|\begin{array}{...}| and \verb|\end{array}| inside of $\boxed{...}$. For example, $\boxed{\begin{array}{ccc}0 & 0 & 1 \\ 1 & 0 & 0 \\ 1 & 0 & 0\end{array}}$.

\textbf{Revised Problem}\\
Consider an $n \times n$ chessboard with $n=10$. Your task is to position $16$ pawns on this board in such a way that no row or column contains $k=6$ consecutive empty squares. It has been determined that this is the minimum number of pawns required to achieve this configuration. Note that the condition $n/2 < k \leq 2n/3$ holds.

Present your solution as an $n \times n$ grid filled with 0s (indicating empty squares) and 1s (indicating squares with pawns), formatted using the LaTeX commands \verb|\begin{array}{...}| and \verb|\end{array}|, encapsulated within $\boxed{...}$. For instance, the format should resemble $\boxed{\begin{array}{ccc}0 & 0 & 1 \\ 1 & 0 & 0 \\ 1 & 0 & 0\end{array}}$.

\subsection{imo-shortlist-2001-c5}
\subsubsection{Variation}
\textbf{Actual Problem}\\
Find $10$ finite sequences $(x_0, x_1, \ldots, x_n)$ such that for every $j$, $0 \leq j \leq n$, $x_j$ equals the number of times $j$ appears in the sequence.

Output the sequences as comma-separated tuples inside of \boxed{}, e.g. \boxed{(1, 2), (1, 2, 3), (1, 1, 1, 1, 0)}.

\textbf{Revised Problem}\\
Identify 10 finite sequences $(x_0, x_1, \ldots, x_n)$ such that, for each index $j$ within the range $0 \leq j \leq n$, the value $x_j$ is equal to the count of the index $j$ appearing in the sequence.

Present the sequences as comma-separated tuples encapsulated inside \boxed{}, for example, \boxed{(1, 2), (1, 2, 3), (1, 1, 1, 1, 0)}.

\subsubsection{Variation}
\textbf{Actual Problem}\\
Find $17$ finite sequences $(x_0, x_1, \ldots, x_n)$ such that for every $j$, $0 \leq j \leq n$, $x_j$ equals the number of times $j$ appears in the sequence.

Output the sequences as comma-separated tuples inside of \boxed{}, e.g. \boxed{(1, 2), (1, 2, 3), (1, 1, 1, 1, 0)}.

\textbf{Revised Problem}\\
Identify 17 distinct finite sequences $(x_0, x_1, \ldots, x_n)$ such that for each $j$, where $0 \leq j \leq n$, the element $x_j$ is equal to the frequency of $j$ within the sequence.

Present the sequences as tuples with elements separated by commas, enclosed within \boxed{}, for instance, \boxed{(1, 2), (1, 2, 3), (1, 1, 1, 1, 0)}.

\subsubsection{Variation}
\textbf{Actual Problem}\\
Find $9$ finite sequences $(x_0, x_1, \ldots, x_n)$ such that for every $j$, $0 \leq j \leq n$, $x_j$ equals the number of times $j$ appears in the sequence.

Output the sequences as comma-separated tuples inside of \boxed{}, e.g. \boxed{(1, 2), (1, 2, 3), (1, 1, 1, 1, 0)}.

\textbf{Revised Problem}\\
Identify $9$ distinct finite sequences $(x_0, x_1, \ldots, x_n)$ for which each element $x_j$ (where $0 \leq j \leq n$) represents the count of occurrences of the index $j$ within the sequence.

Present the sequences as tuples separated by commas within \boxed{}, for example, \boxed{(1, 2), (1, 2, 3), (1, 1, 1, 1, 0)}.

\subsubsection{Variation}
\textbf{Actual Problem}\\
Find $6$ finite sequences $(x_0, x_1, \ldots, x_n)$ such that for every $j$, $0 \leq j \leq n$, $x_j$ equals the number of times $j$ appears in the sequence.

Output the sequences as comma-separated tuples inside of \boxed{}, e.g. \boxed{(1, 2), (1, 2, 3), (1, 1, 1, 1, 0)}.

\textbf{Revised Problem}\\
Identify $6$ sequences of numbers $(x_0, x_1, \ldots, x_n)$, where for each index $j$ from $0$ to $n$, the value $x_j$ is equal to how many times the number $j$ appears within that sequence.

Express the sequences as tuples separated by commas, enclosed within \boxed{}, for example, \boxed{(1, 2), (1, 2, 3), (1, 1, 1, 1, 0)}.

\subsection{imo-shortlist-2001-c6}
\subsubsection{Variation}
\textbf{Actual Problem}\\
For a positive integer $n$ define a sequence of zeros and ones to be balanced if it contains $n$ zeros and $n$ ones. 
Two balanced sequences $a$ and $b$ are neighbors if you can move one of the $2n$ symbols of $a$ to another position to form $b$. For instance, when $n = 4$, the balanced sequences $01101001$ and $00110101$ are neighbors because the third (or fourth) zero in the first sequence can be moved to the first or second position to form the second sequence. 
For $n = 4$, find a set $S$ of at most $\frac {1}{n + 1} \binom{2n}{n}$ balanced sequences such that every balanced sequence is equal to or is a neighbor of at least one sequence in $S$.


Output sequences from the set $S$ as a comma-separated list inside of \boxed{}, e.g. \boxed{01101001, 00110101}.

\textbf{Revised Problem}\\
Consider a positive integer $n$. We define a sequence of length $2n$ consisting of exactly $n$ zeros and $n$ ones as balanced. Two such balanced sequences, $a$ and $b$, are considered neighbors if $b$ can be formed by shifting one symbol from $a$ to a different position. For example, when $n = 4$, the sequence $01101001$ can be rearranged by moving the third or fourth zero to the first or second position to become $00110101$, thus making them neighbors. For $n = 4$, identify a collection $S$ of balanced sequences containing no more than $\frac {1}{n + 1} \binom{2n}{n}$ sequences such that each balanced sequence is either included in $S$ or is a neighbor to at least one sequence in $S$.

Present your solution by listing the sequences from the set $S$, separated by commas, enclosed within \boxed{}. For example, \boxed{01101001, 00110101}.

\subsubsection{Variation}
\textbf{Actual Problem}\\
For a positive integer $n$ define a sequence of zeros and ones to be balanced if it contains $n$ zeros and $n$ ones. 
Two balanced sequences $a$ and $b$ are neighbors if you can move one of the $2n$ symbols of $a$ to another position to form $b$. For instance, when $n = 4$, the balanced sequences $01101001$ and $00110101$ are neighbors because the third (or fourth) zero in the first sequence can be moved to the first or second position to form the second sequence. 
For $n = 5$, find a set $S$ of at most $\frac {1}{n + 1} \binom{2n}{n}$ balanced sequences such that every balanced sequence is equal to or is a neighbor of at least one sequence in $S$.


Output sequences from the set $S$ as a comma-separated list inside of \boxed{}, e.g. \boxed{01101001, 00110101}.

\textbf{Revised Problem}\\
Consider a positive integer $n$. A binary sequence is called balanced if it contains $n$ zeros and $n$ ones. Two balanced sequences $a$ and $b$ are said to be adjacent if it's possible to obtain $b$ from $a$ by relocating a single symbol from one position to another. For example, when $n = 4$, the sequences $01101001$ and $00110101$ are adjacent because by moving the third (or fourth) zero in $01101001$ to the first or second position, we can form $00110101$. Determine a collection $S$ of balanced sequences for $n = 5$ such that the size of $S$ is no more than $\frac{1}{n + 1} \binom{2n}{n}$, and every balanced sequence either belongs to $S$ or is adjacent to at least one sequence in $S$.

Present the sequences within the set $S$ as a list separated by commas inside \boxed{}, for instance, \boxed{01101001, 00110101}.

\subsubsection{Variation}
\textbf{Actual Problem}\\
For a positive integer $n$ define a sequence of zeros and ones to be balanced if it contains $n$ zeros and $n$ ones. 
Two balanced sequences $a$ and $b$ are neighbors if you can move one of the $2n$ symbols of $a$ to another position to form $b$. For instance, when $n = 4$, the balanced sequences $01101001$ and $00110101$ are neighbors because the third (or fourth) zero in the first sequence can be moved to the first or second position to form the second sequence. 
For $n = 6$, find a set $S$ of at most $\frac {1}{n + 1} \binom{2n}{n}$ balanced sequences such that every balanced sequence is equal to or is a neighbor of at least one sequence in $S$.


Output sequences from the set $S$ as a comma-separated list inside of \boxed{}, e.g. \boxed{01101001, 00110101}.

\textbf{Revised Problem}\\
Consider a positive integer \( n \). A sequence made up of zeros and ones is called balanced if it has exactly \( n \) zeros and \( n \) ones. Two balanced sequences, denoted as \( a \) and \( b \), are termed neighbors if one of the symbols in \( a \) can be relocated to a different spot, resulting in sequence \( b \). For example, if \( n = 4 \), the sequences \( 01101001 \) and \( 00110101 \) are neighbors because moving the third (or fourth) zero from the first sequence to the first or second position creates the second sequence. For the case when \( n = 6 \), determine a collection \( S \) containing no more than \( \frac{1}{n + 1} \binom{2n}{n} \) balanced sequences such that each balanced sequence is either in \( S \) or is a neighboring sequence to at least one sequence within \( S \).

Present the sequences in the set \( S \) as a list separated by commas, enclosed within \boxed{}, such as \boxed{01101001, 00110101}.

\subsection{imo-shortlist-2001-n6}
\subsubsection{Variation}
\textbf{Actual Problem}\\
Find 100 positive integers not exceeding 25000, such that all pairwise sums of them are different.

Output the answer as a comma separated list inside of $\boxed{...}$. For example $\boxed{1, 2, 3}$.

\textbf{Revised Problem}\\
Identify 100 distinct positive numbers, none exceeding 25000, such that the sum of every possible pair of these numbers is unique.

List your answer as a sequence of numbers separated by commas within $\boxed{...}$. For instance, $\boxed{1, 2, 3}$.

\subsubsection{Variation}
\textbf{Actual Problem}\\
Find 132 positive integers not exceeding 40000, such that all pairwise sums of them are different.

Output the answer as a comma separated list inside of $\boxed{...}$. For example $\boxed{1, 2, 3}$.

\textbf{Revised Problem}\\
Identify 132 integers from the set of positive numbers up to 40000 such that the sum of every possible pair of these selected numbers is distinct.

Present the solution as a list of numbers separated by commas, enclosed within $\boxed{...}$. For instance, $\boxed{1, 2, 3}$.

\subsubsection{Variation}
\textbf{Actual Problem}\\
Find 95 positive integers not exceeding 20000, such that all pairwise sums of them are different.

Output the answer as a comma separated list inside of $\boxed{...}$. For example $\boxed{1, 2, 3}$.

\textbf{Revised Problem}\\
Determine 95 distinct positive integers, each no greater than 20000, such that each possible sum of two different numbers in this collection is unique.

Present the solution as a list of numbers separated by commas within $\boxed{...}$. For instance, $\boxed{1, 2, 3}$.

\subsubsection{Variation}
\textbf{Actual Problem}\\
Find 113 positive integers not exceeding 30000, such that all pairwise sums of them are different.

Output the answer as a comma separated list inside of $\boxed{...}$. For example $\boxed{1, 2, 3}$.

\textbf{Revised Problem}\\
Identify 113 distinct positive numbers, each no greater than 30,000, such that every sum formed by any two of these numbers is unique.

Present the solution as a list of numbers separated by commas, enclosed within $\boxed{...}$. For instance, $\boxed{1, 2, 3}$.

\subsection{imo-shortlist-2002-n4}
\subsubsection{Variation}
\textbf{Actual Problem}\\
Find $m$ such that the equation
$$
\frac{1}{a} + \frac{1}{b} + \frac{1}{c} + \frac{1}{abc} = \frac{m}{a + b + c}
$$
has at least $k = 15$ solutions such that $a \leq b \leq c$.
Then, output $k$ distinct tuples of positive integers $(m, a_i, b_i, c_i)$ for $i$ from 1 to $k$ that satisfy the equation.


Output $k$ comma-separated tuples inside of \boxed{}, e.g. \boxed{(20, 2, 3, 4), (20, 6, 7, 8)}.

\textbf{Revised Problem}\\
Determine the integer \( m \) for which the equation
$$
\frac{1}{a} + \frac{1}{b} + \frac{1}{c} + \frac{1}{abc} = \frac{m}{a + b + c}
$$
results in at least \( k = 15 \) distinct solutions where the integers satisfy \( a \leq b \leq c \).
Following this, provide \( k \) unique sets of positive integers \( (m, a_i, b_i, c_i) \) for \( i = 1, 2, \ldots, k \) that solve the equation.

List \( k \) tuples separated by commas within a \boxed{}, for example, \boxed{(20, 2, 3, 4), (20, 6, 7, 8)}.

\subsubsection{Variation}
\textbf{Actual Problem}\\
Find $m$ such that the equation
$$
\frac{1}{a} + \frac{1}{b} + \frac{1}{c} + \frac{1}{abc} = \frac{m}{a + b + c}
$$
has at least $k = 17$ solutions such that $a \leq b \leq c$.
Then, output $k$ distinct tuples of positive integers $(m, a_i, b_i, c_i)$ for $i$ from 1 to $k$ that satisfy the equation.


Output $k$ comma-separated tuples inside of \boxed{}, e.g. \boxed{(20, 2, 3, 4), (20, 6, 7, 8)}.

\textbf{Revised Problem}\\
Determine the integer \( m \) such that the following equation:
$$
\frac{1}{a} + \frac{1}{b} + \frac{1}{c} + \frac{1}{abc} = \frac{m}{a + b + c}
$$
possesses at least 17 solutions for which \( a \leq b \leq c \) are positive integers. Subsequently, identify and present 17 distinct positive integer triples \((m, a_i, b_i, c_i)\) for \( i \) ranging from 1 to 17 that satisfy the equation.

Present the 17 distinct triples as comma-separated values enclosed within \boxed{}, for example, \boxed{(20, 2, 3, 4), (20, 6, 7, 8)}.

\subsubsection{Variation}
\textbf{Actual Problem}\\
Find $m$ such that the equation
$$
\frac{1}{a} + \frac{1}{b} + \frac{1}{c} + \frac{1}{abc} = \frac{m}{a + b + c}
$$
has at least $k = 13$ solutions such that $a \leq b \leq c$.
Then, output $k$ distinct tuples of positive integers $(m, a_i, b_i, c_i)$ for $i$ from 1 to $k$ that satisfy the equation.


Output $k$ comma-separated tuples inside of \boxed{}, e.g. \boxed{(20, 2, 3, 4), (20, 6, 7, 8)}.

\textbf{Revised Problem}\\
Identify the integer \( m \) for which the equation
$$
\frac{1}{a} + \frac{1}{b} + \frac{1}{c} + \frac{1}{abc} = \frac{m}{a + b + c}
$$
has at least 13 solutions where the integers \( a, b, c \) satisfy \( a \leq b \leq c \). Subsequently, list 13 unique sets of positive integers \((m, a_i, b_i, c_i)\) for \( i \) ranging from 1 to 13 that fulfill the equation.

List 13 unique tuples in the format \boxed{(m, a_1, b_1, c_1), (m, a_2, b_2, c_2), \ldots, (m, a_{13}, b_{13}, c_{13})}, ensuring they are comma-separated.

\subsubsection{Variation}
\textbf{Actual Problem}\\
Find $m$ such that the equation
$$
\frac{1}{a} + \frac{1}{b} + \frac{1}{c} + \frac{1}{abc} = \frac{m}{a + b + c}
$$
has at least $k = 11$ solutions such that $a \leq b \leq c$.
Then, output $k$ distinct tuples of positive integers $(m, a_i, b_i, c_i)$ for $i$ from 1 to $k$ that satisfy the equation.


Output $k$ comma-separated tuples inside of \boxed{}, e.g. \boxed{(20, 2, 3, 4), (20, 6, 7, 8)}.

\textbf{Revised Problem}\\
Determine the value \( m \) such that the equation
$$
\frac{1}{a} + \frac{1}{b} + \frac{1}{c} + \frac{1}{abc} = \frac{m}{a + b + c}
$$
has at least eleven sets of solutions in positive integers where the condition \(a \leq b \leq c\) holds. Then, provide eleven distinct combinations of integers \((m, a_i, b_i, c_i)\) for \(i\) ranging from 1 to 11 that satisfy the equation.

Display the eleven comma-separated integer combinations within \boxed{}, formatted as follows: \boxed{(20, 2, 3, 4), (20, 6, 7, 8)}.

\subsection{imo-shortlist-2003-n2}
\subsubsection{Variation}
\textbf{Actual Problem}\\
Each positive integer $a$ undergoes the following procedure in order to obtain the number $d = d(a)$:
(i) move the last digit of $a$ to the first position to obtain the number $b$;
(ii) square $b$ to obtain the number $c$;
(iii) move the first digit of $c$ to the end to obtain the number $d$.
(All the numbers in the problem are considered to be represented in base 10.) 
For example, for $a = 2003$, we get $b = 3200$, $c = 10240000$, and $d = 02400001 = 2400001 = d(2003)$.
Find $15$ distinct numbers $a$ for which $d(a) = a^2$.


Output the sequence of numbers $a$ as a comma-separated list inside of \boxed, e.g. \boxed{256, 512, 1024}.

\textbf{Revised Problem}\\
Consider a positive integer $a$ that undergoes the following sequence of transformations to determine the number $d = d(a)$:
1. Shift the last digit of $a$ to the front to create a new number $b$.
2. Calculate the square of $b$, resulting in a number $c$.
3. Transfer the first digit of $c$ to its end to form the number $d$.
(All numerical representations are in base 10.) 
For instance, when $a = 2003$, we compute $b = 3200$, $c = 10240000$, and $d = 02400001 = 2400001 = d(2003)$.
Identify 15 unique numbers $a$ such that $d(a) = a^2$.

Present the list of numbers $a$ as a series of values separated by commas and enclosed within \boxed, for example, \boxed{256, 512, 1024}.

\subsubsection{Variation}
\textbf{Actual Problem}\\
Each positive integer $a$ undergoes the following procedure in order to obtain the number $d = d(a)$:
(i) move the last digit of $a$ to the first position to obtain the number $b$;
(ii) square $b$ to obtain the number $c$;
(iii) move the first digit of $c$ to the end to obtain the number $d$.
(All the numbers in the problem are considered to be represented in base 10.) 
For example, for $a = 2003$, we get $b = 3200$, $c = 10240000$, and $d = 02400001 = 2400001 = d(2003)$.
Find $27$ distinct numbers $a$ for which $d(a) = a^2$.


Output the sequence of numbers $a$ as a comma-separated list inside of \boxed, e.g. \boxed{256, 512, 1024}.

\textbf{Revised Problem}\\
Consider a positive integer \( a \) that goes through a series of transformations to yield the number \( d = d(a) \):
1. Shift the last digit of \( a \) to the front to form a new number \( b \).
2. Compute the square of \( b \) to get \( c \).
3. Move the first digit of \( c \) to the end to obtain the number \( d \).

All numbers are in base 10. For instance, if \( a = 2003 \), then \( b = 3200 \), \( c = 10240000 \), and \( d = 02400001 = 2400001 = d(2003) \).
Identify 27 distinct integers \( a \) such that \( d(a) = a^2 \).

Present the sequence of integers \( a \) as a comma-separated list enclosed in \boxed, for example, \boxed{256, 512, 1024}.

\subsubsection{Variation}
\textbf{Actual Problem}\\
Each positive integer $a$ undergoes the following procedure in order to obtain the number $d = d(a)$:
(i) move the last digit of $a$ to the first position to obtain the number $b$;
(ii) square $b$ to obtain the number $c$;
(iii) move the first digit of $c$ to the end to obtain the number $d$.
(All the numbers in the problem are considered to be represented in base 10.) 
For example, for $a = 2003$, we get $b = 3200$, $c = 10240000$, and $d = 02400001 = 2400001 = d(2003)$.
Find $19$ distinct numbers $a$ for which $d(a) = a^2$.


Output the sequence of numbers $a$ as a comma-separated list inside of \boxed, e.g. \boxed{256, 512, 1024}.

\textbf{Revised Problem}\\
For each positive integer $a$, follow these steps to derive the number $d = d(a)$:
1. Relocate the last digit of $a$ to the front to create the number $b$.
2. Compute $b^2$ to yield the number $c$.
3. Transfer the first digit of $c$ to the end to form the number $d$.
(All numbers are in base 10 notation.) 
For instance, when $a = 2003$, we produce $b = 3200$, then $c = 10240000$, and finally $d = 02400001 = 2400001 = d(2003)$.
Identify 19 unique integers $a$ such that $d(a) = a^2$.

Present the list of integers $a$ as a sequence separated by commas, enclosed within \boxed, for example, \boxed{256, 512, 1024}.

\subsubsection{Variation}
\textbf{Actual Problem}\\
Each positive integer $a$ undergoes the following procedure in order to obtain the number $d = d(a)$:
(i) move the last digit of $a$ to the first position to obtain the number $b$;
(ii) square $b$ to obtain the number $c$;
(iii) move the first digit of $c$ to the end to obtain the number $d$.
(All the numbers in the problem are considered to be represented in base 10.) 
For example, for $a = 2003$, we get $b = 3200$, $c = 10240000$, and $d = 02400001 = 2400001 = d(2003)$.
Find $16$ distinct numbers $a$ for which $d(a) = a^2$.


Output the sequence of numbers $a$ as a comma-separated list inside of \boxed, e.g. \boxed{256, 512, 1024}.

\textbf{Revised Problem}\\
Consider a positive integer \(a\). Follow these steps to derive a number \(d = d(a)\):
1. Take the last digit of \(a\) and place it at the start to form a new number \(b\).
2. Compute the square of \(b\) to obtain \(c\).
3. Shift the first digit of \(c\) to the end to produce the number \(d\).

All calculations are performed in base 10. As an illustration, if \(a = 2003\), then \(b = 3200\), \(c = 10240000\), and \(d = 02400001 = 2400001 = d(2003)\).

Your task is to identify 16 unique integers \(a\) such that \(d(a) = a^2\).

Present your answer as a list of integers \(a\), separated by commas, enclosed within a \boxed command, for example: \boxed{256, 512, 1024}.

\subsection{imo-shortlist-2003-n3}
\subsubsection{Variation}
\textbf{Actual Problem}\\
There are infinitely many different pairs of positive integers $(a,b)$ such that $a>b>1$ and $\frac{(a^2)}{(2ab^2) - b^3 + 1}$ is a positive integer. Find $10$ such pairs.

Output the sequence of pairs $(a,b)$ as a comma-separated list inside of \boxed, e.g. \boxed{(10, 5), (20, 10)}.

\textbf{Revised Problem}\\
Identify ten distinct pairs of positive integers \( (a, b) \) where \( a > b > 1 \), such that the expression \( \frac{a^2}{2ab^2 - b^3 + 1} \) results in a positive integer.

Present the list of pairs \( (a, b) \) as a comma-separated sequence within a \boxed, for example, \boxed{(10, 5), (20, 10)}.

\subsubsection{Variation}
\textbf{Actual Problem}\\
There are infinitely many different pairs of positive integers $(a,b)$ such that $a>b>1$ and $\frac{(a^2)}{(2ab^2) - b^3 + 1}$ is a positive integer. Find $22$ such pairs.

Output the sequence of pairs $(a,b)$ as a comma-separated list inside of \boxed, e.g. \boxed{(10, 5), (20, 10)}.

\textbf{Revised Problem}\\
Consider an infinite number of distinct pairs of positive integers $(a,b)$ satisfying $a > b > 1$ where the expression $\frac{a^2}{2ab^2 - b^3 + 1}$ results in a positive integer. Your task is to determine 22 such pairs.

Present the sequence of pairs $(a,b)$ as a list separated by commas and enclosed in \boxed, for instance, \boxed{(10, 5), (20, 10)}.

\subsubsection{Variation}
\textbf{Actual Problem}\\
There are infinitely many different pairs of positive integers $(a,b)$ such that $a>b>1$ and $\frac{(a^2)}{(2ab^2) - b^3 + 1}$ is a positive integer. Find $14$ such pairs.

Output the sequence of pairs $(a,b)$ as a comma-separated list inside of \boxed, e.g. \boxed{(10, 5), (20, 10)}.

\textbf{Revised Problem}\\
There is an infinite number of distinct pairs of positive integers $(a,b)$ such that $a$ is greater than $b$, and $b$ is greater than $1$, where the expression $\frac{a^2}{2ab^2 - b^3 + 1}$ results in a positive integer. Identify $14$ such pairs.

Present the sequence of pairs $(a,b)$ enclosed within \boxed, separated by commas, for example, \boxed{(10, 5), (20, 10)}.

\subsubsection{Variation}
\textbf{Actual Problem}\\
There are infinitely many different pairs of positive integers $(a,b)$ such that $a>b>1$ and $\frac{(a^2)}{(2ab^2) - b^3 + 1}$ is a positive integer. Find $11$ such pairs.

Output the sequence of pairs $(a,b)$ as a comma-separated list inside of \boxed, e.g. \boxed{(10, 5), (20, 10)}.

\textbf{Revised Problem}\\
Identify an infinite number of distinct pairs of positive integers \((a, b)\) such that \(a > b > 1\) and the fraction \(\frac{a^2}{2ab^2 - b^3 + 1}\) is an integer greater than zero. Provide 11 example pairs.

Present the 11 pairs \((a, b)\) as a list separated by commas, enclosed in \boxed, for example, \boxed{(10, 5), (20, 10)}.

\subsection{imo-shortlist-2005-c8}
\subsubsection{Variation}
\textbf{Actual Problem}\\
Let $M=\{A_1, A_2, \ldots, A_n\}$ be a convex $n$-gon, $n \geq 4$. 
The coloring is "best" if it colors some $n-3$ of the diagonals of $M$ green, and some other $n-3$ of its diagonals red, so that no two diagonals of the same color meet inside $M$, and its "score", i.e., the number of intersection points of green and red diagonals inside $M$, is maximized.
For $n=15$ the maximum score is $108$.
Find one coloring that achieves this score. 


Output the list of green diagonals, followed by the list of red diagonals, separated by a comma, inside of \boxed. Each list of diagonals should contain comma-separated pairs $(a,b)$, referring to the diagonal $(A_a, A_b)$. For example, a well-formatted output is \boxed{((1,3), (2,4)), ((2,5), (3,5))}.

\textbf{Revised Problem}\\
Consider a convex polygon with vertices $A_1, A_2, \ldots, A_{15}$. The task is to assign colors to some of its diagonals such that exactly 12 diagonals are colored green and another 12 are colored red. The coloring must ensure that no two diagonals of the same color intersect inside the polygon. The goal is to maximize the number of intersection points between the green and red diagonals. Determine a coloring configuration that achieves the maximum possible number of 108 intersection points inside the polygon.

Provide your solution by listing the green diagonals first, followed by the red diagonals, all enclosed in \boxed. Each list should consist of pairs $(a,b)$, which denote the diagonal $(A_a, A_b)$, with pairs separated by commas. For illustration, a correctly formatted solution looks like \boxed{((1,3), (2,4)), ((2,5), (3,5))}.

\subsubsection{Variation}
\textbf{Actual Problem}\\
Let $M=\{A_1, A_2, \ldots, A_n\}$ be a convex $n$-gon, $n \geq 4$. 
The coloring is "best" if it colors some $n-3$ of the diagonals of $M$ green, and some other $n-3$ of its diagonals red, so that no two diagonals of the same color meet inside $M$, and its "score", i.e., the number of intersection points of green and red diagonals inside $M$, is maximized.
For $n=20$ the maximum score is $217$.
Find one coloring that achieves this score. 


Output the list of green diagonals, followed by the list of red diagonals, separated by a comma, inside of \boxed. Each list of diagonals should contain comma-separated pairs $(a,b)$, referring to the diagonal $(A_a, A_b)$. For example, a well-formatted output is \boxed{((1,3), (2,4)), ((2,5), (3,5))}.

\textbf{Revised Problem}\\
Consider a convex polygon with $n$ sides, where $n \geq 4$, represented as $M = \{A_1, A_2, \ldots, A_n\}$. A coloring scheme is considered "optimal" if it assigns green to $n-3$ diagonals and red to another set of $n-3$ diagonals such that no two diagonals of the same color intersect within the polygon, while maximizing the "score," which is defined as the number of intersection points between green and red diagonals within the polygon. For $n=20$, this maximum score is $217$. Determine a coloring method that achieves this maximum score.

Present the green diagonals first, followed by the red diagonals, both lists enclosed within \boxed. Each list should consist of comma-separated pairs $(a, b)$, where each pair represents a diagonal $(A_a, A_b)$. For instance, an example of correctly formatted output is \boxed{((1,3), (2,4)), ((2,5), (3,5))}.

\subsubsection{Variation}
\textbf{Actual Problem}\\
Let $M=\{A_1, A_2, \ldots, A_n\}$ be a convex $n$-gon, $n \geq 4$. 
The coloring is "best" if it colors some $n-3$ of the diagonals of $M$ green, and some other $n-3$ of its diagonals red, so that no two diagonals of the same color meet inside $M$, and its "score", i.e., the number of intersection points of green and red diagonals inside $M$, is maximized.
For $n=16$ the maximum score is $127$.
Find one coloring that achieves this score. 


Output the list of green diagonals, followed by the list of red diagonals, separated by a comma, inside of \boxed. Each list of diagonals should contain comma-separated pairs $(a,b)$, referring to the diagonal $(A_a, A_b)$. For example, a well-formatted output is \boxed{((1,3), (2,4)), ((2,5), (3,5))}.

\textbf{Revised Problem}\\
Consider a convex polygon $M$ with vertices $A_1, A_2, \ldots, A_n$, where $n \geq 4$. A coloring is deemed optimal if it involves coloring precisely $n-3$ diagonals of $M$ in green and another $n-3$ diagonals in red, ensuring that no two diagonals of the same color intersect within $M$. The aim is to maximize the "score," which is defined as the number of intersection points inside $M$ between the green and red diagonals. When $n=16$, the highest achievable score is $127$. Determine one such coloring configuration that results in this optimal score.

Present the green diagonals first, followed by the red diagonals, both enclosed within a \boxed command. Each set of diagonals should be displayed as comma-separated pairs $(a,b)$, representing the diagonal $(A_a, A_b)$. An example of the correct format is \boxed{((1,3), (2,4)), ((2,5), (3,5))}.

\subsubsection{Variation}
\textbf{Actual Problem}\\
Let $M=\{A_1, A_2, \ldots, A_n\}$ be a convex $n$-gon, $n \geq 4$. 
The coloring is "best" if it colors some $n-3$ of the diagonals of $M$ green, and some other $n-3$ of its diagonals red, so that no two diagonals of the same color meet inside $M$, and its "score", i.e., the number of intersection points of green and red diagonals inside $M$, is maximized.
For $n=14$ the maximum score is $91$.
Find one coloring that achieves this score. 


Output the list of green diagonals, followed by the list of red diagonals, separated by a comma, inside of \boxed. Each list of diagonals should contain comma-separated pairs $(a,b)$, referring to the diagonal $(A_a, A_b)$. For example, a well-formatted output is \boxed{((1,3), (2,4)), ((2,5), (3,5))}.

\textbf{Revised Problem}\\
Consider a convex polygon $M = \{A_1, A_2, \ldots, A_n\}$ with $n$ sides, where $n \geq 4$. We define a "best" diagonal coloring as one that assigns $n-3$ diagonals of $M$ the color green and another $n-3$ diagonals the color red, ensuring that no diagonals sharing a color intersect within the interior of $M$. The "score" of this coloring is the count of points where green and red diagonals intersect inside the polygon, and the goal is to maximize this score. For $n=14$, the optimal score is $91$. Identify a diagonal coloring that achieves this maximal score.

Present the list of diagonals colored green, followed by those colored red, separated by a comma, enclosed in \boxed. Each set of diagonals should be a series of comma-separated pairs $(a,b)$, each representing the diagonal $(A_a, A_b)$. For example, a correctly formatted output looks like \boxed{((1,3), (2,4)), ((2,5), (3,5))}.

\subsection{imo-shortlist-2006-c5}
\subsubsection{Variation}
\textbf{Actual Problem}\\
An $(n, k)$-tournament is a contest with $n$ players held in $k$ rounds such that:

(i) Each player plays in each round, and every two players meet at most once.

(ii) If player A meets player B in round $i$, player C meets player D in round $i$, and player A
meets player C in round $j$, then player B meets player D in round $j$.

Construct a $(24, 5)$-tournament. 


Output a comma-separated list of rounds, inside of \boxed. Each round is itself a list of numbers, where the i-th number indicates the player that player i meets in that round (1-indexed). For example, a well-formatted output for k=2 and n=4 is \boxed{(4, 3, 2, 1), (3, 4, 1, 2)}.

\textbf{Revised Problem}\\
A contest involving $n$ players and held over $k$ rounds is called an $(n, k)$-tournament if it meets the following criteria:

(i) Every player competes in each round, and no pair of players meets more than once throughout the tournament.

(ii) If player A plays against player B in round $i$, and player C plays against player D in the same round $i$, then if player A encounters player C in round $j$, player B must play against player D in round $j$ as well.

Design an $(24, 5)$-tournament according to these criteria.

Provide your answer as a comma-separated list of rounds enclosed in \boxed. Each round should be represented as a list of numbers, where the i-th number specifies the player whom player i competes against in that round (using 1-based indexing). For instance, a correctly formatted solution for k=2 and n=4 would be \boxed{(4, 3, 2, 1), (3, 4, 1, 2)}.

\subsubsection{Variation}
\textbf{Actual Problem}\\
An $(n, k)$-tournament is a contest with $n$ players held in $k$ rounds such that:

(i) Each player plays in each round, and every two players meet at most once.

(ii) If player A meets player B in round $i$, player C meets player D in round $i$, and player A
meets player C in round $j$, then player B meets player D in round $j$.

Construct a $(96, 3)$-tournament. 


Output a comma-separated list of rounds, inside of \boxed. Each round is itself a list of numbers, where the i-th number indicates the player that player i meets in that round (1-indexed). For example, a well-formatted output for k=2 and n=4 is \boxed{(4, 3, 2, 1), (3, 4, 1, 2)}.

\textbf{Revised Problem}\\
A contest involving $n$ participants is organized across $k$ rounds, referred to as an $(n, k)$-tournament, where:

(i) Every participant competes in each round, with no pair of participants facing each other more than once.

(ii) If participant A faces participant B in round $i$, and participant C faces participant D in round $i$, and in another round $j$, participant A competes against participant C, then in round $j$, participant B must compete against participant D.

Design a $(96, 3)$-tournament.

Provide the solution as a comma-separated sequence of rounds, enclosed within \boxed. Each round should be represented as a sequence of numbers, where the i-th number corresponds to the opponent of player i in that round, using 1-based indexing. For illustration, a correctly formatted response for k=2 and n=4 would be \boxed{(4, 3, 2, 1), (3, 4, 1, 2)}.

\subsubsection{Variation}
\textbf{Actual Problem}\\
An $(n, k)$-tournament is a contest with $n$ players held in $k$ rounds such that:

(i) Each player plays in each round, and every two players meet at most once.

(ii) If player A meets player B in round $i$, player C meets player D in round $i$, and player A
meets player C in round $j$, then player B meets player D in round $j$.

Construct a $(28, 3)$-tournament. 


Output a comma-separated list of rounds, inside of \boxed. Each round is itself a list of numbers, where the i-th number indicates the player that player i meets in that round (1-indexed). For example, a well-formatted output for k=2 and n=4 is \boxed{(4, 3, 2, 1), (3, 4, 1, 2)}.

\textbf{Revised Problem}\\
An $(n, k)$-tournament is a competition involving $n$ participants conducted across $k$ rounds, characterized by the following conditions:

(i) Every participant competes in each round, with any pair of participants confronting each other no more than once throughout the tournament.

(ii) If participant A competes against participant B in the $i^{th}$ round, and participant C faces participant D in the same round, then if participant A encounters participant C in the $j^{th}$ round, participant B must compete against participant D in that same $j^{th}$ round.

Your task is to design a $(28, 3)$-tournament.

Present the output as a comma-separated sequence of rounds enclosed in \boxed. Each round should be represented by a list of numbers, where the number in the i-th position specifies the opponent of player i during that round, using 1-based indexing. For instance, an appropriately formatted output for k=2 and n=4 would be \boxed{(4, 3, 2, 1), (3, 4, 1, 2)}.

\subsubsection{Variation}
\textbf{Actual Problem}\\
An $(n, k)$-tournament is a contest with $n$ players held in $k$ rounds such that:

(i) Each player plays in each round, and every two players meet at most once.

(ii) If player A meets player B in round $i$, player C meets player D in round $i$, and player A
meets player C in round $j$, then player B meets player D in round $j$.

Construct a $(60, 3)$-tournament. 


Output a comma-separated list of rounds, inside of \boxed. Each round is itself a list of numbers, where the i-th number indicates the player that player i meets in that round (1-indexed). For example, a well-formatted output for k=2 and n=4 is \boxed{(4, 3, 2, 1), (3, 4, 1, 2)}.

\textbf{Revised Problem}\\
In an $(n, k)$-tournament, there are $n$ participants and the competition unfolds over $k$ rounds with the following rules:

(i) Every competitor participates in each round, and each pair of competitors faces each other at most once.

(ii) If in round $i$, competitor A plays against competitor B, and competitor C plays against competitor D, and in round $j$, competitor A plays against competitor C, then in round $j$, competitor B must face competitor D.

Your task is to design a $(60, 3)$-tournament that adheres to these rules.

Present your solution as a series of rounds enclosed in \boxed, where each round is represented as a list of numbers. Each number at position i specifies the opponent that player i faces in that round (with player numbering starting from 1). For instance, a correct format for $k = 2$ and $n = 4$ would be: \boxed{(4, 3, 2, 1), (3, 4, 1, 2)}.

\subsection{imo-shortlist-2008-a2}
\subsubsection{Variation}
\textbf{Actual Problem}\\
The inequality
$$
\frac{x^2}{(x-1)^2} + \frac{y^2}{(y-1)^2} + \frac{z^2}{(z-1)^2} \geq 1
$$
holds for real numbers $x, y, z \neq 1$ satisfying the condition $xyz = 1$.
There are infinitely many different triples of rational numbers $x, y, z$ for which this inequality turns into equality.
Find $20$ such triples.


Output a comma-separated list of triples of fractions (or integers), within a single \boxed environment, for example for n=2: \boxed{(\frac{1}{2},\frac{3}{4},-1),(\frac{1}{3},3,\frac{-4}{6})}.

\textbf{Revised Problem}\\
Consider the inequality:
$$
\frac{x^2}{(x-1)^2} + \frac{y^2}{(y-1)^2} + \frac{z^2}{(z-1)^2} \geq 1
$$
which holds for all real numbers \(x, y, z \neq 1\) under the condition that \(xyz = 1\).
There is an infinite number of different rational triples \((x, y, z)\) where this inequality becomes an equality.
Determine 20 such rational triples.

Present your answer as a comma-separated sequence of triples in the form of fractions (or integers), all enclosed in a single \boxed environment. For instance, for \(n=2\), the format should be \boxed{(\frac{1}{2},\frac{3}{4},-1),(\frac{1}{3},3,\frac{-4}{6})}.

\subsubsection{Variation}
\textbf{Actual Problem}\\
The inequality
$$
\frac{x^2}{(x-1)^2} + \frac{y^2}{(y-1)^2} + \frac{z^2}{(z-1)^2} \geq 1
$$
holds for real numbers $x, y, z \neq 1$ satisfying the condition $xyz = 1$.
There are infinitely many different triples of rational numbers $x, y, z$ for which this inequality turns into equality.
Find $22$ such triples.


Output a comma-separated list of triples of fractions (or integers), within a single \boxed environment, for example for n=2: \boxed{(\frac{1}{2},\frac{3}{4},-1),(\frac{1}{3},3,\frac{-4}{6})}.

\textbf{Revised Problem}\\
Consider the inequality:
$$
\frac{x^2}{(x-1)^2} + \frac{y^2}{(y-1)^2} + \frac{z^2}{(z-1)^2} \geq 1
$$
This inequality is valid for real numbers \( x, y, z \neq 1 \) such that the product \( xyz = 1 \). There exist infinitely many sets of rational numbers \( x, y, z \) where this inequality holds as an equality. Your task is to identify 22 distinct triples of such rational numbers.

Provide your answer as a list of 22 triples of fractions (or integers), separated by commas, all enclosed within a single \boxed environment. For instance, if \( n = 2 \), the format should look like: \boxed{(\frac{1}{2},\frac{3}{4},-1),(\frac{1}{3},3,\frac{-4}{6})}.

\subsubsection{Variation}
\textbf{Actual Problem}\\
The inequality
$$
\frac{x^2}{(x-1)^2} + \frac{y^2}{(y-1)^2} + \frac{z^2}{(z-1)^2} \geq 1
$$
holds for real numbers $x, y, z \neq 1$ satisfying the condition $xyz = 1$.
There are infinitely many different triples of rational numbers $x, y, z$ for which this inequality turns into equality.
Find $14$ such triples.


Output a comma-separated list of triples of fractions (or integers), within a single \boxed environment, for example for n=2: \boxed{(\frac{1}{2},\frac{3}{4},-1),(\frac{1}{3},3,\frac{-4}{6})}.

\textbf{Revised Problem}\\
Determine the triples of rational numbers $(x, y, z)$ such that the following holds true:
$$
\frac{x^2}{(x-1)^2} + \frac{y^2}{(y-1)^2} + \frac{z^2}{(z-1)^2} = 1
$$
under the condition that $x, y, z$ are real numbers not equal to $1$ and satisfy $xyz = 1$. You are required to identify $14$ distinct triples of such rational numbers.

Provide your answer as a single \boxed environment containing a list of $14$ triples, each represented by fractions or integers, separated by commas. For instance, for $n=2$, the format should look like: \boxed{(\frac{1}{2},\frac{3}{4},-1),(\frac{1}{3},3,\frac{-4}{6})}.

\subsubsection{Variation}
\textbf{Actual Problem}\\
The inequality
$$
\frac{x^2}{(x-1)^2} + \frac{y^2}{(y-1)^2} + \frac{z^2}{(z-1)^2} \geq 1
$$
holds for real numbers $x, y, z \neq 1$ satisfying the condition $xyz = 1$.
There are infinitely many different triples of rational numbers $x, y, z$ for which this inequality turns into equality.
Find $11$ such triples.


Output a comma-separated list of triples of fractions (or integers), within a single \boxed environment, for example for n=2: \boxed{(\frac{1}{2},\frac{3}{4},-1),(\frac{1}{3},3,\frac{-4}{6})}.

\textbf{Revised Problem}\\
Consider the inequality:
$$
\frac{x^2}{(x-1)^2} + \frac{y^2}{(y-1)^2} + \frac{z^2}{(z-1)^2} \geq 1,
$$
which must hold true for any real numbers \(x, y, z \neq 1\) such that \(xyz = 1\).
There are infinitely many sets of rational numbers \(x, y, z\) for which this inequality is an equality. Identify 11 such sets.

Present your answer as a single \boxed environment containing a list of comma-separated triples of fractions or integers. For instance, if you find 2 sets, format it as \boxed{(\frac{1}{2},\frac{3}{4},-1),(\frac{1}{3},3,\frac{-4}{6})}.

\subsection{imo-shortlist-2009-c2}
\subsubsection{Variation}
\textbf{Actual Problem}\\
For any integer $n \geq 2$, let $N(n)$ be the maximal number of triples $(a_i, b_i, c_i), i=1,\ldots,N(n)$, consisting of nonnegative integers $a_i, b_i$, and $c_i$ such that the following two conditions are satisfied:

(1) $a_i+b_i+c_i=n$ for all $i=1,\ldots,N(n)$,

(2) if $i \neq j$, then $a_i \neq a_j$, $b_i \neq b_j$, and $c_i \neq c_j$.

For $n = 30$, it can be shown that $N(n)=21$. Find $21$ such triples.


Output a comma-separated list of triples, inside of \boxed, for example \boxed{(2,3,4), (5,6,7)}.

\textbf{Revised Problem}\\
Given an integer \( n \geq 2 \), define \( N(n) \) as the largest possible number of distinct triples \((a_i, b_i, c_i)\) of nonnegative integers such that:

1. Each triple satisfies \(a_i + b_i + c_i = n\) for all \(i = 1, \ldots, N(n)\).

2. For any two different triples \(i\) and \(j\), the conditions \(a_i \neq a_j\), \(b_i \neq b_j\), and \(c_i \neq c_j\) hold.

For \(n = 30\), it is established that \(N(n) = 21\). Identify 21 such triples.

Present your answer as a list of triples separated by commas, enclosed within a \boxed format, like \boxed{(1,2,3), (4,5,6)}.

\subsubsection{Variation}
\textbf{Actual Problem}\\
For any integer $n \geq 2$, let $N(n)$ be the maximal number of triples $(a_i, b_i, c_i), i=1,\ldots,N(n)$, consisting of nonnegative integers $a_i, b_i$, and $c_i$ such that the following two conditions are satisfied:

(1) $a_i+b_i+c_i=n$ for all $i=1,\ldots,N(n)$,

(2) if $i \neq j$, then $a_i \neq a_j$, $b_i \neq b_j$, and $c_i \neq c_j$.

For $n = 47$, it can be shown that $N(n)=32$. Find $32$ such triples.


Output a comma-separated list of triples, inside of \boxed, for example \boxed{(2,3,4), (5,6,7)}.

\textbf{Revised Problem}\\
Consider any integer \( n \geq 2 \). Define \( N(n) \) as the largest possible number of distinct triples \((a_i, b_i, c_i)\) composed of nonnegative integers \( a_i, b_i \), and \( c_i \) that fulfill these conditions:

(1) The sum \( a_i + b_i + c_i = n \) holds for every \( i = 1, \ldots, N(n) \).

(2) For any two different triples, \( i \neq j \), the elements of the triples must differ, meaning \( a_i \neq a_j \), \( b_i \neq b_j \), and \( c_i \neq c_j \).

For \( n = 47 \), it is established that \( N(n) = 32 \). Your task is to find and list 32 such triples.

Provide your answer as a list of triples separated by commas, enclosed within \boxed, such as \boxed{(2,3,4), (5,6,7)}.

\subsubsection{Variation}
\textbf{Actual Problem}\\
For any integer $n \geq 2$, let $N(n)$ be the maximal number of triples $(a_i, b_i, c_i), i=1,\ldots,N(n)$, consisting of nonnegative integers $a_i, b_i$, and $c_i$ such that the following two conditions are satisfied:

(1) $a_i+b_i+c_i=n$ for all $i=1,\ldots,N(n)$,

(2) if $i \neq j$, then $a_i \neq a_j$, $b_i \neq b_j$, and $c_i \neq c_j$.

For $n = 24$, it can be shown that $N(n)=17$. Find $17$ such triples.


Output a comma-separated list of triples, inside of \boxed, for example \boxed{(2,3,4), (5,6,7)}.

\textbf{Revised Problem}\\
Consider an integer \(n \geq 2\). Define \(N(n)\) as the largest number of unique triples \((a_i, b_i, c_i)\) of nonnegative integers for \(i = 1, \ldots, N(n)\) such that the following conditions are met:

(1) For each \(i\), the sum \(a_i + b_i + c_i = n\).

(2) For any distinct indices \(i\) and \(j\), the components satisfy \(a_i \neq a_j\), \(b_i \neq b_j\), and \(c_i \neq c_j\).

Given \(n = 24\), it is known that \(N(n) = 17\). Identify 17 triples that fulfill these criteria.

Present the triples as a comma-separated sequence enclosed in \boxed, for instance, \boxed{(2,3,4), (5,6,7)}.

\subsubsection{Variation}
\textbf{Actual Problem}\\
For any integer $n \geq 2$, let $N(n)$ be the maximal number of triples $(a_i, b_i, c_i), i=1,\ldots,N(n)$, consisting of nonnegative integers $a_i, b_i$, and $c_i$ such that the following two conditions are satisfied:

(1) $a_i+b_i+c_i=n$ for all $i=1,\ldots,N(n)$,

(2) if $i \neq j$, then $a_i \neq a_j$, $b_i \neq b_j$, and $c_i \neq c_j$.

For $n = 50$, it can be shown that $N(n)=34$. Find $34$ such triples.


Output a comma-separated list of triples, inside of \boxed, for example \boxed{(2,3,4), (5,6,7)}.

\textbf{Revised Problem}\\
Consider an integer \( n \geq 2 \). Define \( N(n) \) as the greatest possible number of triples \((a_i, b_i, c_i)\), where each \( a_i, b_i, \) and \( c_i \) are nonnegative integers, satisfying the following:

(1) For every \( i \) from 1 through \( N(n) \), the equation \( a_i + b_i + c_i = n \) holds true.

(2) For any \( i \neq j \), it must be the case that \( a_i \neq a_j \), \( b_i \neq b_j \), and \( c_i \neq c_j \).

Given that \( n = 50 \), it is established that \( N(n) = 34 \). Identify 34 such triples.

Present the triples as a comma-separated list enclosed within \boxed{}, such as \boxed{(2,3,4), (5,6,7)}.

\subsection{imo-shortlist-2010-n1}
\subsubsection{Variation}
\textbf{Actual Problem}\\
Find the least positive integer $n$ for which there exists a set $\{s_1, s_2, \ldots, s_n\}$ consisting of $n$ distinct positive integers such that

$$
\left(1 - \frac{1}{s_1}\right) \left(1 - \frac{1}{s_2}\right) \ldots \left(1 - \frac{1}{s_n}\right) = \frac{42}{2010}.
$$

Then, output one such set of $n$ elements.


Output a comma-separated list of integers within a \boxed environment, for example: \boxed{1, 2, 3, 4}.

\textbf{Revised Problem}\\
Determine the smallest positive integer \( n \) such that there exists a collection of \( n \) distinct positive integers \(\{s_1, s_2, \ldots, s_n\}\) satisfying the equation

$$
\prod_{i=1}^{n} \left(1 - \frac{1}{s_i}\right) = \frac{42}{2010}.
$$

Subsequently, provide an example of a set with \( n \) elements that meets this requirement.

Present your answer as a sequence of integers separated by commas and enclosed in a \boxed environment, such as \boxed{1, 2, 3, 4}.

\subsubsection{Variation}
\textbf{Actual Problem}\\
Find the least positive integer $n$ for which there exists a set $\{s_1, s_2, \ldots, s_n\}$ consisting of $n$ distinct positive integers such that

$$
\left(1 - \frac{1}{s_1}\right) \left(1 - \frac{1}{s_2}\right) \ldots \left(1 - \frac{1}{s_n}\right) = \frac{51}{2010}.
$$

Then, output one such set of $n$ elements.


Output a comma-separated list of integers within a \boxed environment, for example: \boxed{1, 2, 3, 4}.

\textbf{Revised Problem}\\
Determine the smallest positive integer \( n \) such that there exists a collection \(\{s_1, s_2, \ldots, s_n\}\) of \( n \) distinct positive integers satisfying the equation

$$
\left(1 - \frac{1}{s_1}\right) \left(1 - \frac{1}{s_2}\right) \cdots \left(1 - \frac{1}{s_n}\right) = \frac{51}{2010}.
$$

After identifying this \( n \), provide an example of one such collection with \( n \) elements.

Present the integers in a comma-separated format enclosed in a \boxed environment, for example: \boxed{5, 6, 7, 8}.

\subsection{imo-shortlist-2011-a1}
\subsubsection{Variation}
\textbf{Actual Problem}\\
For any set $A = {a_1, a_2, a_3, a_4}$ of four distinct positive integers with sum $s_A = a_1 + a_2 + a_3 + a_4$,
let $p_A$ denote the number of pairs $(i, j)$ with $1 \leq i < j \leq 4$ for which $a_i + a_j$ divides $s_A$. 

Among all sets of four distinct positive integers, the maximal value of $p_A$ is $4$.
Output $21$ distinct sets $A$ for which $p_A=4$ and additionally for each set it holds that $a_1 \leq 11$. 


Output a comma-separated list of 4-tuples of integers inside of \boxed, for example \boxed{((2,3,4,5), (4,5,6,7))}.

\textbf{Revised Problem}\\
Consider any set $A = \{a_1, a_2, a_3, a_4\}$, where each $a_i$ is a distinct positive integer, and the total sum is $s_A = a_1 + a_2 + a_3 + a_4$. Define $p_A$ as the count of pairs $(i, j)$ with $1 \leq i < j \leq 4$ such that the sum $a_i + a_j$ evenly divides $s_A$. 

The highest possible value of $p_A$ for any such set of four distinct integers is $4$. Your task is to identify and list 21 unique sets $A$ satisfying $p_A = 4$, with the additional constraint that $a_1 \leq 11$.

Present your solution as a comma-separated list of 4-tuples within \boxed, such as \boxed{((2,3,4,5), (4,5,6,7))}.

\subsubsection{Variation}
\textbf{Actual Problem}\\
For any set $A = {a_1, a_2, a_3, a_4}$ of four distinct positive integers with sum $s_A = a_1 + a_2 + a_3 + a_4$,
let $p_A$ denote the number of pairs $(i, j)$ with $1 \leq i < j \leq 4$ for which $a_i + a_j$ divides $s_A$. 

Among all sets of four distinct positive integers, the maximal value of $p_A$ is $4$.
Output $22$ distinct sets $A$ for which $p_A=4$ and additionally for each set it holds that $a_1 \leq 11$. 


Output a comma-separated list of 4-tuples of integers inside of \boxed, for example \boxed{((2,3,4,5), (4,5,6,7))}.

\textbf{Revised Problem}\\
Consider a set $A = \{a_1, a_2, a_3, a_4\}$ consisting of four distinct positive integers. Define the total sum of these integers as $s_A = a_1 + a_2 + a_3 + a_4$. Let $p_A$ represent the count of pairs $(i, j)$ where $1 \leq i < j \leq 4$ and the sum $a_i + a_j$ is a divisor of $s_A$.

The maximum achievable value for $p_A$ among all possible sets of four distinct positive integers is $4$. You are required to identify 22 unique sets $A$ such that $p_A=4$, ensuring that for each set, the smallest integer $a_1 \leq 11$.

Provide your answer as a comma-separated list of 4-tuples of integers enclosed within \boxed, such as \boxed{((2,3,4,5), (4,5,6,7))}.

\subsubsection{Variation}
\textbf{Actual Problem}\\
For any set $A = {a_1, a_2, a_3, a_4}$ of four distinct positive integers with sum $s_A = a_1 + a_2 + a_3 + a_4$,
let $p_A$ denote the number of pairs $(i, j)$ with $1 \leq i < j \leq 4$ for which $a_i + a_j$ divides $s_A$. 

Among all sets of four distinct positive integers, the maximal value of $p_A$ is $4$.
Output $14$ distinct sets $A$ for which $p_A=4$ and additionally for each set it holds that $a_1 \leq 7$. 


Output a comma-separated list of 4-tuples of integers inside of \boxed, for example \boxed{((2,3,4,5), (4,5,6,7))}.

\textbf{Revised Problem}\\
Consider any collection $A = {a_1, a_2, a_3, a_4}$ of four unique positive integers with the total $s_A = a_1 + a_2 + a_3 + a_4$. Define $p_A$ as the count of pairs $(i, j)$ such that $1 \leq i < j \leq 4$ and $a_i + a_j$ divides $s_A$.

Out of all possible collections of four unique positive integers, the highest possible value for $p_A$ is $4$. Provide $14$ distinct collections $A$ where $p_A=4$ and, in each case, $a_1 \leq 7$.

Present your answer as a comma-separated sequence of 4-tuples of integers enclosed within \boxed, for instance \boxed{((2,3,4,5), (4,5,6,7))}.

\subsubsection{Variation}
\textbf{Actual Problem}\\
For any set $A = {a_1, a_2, a_3, a_4}$ of four distinct positive integers with sum $s_A = a_1 + a_2 + a_3 + a_4$,
let $p_A$ denote the number of pairs $(i, j)$ with $1 \leq i < j \leq 4$ for which $a_i + a_j$ divides $s_A$. 

Among all sets of four distinct positive integers, the maximal value of $p_A$ is $4$.
Output $11$ distinct sets $A$ for which $p_A=4$ and additionally for each set it holds that $a_1 \leq 6$. 


Output a comma-separated list of 4-tuples of integers inside of \boxed, for example \boxed{((2,3,4,5), (4,5,6,7))}.

\textbf{Revised Problem}\\
Consider a set $A = \{a_1, a_2, a_3, a_4\}$ consisting of four different positive integers, where the total sum $s_A = a_1 + a_2 + a_3 + a_4$. Let $p_A$ represent the count of pairs $(i, j)$, satisfying $1 \leq i < j \leq 4$, for which the sum $a_i + a_j$ is a divisor of $s_A$.

Determine the maximum possible value of $p_A$ among all such sets of four distinct positive integers, which is known to be $4$. Provide $11$ distinct sets $A$ achieving $p_A=4$, with the additional condition that $a_1 \leq 6$.

Present the sets as a comma-separated sequence of 4-tuples enclosed within \boxed, such as \boxed{((2,3,4,5), (4,5,6,7))}.

\subsection{imo-shortlist-2012-c2}
\subsubsection{Variation}
\textbf{Actual Problem}\\
Let $n \geq 1$ be an integer. We are interested in finding the maximum number of disjoint pairs of elements of the
set $\{1, 2, \ldots, n\}$ such that the sums of the different pairs are different integers not exceeding $n$. 
For $n=70$ such set of pairs is of size $27$. Find one such set.

Output a comma-separated list of pairs inside of \boxed, for example \boxed{((2,3), (4,5), (1, 6))}.

\textbf{Revised Problem}\\
Consider an integer $n \geq 1$. Your goal is to determine the largest collection of non-overlapping pairs drawn from the set $\{1, 2, \ldots, n\}$ so that each pair's sum is a distinct integer that does not exceed $n$. For the case when $n=70$, one such collection of pairs has a size of $27$. Provide an example of such a collection.

List the pairs in a comma-separated format enclosed in \boxed, such as \boxed{((2,3), (4,5), (1, 6))}.

\subsubsection{Variation}
\textbf{Actual Problem}\\
Let $n \geq 1$ be an integer. We are interested in finding the maximum number of disjoint pairs of elements of the
set $\{1, 2, \ldots, n\}$ such that the sums of the different pairs are different integers not exceeding $n$. 
For $n=72$ such set of pairs is of size $28$. Find one such set.

Output a comma-separated list of pairs inside of \boxed, for example \boxed{((2,3), (4,5), (1, 6))}.

\textbf{Revised Problem}\\
Consider an integer \( n \geq 1\). We aim to determine the largest collection of non-overlapping pairs from the set \(\{1, 2, \ldots, n\}\) such that the sum of each pair is a unique integer that does not exceed \(n\). When \(n=72\), one such collection contains 28 pairs. Identify one such collection.

Provide a comma-separated list of pairs enclosed within \boxed, for instance \boxed{((2,3), (4,5), (1, 6))}.

\subsubsection{Variation}
\textbf{Actual Problem}\\
Let $n \geq 1$ be an integer. We are interested in finding the maximum number of disjoint pairs of elements of the
set $\{1, 2, \ldots, n\}$ such that the sums of the different pairs are different integers not exceeding $n$. 
For $n=40$ such set of pairs is of size $15$. Find one such set.

Output a comma-separated list of pairs inside of \boxed, for example \boxed{((2,3), (4,5), (1, 6))}.

\textbf{Revised Problem}\\
Consider an integer $n \geq 1$. We seek the largest number of non-overlapping pairs from the set $\{1, 2, \ldots, n\}$ such that each pair's sum is a unique integer that is at most $n$. For the specific case of $n = 40$, it is established that a collection of 15 such pairs exists. Identify one possible collection.

Present your answer as a list of pairs separated by commas within a \boxed, for instance \boxed{((2,3), (4,5), (1, 6))}.

\subsubsection{Variation}
\textbf{Actual Problem}\\
Let $n \geq 1$ be an integer. We are interested in finding the maximum number of disjoint pairs of elements of the
set $\{1, 2, \ldots, n\}$ such that the sums of the different pairs are different integers not exceeding $n$. 
For $n=30$ such set of pairs is of size $11$. Find one such set.

Output a comma-separated list of pairs inside of \boxed, for example \boxed{((2,3), (4,5), (1, 6))}.

\textbf{Revised Problem}\\
Consider an integer \( n \geq 1 \). Our goal is to determine the largest number of non-overlapping pairs from the set \(\{1, 2, \ldots, n\}\) such that each pair sums to a unique integer, with all sums being distinct and less than or equal to \( n \). For \( n = 30 \), it is known that such a collection of pairs can have a maximum size of 11. Identify one such collection.

Present the solution as a comma-separated list of pairs enclosed in \boxed, such as \boxed{((2,3), (4,5), (1, 6))}.

\subsection{imo-shortlist-2014-c3}
\subsubsection{Variation}
\textbf{Actual Problem}\\
Let $n \geq 2$ be an integer. Consider an $n \times n$ chessboard divided into $n^2$ unit squares.
We call a configuration of $n$ rooks on this board \textit{happy} if every row and every column contains exactly one rook. 
For $n=22$ and $k=5$, find a happy configuration of rooks such that there is no $k \times k$ square without a rook on any of its $k^2$ unit squares.


Output a comma-separated list of 1-based rook indices (so (1, 1) is the top left corner of the board) inside of \boxed, for example \boxed{((1,1), (4,5), (1,6))}.

\textbf{Revised Problem}\\
Consider an integer $n \geq 2$, and an $n \times n$ chessboard, which is made up of $n^2$ individual unit squares. A placement of $n$ rooks on this board is termed \textit{happy} if each row and each column has precisely one rook. When $n=22$ and $k=5$, determine a happy arrangement of rooks such that every possible $k \times k$ section of the board has at least one rook within its $k^2$ squares.

Present your answer as a comma-separated list of rook positions using 1-based indices (where (1, 1) corresponds to the top-left square of the board) inside a \boxed environment, such as \boxed{((1,1), (4,5), (1,6))}.

\subsubsection{Variation}
\textbf{Actual Problem}\\
Let $n \geq 2$ be an integer. Consider an $n \times n$ chessboard divided into $n^2$ unit squares.
We call a configuration of $n$ rooks on this board \textit{happy} if every row and every column contains exactly one rook. 
For $n=19$ and $k=5$, find a happy configuration of rooks such that there is no $k \times k$ square without a rook on any of its $k^2$ unit squares.


Output a comma-separated list of 1-based rook indices (so (1, 1) is the top left corner of the board) inside of \boxed, for example \boxed{((1,1), (4,5), (1,6))}.

\textbf{Revised Problem}\\
Given an integer $n \geq 2$, imagine an $n \times n$ chessboard split into $n^2$ individual squares. A placement of $n$ rooks on this board is termed as \textit{happy} if there is precisely one rook in each row and each column. For the case where $n=19$ and $k=5$, determine a happy placement of rooks so that no $5 \times 5$ section of the board is without at least one rook.

Provide the positions of the rooks as a comma-separated list of 1-based indices (for instance, (1, 1) represents the top-left corner of the board) within \boxed, like \boxed{((1,1), (4,5), (1,6))}.

\subsubsection{Variation}
\textbf{Actual Problem}\\
Let $n \geq 2$ be an integer. Consider an $n \times n$ chessboard divided into $n^2$ unit squares.
We call a configuration of $n$ rooks on this board \textit{happy} if every row and every column contains exactly one rook. 
For $n=16$ and $k=4$, find a happy configuration of rooks such that there is no $k \times k$ square without a rook on any of its $k^2$ unit squares.


Output a comma-separated list of 1-based rook indices (so (1, 1) is the top left corner of the board) inside of \boxed, for example \boxed{((1,1), (4,5), (1,6))}.

\textbf{Revised Problem}\\
Let $n \geq 2$ be an integer. Imagine an $n \times n$ chessboard composed of $n^2$ individual squares. We define a configuration of $n$ rooks on this board as \textit{happy} if there is exactly one rook in each row and each column. For $n=16$ and $k=4$, identify a happy configuration of rooks such that every $k \times k$ block on the board contains at least one rook in each of its $k^2$ squares.

Provide the answer as a comma-separated list of 1-based rook coordinates (where (1, 1) represents the top left corner) enclosed in \boxed, such as \boxed{((1,1), (4,5), (1,6))}.

\subsubsection{Variation}
\textbf{Actual Problem}\\
Let $n \geq 2$ be an integer. Consider an $n \times n$ chessboard divided into $n^2$ unit squares.
We call a configuration of $n$ rooks on this board \textit{happy} if every row and every column contains exactly one rook. 
For $n=15$ and $k=4$, find a happy configuration of rooks such that there is no $k \times k$ square without a rook on any of its $k^2$ unit squares.


Output a comma-separated list of 1-based rook indices (so (1, 1) is the top left corner of the board) inside of \boxed, for example \boxed{((1,1), (4,5), (1,6))}.

\textbf{Revised Problem}\\
Let $n \geq 2$ be a whole number. Imagine an $n \times n$ chessboard, which is split into $n^2$ individual squares. We define a placement of $n$ rooks on this board as \textit{happy} if each row and each column has one and only one rook. For $n=15$ and $k=4$, determine a happy arrangement of rooks such that no $k \times k$ subgrid is entirely without a rook in any of its $k^2$ individual squares.

Provide the output as a comma-separated list of rook positions using 1-based indices (thus, (1, 1) refers to the top left square of the board) within the \boxed command, for instance \boxed{((1,1), (4,5), (1,6))}.

\subsection{imo-shortlist-2014-n2}
\subsubsection{Variation}
\textbf{Actual Problem}\\
Find 25 distinct pairs $(x, y)$ of positive integers such that $x<y$ and
$$
\sqrt[3]{7x^2-13xy+7y^2} = |x-y|+1.
$$


Output a comma-separated list of pairs (x, y), inside of \boxed, for example \boxed{(2,3), (5,6)}.

\textbf{Revised Problem}\\
Identify 25 distinct pairs \((x, y)\) of positive integers where \(x < y\) and the following holds true:
\[
\sqrt[3]{7x^2 - 13xy + 7y^2} = |x-y| + 1.
\]

Provide the result as a list of pairs \((x, y)\), separated by commas, within a \boxed{ }, for instance, \boxed{(2,3), (5,6)}.

\subsubsection{Variation}
\textbf{Actual Problem}\\
Find 32 distinct pairs $(x, y)$ of positive integers such that $x<y$ and
$$
\sqrt[3]{7x^2-13xy+7y^2} = |x-y|+1.
$$


Output a comma-separated list of pairs (x, y), inside of \boxed, for example \boxed{(2,3), (5,6)}.

\textbf{Revised Problem}\\
Identify 32 unique pairs \((x, y)\) of positive integers such that \(x < y\) and the following equation holds:
$$
\sqrt[3]{7x^2 - 13xy + 7y^2} = |x - y| + 1.
$$

Present your answer as a comma-separated sequence of pairs \((x, y)\) within \boxed, for instance, \boxed{(2,3), (5,6)}.

\subsubsection{Variation}
\textbf{Actual Problem}\\
Find 24 distinct pairs $(x, y)$ of positive integers such that $x<y$ and
$$
\sqrt[3]{7x^2-13xy+7y^2} = |x-y|+1.
$$


Output a comma-separated list of pairs (x, y), inside of \boxed, for example \boxed{(2,3), (5,6)}.

\textbf{Revised Problem}\\
Identify 24 unique pairs \((x, y)\) of positive integers where \(x < y\), meeting the requirement:
$$
\sqrt[3]{7x^2 - 13xy + 7y^2} = |x - y| + 1.
$$

Present the solution as a list of pairs \((x, y)\), separated by commas, and enclosed within \boxed, for instance \boxed{(2,3), (5,6)}.

\subsubsection{Variation}
\textbf{Actual Problem}\\
Find 21 distinct pairs $(x, y)$ of positive integers such that $x<y$ and
$$
\sqrt[3]{7x^2-13xy+7y^2} = |x-y|+1.
$$


Output a comma-separated list of pairs (x, y), inside of \boxed, for example \boxed{(2,3), (5,6)}.

\textbf{Revised Problem}\\
Identify 21 unique pairs of positive integers \((x, y)\) with \(x < y\) that satisfy the equation:

\[
\sqrt[3]{7x^2 - 13xy + 7y^2} = |x-y| + 1.
\]

Provide your answer as a list of pairs \((x, y)\), separated by commas, within a \boxed environment, such as \boxed{(2,3), (5,6)}.

\subsection{imo-shortlist-2016-a5}
\subsubsection{Variation}
\textbf{Actual Problem}\\
Given $n = 10000000000000000000$, find a fraction $\frac{a}{b}$ where $a$ and $b$ are integers satisfying $0 < b \leq \sqrt{n}+1$ and $\sqrt{n} \leq \frac{a}{b} \leq \sqrt{n+1}$.

Output the answer as a fraction inside of $\boxed{...}$. For example $\boxed{\frac{1}{2}}$.

\textbf{Revised Problem}\\
For $n = 10000000000000000000$, determine a fraction $\frac{a}{b}$ with integer values of $a$ and $b$ such that $0 < b \leq \sqrt{n} + 1$ and $\sqrt{n} \leq \frac{a}{b} \leq \sqrt{n+1}$.

Present the solution as a fraction enclosed in $\boxed{...}$. For instance, $\boxed{\frac{1}{2}}$.

\subsubsection{Variation}
\textbf{Actual Problem}\\
Given $n = 17758176404715800194$, find a fraction $\frac{a}{b}$ where $a$ and $b$ are integers satisfying $0 < b \leq \sqrt{n}+1$ and $\sqrt{n} \leq \frac{a}{b} \leq \sqrt{n+1}$.

Output the answer as a fraction inside of $\boxed{...}$. For example $\boxed{\frac{1}{2}}$.

\textbf{Revised Problem}\\
For the given value \(n = 17758176404715800194\), determine a fraction \(\frac{a}{b}\) where \(a\) and \(b\) are integers such that \(0 < b \leq \sqrt{n}+1\) and \(\sqrt{n} \leq \frac{a}{b} \leq \sqrt{n+1}\).

Present the solution as a fraction formatted within a box, like this: \(\boxed{\frac{1}{2}}\).

\subsubsection{Variation}
\textbf{Actual Problem}\\
Given $n = 24089154938208861744$, find a fraction $\frac{a}{b}$ where $a$ and $b$ are integers satisfying $0 < b \leq \sqrt{n}+1$ and $\sqrt{n} \leq \frac{a}{b} \leq \sqrt{n+1}$.

Output the answer as a fraction inside of $\boxed{...}$. For example $\boxed{\frac{1}{2}}$.

\textbf{Revised Problem}\\
Given $n = 24089154938208861744$, determine a fraction $\frac{a}{b}$ where $a$ and $b$ are whole numbers such that $0 < b \leq \sqrt{n}+1$ and $\sqrt{n} \leq \frac{a}{b} \leq \sqrt{n+1}$.

Present your solution as a fraction enclosed in $\boxed{...}$. For instance $\boxed{\frac{1}{2}}$.

\subsubsection{Variation}
\textbf{Actual Problem}\\
Given $n = 11043188731678011336$, find a fraction $\frac{a}{b}$ where $a$ and $b$ are integers satisfying $0 < b \leq \sqrt{n}+1$ and $\sqrt{n} \leq \frac{a}{b} \leq \sqrt{n+1}$.

Output the answer as a fraction inside of $\boxed{...}$. For example $\boxed{\frac{1}{2}}$.

\textbf{Revised Problem}\\
Given the number \(n = 11043188731678011336\), identify a fraction \(\frac{a}{b}\) composed of integers \(a\) and \(b\) such that \(0 < b \leq \sqrt{n} + 1\) and the condition \(\sqrt{n} \leq \frac{a}{b} \leq \sqrt{n+1}\) holds true.

Express the solution in the format of a fraction enclosed in \(\boxed{...}\). For instance, \(\boxed{\frac{1}{2}}\).

\subsection{imo-shortlist-2016-c4}
\subsubsection{Variation}
\textbf{Actual Problem}\\
Given $n = 9$, fill the cells of an $n \times n$ table with one of the letters $I,M$ and $O$ in such a way that:
(i) in each row and each column, one third of the entries are $I$, one third are $M$ and one third are $O$; and
(ii) in any diagonal, if the number of entries on the diagonal is a multiple of three, then one third of the entries are $I$, one third are $M$ and one third are $O$.

Output the answer between \verb|\begin{array}{...}| and \verb|\end{array}| inside of $\boxed{...}$. For example, $\boxed{\begin{array}{ccc}1 & 2 & 3 \\ 4 & 5 & 6 \\ 7 & 8 & 9\end{array}}$.

\textbf{Revised Problem}\\
Let \( n = 9 \). Populate a \( 9 \times 9 \) grid using the letters \( I, M, \) and \( O \) such that:
1. Each row and each column contains exactly one third of each letter (\( I, M, \) and \( O \)).
2. For any diagonal whose length is a multiple of three, ensure that one third of the entries are \( I \), one third are \( M \), and one third are \( O \).

Present your solution within \verb|\begin{array}{...}| and \verb|\end{array}| inside of $\boxed{...}$. For example, $\boxed{\begin{array}{ccc}1 & 2 & 3 \\ 4 & 5 & 6 \\ 7 & 8 & 9\end{array}}$.

\subsubsection{Variation}
\textbf{Actual Problem}\\
Given $n = 36$, fill the cells of an $n \times n$ table with one of the letters $I,M$ and $O$ in such a way that:
(i) in each row and each column, one third of the entries are $I$, one third are $M$ and one third are $O$; and
(ii) in any diagonal, if the number of entries on the diagonal is a multiple of three, then one third of the entries are $I$, one third are $M$ and one third are $O$.

Output the answer between \verb|\begin{array}{...}| and \verb|\end{array}| inside of $\boxed{...}$. For example, $\boxed{\begin{array}{ccc}1 & 2 & 3 \\ 4 & 5 & 6 \\ 7 & 8 & 9\end{array}}$.

\textbf{Revised Problem}\\
Consider an $n \times n$ grid where $n = 36$. Populate each cell with one of the letters 'I', 'M', or 'O' such that:
(i) each row and each column contains equal numbers of 'I', 'M', and 'O'; and
(ii) for any diagonal whose length is a multiple of three, the number of 'I's, 'M's, and 'O's are equal.

Present your solution enclosed within \verb|\begin{array}{...}| and \verb|\end{array}| using the $\boxed{...}$ environment. For instance, $\boxed{\begin{array}{ccc}1 & 2 & 3 \\ 4 & 5 & 6 \\ 7 & 8 & 9\end{array}}$.

\subsubsection{Variation}
\textbf{Actual Problem}\\
Given $n = 18$, fill the cells of an $n \times n$ table with one of the letters $I,M$ and $O$ in such a way that:
(i) in each row and each column, one third of the entries are $I$, one third are $M$ and one third are $O$; and
(ii) in any diagonal, if the number of entries on the diagonal is a multiple of three, then one third of the entries are $I$, one third are $M$ and one third are $O$.

Output the answer between \verb|\begin{array}{...}| and \verb|\end{array}| inside of $\boxed{...}$. For example, $\boxed{\begin{array}{ccc}1 & 2 & 3 \\ 4 & 5 & 6 \\ 7 & 8 & 9\end{array}}$.

\textbf{Revised Problem}\\
Consider an $18 \times 18$ grid. Populate each cell with one of the letters I, M, or O such that: 
(i) every row and column contains each of the letters I, M, and O exactly one-third of the time; and 
(ii) for any diagonal whose length is divisible by three, one-third of the cells contain I, one-third contain M, and one-third contain O.

Present your solution encapsulated within \verb|\begin{array}{...}| and \verb|\end{array}|, enclosed in $\boxed{...}$. For instance, $\boxed{\begin{array}{ccc}1 & 2 & 3 \\ 4 & 5 & 6 \\ 7 & 8 & 9\end{array}}$.

\subsubsection{Variation}
\textbf{Actual Problem}\\
Given $n = 45$, fill the cells of an $n \times n$ table with one of the letters $I,M$ and $O$ in such a way that:
(i) in each row and each column, one third of the entries are $I$, one third are $M$ and one third are $O$; and
(ii) in any diagonal, if the number of entries on the diagonal is a multiple of three, then one third of the entries are $I$, one third are $M$ and one third are $O$.

Output the answer between \verb|\begin{array}{...}| and \verb|\end{array}| inside of $\boxed{...}$. For example, $\boxed{\begin{array}{ccc}1 & 2 & 3 \\ 4 & 5 & 6 \\ 7 & 8 & 9\end{array}}$.

\textbf{Revised Problem}\\
Consider an \( n \times n \) grid where \( n = 45 \). Populate this grid with the letters \( I \), \( M \), and \( O \) such that:
(i) Each row and column contains an equal number of each letter, with exactly one-third being \( I \), one-third being \( M \), and one-third being \( O \);
(ii) For any diagonal whose length is a multiple of three, ensure that it contains one-third \( I \), one-third \( M \), and one-third \( O\).

Present your solution within \verb|\begin{array}{...}| and \verb|\end{array}| enclosed in $\boxed{...}$. For example, $\boxed{\begin{array}{ccc}1 & 2 & 3 \\ 4 & 5 & 6 \\ 7 & 8 & 9\end{array}}$.

\subsection{imo-shortlist-2017-n3}
\subsubsection{Variation}
\textbf{Actual Problem}\\
Given $n = 30$, find integers $a_1,a_2,\ldots, a_n$ whose sum is not divisible by $n$, 
and there does not exist an index $1 \leq i \leq n$ such that none of the numbers $a_i,a_i+a_{i+1},\ldots,a_i+a_{i+1}+\ldots+a_{i+n-1}$ is divisible by $n$. 
Here, we let $a_i=a_{i-n}$ when $i >n$.


Output the answer as a comma separated list inside of $\boxed{...}$. For example $\boxed{1, 2, 3}$.

\textbf{Revised Problem}\\
For \( n = 30 \), determine a sequence of integers \( b_1, b_2, \ldots, b_n \) such that the total sum of these integers is not a multiple of \( n \). Additionally, ensure that for every index \( 1 \leq j \leq n \), at least one of the sums \( b_j, b_j+b_{j+1}, \ldots, b_j+b_{j+1}+\ldots+b_{j+n-1} \) is divisible by \( n \). Here, we assume that \( b_j = b_{j-n} \) when \( j > n \).

Present the solution as a sequence of numbers separated by commas within $\boxed{...}$. For instance, $\boxed{1, 2, 3}$.

\subsubsection{Variation}
\textbf{Actual Problem}\\
Given $n = 35$, find integers $a_1,a_2,\ldots, a_n$ whose sum is not divisible by $n$, 
and there does not exist an index $1 \leq i \leq n$ such that none of the numbers $a_i,a_i+a_{i+1},\ldots,a_i+a_{i+1}+\ldots+a_{i+n-1}$ is divisible by $n$. 
Here, we let $a_i=a_{i-n}$ when $i >n$.


Output the answer as a comma separated list inside of $\boxed{...}$. For example $\boxed{1, 2, 3}$.

\textbf{Revised Problem}\\
For $n = 35$, determine integers $b_1, b_2, \ldots, b_n$ such that their sum is not divisible by $n$, and there is no index $1 \leq j \leq n$ for which all numbers in the sequence $b_j, b_j + b_{j+1}, \ldots, b_j + b_{j+1} + \ldots + b_{j+n-1}$ are divisible by $n$. Assume that $b_i = b_{i-n}$ when $i > n$.

Present your answer in the form of a comma-separated list enclosed in $\boxed{...}$. For instance, $\boxed{1, 2, 3}$.

\subsubsection{Variation}
\textbf{Actual Problem}\\
Given $n = 25$, find integers $a_1,a_2,\ldots, a_n$ whose sum is not divisible by $n$, 
and there does not exist an index $1 \leq i \leq n$ such that none of the numbers $a_i,a_i+a_{i+1},\ldots,a_i+a_{i+1}+\ldots+a_{i+n-1}$ is divisible by $n$. 
Here, we let $a_i=a_{i-n}$ when $i >n$.


Output the answer as a comma separated list inside of $\boxed{...}$. For example $\boxed{1, 2, 3}$.

\textbf{Revised Problem}\\
For \( n = 25 \), determine a sequence of integers \( a_1, a_2, \ldots, a_{25} \) such that:
1. The total sum \( a_1 + a_2 + \cdots + a_{25} \) is not a multiple of 25.
2. For every index \( i \) (where \( 1 \leq i \leq 25 \)), there is at least one sum among \( a_i, a_i + a_{i+1}, \ldots, a_i + a_{i+1} + \cdots + a_{i+24} \) that is divisible by 25. Note that the indices wrap around cyclically, so \( a_i = a_{i-25} \) if \( i > 25 \).

Present your solution as a comma-separated list within the format $\boxed{...}$. For instance, $\boxed{1, 2, 3}$.

\subsubsection{Variation}
\textbf{Actual Problem}\\
Given $n = 21$, find integers $a_1,a_2,\ldots, a_n$ whose sum is not divisible by $n$, 
and there does not exist an index $1 \leq i \leq n$ such that none of the numbers $a_i,a_i+a_{i+1},\ldots,a_i+a_{i+1}+\ldots+a_{i+n-1}$ is divisible by $n$. 
Here, we let $a_i=a_{i-n}$ when $i >n$.


Output the answer as a comma separated list inside of $\boxed{...}$. For example $\boxed{1, 2, 3}$.

\textbf{Revised Problem}\\
Suppose \( n = 21 \). Determine integers \( b_1, b_2, \ldots, b_n \) such that the sum of these integers is not divisible by \( n \), and for every possible starting position \( 1 \leq j \leq n \), there exists a subsequence sum \( b_j, b_j+b_{j+1}, \ldots, b_j+b_{j+1}+\cdots+b_{j+n-1} \) that is divisible by \( n \). Note that \( b_j = b_{j-n} \) when \( j > n \).

Present the solution as a comma-separated sequence enclosed in $\boxed{...}$. For instance, $\boxed{1, 2, 3}$.

\subsection{imo-shortlist-2017-n6}
\subsubsection{Variation}
\textbf{Actual Problem}\\
Find $10$ different triples of positive rational numbers $(x, y, z)$ such that both $x + y + z$ and $\frac{1}{x} + \frac{1}{y} + \frac{1}{z}$ are integers.
Additionally, for each $i = 1, 2, \ldots, n$, in the $i$-th triple of your solution at least one of $x, y, z$ should have numerator or denominator with at least $i$ digits.
Note that one can show that there are infintely many such triples.


Output a comma-separated list of different triples $(x, y, z)$, inside of \boxed, for example \boxed{(\frac{1}{2},\frac{1}{3},\frac{1}{6}), (\frac{1}{3},\frac{1}{4},\frac{1}{12})}.

\textbf{Revised Problem}\\
Identify $10$ distinct sets of positive rational numbers $(x, y, z)$ such that both the sum $x + y + z$ and the sum of reciprocals $\frac{1}{x} + \frac{1}{y} + \frac{1}{z}$ are integers. Furthermore, for each set indexed by $i = 1, 2, \ldots, n$, at least one element among $x, y, z$ should have either a numerator or a denominator with a minimum of $i$ digits. It can be demonstrated that there are infinitely many such sets.

Provide a comma-separated sequence of distinct triples $(x, y, z)$, enclosed within \boxed, for instance \boxed{(\frac{1}{2},\frac{1}{3},\frac{1}{6}), (\frac{1}{3},\frac{1}{4},\frac{1}{12})}.

\subsubsection{Variation}
\textbf{Actual Problem}\\
Find $17$ different triples of positive rational numbers $(x, y, z)$ such that both $x + y + z$ and $\frac{1}{x} + \frac{1}{y} + \frac{1}{z}$ are integers.
Additionally, for each $i = 1, 2, \ldots, n$, in the $i$-th triple of your solution at least one of $x, y, z$ should have numerator or denominator with at least $i$ digits.
Note that one can show that there are infintely many such triples.


Output a comma-separated list of different triples $(x, y, z)$, inside of \boxed, for example \boxed{(\frac{1}{2},\frac{1}{3},\frac{1}{6}), (\frac{1}{3},\frac{1}{4},\frac{1}{12})}.

\textbf{Revised Problem}\\
Identify $17$ distinct sets of positive rational numbers $(x, y, z)$ such that both the sum $x + y + z$ and the reciprocal sum $\frac{1}{x} + \frac{1}{y} + \frac{1}{z}$ are integer values. Furthermore, for each set indexed by $i = 1, 2, \ldots, n$, at least one of the numbers $x, y, z$ must have a numerator or a denominator consisting of at least $i$ digits. Note that it can be shown that there are infinitely many such sets of numbers.

Present your answer as a list of distinct sets of numbers $(x, y, z)$, separated by commas, and enclosed within \boxed, such as \boxed{(\frac{1}{2},\frac{1}{3},\frac{1}{6}), (\frac{1}{3},\frac{1}{4},\frac{1}{12})}.

\subsubsection{Variation}
\textbf{Actual Problem}\\
Find $9$ different triples of positive rational numbers $(x, y, z)$ such that both $x + y + z$ and $\frac{1}{x} + \frac{1}{y} + \frac{1}{z}$ are integers.
Additionally, for each $i = 1, 2, \ldots, n$, in the $i$-th triple of your solution at least one of $x, y, z$ should have numerator or denominator with at least $i$ digits.
Note that one can show that there are infintely many such triples.


Output a comma-separated list of different triples $(x, y, z)$, inside of \boxed, for example \boxed{(\frac{1}{2},\frac{1}{3},\frac{1}{6}), (\frac{1}{3},\frac{1}{4},\frac{1}{12})}.

\textbf{Revised Problem}\\
Identify $9$ unique sets of positive rational numbers $(x, y, z)$ such that both $x + y + z$ and $\frac{1}{x} + \frac{1}{y} + \frac{1}{z}$ are integer values. Furthermore, for each $i = 1, 2, \ldots, 9$, in the $i$-th set of numbers, at least one element among $x, y, z$ must have a numerator or denominator with a minimum of $i$ digits. Note that it is possible to demonstrate that there are infinite such sets.

Present your solution as a list of distinct triples $(x, y, z)$ separated by commas, enclosed in \boxed. For example, \boxed{(\frac{1}{2},\frac{1}{3},\frac{1}{6}), (\frac{1}{3},\frac{1}{4},\frac{1}{12})}.

\subsubsection{Variation}
\textbf{Actual Problem}\\
Find $6$ different triples of positive rational numbers $(x, y, z)$ such that both $x + y + z$ and $\frac{1}{x} + \frac{1}{y} + \frac{1}{z}$ are integers.
Additionally, for each $i = 1, 2, \ldots, n$, in the $i$-th triple of your solution at least one of $x, y, z$ should have numerator or denominator with at least $i$ digits.
Note that one can show that there are infintely many such triples.


Output a comma-separated list of different triples $(x, y, z)$, inside of \boxed, for example \boxed{(\frac{1}{2},\frac{1}{3},\frac{1}{6}), (\frac{1}{3},\frac{1}{4},\frac{1}{12})}.

\textbf{Revised Problem}\\
Identify $6$ distinct sets of positive rational numbers $(x, y, z)$ such that both the sum $x + y + z$ and the sum of their reciprocals $\frac{1}{x} + \frac{1}{y} + \frac{1}{z}$ are whole numbers. Furthermore, for each set indexed by $i = 1, 2, \ldots, n$, in the $i$-th set of numbers, at least one among $x, y, z$ must have its numerator or denominator consisting of at least $i$ digits. It can be demonstrated that there are countless such sets of numbers.

Provide your answer as a list of distinct sets $(x, y, z)$, separated by commas and enclosed within \boxed. For example: \boxed{(\frac{1}{2},\frac{1}{3},\frac{1}{6}), (\frac{1}{3},\frac{1}{4},\frac{1}{12})}.

\subsection{imo-shortlist-2018-c1}
\subsubsection{Variation}
\textbf{Actual Problem}\\
Given $n = 10$, find a set $S$ of $2n$ distinct positive integers
satisfying the following property: For every $m = 2, 3, \ldots, n$ the set $S$ can be partitioned into
two subsets with equal sums of elements, with one of the subsets being of cardinality $m$.

Output the answer as a comma separated list inside of $\boxed{...}$. For example $\boxed{1, 2, 3}$.

\textbf{Revised Problem}\\
Given \( n = 10 \), identify a set \( S \) composed of \( 2n \) unique positive integers that meets this requirement: For each integer \( m \) ranging from 2 to \( n \), there exists a way to divide \( S \) into two subsets that have equal sums, with one subset having exactly \( m \) elements.

Present your solution as a sequence of numbers separated by commas, encapsulated within \(\boxed{...}\). For instance, \(\boxed{1, 2, 3}\).

\subsubsection{Variation}
\textbf{Actual Problem}\\
Given $n = 14$, find a set $S$ of $2n$ distinct positive integers
satisfying the following property: For every $m = 2, 3, \ldots, n$ the set $S$ can be partitioned into
two subsets with equal sums of elements, with one of the subsets being of cardinality $m$.

Output the answer as a comma separated list inside of $\boxed{...}$. For example $\boxed{1, 2, 3}$.

\textbf{Revised Problem}\\
For \(n = 14\), identify a collection \(S\) consisting of \(2n\) unique positive integers such that for every integer \(m\) ranging from 2 to \(n\), the set \(S\) can be divided into two groups with equal sums, and one of these groups has exactly \(m\) elements.

Present your solution as a series of numbers separated by commas within the format $\boxed{...}$. For instance, $\boxed{1, 2, 3}$.

\subsubsection{Variation}
\textbf{Actual Problem}\\
Given $n = 9$, find a set $S$ of $2n$ distinct positive integers
satisfying the following property: For every $m = 2, 3, \ldots, n$ the set $S$ can be partitioned into
two subsets with equal sums of elements, with one of the subsets being of cardinality $m$.

Output the answer as a comma separated list inside of $\boxed{...}$. For example $\boxed{1, 2, 3}$.

\textbf{Revised Problem}\\
For $n = 9$, identify a collection $S$ comprising $2n$ unique positive integers such that the following condition is met: For each $m$ ranging from 2 to $n$, it is possible to divide $S$ into two groups that have equal sums, where one of the groups contains exactly $m$ elements.

Present your solution as a list of numbers separated by commas within $\boxed{...}$. For example, $\boxed{1, 2, 3}$.

\subsubsection{Variation}
\textbf{Actual Problem}\\
Given $n = 12$, find a set $S$ of $2n$ distinct positive integers
satisfying the following property: For every $m = 2, 3, \ldots, n$ the set $S$ can be partitioned into
two subsets with equal sums of elements, with one of the subsets being of cardinality $m$.

Output the answer as a comma separated list inside of $\boxed{...}$. For example $\boxed{1, 2, 3}$.

\textbf{Revised Problem}\\
Given $n = 12$, identify a collection $S$ consisting of $2n$ distinct positive integers that fulfills the following condition: For every integer $m$ such that $m$ ranges from 2 to $n$, it is possible to divide $S$ into two groups where the sums of the integers in each group are equal, and one group contains exactly $m$ integers.

Present the solution as a sequence of numbers separated by commas within $\boxed{...}$. For instance, $\boxed{1, 2, 3}$.

\subsection{imo-shortlist-2020-a3}
\subsubsection{Variation}
\textbf{Actual Problem}\\
Find any positive real numbers a, b, c, d such that $(a + c)(b + d) = ac + bd$ and $\frac{a}{b} + \frac{b}{c} + \frac{c}{d} + \frac{d}{a} = 8$.

Output the answer as a comma separated list inside of $\boxed{...}$. For example $\boxed{1, 2, 3}$.

\textbf{Revised Problem}\\
Identify any positive real numbers \(a, b, c, d\) such that the equation \((a + c)(b + d) = ac + bd\) holds, and the expression \(\frac{a}{b} + \frac{b}{c} + \frac{c}{d} + \frac{d}{a}\) equals 8.

Present the solution as a list of numbers separated by commas within a box, like this: \(\boxed{1, 2, 3}\).

\subsection{imo-shortlist-2020-n1}
\subsubsection{Variation}
\textbf{Actual Problem}\\
Given a positive integer $k = 10$, find a prime number $p$ and distinct integers $a_1, a_2, \ldots, a_{k+3} \in \{1, 2, \ldots, p-1\}$ such that $p$ divides $a_ia_{i+1}a_{i+2}a_{i+3}-i$ for all $i = 1, 2, \ldots, k$.

Output the answer as a comma separated list $(p, a_1, a_2, \ldots, a_13)$ inside of $\boxed{...}$. For example $\boxed{17, 1, 2, 3, 4, 5}$.

\textbf{Revised Problem}\\
Given a positive integer \( k = 10 \), identify a prime number \( p \) and a sequence of distinct integers \( b_1, b_2, \ldots, b_{k+3} \) selected from the set \(\{1, 2, \ldots, p-1\}\) such that for each index \( i = 1, 2, \ldots, k \), the number \( b_i b_{i+1} b_{i+2} b_{i+3} - i \) is divisible by \( p \).

Present the solution as a comma-separated sequence \( (p, b_1, b_2, \ldots, b_{13}) \) enclosed within \(\boxed{...}\). For instance, \(\boxed{17, 1, 2, 3, 4, 5}\).

\subsubsection{Variation}
\textbf{Actual Problem}\\
Given a positive integer $k = 29$, find a prime number $p$ and distinct integers $a_1, a_2, \ldots, a_{k+3} \in \{1, 2, \ldots, p-1\}$ such that $p$ divides $a_ia_{i+1}a_{i+2}a_{i+3}-i$ for all $i = 1, 2, \ldots, k$.

Output the answer as a comma separated list $(p, a_1, a_2, \ldots, a_32)$ inside of $\boxed{...}$. For example $\boxed{17, 1, 2, 3, 4, 5}$.

\textbf{Revised Problem}\\
Let $k = 29$. Determine a prime number $p$ along with a set of distinct integers $a_1, a_2, \ldots, a_{k+3}$ such that each $a_i$ is within the set $\{1, 2, \ldots, p-1\}$, and for every index $i$ from 1 to $k$, the expression $a_ia_{i+1}a_{i+2}a_{i+3} - i$ is divisible by $p$.

Present the solution in the format of a comma-separated list $(p, a_1, a_2, \ldots, a_{32})$ enclosed within $\boxed{...}$. For instance, use the format $\boxed{17, 1, 2, 3, 4, 5}$.

\subsubsection{Variation}
\textbf{Actual Problem}\\
Given a positive integer $k = 13$, find a prime number $p$ and distinct integers $a_1, a_2, \ldots, a_{k+3} \in \{1, 2, \ldots, p-1\}$ such that $p$ divides $a_ia_{i+1}a_{i+2}a_{i+3}-i$ for all $i = 1, 2, \ldots, k$.

Output the answer as a comma separated list $(p, a_1, a_2, \ldots, a_16)$ inside of $\boxed{...}$. For example $\boxed{17, 1, 2, 3, 4, 5}$.

\textbf{Revised Problem}\\
Consider a positive integer \( k = 13 \). Identify a prime number \( p \) and a sequence of distinct integers \( a_1, a_2, \ldots, a_{k+3} \) from the set \( \{1, 2, \ldots, p-1\} \) such that for every \( i \) from 1 to \( k \), the expression \( a_i a_{i+1} a_{i+2} a_{i+3} - i \) is divisible by \( p \).

Present the answer as a list of numbers separated by commas: \((p, a_1, a_2, \ldots, a_{16})\), enclosed within \(\boxed{...}\). For instance, \(\boxed{17, 1, 2, 3, 4, 5}\).

\subsubsection{Variation}
\textbf{Actual Problem}\\
Given a positive integer $k = 8$, find a prime number $p$ and distinct integers $a_1, a_2, \ldots, a_{k+3} \in \{1, 2, \ldots, p-1\}$ such that $p$ divides $a_ia_{i+1}a_{i+2}a_{i+3}-i$ for all $i = 1, 2, \ldots, k$.

Output the answer as a comma separated list $(p, a_1, a_2, \ldots, a_11)$ inside of $\boxed{...}$. For example $\boxed{17, 1, 2, 3, 4, 5}$.

\textbf{Revised Problem}\\
For a given positive integer \( k = 8 \), identify a prime number \( p \) and unique integers \( a_1, a_2, \ldots, a_{k+3} \) chosen from the set \(\{1, 2, \ldots, p-1\}\) such that the expression \( a_ia_{i+1}a_{i+2}a_{i+3} - i \) is divisible by \( p \) for every integer \( i \) from 1 to \( k \).

Present your solution as a list in the format \((p, a_1, a_2, \ldots, a_{11})\) enclosed in \(\boxed{...}\). For example, \(\boxed{17, 1, 2, 3, 4, 5}\).

\subsection{imo-shortlist-2021-a3}
\subsubsection{Variation}
\textbf{Actual Problem}\\
Given a positive integer $n = 30$, find a permutation $(a_1,\dots,a_n)$ of $\{1,\dots,n\}$ such that the 
the value of 
$$ \sum_{k=1}^{n}\left\lfloor\frac{a_k}{k}\right\rfloor $$ 
equals to $1+\lfloor \log_2(n) \rfloor$.
Note that this is the smallest possible value that any permutation can achieve.


Output the answer as a comma separated list inside of $\boxed{...}$. For example $\boxed{1, 2, 3}$.

\textbf{Revised Problem}\\
Consider a positive integer $n = 30$. Determine a permutation $(a_1, a_2, \ldots, a_n)$ of the set $\{1, 2, \ldots, n\}$ such that the expression 
$$ \sum_{k=1}^{n}\left\lfloor\frac{a_k}{k}\right\rfloor $$ 
is equal to $1 + \lfloor \log_2(n) \rfloor$. Note that this represents the minimum possible value achievable by any permutation.

Present your solution as a sequence of numbers separated by commas within $\boxed{...}$. For instance, format it as $\boxed{1, 2, 3}$.

\subsubsection{Variation}
\textbf{Actual Problem}\\
Given a positive integer $n = 37$, find a permutation $(a_1,\dots,a_n)$ of $\{1,\dots,n\}$ such that the 
the value of 
$$ \sum_{k=1}^{n}\left\lfloor\frac{a_k}{k}\right\rfloor $$ 
equals to $1+\lfloor \log_2(n) \rfloor$.
Note that this is the smallest possible value that any permutation can achieve.


Output the answer as a comma separated list inside of $\boxed{...}$. For example $\boxed{1, 2, 3}$.

\textbf{Revised Problem}\\
Consider a positive integer \( n = 37 \). Your task is to determine a sequence \((a_1, a_2, \ldots, a_n)\) which is a permutation of the numbers \(\{1, 2, \ldots, n\}\), such that the sum
\[ \sum_{k=1}^{n}\left\lfloor\frac{a_k}{k}\right\rfloor \]
is equal to \(1 + \lfloor \log_2(n) \rfloor\).
It is important to note that this value is the minimum that can be achieved with any permutation.

Present the permutation as a list of numbers separated by commas, enclosed within $\boxed{...}$. For instance, write $\boxed{1, 2, 3}$.

\subsubsection{Variation}
\textbf{Actual Problem}\\
Given a positive integer $n = 29$, find a permutation $(a_1,\dots,a_n)$ of $\{1,\dots,n\}$ such that the 
the value of 
$$ \sum_{k=1}^{n}\left\lfloor\frac{a_k}{k}\right\rfloor $$ 
equals to $1+\lfloor \log_2(n) \rfloor$.
Note that this is the smallest possible value that any permutation can achieve.


Output the answer as a comma separated list inside of $\boxed{...}$. For example $\boxed{1, 2, 3}$.

\textbf{Revised Problem}\\
For a positive integer \( n = 29 \), identify a permutation \((a_1, a_2, \ldots, a_n)\) of the set \(\{1, 2, \ldots, n\}\) such that the sum
\[ \sum_{k=1}^{n} \left\lfloor \frac{a_k}{k} \right\rfloor \]
equals \(1 + \lfloor \log_2(n) \rfloor\).
This specified sum represents the minimum value achievable by any permutation.

Present your solution as a list separated by commas, enclosed within \(\boxed{...}\). For instance, \(\boxed{1, 2, 3}\).

\subsubsection{Variation}
\textbf{Actual Problem}\\
Given a positive integer $n = 26$, find a permutation $(a_1,\dots,a_n)$ of $\{1,\dots,n\}$ such that the 
the value of 
$$ \sum_{k=1}^{n}\left\lfloor\frac{a_k}{k}\right\rfloor $$ 
equals to $1+\lfloor \log_2(n) \rfloor$.
Note that this is the smallest possible value that any permutation can achieve.


Output the answer as a comma separated list inside of $\boxed{...}$. For example $\boxed{1, 2, 3}$.

\textbf{Revised Problem}\\
Consider a positive integer $n = 26$. Your task is to determine a permutation $(a_1, a_2, \ldots, a_n)$ of the set $\{1, 2, \ldots, n\}$ such that the sum
$$ \sum_{k=1}^{n} \left\lfloor \frac{a_k}{k} \right\rfloor $$
is precisely equal to $1 + \lfloor \log_2(n) \rfloor$. This specified sum represents the minimum possible value achievable by any permutation.

Present your result in the form of a list, separated by commas, and enclosed within $\boxed{...}$. For example, $\boxed{1, 2, 3}$.

\subsection{imo-shortlist-2022-a5}
\subsubsection{Variation}
\textbf{Actual Problem}\\
For $n = 4$, find n real numbers $a_1 < a_2 < \cdots < a_n$ and a real number $r > 0$ such that the $\frac{1}{2}n(n-1)$ differences $a_j - a_i$ for $1 \leq i < j \leq n$ are equal, in some order, to the numbers $r, r^2, \ldots, r^{\frac{1}{2}n(n-1)}$.

Output the sequence $(r, a_1, a_2, \ldots, a_n)$ as a comma-separated list of numbers inside of inside of $\boxed{...}$. For example $\boxed{1.5, 1, 2, 3}$.

\textbf{Revised Problem}\\
For $n = 4$, identify four real numbers $a_1 < a_2 < \cdots < a_4$ and a positive real number $r$ such that the differences $a_j - a_i$ for all pairs $1 \leq i < j \leq 4$ are identical, in some order, to the numbers $r, r^2, \ldots, r^6$.

Present the result as a sequence $(r, a_1, a_2, a_3, a_4)$ in a comma-separated format within $\boxed{...}$. For example, $\boxed{1.5, 1, 2, 3, 4}$.

\subsubsection{Variation}
\textbf{Actual Problem}\\
For $n = 3$, find n real numbers $a_1 < a_2 < \cdots < a_n$ and a real number $r > 0$ such that the $\frac{1}{2}n(n-1)$ differences $a_j - a_i$ for $1 \leq i < j \leq n$ are equal, in some order, to the numbers $r, r^2, \ldots, r^{\frac{1}{2}n(n-1)}$.

Output the sequence $(r, a_1, a_2, \ldots, a_n)$ as a comma-separated list of numbers inside of inside of $\boxed{...}$. For example $\boxed{1.5, 1, 2, 3}$.

\textbf{Revised Problem}\\
Determine three real numbers \( b_1 < b_2 < b_3 \) and a positive real number \( s \) such that the three differences \( b_j - b_i \) for \( i < j \) are, in some order, equal to \( s, s^2, \) and \( s^3 \).

Present the solution in the format \((s, b_1, b_2, b_3)\) as a comma-separated list of values enclosed in \(\boxed{...}\). For example, \(\boxed{1.5, 1, 2, 3}\).

\subsection{imo-shortlist-2022-c1}
\subsubsection{Variation}
\textbf{Actual Problem}\\
A $\pm 1$-sequence is a sequence of $62$ numbers $a_1, \ldots, a_{62}$ each equal to either $+1$ or $-1$. 
Given $C = 16$, find a $\pm 1$-sequence such that for any integer $k$ and indices $1 \le t_1 < \ldots < t_k \le 62$ so that $t_{i+1} - t_i \le 2$ for all $i$, it holds that $$\left| \sum_{i = 1}^{k} a_{t_i} \right| \leq C.$$
Note that one could prove that $C = 16$ is the smallest value of $C$ such that such sequence exists.


Output the answer as a comma separated list inside of $\boxed{...}$. For example $\boxed{1, 2, 3}$.

\textbf{Revised Problem}\\
Consider a sequence of 62 numbers, each either +1 or -1, called a $\pm 1$-sequence. Your task is to find such a sequence where for any integer $k$ and indices $1 \leq t_1 < \ldots < t_k \leq 62$, such that the difference between consecutive indices $t_{i+1} - t_i$ is at most 2 for all $i$, it satisfies the condition $$\left| \sum_{i = 1}^{k} a_{t_i} \right| \leq 16.$$ It can be shown that the smallest possible value of $C$ for which such a sequence exists is $C = 16$.

Present your answer as a sequence of numbers separated by commas within $\boxed{...}$. For instance, $\boxed{1, 2, 3}$.

\subsubsection{Variation}
\textbf{Actual Problem}\\
A $\pm 1$-sequence is a sequence of $90$ numbers $a_1, \ldots, a_{90}$ each equal to either $+1$ or $-1$. 
Given $C = 23$, find a $\pm 1$-sequence such that for any integer $k$ and indices $1 \le t_1 < \ldots < t_k \le 90$ so that $t_{i+1} - t_i \le 2$ for all $i$, it holds that $$\left| \sum_{i = 1}^{k} a_{t_i} \right| \leq C.$$
Note that one could prove that $C = 23$ is the smallest value of $C$ such that such sequence exists.


Output the answer as a comma separated list inside of $\boxed{...}$. For example $\boxed{1, 2, 3}$.

\textbf{Revised Problem}\\
A sequence of length 90 is composed of elements $a_1, a_2, \ldots, a_{90}$ where each element is either $+1$ or $-1$. Let $C = 23$. You are required to create a sequence such that for any integer $k$ and any set of indices $1 \le t_1 < t_2 < \cdots < t_k \le 90$ that satisfy $t_{i+1} - t_i \le 2$ for all $i$, the inequality $$\left| \sum_{i=1}^{k} a_{t_i} \right| \leq C$$ holds true. It is known that $C = 23$ is the minimum such value for which a sequence meeting these criteria exists.

Present the sequence as a list of numbers separated by commas within a box, like this: $\boxed{1, 2, 3}$.

\subsubsection{Variation}
\textbf{Actual Problem}\\
A $\pm 1$-sequence is a sequence of $58$ numbers $a_1, \ldots, a_{58}$ each equal to either $+1$ or $-1$. 
Given $C = 15$, find a $\pm 1$-sequence such that for any integer $k$ and indices $1 \le t_1 < \ldots < t_k \le 58$ so that $t_{i+1} - t_i \le 2$ for all $i$, it holds that $$\left| \sum_{i = 1}^{k} a_{t_i} \right| \leq C.$$
Note that one could prove that $C = 15$ is the smallest value of $C$ such that such sequence exists.


Output the answer as a comma separated list inside of $\boxed{...}$. For example $\boxed{1, 2, 3}$.

\textbf{Revised Problem}\\
A $\pm 1$-sequence consists of $58$ numbers $a_1, a_2, \ldots, a_{58}$, where each number is either $+1$ or $-1$. Determine a sequence where the constant $C = 15$ satisfies the condition that for any integer $k$ and indices $1 \le t_1 < t_2 < \ldots < t_k \le 58$ with $t_{i+1} - t_i \le 2$ for all $i$, the inequality $$\left| \sum_{i=1}^{k} a_{t_i} \right| \leq C$$ holds. It can be demonstrated that $C = 15$ is the minimum value for which such a sequence is possible.

Present the solution as a series of numbers separated by commas within $\boxed{...}$. For instance, $\boxed{1, 2, 3}$.

\subsubsection{Variation}
\textbf{Actual Problem}\\
A $\pm 1$-sequence is a sequence of $46$ numbers $a_1, \ldots, a_{46}$ each equal to either $+1$ or $-1$. 
Given $C = 12$, find a $\pm 1$-sequence such that for any integer $k$ and indices $1 \le t_1 < \ldots < t_k \le 46$ so that $t_{i+1} - t_i \le 2$ for all $i$, it holds that $$\left| \sum_{i = 1}^{k} a_{t_i} \right| \leq C.$$
Note that one could prove that $C = 12$ is the smallest value of $C$ such that such sequence exists.


Output the answer as a comma separated list inside of $\boxed{...}$. For example $\boxed{1, 2, 3}$.

\textbf{Revised Problem}\\
Consider a sequence consisting of 46 elements, each of which is either +1 or -1. Define this sequence as $b_1, b_2, \ldots, b_{46}$. Given that $D = 12$, determine a sequence where, for any integer $m$ and any selected indices $1 \le s_1 < \dots < s_m \le 46$ such that $s_{j+1} - s_j \le 2$ for every $j$, the following condition is satisfied:
$$\left| \sum_{j=1}^{m} b_{s_j} \right| \leq D.$$
It is known that $D = 12$ is the minimum value making such a sequence possible.

Present the sequence as a list of numbers separated by commas, enclosed in $\boxed{...}$. For instance, $\boxed{1, 2, 3}$.

\subsection{imo-shortlist-2022-c8}
\subsubsection{Variation}
\textbf{Actual Problem}\\
Let $n$ be a positive integer. A Nordic square is an $n \times n$ board containing all the integers from $1$ to $n^2$ so that each cell contains exactly one number. Two different cells are considered adjacent if they share a common side. Every cell that is adjacent only to cells containing larger numbers is called a valley. An uphill path is a sequence of one or more cells such that:

(i) the first cell in the sequence is a valley,
(ii) each subsequent cell in the sequence is adjacent to the previous cell, and
(iii) the numbers written in the cells in the sequence are in increasing order.

Given $n = 19$ find a Nordic square with $685$ uphill paths (one can show that this is the minimum number).


Output the answer between \verb|\begin{array}{...}| and \verb|\end{array}| inside of $\boxed{...}$. For example, $\boxed{\begin{array}{ccc}1 & 2 & 3 \\ 4 & 5 & 6 \\ 7 & 8 & 9\end{array}}$.

\textbf{Revised Problem}\\
Consider $n$ as a positive integer. A Nordic square refers to an $n \times n$ grid that contains all integers from $1$ to $n^2$, placing exactly one number in each cell. Two different cells are deemed adjacent if they share a common edge. A cell is termed a valley if it is adjacent only to cells with larger numbers. An uphill path consists of one or more cells satisfying the following conditions:

(i) The initial cell in the path is a valley,
(ii) Each subsequent cell in the path is adjacent to the one before it, and
(iii) The numbers in the path's cells are in ascending order.

For $n = 19$, determine a Nordic square that results in $685$ uphill paths, which is the minimum number achievable.

Present the solution using \verb|\begin{array}{...}| and \verb|\end{array}| within $\boxed{...}$. For instance, like this: $\boxed{\begin{array}{ccc}1 & 2 & 3 \\ 4 & 5 & 6 \\ 7 & 8 & 9\end{array}}$.

\subsubsection{Variation}
\textbf{Actual Problem}\\
Let $n$ be a positive integer. A Nordic square is an $n \times n$ board containing all the integers from $1$ to $n^2$ so that each cell contains exactly one number. Two different cells are considered adjacent if they share a common side. Every cell that is adjacent only to cells containing larger numbers is called a valley. An uphill path is a sequence of one or more cells such that:

(i) the first cell in the sequence is a valley,
(ii) each subsequent cell in the sequence is adjacent to the previous cell, and
(iii) the numbers written in the cells in the sequence are in increasing order.

Given $n = 22$ find a Nordic square with $925$ uphill paths (one can show that this is the minimum number).


Output the answer between \verb|\begin{array}{...}| and \verb|\end{array}| inside of $\boxed{...}$. For example, $\boxed{\begin{array}{ccc}1 & 2 & 3 \\ 4 & 5 & 6 \\ 7 & 8 & 9\end{array}}$.

\textbf{Revised Problem}\\
Consider a positive integer \( n \). A Nordic square is defined as an \( n \times n \) grid filled with integers from \( 1 \) to \( n^2 \), ensuring each number appears exactly once. In this grid, two distinct cells are deemed adjacent if they share a side. A cell is classified as a valley if all adjacent cells contain numbers greater than the number in this cell. Define an uphill path as a sequence of one or more cells where:

(i) the initial cell is a valley,
(ii) each following cell is adjacent to the one before it, and
(iii) the sequence of numbers in these cells strictly increases.

For \( n = 22 \), determine a Nordic square that results in exactly \( 925 \) uphill paths (proven to be the least possible number).

Present your solution using the format \verb|\begin{array}{...}| and \verb|\end{array}| enclosed within $\boxed{...}$. For instance, $\boxed{\begin{array}{ccc}1 & 2 & 3 \\ 4 & 5 & 6 \\ 7 & 8 & 9\end{array}}$.

\subsubsection{Variation}
\textbf{Actual Problem}\\
Let $n$ be a positive integer. A Nordic square is an $n \times n$ board containing all the integers from $1$ to $n^2$ so that each cell contains exactly one number. Two different cells are considered adjacent if they share a common side. Every cell that is adjacent only to cells containing larger numbers is called a valley. An uphill path is a sequence of one or more cells such that:

(i) the first cell in the sequence is a valley,
(ii) each subsequent cell in the sequence is adjacent to the previous cell, and
(iii) the numbers written in the cells in the sequence are in increasing order.

Given $n = 14$ find a Nordic square with $365$ uphill paths (one can show that this is the minimum number).


Output the answer between \verb|\begin{array}{...}| and \verb|\end{array}| inside of $\boxed{...}$. For example, $\boxed{\begin{array}{ccc}1 & 2 & 3 \\ 4 & 5 & 6 \\ 7 & 8 & 9\end{array}}$.

\textbf{Revised Problem}\\
Consider a positive integer \( n \). A Nordic square is defined as an \( n \times n \) grid that includes every integer from 1 to \( n^2 \), with each number appearing exactly once in a distinct cell. Two cells are said to be adjacent if they share a side. A cell is called a valley if all of its adjacent cells contain higher numbers. An uphill path is defined as a sequence of cells meeting the following conditions:

(i) The starting cell of the sequence is a valley,
(ii) Each next cell in the sequence must be adjacent to the prior cell,
(iii) The sequence's numbers must be in strictly increasing order.

For \( n = 14 \), determine a configuration of a Nordic square that results in exactly 365 uphill paths, which is known to be the minimum possible.

Present the solution within \verb|\begin{array}{...}| and \verb|\end{array}| wrapped in $\boxed{...}$. For instance, $\boxed{\begin{array}{ccc}1 & 2 & 3 \\ 4 & 5 & 6 \\ 7 & 8 & 9\end{array}}$.

\subsubsection{Variation}
\textbf{Actual Problem}\\
Let $n$ be a positive integer. A Nordic square is an $n \times n$ board containing all the integers from $1$ to $n^2$ so that each cell contains exactly one number. Two different cells are considered adjacent if they share a common side. Every cell that is adjacent only to cells containing larger numbers is called a valley. An uphill path is a sequence of one or more cells such that:

(i) the first cell in the sequence is a valley,
(ii) each subsequent cell in the sequence is adjacent to the previous cell, and
(iii) the numbers written in the cells in the sequence are in increasing order.

Given $n = 11$ find a Nordic square with $221$ uphill paths (one can show that this is the minimum number).


Output the answer between \verb|\begin{array}{...}| and \verb|\end{array}| inside of $\boxed{...}$. For example, $\boxed{\begin{array}{ccc}1 & 2 & 3 \\ 4 & 5 & 6 \\ 7 & 8 & 9\end{array}}$.

\textbf{Revised Problem}\\
Consider a positive integer $n$. A Nordic square is defined as an $n \times n$ grid filled with the integers from $1$ to $n^2$, where each cell contains a unique number. Two cells are considered adjacent if they are next to each other horizontally or vertically. A cell is called a valley if it is adjacent only to cells with larger numbers. An uphill path is a sequence of one or more cells that satisfies the following conditions:

(i) It begins with a valley,
(ii) Each subsequent cell in the path is adjacent to the previous one, and
(iii) The numbers in the sequence appear in strictly increasing order.

For $n = 11$, find a Nordic square such that the number of uphill paths is 221, which is the smallest possible number.

Present the solution enclosed within \verb|\begin{array}{...}| and \verb|\end{array}| inside of $\boxed{...}$. For example, it should be formatted as $\boxed{\begin{array}{ccc}1 & 2 & 3 \\ 4 & 5 & 6 \\ 7 & 8 & 9\end{array}}$.

\subsection{imo-shortlist-2023-a5}
\subsubsection{Variation}
\textbf{Actual Problem}\\
Let $n = 59$ and $a_1, a_2, \ldots, a_n$ be positive integers such that
(i) $a_1, a_2, \ldots, a_n$ is a permutation of $1, 2, \ldots, n$, and
(ii) $|a_1 - a_2|, |a_2 - a_3|, \ldots, |a_{n-1} - a_{n}|$ is a permutation of $1, 2, \ldots, {n-1}$.
Find one sequence such that $\max(a_1, a_n) \leq 16$.

Output the answer as a comma separated list inside of $\boxed{...}$. For example $\boxed{1, 2, 3}$.

\textbf{Revised Problem}\\
Consider $n = 59$ and let $a_1, a_2, \ldots, a_n$ represent positive integers such that:
(i) The sequence $a_1, a_2, \ldots, a_n$ is a rearrangement of the integers $1, 2, \ldots, n$.
(ii) The sequence of absolute differences $|a_1 - a_2|, |a_2 - a_3|, \ldots, |a_{n-1} - a_{n}|$ is a rearrangement of the integers $1, 2, \ldots, n-1$.
Identify one such sequence where the maximum of $a_1$ and $a_n$ is at most 16.

Present the solution as a list of numbers separated by commas within $\boxed{...}$. For example, $\boxed{1, 2, 3}$.

\subsubsection{Variation}
\textbf{Actual Problem}\\
Let $n = 431$ and $a_1, a_2, \ldots, a_n$ be positive integers such that
(i) $a_1, a_2, \ldots, a_n$ is a permutation of $1, 2, \ldots, n$, and
(ii) $|a_1 - a_2|, |a_2 - a_3|, \ldots, |a_{n-1} - a_{n}|$ is a permutation of $1, 2, \ldots, {n-1}$.
Find one sequence such that $\max(a_1, a_n) \leq 109$.

Output the answer as a comma separated list inside of $\boxed{...}$. For example $\boxed{1, 2, 3}$.

\textbf{Revised Problem}\\
Consider $n = 431$ and $b_1, b_2, \ldots, b_n$ as a set of positive integers meeting the conditions:
(i) The sequence $b_1, b_2, \ldots, b_n$ is a rearrangement of the numbers $1, 2, \ldots, n$, and
(ii) The values $|b_1 - b_2|, |b_2 - b_3|, \ldots, |b_{n-1} - b_n|$ are a rearrangement of $1, 2, \ldots, n-1$.
Find a sequence such that $\max(b_1, b_n) \leq 109$.

Present your solution as a list separated by commas within $\boxed{...}$. For instance, $\boxed{1, 2, 3}$.

\subsubsection{Variation}
\textbf{Actual Problem}\\
Let $n = 175$ and $a_1, a_2, \ldots, a_n$ be positive integers such that
(i) $a_1, a_2, \ldots, a_n$ is a permutation of $1, 2, \ldots, n$, and
(ii) $|a_1 - a_2|, |a_2 - a_3|, \ldots, |a_{n-1} - a_{n}|$ is a permutation of $1, 2, \ldots, {n-1}$.
Find one sequence such that $\max(a_1, a_n) \leq 45$.

Output the answer as a comma separated list inside of $\boxed{...}$. For example $\boxed{1, 2, 3}$.

\textbf{Revised Problem}\\
Consider $n = 175$ and let $b_1, b_2, \ldots, b_n$ be a series of positive integers such that:
(i) $b_1, b_2, \ldots, b_n$ is a rearrangement of the set $1, 2, \ldots, n$, and
(ii) The sequence $|b_1 - b_2|, |b_2 - b_3|, \ldots, |b_{n-1} - b_{n}|$ forms a rearrangement of the numbers $1, 2, \ldots, n-1$.
Find at least one sequence where the largest element among $b_1$ and $b_n$ does not exceed 45.

Present your solution as a list of numbers separated by commas within $\boxed{...}$. For instance, format it like $\boxed{1, 2, 3}$.

\subsubsection{Variation}
\textbf{Actual Problem}\\
Let $n = 95$ and $a_1, a_2, \ldots, a_n$ be positive integers such that
(i) $a_1, a_2, \ldots, a_n$ is a permutation of $1, 2, \ldots, n$, and
(ii) $|a_1 - a_2|, |a_2 - a_3|, \ldots, |a_{n-1} - a_{n}|$ is a permutation of $1, 2, \ldots, {n-1}$.
Find one sequence such that $\max(a_1, a_n) \leq 25$.

Output the answer as a comma separated list inside of $\boxed{...}$. For example $\boxed{1, 2, 3}$.

\textbf{Revised Problem}\\
Consider $n = 95$ and let $b_1, b_2, \ldots, b_n$ represent positive integers such that:
(i) The sequence $b_1, b_2, \ldots, b_n$ is a rearrangement of the numbers $1, 2, \ldots, n$, and
(ii) The differences $|b_1 - b_2|, |b_2 - b_3|, \ldots, |b_{n-1} - b_{n}|$ form a rearrangement of $1, 2, \ldots, n-1$.
Identify a sequence where the highest value between $b_1$ and $b_n$ does not exceed 25.

Present your answer as a sequence in a comma-separated format within $\boxed{...}$. For instance, $\boxed{1, 2, 3}$.

\subsection{imo-shortlist-2023-c2}
\subsubsection{Variation}
\textbf{Actual Problem}\\
One can show that the maximal length $L$ of a sequence $a_1, a_2, \ldots, a_L$ of positive integers satisfying both:

(i) every term in the sequence is less than or equal to $2^5$, and

(ii) there does not exist a consecutive subsequence $a_i, a_{i+1}, \ldots, a_{j}$ (where $1 \leq i \leq j \leq L$) with
a choice of signs $s_i, s_{i+1}, \ldots, s_{j} \in \{1, -1\}$ such that $s_i a_i + s_{i+1} a_{i+1} + \cdots + s_{j} a_{j} = 0$,

is $L = 63$.

Find one such sequence $a_1, a_2, \ldots, a_L$.

Output the answer as a comma separated list inside of $\boxed{...}$. For example $\boxed{1, 2, 3}$.

\textbf{Revised Problem}\\
Determine the longest possible sequence of positive integers \( b_1, b_2, \ldots, b_M \) such that:

(i) each number in the sequence is at most \( 32 \), and

(ii) for every consecutive subsequence \( b_k, b_{k+1}, \ldots, b_{n} \) (where \( 1 \leq k \leq n \leq M \)), there is no selection of signs \( t_k, t_{k+1}, \ldots, t_{n} \in \{1, -1\} \) that makes \( t_k b_k + t_{k+1} b_{k+1} + \cdots + t_{n} b_{n} = 0 \).

It is known that \( M = 63 \) for such a sequence.

Provide an example of a sequence \( b_1, b_2, \ldots, b_M \) that meets these criteria.

Present the solution as a list separated by commas within a box, like this: \(\boxed{1, 2, 3}\).

\subsubsection{Variation}
\textbf{Actual Problem}\\
One can show that the maximal length $L$ of a sequence $a_1, a_2, \ldots, a_L$ of positive integers satisfying both:

(i) every term in the sequence is less than or equal to $2^6$, and

(ii) there does not exist a consecutive subsequence $a_i, a_{i+1}, \ldots, a_{j}$ (where $1 \leq i \leq j \leq L$) with
a choice of signs $s_i, s_{i+1}, \ldots, s_{j} \in \{1, -1\}$ such that $s_i a_i + s_{i+1} a_{i+1} + \cdots + s_{j} a_{j} = 0$,

is $L = 127$.

Find one such sequence $a_1, a_2, \ldots, a_L$.

Output the answer as a comma separated list inside of $\boxed{...}$. For example $\boxed{1, 2, 3}$.

\textbf{Revised Problem}\\
Determine the longest possible sequence of positive integers, \( b_1, b_2, \ldots, b_M \), such that:

(i) Each integer in the sequence is at most \(2^6\), and

(ii) For any consecutive subsequence \( b_k, b_{k+1}, \ldots, b_l \) (where \(1 \leq k \leq l \leq M\)), there is no selection of signs \( t_k, t_{k+1}, \ldots, t_l \in \{1, -1\} \) that makes \( t_k b_k + t_{k+1} b_{k+1} + \cdots + t_l b_l = 0 \).

It is known that such a sequence can have a maximum length of \( M = 127 \).

Construct one such sequence \( b_1, b_2, \ldots, b_M \).

Present your solution as a list separated by commas, enclosed within $\boxed{...}$. For instance, $\boxed{1, 2, 3}$.

\subsubsection{Variation}
\textbf{Actual Problem}\\
One can show that the maximal length $L$ of a sequence $a_1, a_2, \ldots, a_L$ of positive integers satisfying both:

(i) every term in the sequence is less than or equal to $2^4$, and

(ii) there does not exist a consecutive subsequence $a_i, a_{i+1}, \ldots, a_{j}$ (where $1 \leq i \leq j \leq L$) with
a choice of signs $s_i, s_{i+1}, \ldots, s_{j} \in \{1, -1\}$ such that $s_i a_i + s_{i+1} a_{i+1} + \cdots + s_{j} a_{j} = 0$,

is $L = 31$.

Find one such sequence $a_1, a_2, \ldots, a_L$.

Output the answer as a comma separated list inside of $\boxed{...}$. For example $\boxed{1, 2, 3}$.

\textbf{Revised Problem}\\
Determine the longest possible sequence \( a_1, a_2, \ldots, a_L \) of positive integers where:

(i) each term in this sequence is no greater than \( 16 \), and

(ii) there is no consecutive subsequence \( a_i, a_{i+1}, \ldots, a_j \) (for \( 1 \leq i \leq j \leq L \)) for which there exists a set of signs \( s_i, s_{i+1}, \ldots, s_j \) where each \( s \) is either \( 1 \) or \( -1 \), such that the sum \( s_i a_i + s_{i+1} a_{i+1} + \cdots + s_j a_j = 0 \).

It is established that the maximum length \( L \) for such a sequence is 31. Provide one example of such a sequence \( a_1, a_2, \ldots, a_L \).

Present the sequence as a list separated by commas, enclosed within a box like this: \(\boxed{...}\). For instance, \(\boxed{1, 2, 3}\).

\subsubsection{Variation}
\textbf{Actual Problem}\\
One can show that the maximal length $L$ of a sequence $a_1, a_2, \ldots, a_L$ of positive integers satisfying both:

(i) every term in the sequence is less than or equal to $2^3$, and

(ii) there does not exist a consecutive subsequence $a_i, a_{i+1}, \ldots, a_{j}$ (where $1 \leq i \leq j \leq L$) with
a choice of signs $s_i, s_{i+1}, \ldots, s_{j} \in \{1, -1\}$ such that $s_i a_i + s_{i+1} a_{i+1} + \cdots + s_{j} a_{j} = 0$,

is $L = 15$.

Find one such sequence $a_1, a_2, \ldots, a_L$.

Output the answer as a comma separated list inside of $\boxed{...}$. For example $\boxed{1, 2, 3}$.

\textbf{Revised Problem}\\
Consider the task of determining the longest possible sequence of positive integers \(b_1, b_2, \ldots, b_M\) such that:

(i) every element in the sequence is no greater than \(8\), and

(ii) there is no consecutive subsequence \(b_k, b_{k+1}, \ldots, b_m\) (where \(1 \leq k \leq m \leq M\)) for which you can choose signs \(t_k, t_{k+1}, \ldots, t_m \in \{1, -1\}\) such that the sum \(t_k b_k + t_{k+1} b_{k+1} + \cdots + t_m b_m\) equals zero.

The maximum value of \(M\) for which such a sequence exists is \(M = 15\).

Provide an example of such a sequence \(b_1, b_2, \ldots, b_M\).

Present the solution as a sequence of numbers separated by commas, enclosed within a box. For instance, \(\boxed{1, 2, 3}\).

\section{jbmo}
\subsection{jbmo-shortlist-2008-c1}
\subsubsection{Variation}
\textbf{Actual Problem}\\
On a $5 \times 5$ board, $n$ white markers are positioned, each marker in a distinct $1 \times 1$ square. A smart child got an assignment to recolor in black as many markers as possible, in the following manner: a white marker is taken from the board, it is colored in black, and then put back on the board on an empty square such that none of the neighboring squares contains a white marker (two squares are called neighboring if they share a common side).
If it is possible for the child to succeed in coloring all the markers black, we say that the initial positioning of the markers was good. Exhibit a good initial positioning with the maximum of $20$ white markers, and the sequences of moves to color them all black.

As your output, present a tuple of 2 lists. The first list should be a list of 1-indexed coordinates for where the white markers were initially placed. The second one should contain a sequence of quadruples in the form $(a,b,c,d)$, which represent a coloring of the marker at coordinates $(a,b)$, which is then placed at a valid position $(c,d)$. Output the answer as 2 comma-separated lists of comma-separated tuples in in $\boxed{...}$. For example, for a $3\times 3$ board: $\boxed{([(1, 1), (1,2)], [(1,1,3,3), (1,2,1,1)])}$

\textbf{Revised Problem}\\
Consider a $5 \times 5$ grid with $n$ white markers, each placed in a separate $1 \times 1$ cell. A child has the challenge of recoloring as many markers as possible black, following these rules: a white marker is removed from the grid, painted black, and then placed back onto an empty cell so that no adjacent cells have a white marker (adjacent cells share a side). If the child can successfully recolor all markers black, the starting placement of markers is deemed successful. Identify a successful starting arrangement that uses the maximum number of 20 white markers, along with the steps taken to recolor them all black.

Present your solution as a tuple consisting of two lists. The first list should contain 1-indexed coordinates representing the initial positions of the white markers. The second list should describe each move as a quadruple $(a,b,c,d)$, indicating the recoloring of the marker from $(a,b)$ and its placement at $(c,d)$. Format the answer as two lists of tuples separated by commas within $\boxed{...}$. For instance, for a $3\times 3$ grid, the output may look like $\boxed{([(1, 1), (1,2)], [(1,1,3,3), (1,2,1,1)])}$.

\subsubsection{Variation}
\textbf{Actual Problem}\\
On a $8 \times 8$ board, $n$ white markers are positioned, each marker in a distinct $1 \times 1$ square. A smart child got an assignment to recolor in black as many markers as possible, in the following manner: a white marker is taken from the board, it is colored in black, and then put back on the board on an empty square such that none of the neighboring squares contains a white marker (two squares are called neighboring if they share a common side).
If it is possible for the child to succeed in coloring all the markers black, we say that the initial positioning of the markers was good. Exhibit a good initial positioning with the maximum of $56$ white markers, and the sequences of moves to color them all black.

As your output, present a tuple of 2 lists. The first list should be a list of 1-indexed coordinates for where the white markers were initially placed. The second one should contain a sequence of quadruples in the form $(a,b,c,d)$, which represent a coloring of the marker at coordinates $(a,b)$, which is then placed at a valid position $(c,d)$. Output the answer as 2 comma-separated lists of comma-separated tuples in in $\boxed{...}$. For example, for a $3\times 3$ board: $\boxed{([(1, 1), (1,2)], [(1,1,3,3), (1,2,1,1)])}$

\textbf{Revised Problem}\\
Consider a chessboard of dimension $8 \times 8$, where $n$ white markers are initially placed, each occupying a unique $1 \times 1$ square. A clever child is tasked with changing as many markers as possible to black using the following rule: a white marker is removed from its position, painted black, and then placed back onto a vacant square such that no adjacent square (sharing an edge) contains a white marker. If the child can manage to transform all markers to black, the starting arrangement of the markers is deemed valid. Provide an example of a valid starting arrangement with the maximum 56 white markers and the sequence of moves required to turn them all black.

Present your output as a tuple containing two lists. The first list should specify the 1-indexed coordinates where the white markers were originally placed. The second list should detail a series of quadruples in the format $(a,b,c,d)$, indicating that the marker at position $(a,b)$ is painted and then relocated to the position $(c,d)$. Format the output as two lists of tuples, separated by commas, enclosed in $\boxed{...}$. For instance, for a $3\times 3$ board: $\boxed{([(1, 1), (1,2)], [(1,1,3,3), (1,2,1,1)])}$

\subsubsection{Variation}
\textbf{Actual Problem}\\
On a $6 \times 6$ board, $n$ white markers are positioned, each marker in a distinct $1 \times 1$ square. A smart child got an assignment to recolor in black as many markers as possible, in the following manner: a white marker is taken from the board, it is colored in black, and then put back on the board on an empty square such that none of the neighboring squares contains a white marker (two squares are called neighboring if they share a common side).
If it is possible for the child to succeed in coloring all the markers black, we say that the initial positioning of the markers was good. Exhibit a good initial positioning with the maximum of $30$ white markers, and the sequences of moves to color them all black.

As your output, present a tuple of 2 lists. The first list should be a list of 1-indexed coordinates for where the white markers were initially placed. The second one should contain a sequence of quadruples in the form $(a,b,c,d)$, which represent a coloring of the marker at coordinates $(a,b)$, which is then placed at a valid position $(c,d)$. Output the answer as 2 comma-separated lists of comma-separated tuples in in $\boxed{...}$. For example, for a $3\times 3$ board: $\boxed{([(1, 1), (1,2)], [(1,1,3,3), (1,2,1,1)])}$

\textbf{Revised Problem}\\
Consider a $6 \times 6$ grid where $n$ white markers are placed, each occupying its own distinct $1 \times 1$ cell. A clever child has been tasked with the challenge of turning as many markers black as possible. The process involves picking up a white marker, painting it black, and then placing it back on the grid in an empty cell, such that no adjacent cells (cells sharing a side) contain a white marker. If the child can successfully convert all white markers to black under these conditions, the initial setup of the markers is deemed effective. Present an effective initial setup featuring a maximum of $30$ white markers, along with the sequence of steps required to convert all the markers to black.

Your output should be a tuple containing two lists. The first list must include 1-indexed coordinates indicating the initial placement of the white markers. The second list should consist of a series of quadruples formatted as $(a,b,c,d)$, indicating that the marker initially at position $(a,b)$ is painted black and then moved to a new valid position $(c,d)$. Present your solution in two comma-separated lists of comma-separated tuples inside $\boxed{...}$. For instance, for a $3\times 3$ grid: $\boxed{([(1, 1), (1,2)], [(1,1,3,3), (1,2,1,1)])}$

\subsubsection{Variation}
\textbf{Actual Problem}\\
On a $9 \times 9$ board, $n$ white markers are positioned, each marker in a distinct $1 \times 1$ square. A smart child got an assignment to recolor in black as many markers as possible, in the following manner: a white marker is taken from the board, it is colored in black, and then put back on the board on an empty square such that none of the neighboring squares contains a white marker (two squares are called neighboring if they share a common side).
If it is possible for the child to succeed in coloring all the markers black, we say that the initial positioning of the markers was good. Exhibit a good initial positioning with the maximum of $72$ white markers, and the sequences of moves to color them all black.

As your output, present a tuple of 2 lists. The first list should be a list of 1-indexed coordinates for where the white markers were initially placed. The second one should contain a sequence of quadruples in the form $(a,b,c,d)$, which represent a coloring of the marker at coordinates $(a,b)$, which is then placed at a valid position $(c,d)$. Output the answer as 2 comma-separated lists of comma-separated tuples in in $\boxed{...}$. For example, for a $3\times 3$ board: $\boxed{([(1, 1), (1,2)], [(1,1,3,3), (1,2,1,1)])}$

\textbf{Revised Problem}\\
Consider a $9 \times 9$ board with $n$ white markers, each occupying a unique $1 \times 1$ cell. A clever child must complete a task to convert as many markers to black as possible, adhering to the following guideline: a white marker is removed from the board, recolored black, and repositioned on an unoccupied cell such that no adjacent cell contains a white marker (two cells are adjacent if they share an edge). If the child can successfully repaint all the markers black, we define the initial placement of the markers as optimal. Identify an optimal starting arrangement with a maximum of 72 white markers, and outline the moves that transform them all to black.

Your output should be a tuple containing two lists. The first list should hold the 1-based coordinates of the initial positions of the white markers. The second list should consist of sequences of four numbers formatted as $(a,b,c,d)$, where a marker at position $(a,b)$ is recolored and repositioned at $(c,d)$. Present the output as two lists of tuples, separated by commas and enclosed within $\boxed{...}$. For example, for a $3\times 3$ board: $\boxed{([(1, 1), (1,2)], [(1,1,3,3), (1,2,1,1)])}$

\subsection{jbmo-shortlist-2016-c2}
\subsubsection{Variation}
\textbf{Actual Problem}\\
The natural numbers from $1$ to $50$ are written down on the blackboard. Find which 25 numbers should be erased off the board, in order for the sum of any two of the remaining numbers to not be a prime.

Output the answer as a comma separated list inside of $\boxed{...}$. For example $\boxed{1, 2, 3}$.

\textbf{Revised Problem}\\
On a blackboard, all natural numbers from 1 to 50 are displayed. Identify the 25 numbers that should be removed so that for any two numbers left on the board, their sum is never a prime number.

Present your answer as a list of numbers separated by commas within $\boxed{...}$. For instance, $\boxed{1, 2, 3}$.

\subsection{jbmo-shortlist-2018-a7}
\subsubsection{Variation}
\textbf{Actual Problem}\\
Let $A$ be a set of positive integers satisfying the following:

$a.)$ If $n \in A$, then $n \le 2018$.

$b.)$ If $S \subset A$ such that $|S|=3$, then there exists $m,n \in S$ such that $|n-m| \ge \sqrt{n}+\sqrt{m}$.

Give a construction for $A$ with cardinality $\sqrt{N}=44$.

Output the answer as a comma separated list inside of $\boxed{...}$. For example $\boxed{1, 2, 3}$.

\textbf{Revised Problem}\\
Consider a set \( A \) of positive integers with the following properties:

a.) Any integer \( n \) in \( A \) satisfies \( n \leq 2018 \).

b.) For every subset \( S \) of \( A \) with exactly 3 elements, there are two elements \( m \) and \( n \) in \( S \) such that the condition \( |n - m| \geq \sqrt{n} + \sqrt{m} \) holds.

Construct a set \( A \) such that the number of elements in \( A \) is 44, given that \( \sqrt{N} = 44 \).

Present your solution as a sequence of numbers separated by commas within the notation \(\boxed{...}\). For example, \(\boxed{1, 2, 3}\).

\subsubsection{Variation}
\textbf{Actual Problem}\\
Let $A$ be a set of positive integers satisfying the following:

$a.)$ If $n \in A$, then $n \le 168$.

$b.)$ If $S \subset A$ such that $|S|=3$, then there exists $m,n \in S$ such that $|n-m| \ge \sqrt{n}+\sqrt{m}$.

Give a construction for $A$ with cardinality $\sqrt{N}=12$.

Output the answer as a comma separated list inside of $\boxed{...}$. For example $\boxed{1, 2, 3}$.

\textbf{Revised Problem}\\
Consider a set \( A \) consisting of positive integers that adhere to the following requirements:

a.) For every element \( n \) in \( A \), it holds that \( n \le 168 \).

b.) For any subset \( S \) of \( A \) with exactly three elements, there exist two elements \( m \) and \( n \) in \( S \) such that the inequality \( |n-m| \ge \sqrt{n} + \sqrt{m} \) is satisfied.

Construct a set \( A \) such that the number of elements in \( A \) is 144.

Present your solution as a list of numbers separated by commas, enclosed within \(\boxed{...}\). For example, \(\boxed{1, 2, 3}\).

\subsubsection{Variation}
\textbf{Actual Problem}\\
Let $A$ be a set of positive integers satisfying the following:

$a.)$ If $n \in A$, then $n \le 80$.

$b.)$ If $S \subset A$ such that $|S|=3$, then there exists $m,n \in S$ such that $|n-m| \ge \sqrt{n}+\sqrt{m}$.

Give a construction for $A$ with cardinality $\sqrt{N}=8$.

Output the answer as a comma separated list inside of $\boxed{...}$. For example $\boxed{1, 2, 3}$.

\textbf{Revised Problem}\\
Consider a set \( A \) of positive integers defined by these conditions:

a.) Every element \( n \) in \( A \) satisfies \( n \le 80 \).

b.) For any subset \( S \subset A \) with exactly 3 elements, there exist two numbers \( m, n \in S \) such that \( |n-m| \ge \sqrt{n} + \sqrt{m} \).

Construct a set \( A \) such that its size is \( \sqrt{N} = 8 \).

Present the solution as a list of numbers separated by commas and enclosed in \(\boxed{...}\). For instance, \(\boxed{1, 2, 3}\).

\subsubsection{Variation}
\textbf{Actual Problem}\\
Let $A$ be a set of positive integers satisfying the following:

$a.)$ If $n \in A$, then $n \le 48$.

$b.)$ If $S \subset A$ such that $|S|=3$, then there exists $m,n \in S$ such that $|n-m| \ge \sqrt{n}+\sqrt{m}$.

Give a construction for $A$ with cardinality $\sqrt{N}=6$.

Output the answer as a comma separated list inside of $\boxed{...}$. For example $\boxed{1, 2, 3}$.

\textbf{Revised Problem}\\
Consider a set \( B \) of positive integers fulfilling the conditions below:

a.) If \( x \) is an element of \( B \), then \( x \leq 48 \).

b.) For any subset \( T \subset B \) with exactly three elements, there are elements \( p, q \in T \) such that \( |p-q| \geq \sqrt{p} + \sqrt{q} \).

Construct a set \( B \) such that its size is \( \sqrt{M} = 6 \).

Present your solution as a list of numbers separated by commas, enclosed in \(\boxed{...}\). For instance, \(\boxed{1, 2, 3}\).

\subsection{jbmo-shortlist-2018-n4}
\subsubsection{Variation}
\textbf{Actual Problem}\\
There are infinitely many distinct positive integers $n$ such that $\frac{4^n+2^n+1}{n^2 + n + 1}$ is a positive integer. Give examples of $7$ such numbers.

Output the answer as a comma separated list inside of $\boxed{...}$. For example $\boxed{1, 2, 3}$.

\textbf{Revised Problem}\\
Identify infinitely many distinct positive integers \( n \) for which \(\frac{4^n + 2^n + 1}{n^2 + n + 1}\) results in a positive integer. Provide 7 such examples.

Present your answer as a list of numbers separated by commas within a \(\boxed{...}\). For instance, \(\boxed{1, 2, 3}\).

\subsubsection{Variation}
\textbf{Actual Problem}\\
There are infinitely many distinct positive integers $n$ such that $\frac{4^n+2^n+1}{n^2 + n + 1}$ is a positive integer. Give examples of $10$ such numbers.

Output the answer as a comma separated list inside of $\boxed{...}$. For example $\boxed{1, 2, 3}$.

\textbf{Revised Problem}\\
Identify ten distinct positive integers \( n \) such that the expression \(\frac{4^n + 2^n + 1}{n^2 + n + 1}\) evaluates to a positive integer. Note that there are infinitely many such integers.

Present the answer as a comma-separated list enclosed within \(\boxed{...}\). For instance, \(\boxed{1, 2, 3}\).

\subsubsection{Variation}
\textbf{Actual Problem}\\
There are infinitely many distinct positive integers $n$ such that $\frac{4^n+2^n+1}{n^2 + n + 1}$ is a positive integer. Give examples of $8$ such numbers.

Output the answer as a comma separated list inside of $\boxed{...}$. For example $\boxed{1, 2, 3}$.

\textbf{Revised Problem}\\
Find infinitely many different positive integers \( n \) for which the expression \(\frac{4^n + 2^n + 1}{n^2 + n + 1}\) results in a whole number. Provide 8 specific examples of such integers.

Present the answer as a list of numbers separated by commas, enclosed within \(\boxed{...}\). For instance, \(\boxed{1, 2, 3}\).

\subsubsection{Variation}
\textbf{Actual Problem}\\
There are infinitely many distinct positive integers $n$ such that $\frac{4^n+2^n+1}{n^2 + n + 1}$ is a positive integer. Give examples of $9$ such numbers.

Output the answer as a comma separated list inside of $\boxed{...}$. For example $\boxed{1, 2, 3}$.

\textbf{Revised Problem}\\
Identify infinitely many distinct positive integers \( n \) such that the expression \(\frac{4^n + 2^n + 1}{n^2 + n + 1}\) evaluates to a positive integer. List \( 9 \) such values of \( n \).

Provide the solution as a list of numbers, separated by commas and enclosed within a boxed format, like this: \(\boxed{1, 2, 3}\).

\subsection{jbmo-shortlist-2018-p3}
\subsubsection{Variation}
\textbf{Actual Problem}\\
Find nonzero rational numbers $x_1,x_2,\ldots,x_{120}$ that are not all equal and satisfy 
$$x_1+\frac{4}{x_2}=x_2+\frac{4}{x_3}=x_3+\frac{4}{x_4}=\ldots=x_{120-1}+\frac{4}{x_{120}}=x_{120}+\frac{4}{x_1}.$$

Output the answer as a comma separated list inside of $\boxed{...}$. For example $\boxed{1, 2, 3}$.

\textbf{Revised Problem}\\
Identify nonzero rational values $x_1, x_2, \ldots, x_{120}$ that are distinct from each other and fulfill the condition:
$$x_1 + \frac{4}{x_2} = x_2 + \frac{4}{x_3} = x_3 + \frac{4}{x_4} = \ldots = x_{119} + \frac{4}{x_{120}} = x_{120} + \frac{4}{x_1}.$$

Present your solution as a list separated by commas enclosed in $\boxed{...}$. For example, $\boxed{1, 2, 3}$.

\subsubsection{Variation}
\textbf{Actual Problem}\\
Find nonzero rational numbers $x_1,x_2,\ldots,x_{222}$ that are not all equal and satisfy 
$$x_1+\frac{4}{x_2}=x_2+\frac{4}{x_3}=x_3+\frac{4}{x_4}=\ldots=x_{222-1}+\frac{4}{x_{222}}=x_{222}+\frac{4}{x_1}.$$

Output the answer as a comma separated list inside of $\boxed{...}$. For example $\boxed{1, 2, 3}$.

\textbf{Revised Problem}\\
Identify nonzero rational numbers $y_1, y_2, \ldots, y_{222}$ that do not all share the same value and meet the condition 
$$y_1+\frac{4}{y_2}=y_2+\frac{4}{y_3}=y_3+\frac{4}{y_4}=\ldots=y_{221}+\frac{4}{y_{222}}=y_{222}+\frac{4}{y_1}.$$

Provide your solution as a sequence of numbers separated by commas, enclosed in $\boxed{...}$. For instance, $\boxed{1, 2, 3}$.

\subsubsection{Variation}
\textbf{Actual Problem}\\
Find nonzero rational numbers $x_1,x_2,\ldots,x_{84}$ that are not all equal and satisfy 
$$x_1+\frac{4}{x_2}=x_2+\frac{4}{x_3}=x_3+\frac{4}{x_4}=\ldots=x_{84-1}+\frac{4}{x_{84}}=x_{84}+\frac{4}{x_1}.$$

Output the answer as a comma separated list inside of $\boxed{...}$. For example $\boxed{1, 2, 3}$.

\textbf{Revised Problem}\\
Determine a set of nonzero rational numbers \( y_1, y_2, \ldots, y_{84} \), such that they are not all identical and fulfill the condition:
$$ y_1 + \frac{4}{y_2} = y_2 + \frac{4}{y_3} = y_3 + \frac{4}{y_4} = \ldots = y_{83} + \frac{4}{y_{84}} = y_{84} + \frac{4}{y_1}. $$

Present your solution in the form of a comma-separated list within $\boxed{...}$. For instance, $\boxed{1, 2, 3}$.

\subsubsection{Variation}
\textbf{Actual Problem}\\
Find nonzero rational numbers $x_1,x_2,\ldots,x_{225}$ that are not all equal and satisfy 
$$x_1+\frac{4}{x_2}=x_2+\frac{4}{x_3}=x_3+\frac{4}{x_4}=\ldots=x_{225-1}+\frac{4}{x_{225}}=x_{225}+\frac{4}{x_1}.$$

Output the answer as a comma separated list inside of $\boxed{...}$. For example $\boxed{1, 2, 3}$.

\textbf{Revised Problem}\\
Identify nonzero rational numbers \(y_1, y_2, \ldots, y_{225}\) that are distinct and satisfy the condition:
\[ y_1 + \frac{4}{y_2} = y_2 + \frac{4}{y_3} = y_3 + \frac{4}{y_4} = \cdots = y_{224} + \frac{4}{y_{225}} = y_{225} + \frac{4}{y_1}. \]

Present the solution as a list of numbers separated by commas within a box, like \(\boxed{1, 2, 3}\).

\subsection{jbmo-shortlist-2019-c4}
\subsubsection{Variation}
\textbf{Actual Problem}\\
A $5 \times 50$ table is divided into $N=50$ unit square cells, where $3N+2=152$ of them are coloured black and the rest are coloured white. Two unit square cells are called adjacent if they share a common side. Each of the unit square cells has at most two adjacent black unit square cells. Find such a board configuration.

Output the board where cells are denoted with "w" or "b", namely "b" if the cell is colored black and $w$ if it is white. Output the answer between \verb|\begin{array}{...}| and \verb|\end{array}| inside of $\boxed{...}$. For example, $\boxed{\begin{array}{ccc}w & b & b \\ b & b & w \\ w & w & w\end{array}}$.

\textbf{Revised Problem}\\
A grid measuring $5 \times 50$ is split into unit square cells, totaling $N=50$, where $3N+2=152$ of these cells are black, and the remaining cells are white. Cells are considered adjacent if they have a common side. Each unit square cell is allowed to have up to two neighboring black cells. Determine a possible configuration for this grid.

Represent the grid with cells labeled as "w" for white and "b" for black. Place your solution within \verb|\begin{array}{...}| and \verb|\end{array}|, inside a $\boxed{...}$. For instance, $\boxed{\begin{array}{ccc}w & b & b \\ b & b & w \\ w & w & w\end{array}}$.

\subsubsection{Variation}
\textbf{Actual Problem}\\
A $5 \times 44$ table is divided into $N=44$ unit square cells, where $3N+2=134$ of them are coloured black and the rest are coloured white. Two unit square cells are called adjacent if they share a common side. Each of the unit square cells has at most two adjacent black unit square cells. Find such a board configuration.

Output the board where cells are denoted with "w" or "b", namely "b" if the cell is colored black and $w$ if it is white. Output the answer between \verb|\begin{array}{...}| and \verb|\end{array}| inside of $\boxed{...}$. For example, $\boxed{\begin{array}{ccc}w & b & b \\ b & b & w \\ w & w & w\end{array}}$.

\textbf{Revised Problem}\\
Consider a $5 \times 44$ grid composed of $N=44$ individual unit squares. In this grid, $3N + 2 = 134$ results in a total of 44 black squares, with the remainder being white. Cells are considered adjacent if they share one side. Each black cell is allowed to have no more than two black cells adjacent to it. Design a layout of this grid.

Depict the grid using "w" for white cells and "b" for black cells. Present your solution within the syntax \verb|\begin{array}{...}| and \verb|\end{array}|, encapsulated by $\boxed{...}$. For instance, $\boxed{\begin{array}{ccc}w & b & b \\ b & b & w \\ w & w & w\end{array}}$.

\subsubsection{Variation}
\textbf{Actual Problem}\\
A $5 \times 28$ table is divided into $N=28$ unit square cells, where $3N+2=86$ of them are coloured black and the rest are coloured white. Two unit square cells are called adjacent if they share a common side. Each of the unit square cells has at most two adjacent black unit square cells. Find such a board configuration.

Output the board where cells are denoted with "w" or "b", namely "b" if the cell is colored black and $w$ if it is white. Output the answer between \verb|\begin{array}{...}| and \verb|\end{array}| inside of $\boxed{...}$. For example, $\boxed{\begin{array}{ccc}w & b & b \\ b & b & w \\ w & w & w\end{array}}$.

\textbf{Revised Problem}\\
Consider a $5 \times 28$ grid filled with $N=28$ unit square cells, and it is given that $3N+2=86$ of these cells are black, with the remainder being white. A pair of unit square cells are said to be adjacent if they share a common side. Each black cell can have at most two black neighbors. Determine a configuration of this grid that satisfies these criteria.

Present the grid using "w" for white cells and "b" for black cells. Your output should be enclosed within \verb|\begin{array}{...}| and \verb|\end{array}|, placed inside $\boxed{...}$. For instance, $\boxed{\begin{array}{ccc}w & b & b \\ b & b & w \\ w & w & w\end{array}}$.

\subsubsection{Variation}
\textbf{Actual Problem}\\
A $5 \times 23$ table is divided into $N=23$ unit square cells, where $3N+2=71$ of them are coloured black and the rest are coloured white. Two unit square cells are called adjacent if they share a common side. Each of the unit square cells has at most two adjacent black unit square cells. Find such a board configuration.

Output the board where cells are denoted with "w" or "b", namely "b" if the cell is colored black and $w$ if it is white. Output the answer between \verb|\begin{array}{...}| and \verb|\end{array}| inside of $\boxed{...}$. For example, $\boxed{\begin{array}{ccc}w & b & b \\ b & b & w \\ w & w & w\end{array}}$.

\textbf{Revised Problem}\\
A grid measuring 5 rows by 23 columns contains a total of 115 unit square cells. Out of these, 71 cells are colored black, as determined by \(3N + 2 = 71\), and the remaining cells are colored white. Two cells are considered adjacent if they share a side. Design a configuration such that no black cell has more than two black neighbors. Provide a layout of such a grid.

Present the grid using "w" for white cells and "b" for black cells. Enclose your solution within \verb|\begin{array}{...}| and \verb|\end{array}|, all wrapped in $\boxed{...}$. For instance, $\boxed{\begin{array}{ccc}w & b & b \\ b & b & w \\ w & w & w\end{array}}$.

\subsection{jbmo-shortlist-2021-c5}
\subsubsection{Variation}
\textbf{Actual Problem}\\
Alice and Bob play a game together as a team on a $15 \times 15$ board with all unit squares initially white. Alice sets up the game by coloring exactly $k$ of the unit squares red at the beginning. After that, a legal move for Bob is to choose a row with at least $11$ red squares or a column with at least $7$ red squares and color all of the remaining squares in it red. Give an example such that Alice can set up a game in such a way that Bob can color the entire board red after finitely many moves with exactly $77$ initial red squares?

Output the board where cells are denoted with "o" or "r", namely "r" if the cell is colored red and "o" otherwise. Output the answer between \verb|\begin{array}{...}| and \verb|\end{array}| inside of $\boxed{...}$. For example, $\boxed{\begin{array}{ccc}o & r & r \\ r & r & o \\ o & o & o\end{array}}$.

\textbf{Revised Problem}\\
Alice and Bob are collaborating on a game played on a $15 \times 15$ grid where each small square begins as white. Alice's role is to initially paint exactly $k$ squares red. Following this, Bob can make a valid move by selecting either a row containing at least $11$ red squares or a column with a minimum of $7$ red squares, then coloring all other squares in that row or column red. Provide an example of how Alice can arrange the initial $77$ red squares so that Bob can eventually turn the whole board red through a finite sequence of moves.

Present the grid using "o" for white squares and "r" for red squares, formatted between \verb|\begin{array}{...}| and \verb|\end{array}| enclosed within $\boxed{...}$. For instance, $\boxed{\begin{array}{ccc}o & r & r \\ r & r & o \\ o & o & o\end{array}}$.

\subsubsection{Variation}
\textbf{Actual Problem}\\
Alice and Bob play a game together as a team on a $10 \times 10$ board with all unit squares initially white. Alice sets up the game by coloring exactly $k$ of the unit squares red at the beginning. After that, a legal move for Bob is to choose a row with at least $8$ red squares or a column with at least $5$ red squares and color all of the remaining squares in it red. Give an example such that Alice can set up a game in such a way that Bob can color the entire board red after finitely many moves with exactly $40$ initial red squares?

Output the board where cells are denoted with "o" or "r", namely "r" if the cell is colored red and "o" otherwise. Output the answer between \verb|\begin{array}{...}| and \verb|\end{array}| inside of $\boxed{...}$. For example, $\boxed{\begin{array}{ccc}o & r & r \\ r & r & o \\ o & o & o\end{array}}$.

\textbf{Revised Problem}\\
Alice and Bob are collaborating in a game played on a $10 \times 10$ grid where every square starts off colored white. Alice's role is to initially color exactly $k$ squares red. Following Alice's setup, Bob can make moves by selecting a row that contains at least $8$ red squares or a column with a minimum of $5$ red squares, and then coloring all the squares in that row or column red. Provide a configuration where Alice can set up the board such that Bob can turn every square on the grid red in a finite number of steps with precisely $40$ squares initially colored red.

Present the grid using "o" for white squares and "r" for red squares. Format the solution within \verb|\begin{array}{...}| and \verb|\end{array}|, enclosed by $\boxed{...}$. For instance, $\boxed{\begin{array}{ccc}o & r & r \\ r & r & o \\ o & o & o\end{array}}$.

\subsubsection{Variation}
\textbf{Actual Problem}\\
Alice and Bob play a game together as a team on a $6 \times 6$ board with all unit squares initially white. Alice sets up the game by coloring exactly $k$ of the unit squares red at the beginning. After that, a legal move for Bob is to choose a row with at least $4$ red squares or a column with at least $2$ red squares and color all of the remaining squares in it red. Give an example such that Alice can set up a game in such a way that Bob can color the entire board red after finitely many moves with exactly $8$ initial red squares?

Output the board where cells are denoted with "o" or "r", namely "r" if the cell is colored red and "o" otherwise. Output the answer between \verb|\begin{array}{...}| and \verb|\end{array}| inside of $\boxed{...}$. For example, $\boxed{\begin{array}{ccc}o & r & r \\ r & r & o \\ o & o & o\end{array}}$.

\textbf{Revised Problem}\\
Alice and Bob are collaborating on a game using a $6 \times 6$ grid where all squares start as white. Alice begins by selecting exactly $k$ squares to color red. Bob's legal move is to pick a row containing no less than 4 red squares or a column containing no less than 2 red squares and turn all remaining squares in that row or column red. Demonstrate a setup where Alice colors precisely 8 squares red initially, enabling Bob to turn the whole board red through a finite number of moves.

Represent the grid using "o" for white squares and "r" for red squares. Render your solution within \verb|\begin{array}{...}| and \verb|\end{array}| encapsulated by $\boxed{...}$. For instance, $\boxed{\begin{array}{ccc}o & r & r \\ r & r & o \\ o & o & o\end{array}}$.

\subsubsection{Variation}
\textbf{Actual Problem}\\
Alice and Bob play a game together as a team on a $4 \times 4$ board with all unit squares initially white. Alice sets up the game by coloring exactly $k$ of the unit squares red at the beginning. After that, a legal move for Bob is to choose a row with at least $2$ red squares or a column with at least $2$ red squares and color all of the remaining squares in it red. Give an example such that Alice can set up a game in such a way that Bob can color the entire board red after finitely many moves with exactly $4$ initial red squares?

Output the board where cells are denoted with "o" or "r", namely "r" if the cell is colored red and "o" otherwise. Output the answer between \verb|\begin{array}{...}| and \verb|\end{array}| inside of $\boxed{...}$. For example, $\boxed{\begin{array}{ccc}o & r & r \\ r & r & o \\ o & o & o\end{array}}$.

\textbf{Revised Problem}\\
Alice and Bob are collaborating in a game played on a $4 \times 4$ board, starting with all squares colored white. At the outset, Alice colors exactly $k$ of these squares red. Following this setup, Bob can make a legal move by selecting any row or column containing at least $2$ red squares and then coloring every remaining square in that row or column red. Demonstrate a configuration where Alice can initially color exactly $4$ squares red such that Bob is able to turn the entire board red in a finite number of moves.

Present the board using "o" for white squares and "r" for red squares. Display your solution within the \verb|\begin{array}{...}| and \verb|\end{array}| tags, encapsulated by $\boxed{...}$. For instance, $\boxed{\begin{array}{ccc}o & r & r \\ r & r & o \\ o & o & o\end{array}}$.

\subsection{jbmo-shortlist-2023-c1}
\subsubsection{Variation}
\textbf{Actual Problem}\\
We are given a square board with dimensions $17\times 17$, in which each unit cell is colored blue or red. There are exactly $9$ rows in which the majority of cells are blue, and exactly $9$ columns in which the majority of cells are red.  Give an example such that the largest monochromatic square has a side length of $\frac{N-1}{2}=8$.

Output the board where cells are denoted with "b" or "r", depending on the color. Output the answer between \verb|\begin{array}{...}| and \verb|\end{array}| inside of $\boxed{...}$. For example, $\boxed{\begin{array}{ccc}b & r & r \\ r & r & b \\ b & b & b\end{array}}$.

\textbf{Revised Problem}\\
Consider a $17 \times 17$ grid where each cell is filled with either blue or red color. Among the 17 rows, exactly 9 have more blue cells than red, and among the 17 columns, exactly 9 have more red cells than blue. Construct a grid such that the largest square made entirely of one color has a maximum side length of 8.

Present the grid with "b" representing blue cells and "r" representing red cells. Format the grid using \verb|\begin{array}{...}| and \verb|\end{array}| within $\boxed{...}$. For instance, $\boxed{\begin{array}{ccc}b & r & r \\ r & r & b \\ b & b & b\end{array}}$.

\subsubsection{Variation}
\textbf{Actual Problem}\\
We are given a square board with dimensions $21\times 21$, in which each unit cell is colored blue or red. There are exactly $11$ rows in which the majority of cells are blue, and exactly $11$ columns in which the majority of cells are red.  Give an example such that the largest monochromatic square has a side length of $\frac{N-1}{2}=10$.

Output the board where cells are denoted with "b" or "r", depending on the color. Output the answer between \verb|\begin{array}{...}| and \verb|\end{array}| inside of $\boxed{...}$. For example, $\boxed{\begin{array}{ccc}b & r & r \\ r & r & b \\ b & b & b\end{array}}$.

\textbf{Revised Problem}\\
Consider a square grid of size $21\times 21$, where each individual unit is either blue or red. In this grid, there are precisely $11$ rows where blue cells are dominant, and exactly $11$ columns where red cells are most prevalent. Provide an arrangement of the grid such that the largest contiguous square of a single color has a side length equal to $\frac{N-1}{2}=10$.

Present the grid using "b" for blue and "r" for red cells. Format your response between \verb|\begin{array}{...}| and \verb|\end{array}| enclosed within $\boxed{...}$. For instance, $\boxed{\begin{array}{ccc}b & r & r \\ r & r & b \\ b & b & b\end{array}}$.

\subsubsection{Variation}
\textbf{Actual Problem}\\
We are given a square board with dimensions $15\times 15$, in which each unit cell is colored blue or red. There are exactly $8$ rows in which the majority of cells are blue, and exactly $8$ columns in which the majority of cells are red.  Give an example such that the largest monochromatic square has a side length of $\frac{N-1}{2}=7$.

Output the board where cells are denoted with "b" or "r", depending on the color. Output the answer between \verb|\begin{array}{...}| and \verb|\end{array}| inside of $\boxed{...}$. For example, $\boxed{\begin{array}{ccc}b & r & r \\ r & r & b \\ b & b & b\end{array}}$.

\textbf{Revised Problem}\\
Consider a square grid of size $15 \times 15$, where each cell is painted either blue or red. Specifically, there are $8$ rows where blue is the predominant color and $8$ columns where red is the predominant color. Construct an arrangement such that the side of the largest square consisting entirely of one color is $\frac{N-1}{2}=7$.

Represent the grid using "b" for blue and "r" for red squares. Present your solution within \verb|\begin{array}{...}| and \verb|\end{array}| enclosed in $\boxed{...}$. For instance, $\boxed{\begin{array}{ccc}b & r & r \\ r & r & b \\ b & b & b\end{array}}$.

\subsubsection{Variation}
\textbf{Actual Problem}\\
We are given a square board with dimensions $19\times 19$, in which each unit cell is colored blue or red. There are exactly $10$ rows in which the majority of cells are blue, and exactly $10$ columns in which the majority of cells are red.  Give an example such that the largest monochromatic square has a side length of $\frac{N-1}{2}=9$.

Output the board where cells are denoted with "b" or "r", depending on the color. Output the answer between \verb|\begin{array}{...}| and \verb|\end{array}| inside of $\boxed{...}$. For example, $\boxed{\begin{array}{ccc}b & r & r \\ r & r & b \\ b & b & b\end{array}}$.

\textbf{Revised Problem}\\
Consider a $19 \times 19$ grid where each cell is filled with either blue or red. In exactly 10 rows, the number of blue cells outnumbers the red, and in exactly 10 columns, the number of red cells exceeds the blue. Construct an example of such a grid where the largest contiguous block of cells of the same color forms a square with a maximum side length of 9.

Represent the grid with cells marked as "b" for blue or "r" for red. Present your solution enclosed within \verb|\begin{array}{...}| and \verb|\end{array}| wrapped in $\boxed{...}$. For instance, use $\boxed{\begin{array}{ccc}b & r & r \\ r & r & b \\ b & b & b\end{array}}$ to show the grid configuration.

\subsection{jbmo-shortlist-2023-c2}
\subsubsection{Variation}
\textbf{Actual Problem}\\
Consider an increasing sequence of real numbers $a_1 < a_2 < \cdots < a_{25}$ such that all pairwise sums of the elements in the sequence are different. For such a sequence, denote by $M$ the number of pairs $(a_i, a_j)$ such that $a_i < a_j$ and such that $a_i+a_j < a_2+a_{24}$. Find 2 examples for $a_1 < a_2 < \cdots < a_{25}$, namely for $M=44$ and $M=255$. 

Output the answer as two comma-separated sequences of the same length inside $\boxed{...}$, i.e. $\boxed{((1, 2, 3, 4), (5, 6, 7, 8))}$

\textbf{Revised Problem}\\
Imagine a sequence of increasing real numbers $b_1 < b_2 < \cdots < b_{25}$ where every sum of two distinct elements in the sequence is unique. For such a sequence, let $N$ represent the count of pairs $(b_i, b_j)$ that satisfy $b_i < b_j$ and $b_i+b_j < b_2+b_{24}$. Determine two instances of the sequence $b_1 < b_2 < \cdots < b_{25}$ where $N=44$ and $N=255$.

Present the answer as two sequences separated by commas, each enclosed in parentheses, within a box, for example, $\boxed{((1, 2, 3, 4), (5, 6, 7, 8))}$.

\subsubsection{Variation}
\textbf{Actual Problem}\\
Consider an increasing sequence of real numbers $a_1 < a_2 < \cdots < a_{44}$ such that all pairwise sums of the elements in the sequence are different. For such a sequence, denote by $M$ the number of pairs $(a_i, a_j)$ such that $a_i < a_j$ and such that $a_i+a_j < a_2+a_{43}$. Find 2 examples for $a_1 < a_2 < \cdots < a_{44}$, namely for $M=82$ and $M=863$. 

Output the answer as two comma-separated sequences of the same length inside $\boxed{...}$, i.e. $\boxed{((1, 2, 3, 4), (5, 6, 7, 8))}$

\textbf{Revised Problem}\\
Imagine a strictly increasing sequence of real numbers \( b_1 < b_2 < \ldots < b_{44} \) where every pair of elements in the sequence sums to a unique value. For this type of sequence, let \( N \) represent the number of pairs \((b_i, b_j)\) where \( b_i < b_j \) and \( b_i+b_j < b_2+b_{43} \). Determine two sequences for which \( N=82 \) and \( N=863 \).

Display your answer as a pair of comma-separated sequences with the same length within \(\boxed{...}\), for instance, \(\boxed{((1, 2, 3, 4), (5, 6, 7, 8))}\).

\subsubsection{Variation}
\textbf{Actual Problem}\\
Consider an increasing sequence of real numbers $a_1 < a_2 < \cdots < a_{28}$ such that all pairwise sums of the elements in the sequence are different. For such a sequence, denote by $M$ the number of pairs $(a_i, a_j)$ such that $a_i < a_j$ and such that $a_i+a_j < a_2+a_{27}$. Find 2 examples for $a_1 < a_2 < \cdots < a_{28}$, namely for $M=50$ and $M=327$. 

Output the answer as two comma-separated sequences of the same length inside $\boxed{...}$, i.e. $\boxed{((1, 2, 3, 4), (5, 6, 7, 8))}$

\textbf{Revised Problem}\\
Imagine a strictly increasing sequence of real numbers \( b_1, b_2, \ldots, b_{28} \) where each element is distinct and every possible sum of two elements from the sequence is unique. For this sequence, define \( N \) as the count of pairs \((b_i, b_j)\) such that \( b_i < b_j \) and \( b_i + b_j < b_2 + b_{27} \). Determine two specific sequences for which \( N = 50 \) and \( N = 327 \).

Present the solution as two sequences enclosed in \(\boxed{...}\), separated by a comma. For example, \(\boxed{((1, 2, 3, 4), (5, 6, 7, 8))}\).

\subsubsection{Variation}
\textbf{Actual Problem}\\
Consider an increasing sequence of real numbers $a_1 < a_2 < \cdots < a_{23}$ such that all pairwise sums of the elements in the sequence are different. For such a sequence, denote by $M$ the number of pairs $(a_i, a_j)$ such that $a_i < a_j$ and such that $a_i+a_j < a_2+a_{22}$. Find 2 examples for $a_1 < a_2 < \cdots < a_{23}$, namely for $M=40$ and $M=212$. 

Output the answer as two comma-separated sequences of the same length inside $\boxed{...}$, i.e. $\boxed{((1, 2, 3, 4), (5, 6, 7, 8))}$

\textbf{Revised Problem}\\
Imagine a strictly increasing sequence of real numbers \(b_1 < b_2 < \cdots < b_{23}\) where all sums of two different elements in the sequence are distinct. For such a sequence, let \(N\) represent the count of pairs \((b_i, b_j)\) such that \(b_i < b_j\) and \(b_i + b_j < b_2 + b_{22}\). Determine two sequences for \(b_1 < b_2 < \cdots < b_{23}\) where \(N = 40\) for one sequence and \(N = 212\) for the other.

Present the solution as two comma-separated sequences of equal size contained within \(\boxed{...}\). For example, \(\boxed{((1, 2, 3, 4), (5, 6, 7, 8))}\).

\subsection{jbmo-shortlist-2023-n3}
\subsubsection{Variation}
\textbf{Actual Problem}\\
Find a set $A\subseteq \{1, 2, \ldots, 100\}$ with $17$ elements such that for any $a, b \in A$, $a$ divides $b$ if and only if $s(a)$ divides $s(b)$, where $s(k)$ denotes the sum of $k$'s digits.

Output the answer as a comma separated list inside of $\boxed{...}$. For example $\boxed{1, 2, 3}$.

\textbf{Revised Problem}\\
Identify a subset $A$ consisting of 17 numbers from the set $\{1, 2, \ldots, 100\}$ such that for any two numbers $a$ and $b$ in $A$, $a$ is a divisor of $b$ if and only if the sum of the digits of $a$ divides the sum of the digits of $b$. Let $s(k)$ denote the sum of the digits of $k$.

Present the solution as a list of numbers separated by commas within $\boxed{...}$. For instance, $\boxed{1, 2, 3}$.

\section{konhauser}
\subsection{konhauser-2013-1}
\subsubsection{Variation}
\textbf{Actual Problem}\\
Consider the points $A = (4, 0)$, $B = (0, 3)$ and $C=(0,0)$ in the plane. Find a pair of poinst $P$ and $Q$ on ABC that divide the perimeter of the triangle in half and such that PQ divides the area of the triangle in half.

Output the answer as a comma separated list of lists inside of $\boxed{...}$. For example $\boxed{(1,1), (1,\sqrt{2})}$.

\textbf{Revised Problem}\\
Given the triangle with vertices $A = (4, 0)$, $B = (0, 3)$, and $C = (0, 0)$ in the coordinate plane, identify two points $P$ and $Q$ located on the perimeter of $ABC$. The line segment $PQ$ should bisect both the perimeter and the area of triangle $ABC$ equally.

Express the solution as a comma-separated list of ordered pairs enclosed within $\boxed{...}$. For instance, $\boxed{(1,1), (1,\sqrt{2})}$.

\subsubsection{Variation}
\textbf{Actual Problem}\\
Consider the points $A = (48, 0)$, $B = (0, 64)$ and $C=(0,0)$ in the plane. Find a pair of poinst $P$ and $Q$ on ABC that divide the perimeter of the triangle in half and such that PQ divides the area of the triangle in half.

Output the answer as a comma separated list of lists inside of $\boxed{...}$. For example $\boxed{(1,1), (1,\sqrt{2})}$.

\textbf{Revised Problem}\\
In the coordinate plane, there is a triangle with vertices at $A = (48, 0)$, $B = (0, 64)$, and $C = (0, 0)$. Identify two points, $P$ and $Q$, located along the perimeter of the triangle, which divide the total perimeter into two equal parts. Additionally, the line segment $PQ$ should split the total area of the triangle into two equal sections.

Provide your solution formatted as a comma-separated list of coordinate pairs enclosed within $\boxed{...}$. For instance, format your answer as $\boxed{(1,1), (1,\sqrt{2})}$.

\subsubsection{Variation}
\textbf{Actual Problem}\\
Consider the points $A = (16, 0)$, $B = (0, 30)$ and $C=(0,0)$ in the plane. Find a pair of poinst $P$ and $Q$ on ABC that divide the perimeter of the triangle in half and such that PQ divides the area of the triangle in half.

Output the answer as a comma separated list of lists inside of $\boxed{...}$. For example $\boxed{(1,1), (1,\sqrt{2})}$.

\textbf{Revised Problem}\\
Given the coordinates of triangle vertices $A = (16, 0)$, $B = (0, 30)$, and $C=(0,0)$, determine the locations of points $P$ and $Q$ on the perimeter of triangle ABC such that the line segment $PQ$ bisects both the perimeter and the area of the triangle.

Present the solution as a list of coordinate pairs separated by commas, enclosed within $\boxed{...}$. For instance, $\boxed{(1,1), (1,\sqrt{2})}$.

\subsubsection{Variation}
\textbf{Actual Problem}\\
Consider the points $A = (15, 0)$, $B = (0, 8)$ and $C=(0,0)$ in the plane. Find a pair of poinst $P$ and $Q$ on ABC that divide the perimeter of the triangle in half and such that PQ divides the area of the triangle in half.

Output the answer as a comma separated list of lists inside of $\boxed{...}$. For example $\boxed{(1,1), (1,\sqrt{2})}$.

\textbf{Revised Problem}\\
Given the triangle with vertices $A = (15, 0)$, $B = (0, 8)$, and $C = (0,0)$ on a coordinate plane, identify two points $P$ and $Q$ located on the perimeter of triangle ABC. These points should be positioned such that they evenly split the total perimeter and the total area of the triangle.

Express the solution as a list of lists, each representing a point, separated by commas and enclosed within $\boxed{...}$. For instance, the format should look like $\boxed{(1,1), (1,\sqrt{2})}$.

\subsection{konhauser-2014-7}
\subsubsection{Variation}
\textbf{Actual Problem}\\
Find any pair of real numbers $x$ and $y$ that are solutions to the system of equations:

$$
\begin{cases}
3^x - 3^y = 2^y \\
9^x - 6^y = 19^y
\end{cases}
$$


Output the answer as a comma separated list inside of $\boxed{...}$. For example $\boxed{1, 2, 3}$.
You can use valid LaTeX expressions to represent the numbers.

\textbf{Revised Problem}\\
Determine a pair of real numbers $(x, y)$ that satisfy the following set of equations:

$$
\begin{cases}
3^x - 3^y = 2^y \\
9^x - 6^y = 19^y
\end{cases}
$$

Provide the solution as a list of numbers separated by commas enclosed in $\boxed{...}$. For instance, $\boxed{1, 2, 3}$. Make sure to use appropriate LaTeX notation for the numbers.

\subsubsection{Variation}
\textbf{Actual Problem}\\
Find any pair of real numbers $x$ and $y$ that are solutions to the system of equations:

$$
\begin{cases}
8^x - 8^y = 5^y \\
64^x - 40^y = 129^y
\end{cases}
$$


Output the answer as a comma separated list inside of $\boxed{...}$. For example $\boxed{1, 2, 3}$.
You can use valid LaTeX expressions to represent the numbers.

\textbf{Revised Problem}\\
Determine a pair of real numbers $(x, y)$ that satisfy the following system of equations:

$$
\begin{cases}
8^x - 8^y = 5^y \\
64^x - 40^y = 129^y
\end{cases}
$$

Provide your solution in the form of a comma-separated list enclosed within $\boxed{...}$. For instance, $\boxed{1, 2, 3}$. Use valid LaTeX expressions to denote the numbers.

\subsubsection{Variation}
\textbf{Actual Problem}\\
Find any pair of real numbers $x$ and $y$ that are solutions to the system of equations:

$$
\begin{cases}
4^x - 9^y = 2^y \\
16^x - 18^y = 103^y
\end{cases}
$$


Output the answer as a comma separated list inside of $\boxed{...}$. For example $\boxed{1, 2, 3}$.
You can use valid LaTeX expressions to represent the numbers.

\textbf{Revised Problem}\\
Determine a pair of real numbers \(x\) and \(y\) that satisfy the following system of equations:

$$
\begin{cases}
4^x - 9^y = 2^y \\
16^x - 18^y = 103^y
\end{cases}
$$

Present the solution as a list of numbers separated by commas and enclosed in $\boxed{...}$. For instance, write it as $\boxed{1, 2, 3}$. You may use appropriate LaTeX expressions for the numbers.

\subsubsection{Variation}
\textbf{Actual Problem}\\
Find any pair of real numbers $x$ and $y$ that are solutions to the system of equations:

$$
\begin{cases}
8^x - 2^y = 8^y \\
64^x - 16^y = 84^y
\end{cases}
$$


Output the answer as a comma separated list inside of $\boxed{...}$. For example $\boxed{1, 2, 3}$.
You can use valid LaTeX expressions to represent the numbers.

\textbf{Revised Problem}\\
Determine any real number pair $(x, y)$ that satisfies the following system of equations:

$$
\begin{cases}
8^x - 2^y = 8^y \\
64^x - 16^y = 84^y
\end{cases}
$$

Present your solution as a list of numbers separated by commas within a $\boxed{...}$. For instance, $\boxed{1, 2, 3}$. You may employ valid LaTeX notation to denote the numbers.

\subsection{konhauser-2015-2}
\subsubsection{Variation}
\textbf{Actual Problem}\\
A student is given two values of a function, $f(3)$ and $f(7)$, and is asked to find $f(5)$. The function is to be an exponential function, but the student uses a linear function instead, and as a result his answer is exactly one away from the correct answer. The only mistake the student makes is using the wrong type of function. If $f(3)$ is $25$, what could $f(7)$ have been? Give two distinct possible real values.

Output the answer as a comma separated list inside of $\boxed{...}$. For example $\boxed{1.12, 2.13, 3.14}$. You can use real values or valid LaTeX expressions to represent the numbers.

\textbf{Revised Problem}\\
A student is provided with the values \( f(3) \) and \( f(7) \) for a function and needs to determine \( f(5) \). Although the function is exponential, the student mistakenly applies a linear function, resulting in a discrepancy of precisely one unit from the actual answer. Assuming that the only error was using an incorrect function type, find two different potential real values that \( f(7) \) could be if \( f(3) \) equals 25.

Present your answer as a list separated by commas within $\boxed{...}$. For instance, $\boxed{1.12, 2.13, 3.14}$. You may use real numbers or proper LaTeX expressions to represent the values.

\subsubsection{Variation}
\textbf{Actual Problem}\\
A student is given two values of a function, $f(51)$ and $f(51)$, and is asked to find $f(51.0)$. The function is to be an exponential function, but the student uses a linear function instead, and as a result his answer is exactly one away from the correct answer. The only mistake the student makes is using the wrong type of function. If $f(51)$ is $112$, what could $f(51)$ have been? Give two distinct possible real values.

Output the answer as a comma separated list inside of $\boxed{...}$. For example $\boxed{1.12, 2.13, 3.14}$. You can use real values or valid LaTeX expressions to represent the numbers.

\textbf{Revised Problem}\\
A student is provided with two instances of a function's value, both being $f(51)$. The task is to determine $f(51.0)$. The function should be exponential, yet the student inadvertently applies a linear function, leading to an answer that is precisely one unit off from the true solution. The student's sole error is using an incorrect function type. Assuming $f(51)$ is $112$, identify two different possible real values for $f(51)$. Present your findings.

Express your answer as a list of values separated by commas within $\boxed{...}$. For instance, $\boxed{1.12, 2.13, 3.14}$. The values can be presented as real numbers or in valid LaTeX format.

\subsubsection{Variation}
\textbf{Actual Problem}\\
A student is given two values of a function, $f(5)$ and $f(19)$, and is asked to find $f(12.0)$. The function is to be an exponential function, but the student uses a linear function instead, and as a result his answer is exactly one away from the correct answer. The only mistake the student makes is using the wrong type of function. If $f(5)$ is $70$, what could $f(19)$ have been? Give two distinct possible real values.

Output the answer as a comma separated list inside of $\boxed{...}$. For example $\boxed{1.12, 2.13, 3.14}$. You can use real values or valid LaTeX expressions to represent the numbers.

\textbf{Revised Problem}\\
A student is provided with the values of a function at $x = 5$ and $x = 19$, specifically $f(5)$ and $f(19)$, and must determine the value of $f(12.0)$. Although the function is exponential, the student incorrectly uses a linear model, resulting in a calculated value that is precisely one unit off from the actual value. The student's sole error is the choice of function type. Given that $f(5)$ equals $70$, what are two different real values that $f(19)$ could take?

Present your answer as a list separated by commas within $\boxed{...}$. For instance, $\boxed{1.12, 2.13, 3.14}$. Real numbers or appropriate LaTeX representations are acceptable for expressing the values.

\subsubsection{Variation}
\textbf{Actual Problem}\\
A student is given two values of a function, $f(3)$ and $f(9)$, and is asked to find $f(6.0)$. The function is to be an exponential function, but the student uses a linear function instead, and as a result his answer is exactly one away from the correct answer. The only mistake the student makes is using the wrong type of function. If $f(3)$ is $26$, what could $f(9)$ have been? Give two distinct possible real values.

Output the answer as a comma separated list inside of $\boxed{...}$. For example $\boxed{1.12, 2.13, 3.14}$. You can use real values or valid LaTeX expressions to represent the numbers.

\textbf{Revised Problem}\\
A student is provided with two known values of a function, specifically $f(3)$ and $f(9)$, and is tasked with determining $f(6.0)$. Although the function is inherently exponential, the student mistakenly applies a linear function instead, resulting in their calculated answer being precisely one unit different from the accurate result. The sole error made by the student lies in the choice of function type. Given that $f(3)$ equals $26$, identify two different possible real values for $f(9)$.

Present the solution as a series of values separated by commas within $\boxed{...}$. For example, you might express it as $\boxed{1.12, 2.13, 3.14}$. Feel free to use real numbers or appropriate LaTeX expressions to denote the values.

\subsection{konhauser-2016-1}
\subsubsection{Variation}
\textbf{Actual Problem}\\
A footworm grows at a constant rate of one foot per day and stops growing when it reaches one foot. A full-grown worm can be cut into two worms of lengths $x$ and $1-x$, each of which then grows at the constant rate until full-grown. Worms that are not full-grown cannot be cut. Show how to produce, in 1 day, 6 full-grown worms, where you start with one full-grown worm.

Output the answer as a comma separated list of lists inside of $\boxed{...}$. Each list first contains the current time in days, then a list of the worms' current lengths before the cut, and then a list of the worms' current lengths after the cut.

For instance, to cut a worm of length 1 into two worms of length 0.5 at time $\frac{1}{2}$, you would output '$\frac{1}{2}, (1), (0.5, 0.5)$'. A correct solution to produce $4$ worms in 1 day would be:
$\boxed{
    (0, (1), (0.5, 0.5)),
    (\frac{1}{2}, (1, 1), (0.5, 0.5, 1)),
    (\frac{1}{2}, (1, 0.5, 0.5), (0.5, 0.5, 0.5, 0.5)),
}$
Note that you cannot cut more than one worm in a single line. The last line where all worms are full-grown at time 1 must be omitted.

\textbf{Revised Problem}\\
A footworm increases in size at a steady pace of one foot each day and ceases to grow once it reaches a foot in length. A completely matured worm can be divided into two smaller worms measuring $x$ and $1-x$ feet, each of which then grows at the same rate until reaching a full foot. Only fully matured worms can be divided. Demonstrate how to achieve a total of 6 fully grown worms within a single day, starting from just one fully grown worm.

Express the solution as a list of lists, separated by commas, enclosed within $\boxed{...}$. Each list should start with the current day, then include a list of the worms' lengths before division, followed by a list of the worms' lengths after division.

For example, if you divide a worm of length 1 into two worms of length 0.5 at time $\frac{1}{2}$, you should write '$\frac{1}{2}, (1), (0.5, 0.5)$'. A valid method to create 4 worms in a day might be:
$\boxed{
    (0, (1), (0.5, 0.5)),
    (\frac{1}{2}, (1, 1), (0.5, 0.5, 1)),
    (\frac{1}{2}, (1, 0.5, 0.5), (0.5, 0.5, 0.5, 0.5)),
}$
Note that only one worm can be divided at a time in each entry. Omit the final entry where all worms are fully grown at day 1.

\subsubsection{Variation}
\textbf{Actual Problem}\\
A footworm grows at a constant rate of one foot per day and stops growing when it reaches one foot. A full-grown worm can be cut into two worms of lengths $x$ and $1-x$, each of which then grows at the constant rate until full-grown. Worms that are not full-grown cannot be cut. Show how to produce, in $\frac{1023}{1024}$ of a day, 10 full-grown worms, where you start with one full-grown worm.

Output the answer as a comma separated list of lists inside of $\boxed{...}$. Each list first contains the current time in days, then a list of the worms' current lengths before the cut, and then a list of the worms' current lengths after the cut.

For instance, to cut a worm of length 1 into two worms of length 0.5 at time $\frac{1}{2}$, you would output '$\frac{1}{2}, (1), (0.5, 0.5)$'. A correct solution to produce $4$ worms in 1 day would be:
$\boxed{
    (0, (1), (0.5, 0.5)),
    (\frac{1}{2}, (1, 1), (0.5, 0.5, 1)),
    (\frac{1}{2}, (1, 0.5, 0.5), (0.5, 0.5, 0.5, 0.5)),
}$
Note that you cannot cut more than one worm in a single line. The last line where all worms are full-grown at time 1 must be omitted.

\textbf{Revised Problem}\\
A footworm increases in length at a steady pace of one foot per day and ceases to grow upon reaching a length of one foot. When fully grown, a worm can be split into two segments of lengths $x$ and $1-x$. Each segment then grows at a constant rate until reaching full size. Worms that have not yet reached full size cannot be divided. Demonstrate how to generate 10 fully grown worms in $\frac{1023}{1024}$ of a day, beginning with a single fully grown worm.

Present the solution as a comma-separated sequence of lists encapsulated within $\boxed{...}$. Each list includes the current time in days, followed by the list of worms' lengths before the division, and then the list of worms' lengths after the division.

For example, to split a worm of length 1 into two worms of length 0.5 at time $\frac{1}{2}$, you would express it as '$\frac{1}{2}, (1), (0.5, 0.5)$'. An accurate sequence to produce 4 worms in 1 day would be:
$\boxed{
    (0, (1), (0.5, 0.5)),
    (\frac{1}{2}, (1, 1), (0.5, 0.5, 1)),
    (\frac{1}{2}, (1, 0.5, 0.5), (0.5, 0.5, 0.5, 0.5)),
}$
Note that you must not cut more than one worm in a single step. The final step where all worms are fully grown at time 1 should be excluded.

\subsubsection{Variation}
\textbf{Actual Problem}\\
A footworm grows at a constant rate of one foot per day and stops growing when it reaches one foot. A full-grown worm can be cut into two worms of lengths $x$ and $1-x$, each of which then grows at the constant rate until full-grown. Worms that are not full-grown cannot be cut. Show how to produce, in $\frac{31}{32}$ of a day, 5 full-grown worms, where you start with one full-grown worm.

Output the answer as a comma separated list of lists inside of $\boxed{...}$. Each list first contains the current time in days, then a list of the worms' current lengths before the cut, and then a list of the worms' current lengths after the cut.

For instance, to cut a worm of length 1 into two worms of length 0.5 at time $\frac{1}{2}$, you would output '$\frac{1}{2}, (1), (0.5, 0.5)$'. A correct solution to produce $4$ worms in 1 day would be:
$\boxed{
    (0, (1), (0.5, 0.5)),
    (\frac{1}{2}, (1, 1), (0.5, 0.5, 1)),
    (\frac{1}{2}, (1, 0.5, 0.5), (0.5, 0.5, 0.5, 0.5)),
}$
Note that you cannot cut more than one worm in a single line. The last line where all worms are full-grown at time 1 must be omitted.

\textbf{Revised Problem}\\
A footworm increases in length at a steady rate of one foot per day and ceases to grow once it reaches a length of one foot. Upon reaching full length, a worm can be divided into two segments measuring $x$ and $1-x$, each of which subsequently grows at the same constant rate until it becomes full-sized. Worms that have not reached full size cannot be divided. Demonstrate how to create 5 full-sized worms in $\frac{31}{32}$ of a day, beginning with a single full-sized worm.

Present the answer as a sequence of lists, separated by commas, enclosed within $\boxed{...}$. Each list should include the current time in days, followed by a list of the worms' lengths prior to the cut, and then a list of the lengths of the worms post-cut.

For example, to split a worm of length 1 into two segments of length 0.5 at time $\frac{1}{2}$, you would write '$\frac{1}{2}, (1), (0.5, 0.5)$'. A valid solution to generate $4$ worms in 1 day would be:
$\boxed{
    (0, (1), (0.5, 0.5)),
    (\frac{1}{2}, (1, 1), (0.5, 0.5, 1)),
    (\frac{1}{2}, (1, 0.5, 0.5), (0.5, 0.5, 0.5, 0.5)),
}$
Note that only one worm can be divided per step. The final step where all worms are full-sized at time 1 should not be included.

\subsubsection{Variation}
\textbf{Actual Problem}\\
A footworm grows at a constant rate of one foot per day and stops growing when it reaches one foot. A full-grown worm can be cut into two worms of lengths $x$ and $1-x$, each of which then grows at the constant rate until full-grown. Worms that are not full-grown cannot be cut. Show how to produce, in $\frac{255}{256}$ of a day, 8 full-grown worms, where you start with one full-grown worm.

Output the answer as a comma separated list of lists inside of $\boxed{...}$. Each list first contains the current time in days, then a list of the worms' current lengths before the cut, and then a list of the worms' current lengths after the cut.

For instance, to cut a worm of length 1 into two worms of length 0.5 at time $\frac{1}{2}$, you would output '$\frac{1}{2}, (1), (0.5, 0.5)$'. A correct solution to produce $4$ worms in 1 day would be:
$\boxed{
    (0, (1), (0.5, 0.5)),
    (\frac{1}{2}, (1, 1), (0.5, 0.5, 1)),
    (\frac{1}{2}, (1, 0.5, 0.5), (0.5, 0.5, 0.5, 0.5)),
}$
Note that you cannot cut more than one worm in a single line. The last line where all worms are full-grown at time 1 must be omitted.

\textbf{Revised Problem}\\
A footworm increases in length by one foot every day, ceasing growth upon reaching a length of one foot. When a worm is fully grown, it can be split into two worms with lengths $x$ and $1-x$, each resuming growth at the same constant rate until they are fully grown. Worms that haven't reached full maturity cannot be split. Demonstrate a method to generate 8 fully grown worms within $\frac{255}{256}$ of a day, beginning with a single fully grown worm.

Present your answer as a sequence of comma-separated lists enclosed in $\boxed{...}$. Each list should begin with the current time in days, followed by a list of the worms' lengths prior to the cut and then a list of the worms' lengths post-cut.

For example, to split a worm of length 1 into two worms each of length 0.5 at time $\frac{1}{2}$, you would write '$\frac{1}{2}, (1), (0.5, 0.5)$'. An example of a correct approach to creating 4 worms in 1 day would be:
$\boxed{
    (0, (1), (0.5, 0.5)),
    (\frac{1}{2}, (1, 1), (0.5, 0.5, 1)),
    (\frac{1}{2}, (1, 0.5, 0.5), (0.5, 0.5, 0.5, 0.5)),
}$
Ensure that no more than one worm is split per step. The final step where all worms reach maturity at time 1 should not be included.

\subsection{konhauser-2016-3}
\subsubsection{Variation}
\textbf{Actual Problem}\\
Define the crown graph $C_{m, n}$ with $n = 7$ and $m = 7$ to consist of an $n$-cycle with vertices $v_i$ and $m$ additional vertices $u_j$ that are initially isolated. Then edges are added between each $u_i$ and $v_j$. The minimum number of colors required such that the edges of $C_{m, n}$ with a common vertex get different colors is 9. Give such a coloring.

Output the answer as a list of two elements: one matrix between \verb|\begin{array}{...}| and \verb|\end{array}| and one list inside of $\boxed{...}$. A number indicates the color of the edge. The $i$-th element of the $j$-th row in the matrix indicates the color of the edge between $u_j$ and $v_i$. The $i$-th element of the list indicates the color of the edge between $v_i$ and $v_{(i+1) \text{mod}(\text{len}(v_i))}$.

For instance, the following is a valid solution for $C_{{5, 3}}$ with 7 colors:

$\boxed{
\begin{array}{ccc}
3 & 2 & 7 \\
2 & 6 & 4 \\
5 & 1 & 2 \\
4 & 3 & 1 \\
1 & 4 & 3 \\
\end{array},
(7, 5, 6)
}$



\textbf{Revised Problem}\\
Consider the crown graph $C_{m, n}$ where $n = 7$ and $m = 7$. This graph is formed by an $n$-cycle with vertices labeled as $v_i$, and $m$ additional vertices labeled as $u_j$, which are initially isolated. Subsequently, edges are added between each vertex $u_i$ and every vertex $v_j$. Determine the smallest number of colors needed to color the edges of $C_{m, n}$ such that no two edges sharing a vertex have the same color, which is known to be 9. Provide a valid coloring scheme.

Present the solution as a pair consisting of a matrix and a list. Enclose the matrix with \verb|\begin{array}{...}| and \verb|\end{array}|, and place the list within $\boxed{...}$. Each entry in the matrix represents the color of the edge connecting $u_j$ and $v_i$. The elements in the list represent the colors of the edges between consecutive vertices in the $n$-cycle, specifically, between $v_i$ and $v_{(i+1) \text{mod}(\text{len}(v_i))}$.

For example, the following is a valid coloring scheme for $C_{{5, 3}}$ using 7 colors:

$\boxed{
\begin{array}{ccc}
3 & 2 & 7 \\
2 & 6 & 4 \\
5 & 1 & 2 \\
4 & 3 & 1 \\
1 & 4 & 3 \\
\end{array},
(7, 5, 6)
}$

\subsubsection{Variation}
\textbf{Actual Problem}\\
Define the crown graph $C_{m, n}$ with $n = 9$ and $m = 10$ to consist of an $n$-cycle with vertices $v_i$ and $m$ additional vertices $u_j$ that are initially isolated. Then edges are added between each $u_i$ and $v_j$. The minimum number of colors required such that the edges of $C_{m, n}$ with a common vertex get different colors is 12. Give such a coloring.

Output the answer as a list of two elements: one matrix between \verb|\begin{array}{...}| and \verb|\end{array}| and one list inside of $\boxed{...}$. A number indicates the color of the edge. The $i$-th element of the $j$-th row in the matrix indicates the color of the edge between $u_j$ and $v_i$. The $i$-th element of the list indicates the color of the edge between $v_i$ and $v_{(i+1) \text{mod}(\text{len}(v_i))}$.

For instance, the following is a valid solution for $C_{{5, 3}}$ with 7 colors:

$\boxed{
\begin{array}{ccc}
3 & 2 & 7 \\
2 & 6 & 4 \\
5 & 1 & 2 \\
4 & 3 & 1 \\
1 & 4 & 3 \\
\end{array},
(7, 5, 6)
}$



\textbf{Revised Problem}\\
Consider the crown graph $C_{m, n}$ where $n = 9$ and $m = 10$. This graph is formed from an $n$-cycle with vertices labeled $v_i$ and $m$ isolated vertices labeled $u_j$. Each vertex $u_i$ is connected to every vertex $v_j$ by an edge. Determine the least number of colors needed to color the edges such that no two edges sharing a vertex have the same color. It has been established that 12 colors are sufficient. Provide an example of such a coloring.

Present your solution as a list containing two elements: a matrix formatted between \verb|\begin{array}{...}| and \verb|\end{array}|, and a list within $\boxed{...}$. Each number denotes the color assigned to an edge. The $i$-th entry of the $j$-th row in the matrix represents the color of the edge between $u_j$ and $v_i$. In the list, the $i$-th entry specifies the color of the edge connecting $v_i$ and $v_{(i+1) \text{mod}(\text{len}(v_i))}$.

For example, a solution for $C_{{5, 3}}$ using 7 colors is:

$\boxed{
\begin{array}{ccc}
3 & 2 & 7 \\
2 & 6 & 4 \\
5 & 1 & 2 \\
4 & 3 & 1 \\
1 & 4 & 3 \\
\end{array},
(7, 5, 6)
}$

\subsubsection{Variation}
\textbf{Actual Problem}\\
Define the crown graph $C_{m, n}$ with $n = 5$ and $m = 9$ to consist of an $n$-cycle with vertices $v_i$ and $m$ additional vertices $u_j$ that are initially isolated. Then edges are added between each $u_i$ and $v_j$. The minimum number of colors required such that the edges of $C_{m, n}$ with a common vertex get different colors is 11. Give such a coloring.

Output the answer as a list of two elements: one matrix between \verb|\begin{array}{...}| and \verb|\end{array}| and one list inside of $\boxed{...}$. A number indicates the color of the edge. The $i$-th element of the $j$-th row in the matrix indicates the color of the edge between $u_j$ and $v_i$. The $i$-th element of the list indicates the color of the edge between $v_i$ and $v_{(i+1) \text{mod}(\text{len}(v_i))}$.

For instance, the following is a valid solution for $C_{{5, 3}}$ with 7 colors:

$\boxed{
\begin{array}{ccc}
3 & 2 & 7 \\
2 & 6 & 4 \\
5 & 1 & 2 \\
4 & 3 & 1 \\
1 & 4 & 3 \\
\end{array},
(7, 5, 6)
}$



\textbf{Revised Problem}\\
Consider the crown graph $C_{m, n}$, where $n = 5$ and $m = 9$. This graph is formed by starting with a cycle of $n$ vertices labeled as $v_i$, augmented by $m$ additional isolated vertices labeled as $u_j$. We then connect each $u_j$ to every $v_i$. Determine the smallest number of colors required to color the edges of $C_{m, n}$ such that no two edges sharing a vertex have the same color. It is known that 11 colors are sufficient for this configuration. Provide an example of such an edge coloring.

Present the solution as a pair consisting of a matrix enclosed by \verb|\begin{array}{...}| and \verb|\end{array}|, and a list within $\boxed{...}$. Each number represents the color assigned to an edge. The element in the $i$-th row and $j$-th column of the matrix denotes the color of the edge connecting vertex $u_j$ and vertex $v_i$. The $i$-th entry of the list specifies the color of the edge connecting vertex $v_i$ to vertex $v_{(i+1) \bmod n}$.

For example, a valid solution for $C_{5, 3}$ using 7 colors would be:

$\boxed{
\begin{array}{ccc}
3 & 2 & 7 \\
2 & 6 & 4 \\
5 & 1 & 2 \\
4 & 3 & 1 \\
1 & 4 & 3 \\
\end{array},
(7, 5, 6)
}$

\subsubsection{Variation}
\textbf{Actual Problem}\\
Define the crown graph $C_{m, n}$ with $n = 3$ and $m = 4$ to consist of an $n$-cycle with vertices $v_i$ and $m$ additional vertices $u_j$ that are initially isolated. Then edges are added between each $u_i$ and $v_j$. The minimum number of colors required such that the edges of $C_{m, n}$ with a common vertex get different colors is 6. Give such a coloring.

Output the answer as a list of two elements: one matrix between \verb|\begin{array}{...}| and \verb|\end{array}| and one list inside of $\boxed{...}$. A number indicates the color of the edge. The $i$-th element of the $j$-th row in the matrix indicates the color of the edge between $u_j$ and $v_i$. The $i$-th element of the list indicates the color of the edge between $v_i$ and $v_{(i+1) \text{mod}(\text{len}(v_i))}$.

For instance, the following is a valid solution for $C_{{5, 3}}$ with 7 colors:

$\boxed{
\begin{array}{ccc}
3 & 2 & 7 \\
2 & 6 & 4 \\
5 & 1 & 2 \\
4 & 3 & 1 \\
1 & 4 & 3 \\
\end{array},
(7, 5, 6)
}$



\textbf{Revised Problem}\\
Consider the crown graph \( C_{m, n} \) where \( n = 3 \) and \( m = 4 \). This graph consists of a cycle with \( n \) vertices labeled \( v_i \) and \( m \) additional isolated vertices labeled \( u_j \). Edges are then added connecting each \( u_i \) to each \( v_j \). Determine the minimum number of colors needed to color the edges such that no two edges incident to the same vertex have the same color. The minimum number of colors required for this edge coloring is 6. Provide an example of such a coloring scheme.

Present your solution as a list containing two parts: a matrix using \verb|\begin{array}{...}| and \verb|\end{array}|, and a list enclosed in $\boxed{...}$. Each number corresponds to a color assigned to an edge. In the matrix, the \( i \)-th element of the \( j \)-th row represents the color of the edge between \( u_j \) and \( v_i \). Each element in the list denotes the color for the edge between \( v_i \) and \( v_{(i+1) \text{mod}(\text{len}(v_i))} \).

For example, a valid solution for \( C_{5, 3} \) using 7 colors could be represented as:

$\boxed{
\begin{array}{ccc}
3 & 2 & 7 \\
2 & 6 & 4 \\
5 & 1 & 2 \\
4 & 3 & 1 \\
1 & 4 & 3 \\
\end{array},
(7, 5, 6)
}$

\subsection{konhauser-2019-1}
\subsubsection{Variation}
\textbf{Actual Problem}\\
A professor is teaching a projective geometry class with 13 students in it. She wishes for the students (numbered from 1 to 13) to complete 13 group projects, where four students will work together on each project. The assignments must satisfy the following additional rules:

\begin{enumerate}
    \item Each pair of students is assigned to work on exactly one project together.
    \item If $i$ and $j$ are any two numbers between $1$ and $13$ (inclusive), then student $i$ works on project $j$ if and only if student $j$ works on project $i$.
    \item The professor has already assigned students to projects 1, 4, 7, 10, and 13.
\end{enumerate}

Find a valid assignment of students to the remaining eight, and record those assignments in the table below.

1: 2,3,4,5 \\
2: Not assigned yet \\
3: Not assigned yet \\
4: 1,5,10,11 \\
5: Not assigned yet \\
6: Not assigned yet \\
7: 2,6,10,12 \\
8: Not assigned yet \\
9: Not assigned yet \\
10: 4,7,8,10 \\
11: Not assigned yet \\
12: Not assigned yet \\
13: 5,6,8,13 \\

Output the answer as a comma separated list of lists inside of $\boxed{...}$. For instance, $\boxed{(1,2,3),(4,5,6)}$. The $i$-th element of your list should indicate the students assigned to project $i$. Thus, you should include the already assigned projects as well.

\textbf{Revised Problem}\\
A professor is conducting a projective geometry course with 13 students, labeled 1 through 13. She intends to have the students complete 13 projects, each involving a group of four students. The distribution must adhere to the following criteria:

\begin{enumerate}
    \item Every pair of students must collaborate on precisely one project.
    \item If students $i$ and $j$ are any two numbers from 1 to 13, student $i$ participates in project $j$ if and only if student $j$ participates in project $i$.
    \item The professor has already pre-determined which students will be in projects 1, 4, 7, 10, and 13.
\end{enumerate}

Determine how to appropriately assign students to the remaining eight projects, and fill in the table below with these assignments.

1: 2,3,4,5 \\
2: Unassigned \\
3: Unassigned \\
4: 1,5,10,11 \\
5: Unassigned \\
6: Unassigned \\
7: 2,6,10,12 \\
8: Unassigned \\
9: Unassigned \\
10: 4,7,8,10 \\
11: Unassigned \\
12: Unassigned \\
13: 5,6,8,13 \\

Present the solution as a list of lists enclosed in $\boxed{...}$, using commas to separate each sub-list. For example, $\boxed{(1,2,3),(4,5,6)}$. The list's $i$-th element should represent the students assigned to project $i$, including the projects that have already been assigned.

\subsubsection{Variation}
\textbf{Actual Problem}\\
A professor is teaching a projective geometry class with 13 students in it. She wishes for the students (numbered from 1 to 13) to complete 13 group projects, where four students will work together on each project. The assignments must satisfy the following additional rules:

\begin{enumerate}
    \item Each pair of students is assigned to work on exactly one project together.
    \item If $i$ and $j$ are any two numbers between $1$ and $13$ (inclusive), then student $i$ works on project $j$ if and only if student $j$ works on project $i$.
    \item The professor has already assigned students to projects 2, 6, 4, 5, and 7.
\end{enumerate}

Find a valid assignment of students to the remaining eight, and record those assignments in the table below.

1: Not assigned yet \\
2: 11,10,6,12 \\
3: Not assigned yet \\
4: 11,3,5,13 \\
5: 6,4,8,5 \\
6: 2,12,5,1 \\
7: 12,3,8,7 \\
8: Not assigned yet \\
9: Not assigned yet \\
10: Not assigned yet \\
11: Not assigned yet \\
12: Not assigned yet \\
13: Not assigned yet \\

Output the answer as a comma separated list of lists inside of $\boxed{...}$. For instance, $\boxed{(1,2,3),(4,5,6)}$. The $i$-th element of your list should indicate the students assigned to project $i$. Thus, you should include the already assigned projects as well.

\textbf{Revised Problem}\\
A professor is overseeing a projective geometry course with 13 students enrolled, labeled from 1 to 13. She intends for the students to collaborate on 13 different group projects, each involving four students. The assignment of students to projects is subject to these conditions:

\begin{enumerate}
    \item Every pair of students must be assigned to exactly one project together.
    \item For any two numbers $i$ and $j$ ranging from 1 to 13, student $i$ is assigned to project $j$ if and only if student $j$ is part of project $i$.
    \item The projects numbered 2, 6, 4, 5, and 7 already have some students assigned.
\end{enumerate}

Determine a suitable assignment for students to the remaining eight projects, and record these in the table provided below.

1: Not assigned yet \\
2: 11,10,6,12 \\
3: Not assigned yet \\
4: 11,3,5,13 \\
5: 6,4,8,5 \\
6: 2,12,5,1 \\
7: 12,3,8,7 \\
8: Not assigned yet \\
9: Not assigned yet \\
10: Not assigned yet \\
11: Not assigned yet \\
12: Not assigned yet \\
13: Not assigned yet \\

Express your solution as a list of lists, separated by commas, within $\boxed{...}$. For example, $\boxed{(1,2,3),(4,5,6)}$. The list's $i$-th element should represent the students assigned to project $i$, including those projects that have pre-assigned students.

\subsubsection{Variation}
\textbf{Actual Problem}\\
A professor is teaching a projective geometry class with 13 students in it. She wishes for the students (numbered from 1 to 13) to complete 13 group projects, where four students will work together on each project. The assignments must satisfy the following additional rules:

\begin{enumerate}
    \item Each pair of students is assigned to work on exactly one project together.
    \item If $i$ and $j$ are any two numbers between $1$ and $13$ (inclusive), then student $i$ works on project $j$ if and only if student $j$ works on project $i$.
    \item The professor has already assigned students to projects 9, 12, 7, 5, and 3.
\end{enumerate}

Find a valid assignment of students to the remaining eight, and record those assignments in the table below.

1: Not assigned yet \\
2: Not assigned yet \\
3: 6,4,8,3 \\
4: Not assigned yet \\
5: 12,7,8,5 \\
6: Not assigned yet \\
7: 13,4,5,10 \\
8: Not assigned yet \\
9: 13,1,12,6 \\
10: Not assigned yet \\
11: Not assigned yet \\
12: 9,6,5,2 \\
13: Not assigned yet \\

Output the answer as a comma separated list of lists inside of $\boxed{...}$. For instance, $\boxed{(1,2,3),(4,5,6)}$. The $i$-th element of your list should indicate the students assigned to project $i$. Thus, you should include the already assigned projects as well.

\textbf{Revised Problem}\\
A professor is conducting a projective geometry course with 13 students enrolled, numbered from 1 to 13. She plans for them to complete 13 team projects, each involving groups of four students. The project assignments must adhere to the following criteria:

1. Every pair of students must collaborate on exactly one project together.
2. If students $i$ and $j$ are chosen for any numbers between $1$ and $13$ (inclusive), then student $i$ is assigned to project $j$ if and only if student $j$ is assigned to project $i$.
3. The professor has already determined the student groups for projects 9, 12, 7, 5, and 3.

Determine a valid distribution of students for the remaining eight projects, and fill in the table as follows:

1: To be determined \\
2: To be determined \\
3: 6,4,8,3 \\
4: To be determined \\
5: 12,7,8,5 \\
6: To be determined \\
7: 13,4,5,10 \\
8: To be determined \\
9: 13,1,12,6 \\
10: To be determined \\
11: To be determined \\
12: 9,6,5,2 \\
13: To be determined \\

Express your solution as a series of comma-separated lists within $\boxed{...}$. For example, $\boxed{(1,2,3),(4,5,6)}$. The $i$-th entry in your list should specify the students assigned to project $i$. Ensure to include the projects that have already been assigned.

\subsubsection{Variation}
\textbf{Actual Problem}\\
A professor is teaching a projective geometry class with 13 students in it. She wishes for the students (numbered from 1 to 13) to complete 13 group projects, where four students will work together on each project. The assignments must satisfy the following additional rules:

\begin{enumerate}
    \item Each pair of students is assigned to work on exactly one project together.
    \item If $i$ and $j$ are any two numbers between $1$ and $13$ (inclusive), then student $i$ works on project $j$ if and only if student $j$ works on project $i$.
    \item The professor has already assigned students to projects 10, 13, 9, 6, and 1.
\end{enumerate}

Find a valid assignment of students to the remaining eight, and record those assignments in the table below.

1: 11,8,5,1 \\
2: Not assigned yet \\
3: Not assigned yet \\
4: Not assigned yet \\
5: Not assigned yet \\
6: 13,9,5,6 \\
7: Not assigned yet \\
8: Not assigned yet \\
9: 4,8,6,2 \\
10: 4,7,13,11 \\
11: Not assigned yet \\
12: Not assigned yet \\
13: 10,11,6,12 \\

Output the answer as a comma separated list of lists inside of $\boxed{...}$. For instance, $\boxed{(1,2,3),(4,5,6)}$. The $i$-th element of your list should indicate the students assigned to project $i$. Thus, you should include the already assigned projects as well.

\textbf{Revised Problem}\\
In a projective geometry course, a professor has 13 students labeled from 1 to 13. She intends to organize them into 13 separate group projects, with each project involving four students. The following conditions must be met:

\begin{enumerate}
    \item Every pair of students collaborates in precisely one project.
    \item For any pair of students numbered $i$ and $j$ (where $1 \leq i, j \leq 13$), student $i$ is assigned to project $j$ if and only if student $j$ is assigned to project $i$.
    \item Projects 10, 13, 9, 6, and 1 already have student assignments.
\end{enumerate}

Determine a suitable assignment for students to the remaining eight projects and display these assignments in the table below.

1: 11,8,5,1 \\
2: Not assigned yet \\
3: Not assigned yet \\
4: Not assigned yet \\
5: Not assigned yet \\
6: 13,9,5,6 \\
7: Not assigned yet \\
8: Not assigned yet \\
9: 4,8,6,2 \\
10: 4,7,13,11 \\
11: Not assigned yet \\
12: Not assigned yet \\
13: 10,11,6,12 \\

Present your solution as a comma separated list of lists encapsulated within $\boxed{...}$. For example, $\boxed{(1,2,3),(4,5,6)}$. Each element in your list should correspond to the students assigned to the respective project number. Ensure to include the already assigned projects in your final output.

\subsection{konhauser-2020-9}
\subsubsection{Variation}
\textbf{Actual Problem}\\
The following problem was intended to be on the 2020 Konhauser, but when we received it, the ink was smudged on a crucial piece of information.

'Find distinct positive integers $a, b, c, d, e, f$ such that

$$
\begin{aligned}
 \text{lcm}(a, b, c) = 60, \quad \text{lcm}(b, c, d) = 540, \quad \text{lcm}(c, d, e) = 135 \\
   \text{lcm}(d, e, f) = 5454, \quad \text{lcm}(e, f, a) = 1212, \quad \text{lcm}(f, a, b) = \text{illegible}.
\end{aligned}
$$
Here lcm denotes the least common multiple of a set of integers.'

Find a possible value for the last least common multiple along with the integers $(a,b,c,d,e,f)$ that satisfy the conditions.

Output the answer as a comma separated list inside of $\boxed{...}$. For example $\boxed{1, 2, 3}$.
The first number should be the last least common multiple and the numbers after need to be $a,b,c,d,e,f$ in order.

\textbf{Revised Problem}\\
The following problem was supposed to appear in the 2020 Konhauser contest, but a critical part of it became unreadable due to smudged ink.

"Identify distinct positive integers $a, b, c, d, e, f$ such that:

$$
\begin{aligned}
 \text{lcm}(a, b, c) = 60, \quad \text{lcm}(b, c, d) = 540, \quad \text{lcm}(c, d, e) = 135 \\
   \text{lcm}(d, e, f) = 5454, \quad \text{lcm}(e, f, a) = 1212, \quad \text{lcm}(f, a, b) = \text{unreadable}.
\end{aligned}
$$

Here, lcm represents the least common multiple of a group of numbers."

Determine a feasible value for the unreadable least common multiple along with the integers $(a, b, c, d, e, f)$ that fulfill the given conditions.

Provide the solution as a comma-separated sequence enclosed in $\boxed{...}$. For instance, use $\boxed{1, 2, 3}$.
The initial number should correspond to the unreadable least common multiple, followed by the integers $a, b, c, d, e, f$ in that sequence.

\subsubsection{Variation}
\textbf{Actual Problem}\\
The following problem was intended to be on the 2020 Konhauser, but when we received it, the ink was smudged on a crucial piece of information.

'Find distinct positive integers $a, b, c, d, e, f$ such that

$$
\begin{aligned}
 \text{lcm}(a, b, c) = 7011451, \quad \text{lcm}(b, c, d) = 65970742459, \quad \text{lcm}(c, d, e) = 39244939 \\
   \text{lcm}(d, e, f) = 112258779, \quad \text{lcm}(e, f, a) = 489171, \quad \text{lcm}(f, a, b) = \text{illegible}.
\end{aligned}
$$
Here lcm denotes the least common multiple of a set of integers.'

Find a possible value for the last least common multiple along with the integers $(a,b,c,d,e,f)$ that satisfy the conditions.

Output the answer as a comma separated list inside of $\boxed{...}$. For example $\boxed{1, 2, 3}$.
The first number should be the last least common multiple and the numbers after need to be $a,b,c,d,e,f$ in order.

\textbf{Revised Problem}\\
The following problem was meant for the 2020 Konhauser competition, but unfortunately, a crucial detail was obscured due to smudged ink.

"Identify six distinct positive integers $a, b, c, d, e, f$ such that:

$$
\begin{aligned}
 \text{lcm}(a, b, c) = 7011451, \quad \text{lcm}(b, c, d) = 65970742459, \quad \text{lcm}(c, d, e) = 39244939, \\
   \text{lcm}(d, e, f) = 112258779, \quad \text{lcm}(e, f, a) = 489171, \quad \text{lcm}(f, a, b) = \text{unreadable}.
\end{aligned}
$$

where lcm represents the least common multiple of the numbers involved."

Determine a possible value for the unreadable least common multiple and provide the integers $(a, b, c, d, e, f)$ that fulfill all the given conditions.

Present the answer as a list separated by commas within $\boxed{...}$. For instance, like $\boxed{1, 2, 3}$. The first entry should be the unreadable least common multiple, followed by the integers $a, b, c, d, e, f$ in sequence.

\subsubsection{Variation}
\textbf{Actual Problem}\\
The following problem was intended to be on the 2020 Konhauser, but when we received it, the ink was smudged on a crucial piece of information.

'Find distinct positive integers $a, b, c, d, e, f$ such that

$$
\begin{aligned}
 \text{lcm}(a, b, c) = 786379, \quad \text{lcm}(b, c, d) = 3530055331, \quad \text{lcm}(c, d, e) = 29174011 \\
   \text{lcm}(d, e, f) = 16541965, \quad \text{lcm}(e, f, a) = 40535, \quad \text{lcm}(f, a, b) = \text{illegible}.
\end{aligned}
$$
Here lcm denotes the least common multiple of a set of integers.'

Find a possible value for the last least common multiple along with the integers $(a,b,c,d,e,f)$ that satisfy the conditions.

Output the answer as a comma separated list inside of $\boxed{...}$. For example $\boxed{1, 2, 3}$.
The first number should be the last least common multiple and the numbers after need to be $a,b,c,d,e,f$ in order.

\textbf{Revised Problem}\\
The following problem was meant to be included in the 2020 Konhauser, but when it was retrieved, the ink had blurred a vital detail.

'Determine the distinct positive integers $a, b, c, d, e, f$ such that the following conditions are met:

$$
\begin{aligned}
 \text{lcm}(a, b, c) = 786379, \quad \text{lcm}(b, c, d) = 3530055331, \quad \text{lcm}(c, d, e) = 29174011 \\
   \text{lcm}(d, e, f) = 16541965, \quad \text{lcm}(e, f, a) = 40535, \quad \text{lcm}(f, a, b) = \text{unreadable}.
\end{aligned}
$$
where lcm represents the least common multiple of the given integers.'

Identify a potential value for the unreadable least common multiple along with the integers $(a, b, c, d, e, f)$ that satisfy these conditions.

Express the solution as a list of numbers separated by commas enclosed within $\boxed{...}$. For instance, $\boxed{1, 2, 3}$.
The initial number should be the missing least common multiple, followed by the values of $a, b, c, d, e, f$ in sequence.

\subsubsection{Variation}
\textbf{Actual Problem}\\
The following problem was intended to be on the 2020 Konhauser, but when we received it, the ink was smudged on a crucial piece of information.

'Find distinct positive integers $a, b, c, d, e, f$ such that

$$
\begin{aligned}
 \text{lcm}(a, b, c) = 1188239, \quad \text{lcm}(b, c, d) = 9412041119, \quad \text{lcm}(c, d, e) = 55692551 \\
   \text{lcm}(d, e, f) = 265773313, \quad \text{lcm}(e, f, a) = 436189, \quad \text{lcm}(f, a, b) = \text{illegible}.
\end{aligned}
$$
Here lcm denotes the least common multiple of a set of integers.'

Find a possible value for the last least common multiple along with the integers $(a,b,c,d,e,f)$ that satisfy the conditions.

Output the answer as a comma separated list inside of $\boxed{...}$. For example $\boxed{1, 2, 3}$.
The first number should be the last least common multiple and the numbers after need to be $a,b,c,d,e,f$ in order.

\textbf{Revised Problem}\\
The following puzzle was meant to be included in the 2020 Konhauser, but a key piece of data was obscured by ink smudges.

'Identify distinct positive integers $a, b, c, d, e, f$ that fulfill the following conditions:

$$
\begin{aligned}
 \text{lcm}(a, b, c) &= 1188239, \\
 \text{lcm}(b, c, d) &= 9412041119, \\
 \text{lcm}(c, d, e) &= 55692551, \\
 \text{lcm}(d, e, f) &= 265773313, \\
 \text{lcm}(e, f, a) &= 436189, \\
 \text{lcm}(f, a, b) &= \text{obscured}.
\end{aligned}
$$
The term lcm represents the least common multiple of a group of integers.'

Determine a plausible value for the obscured least common multiple, as well as the integers $(a, b, c, d, e, f)$ that satisfy these criteria.

Present your solution as a sequence of values separated by commas enclosed in $\boxed{...}$. For instance, $\boxed{1, 2, 3}$. The first entry should be the determined least common multiple, followed by the integers $a, b, c, d, e, f$ in sequence.

\subsection{konhauser-2021-10}
\subsubsection{Variation}
\textbf{Actual Problem}\\
Give 5 distinct 10-digit numbers $n$ such that the binomial coefficient $\binom{2n}{n}$ is divisible by $2^{10}$ but not by $2^{11}$. Recall that $\binom{2n}{n} = \frac{2n!}{n!^2}$ and $n! = n(n-1)...3\cdot 2 \cdot 1$.

Output the answer as a comma separated list inside of $\boxed{...}$. For example $\boxed{1, 2, 3}$.

\textbf{Revised Problem}\\
Identify 5 unique 10-digit integers \( n \) such that the binomial coefficient \(\binom{2n}{n}\) is divisible by \( 2^{10} \) but not by \( 2^{11} \). Recall that \(\binom{2n}{n} = \frac{(2n)!}{(n!)^2}\) and that \( n! = n(n-1)\cdots 3 \cdot 2 \cdot 1\).

Provide the solution as a list of numbers separated by commas enclosed within a box, for example, \(\boxed{1, 2, 3}\).

\subsubsection{Variation}
\textbf{Actual Problem}\\
Give 18 distinct 20-digit numbers $n$ such that the binomial coefficient $\binom{2n}{n}$ is divisible by $2^{51}$ but not by $2^{52}$. Recall that $\binom{2n}{n} = \frac{2n!}{n!^2}$ and $n! = n(n-1)...3\cdot 2 \cdot 1$.

Output the answer as a comma separated list inside of $\boxed{...}$. For example $\boxed{1, 2, 3}$.

\textbf{Revised Problem}\\
Identify 18 unique 20-digit integers \( n \) such that the binomial coefficient \( \binom{2n}{n} \) can be divided by \( 2^{51} \) but not by \( 2^{52} \). Remember that the expression for the binomial coefficient is given by \( \binom{2n}{n} = \frac{(2n)!}{(n!)^2} \), and the factorial \( n! \) is the product of all positive integers up to \( n \).

Present your solution as a list of numbers separated by commas enclosed within a \(\boxed{...}\). For instance, \(\boxed{1, 2, 3}\).

\subsubsection{Variation}
\textbf{Actual Problem}\\
Give 13 distinct 12-digit numbers $n$ such that the binomial coefficient $\binom{2n}{n}$ is divisible by $2^{7}$ but not by $2^{8}$. Recall that $\binom{2n}{n} = \frac{2n!}{n!^2}$ and $n! = n(n-1)...3\cdot 2 \cdot 1$.

Output the answer as a comma separated list inside of $\boxed{...}$. For example $\boxed{1, 2, 3}$.

\textbf{Revised Problem}\\
Identify 13 unique numbers, each with 12 digits, denoted as $n$, such that the binomial coefficient $\binom{2n}{n}$ is divisible by $2^{7}$ but not by $2^{8}$. Remember, the binomial coefficient is defined as $\binom{2n}{n} = \frac{(2n)!}{(n!)^2}$, where $n!$ is the factorial of $n$, calculated as $n(n-1)(n-2)...3 \cdot 2 \cdot 1$.

Present the answer as a list of numbers separated by commas within $\boxed{...}$. For instance, $\boxed{1, 2, 3}$.

\subsubsection{Variation}
\textbf{Actual Problem}\\
Give 7 distinct 9-digit numbers $n$ such that the binomial coefficient $\binom{2n}{n}$ is divisible by $2^{5}$ but not by $2^{6}$. Recall that $\binom{2n}{n} = \frac{2n!}{n!^2}$ and $n! = n(n-1)...3\cdot 2 \cdot 1$.

Output the answer as a comma separated list inside of $\boxed{...}$. For example $\boxed{1, 2, 3}$.

\textbf{Revised Problem}\\
Identify 7 unique 9-digit integers $n$ such that the binomial coefficient $\binom{2n}{n}$ is divisible by $2^{5}$ but not by $2^{6}$. Keep in mind that the binomial coefficient is defined as $\binom{2n}{n} = \frac{2n!}{n!^2}$, where $n! = n(n-1)\cdots3\cdot2\cdot1$.

Present the solution as a list of numbers separated by commas and enclosed within $\boxed{...}$. For instance, $\boxed{1, 2, 3}$.

\subsection{konhauser-2021-6}
\subsubsection{Variation}
\textbf{Actual Problem}\\
Find a differentiable function $f \colon \mathbb{R} \to \mathbb{R}$ such that $f(2x) - f(x) = 20x^2 - 21x$.

Output the answer as a valid LaTex expression in \boxed{...}. Do not include $f(x)$ or dollar signs. For example, \boxed{\frac{3}{4}x + 2}

\textbf{Revised Problem}\\
Identify a differentiable function \( f: \mathbb{R} \rightarrow \mathbb{R} \) that satisfies the condition \( f(2x) - f(x) = 20x^2 - 21x \).

Present your solution as a legitimate LaTeX expression enclosed in \boxed{...}. Exclude \( f(x) \) from your response and omit any dollar signs. For instance, \boxed{\frac{3}{4}x + 2}

\subsubsection{Variation}
\textbf{Actual Problem}\\
Find a differentiable function $f \colon \mathbb{R} \to \mathbb{R}$ such that $f(5x) - f(x) = 25x^2 - 49x$.

Output the answer as a valid LaTex expression in \boxed{...}. Do not include $f(x)$ or dollar signs. For example, \boxed{\frac{3}{4}x + 2}

\textbf{Revised Problem}\\
Identify a differentiable function \( f \colon \mathbb{R} \to \mathbb{R} \) for which \( f(5x) - f(x) = 25x^2 - 49x \).

Present your solution as a correct LaTeX expression using \boxed{...}. Ensure you do not include \( f(x) \) or dollar symbols. For instance, \boxed{\frac{3}{4}x + 2}.

\subsubsection{Variation}
\textbf{Actual Problem}\\
Find a differentiable function $f \colon \mathbb{R} \to \mathbb{R}$ such that $f(2x) - f(x) = 9x^2 - 37x$.

Output the answer as a valid LaTex expression in \boxed{...}. Do not include $f(x)$ or dollar signs. For example, \boxed{\frac{3}{4}x + 2}

\textbf{Revised Problem}\\
Identify a differentiable function $f: \mathbb{R} \rightarrow \mathbb{R}$ such that the equation $f(2x) - f(x) = 9x^2 - 37x$ holds true.

Present the solution as a valid LaTeX expression enclosed in \boxed{...}. Exclude $f(x)$ and do not use dollar signs. For instance, \boxed{\frac{3}{4}x + 2}

\subsubsection{Variation}
\textbf{Actual Problem}\\
Find a differentiable function $f \colon \mathbb{R} \to \mathbb{R}$ such that $f(2x) - f(x) = 4x^2 - 6x$.

Output the answer as a valid LaTex expression in \boxed{...}. Do not include $f(x)$ or dollar signs. For example, \boxed{\frac{3}{4}x + 2}

\textbf{Revised Problem}\\
Determine a continuously differentiable function $f: \mathbb{R} \to \mathbb{R}$ that satisfies the equation $f(2x) - f(x) = 4x^2 - 6x$ for every real number $x$.

Present your solution as a valid LaTeX expression inside \boxed{...}. Avoid including $f(x)$ or dollar symbols. For instance, use \boxed{\frac{3}{4}x + 2}.

\subsection{konhauser-2023-3}
\subsubsection{Variation}
\textbf{Actual Problem}\\
A Lockridge Pyramid is a triangular array of numbers where each entry (except in the bottom row) is the sum of the two entries below it. For instance,

$\begin{array}{cccc}
    20 \\
    9 & 11 \\
    3 &  6 & 5 \\
    2 &  1 & 5 & 0 \\
\end{array}$

Generate 14 Lockridge Pyramids so that the bottom row consists of only 0s and 1s, and the number at the top of the pyramid is 10. Output the bottom row for each of these pyramids.

Output the bottom rows of these pyramids as a comma separated list of lists inside of \boxed{...}. For example, \boxed{(1,1,0),(0,1,0),(0,0,1,1)}

\textbf{Revised Problem}\\
A Lockridge Pyramid is a triangular arrangement of numbers where each number (except those in the lowest row) is derived by adding the two numbers directly beneath it. For example, consider the pyramid:

$\begin{array}{cccc}
    20 \\
    9 & 11 \\
    3 &  6 & 5 \\
    2 &  1 & 5 & 0 \\
\end{array}$

Create 14 distinct Lockridge Pyramids such that the lowest row consists only of 0s and 1s, and the number at the apex of the pyramid equals 10. Present the bottom row for each pyramid.

Display the bottom rows of these pyramids as a series of lists separated by commas, encapsulated within \boxed{...}. For instance, \boxed{(1,1,0),(0,1,0),(0,0,1,1)}

\subsubsection{Variation}
\textbf{Actual Problem}\\
A Lockridge Pyramid is a triangular array of numbers where each entry (except in the bottom row) is the sum of the two entries below it. For instance,

$\begin{array}{cccc}
    20 \\
    9 & 11 \\
    3 &  6 & 5 \\
    2 &  1 & 5 & 0 \\
\end{array}$

Generate 11 Lockridge Pyramids so that the bottom row consists of only 0s and 1s, and the number at the top of the pyramid is 14. Output the bottom row for each of these pyramids.

Output the bottom rows of these pyramids as a comma separated list of lists inside of \boxed{...}. For example, \boxed{(1,1,0),(0,1,0),(0,0,1,1)}

\textbf{Revised Problem}\\
In a Lockridge Pyramid, each number in the triangle, except those in the bottom row, is calculated as the sum of the two numbers directly beneath it. For example:

$\begin{array}{cccc}
    20 \\
    9 & 11 \\
    3 &  6 & 5 \\
    2 &  1 & 5 & 0 \\
\end{array}$

Construct 11 Lockridge Pyramids where each bottom row consists exclusively of 0s and 1s, and the topmost number of each pyramid equals 14. Present the bottom rows for each pyramid.

List the bottom rows of these pyramids as a sequence of lists, separated by commas and enclosed within \boxed{...}. For instance, \boxed{(1,1,0),(0,1,0),(0,0,1,1)}

\subsubsection{Variation}
\textbf{Actual Problem}\\
A Lockridge Pyramid is a triangular array of numbers where each entry (except in the bottom row) is the sum of the two entries below it. For instance,

$\begin{array}{cccc}
    20 \\
    9 & 11 \\
    3 &  6 & 5 \\
    2 &  1 & 5 & 0 \\
\end{array}$

Generate 18 Lockridge Pyramids so that the bottom row consists of only 0s and 1s, and the number at the top of the pyramid is 11. Output the bottom row for each of these pyramids.

Output the bottom rows of these pyramids as a comma separated list of lists inside of \boxed{...}. For example, \boxed{(1,1,0),(0,1,0),(0,0,1,1)}

\textbf{Revised Problem}\\
In a Lockridge Pyramid, numbers are arranged in a triangular form where each number, except those in the last row, is the sum of the two numbers directly beneath it. Consider the example below:

$\begin{array}{cccc}
    20 \\
    9 & 11 \\
    3 &  6 & 5 \\
    2 &  1 & 5 & 0 \\
\end{array}$

Your task is to create 18 Lockridge Pyramids such that each pyramid's bottom row comprises only 0s and 1s, and the topmost number of each pyramid equals 11. Provide the bottom row for each pyramid.

Present the bottom rows of these pyramids as comma-separated elements within lists, enclosed by \boxed{...}. For instance, \boxed{(1,1,0),(0,1,0),(0,0,1,1)}

\subsubsection{Variation}
\textbf{Actual Problem}\\
A Lockridge Pyramid is a triangular array of numbers where each entry (except in the bottom row) is the sum of the two entries below it. For instance,

$\begin{array}{cccc}
    20 \\
    9 & 11 \\
    3 &  6 & 5 \\
    2 &  1 & 5 & 0 \\
\end{array}$

Generate 13 Lockridge Pyramids so that the bottom row consists of only 0s and 1s, and the number at the top of the pyramid is 10. Output the bottom row for each of these pyramids.

Output the bottom rows of these pyramids as a comma separated list of lists inside of \boxed{...}. For example, \boxed{(1,1,0),(0,1,0),(0,0,1,1)}

\textbf{Revised Problem}\\
A Lockridge Pyramid is a triangle-shaped arrangement of numbers where each number, except those in the last row, is the sum of the two numbers directly beneath it. Consider the following example:

$\begin{array}{cccc}
    20 \\
    9 & 11 \\
    3 &  6 & 5 \\
    2 &  1 & 5 & 0 \\
\end{array}$

Your task is to create 13 distinct Lockridge Pyramids where the lowest row contains only the digits 0 and 1, and the number at the very top of the pyramid equals 10. Provide the bottom row for each of these pyramids.

Present the bottom rows of these pyramids as a sequence of lists, with each list separated by commas and enclosed within \boxed{...}. For instance, \boxed{(1,1,0),(0,1,0),(0,0,1,1)}

\section{misc}
\subsection{misc-balticway-2005-19}
\subsubsection{Variation}
\textbf{Actual Problem}\\
Find $10$ unique sets of $15$ different positive square numbers such that their sum is also a square number.

Output the sequences as a list of comma-separated tuples of increasing integers inside of \boxed, e.g. \boxed{(1, 2, 3, 4, 5), (7, 10, 11, 12, 14), (2, 3, 4, 9, 10)}.

\textbf{Revised Problem}\\
Determine 10 distinct collections, each consisting of 15 unique positive perfect squares, such that the sum of the squares in each collection is itself a perfect square.

Present the solutions as a list of tuples containing integers in ascending order enclosed in \boxed, such as \boxed{(1, 2, 3, 4, 5), (7, 10, 11, 12, 14), (2, 3, 4, 9, 10)}.

\subsubsection{Variation}
\textbf{Actual Problem}\\
Find $9$ unique sets of $10$ different positive square numbers such that their sum is also a square number.

Output the sequences as a list of comma-separated tuples of increasing integers inside of \boxed, e.g. \boxed{(1, 2, 3, 4, 5), (7, 10, 11, 12, 14), (2, 3, 4, 9, 10)}.

\textbf{Revised Problem}\\
Identify $9$ distinct groups, each consisting of $10$ different positive integers that are perfect squares, such that the total of each group also results in a perfect square.

Present the sequences as a list of tuples where numbers are in ascending order, enclosed within \boxed, for example, \boxed{(1, 2, 3, 4, 5), (7, 10, 11, 12, 14), (2, 3, 4, 9, 10)}.

\subsubsection{Variation}
\textbf{Actual Problem}\\
Find $5$ unique sets of $7$ different positive square numbers such that their sum is also a square number.

Output the sequences as a list of comma-separated tuples of increasing integers inside of \boxed, e.g. \boxed{(1, 2, 3, 4, 5), (7, 10, 11, 12, 14), (2, 3, 4, 9, 10)}.

\textbf{Revised Problem}\\
Identify five distinct groups of seven different positive squares such that the total of each group is itself a square number.

Present the groups as a list of tuples of increasing numbers, separated by commas and enclosed within \boxed, for example, \boxed{(1, 4, 9, 16, 25), (49, 100, 121, 144, 169), (4, 9, 16, 81, 100)}.

\subsubsection{Variation}
\textbf{Actual Problem}\\
Find $3$ unique sets of $7$ different positive square numbers such that their sum is also a square number.

Output the sequences as a list of comma-separated tuples of increasing integers inside of \boxed, e.g. \boxed{(1, 2, 3, 4, 5), (7, 10, 11, 12, 14), (2, 3, 4, 9, 10)}.

\textbf{Revised Problem}\\
Identify $3$ distinct groups, each consisting of $7$ unique positive squares, such that the total sum of each group forms a perfect square.

Present the solutions as a sequence of comma-separated tuples containing integers in ascending order within \boxed, for example, \boxed{(1, 4, 9, 16, 25), (4, 9, 16, 25, 36), (1, 9, 16, 25, 49)}.

\subsection{misc-handout-nz1}
\subsubsection{Variation}
\textbf{Actual Problem}\\
The two pairs of consecutive natural numbers $(8,9)$ and $(288,289)$ have the following property: in each pair, each number contains each of its prime factors to a power not less than 2. There are infinitely many such pairs of consecutive natural numbers. 
Give 8 examples of such distinct pairs.

Output the answer as a list of numbers inside of $\boxed{...}$, where you should only give the first number for each pair. For example, $\boxed{[3, 6, 11]}$.

\textbf{Revised Problem}\\
Consider the pairs of consecutive natural numbers $(8,9)$ and $(288,289)$, where each number within a pair is a product of prime numbers each raised to a power of at least 2. There are infinitely many such pairs of consecutive integers. Identify 8 distinct pairs of consecutive numbers sharing this characteristic.

Present your solution as a list within $\boxed{...}$, displaying only the first number of each pair. For instance, $\boxed{[3, 6, 11]}$.

\subsubsection{Variation}
\textbf{Actual Problem}\\
The two pairs of consecutive natural numbers $(8,9)$ and $(288,289)$ have the following property: in each pair, each number contains each of its prime factors to a power not less than 2. There are infinitely many such pairs of consecutive natural numbers. 
Give 9 examples of such distinct pairs.

Output the answer as a list of numbers inside of $\boxed{...}$, where you should only give the first number for each pair. For example, $\boxed{[3, 6, 11]}$.

\textbf{Revised Problem}\\
Consider the consecutive natural number pairs $(8,9)$ and $(288,289)$. Each number within these pairs has the property that all of its prime factors are raised to a power of at least 2. There are infinitely many pairs of consecutive numbers with this property. Provide 9 examples of such unique pairs.

Present your answer as a list of numbers enclosed in $\boxed{...}$, including only the first number from each pair. For instance, $\boxed{[3, 6, 11]}$.

\subsubsection{Variation}
\textbf{Actual Problem}\\
The two pairs of consecutive natural numbers $(8,9)$ and $(288,289)$ have the following property: in each pair, each number contains each of its prime factors to a power not less than 2. There are infinitely many such pairs of consecutive natural numbers. 
Give 10 examples of such distinct pairs.

Output the answer as a list of numbers inside of $\boxed{...}$, where you should only give the first number for each pair. For example, $\boxed{[3, 6, 11]}$.

\textbf{Revised Problem}\\
Consider the pairs of consecutive natural numbers like $(8,9)$ and $(288,289)$, where both numbers in each pair have all their prime factors raised to a power of 2 or greater. There are infinitely many such pairs of consecutive natural numbers. 
Provide 10 distinct examples of these pairs.

Write your answer as a list of the first numbers in each pair, enclosed in $\boxed{...}$. For instance, if the pairs are $(3,4)$, $(6,7)$, $(11,12)$, you should write $\boxed{[3, 6, 11]}$.

\subsection{misc-vwo-2019-4}
\subsubsection{Variation}
\textbf{Actual Problem}\\
There are $n = 25$ hoops in a circle. Rik numbers all hoops with a natural number between $1$ and $n$ such that each number occurs exactly once. After that, he makes a walk from hoop to hoop. He starts in hoop with number 1 and follows the following rule: If he arrives in hoop $k$, he moves to the hoop that is $k$ hoops away in the clockwise direction and skips the intermediate hoops. The walk ends when Rik arrives in a hoop he has already visited. The length of the walk is the number of hoops he has visited along the way.

Give a numbering of the $25$ hoops such that the length of the walk is $25$.

Output the answer as a comma separated list inside of $\boxed{...}$. For example $\boxed{1, 2, 3}$.

\textbf{Revised Problem}\\
Consider a circle of $25$ hoops, each marked with a unique natural number from $1$ to $25$. Rik numbers the hoops so that each number appears exactly once. Starting at the hoop labeled with the number 1, he proceeds by moving to a hoop that is exactly the number of positions indicated by the current hoop's label in a clockwise direction, skipping all intermediate hoops. This walking sequence concludes when Rik revisits a previously encountered hoop. Determine a labeling of the $25$ hoops that results in a walk covering all $25$ hoops before returning to any hoop for a second time.

Present the solution as a sequence in a comma-separated list enclosed within $\boxed{...}$. For example, express the answer as $\boxed{1, 2, 3}$.

\subsubsection{Variation}
\textbf{Actual Problem}\\
There are $n = 49$ hoops in a circle. Rik numbers all hoops with a natural number between $1$ and $n$ such that each number occurs exactly once. After that, he makes a walk from hoop to hoop. He starts in hoop with number 1 and follows the following rule: If he arrives in hoop $k$, he moves to the hoop that is $k$ hoops away in the clockwise direction and skips the intermediate hoops. The walk ends when Rik arrives in a hoop he has already visited. The length of the walk is the number of hoops he has visited along the way.

Give a numbering of the $49$ hoops such that the length of the walk is $48$.

Output the answer as a comma separated list inside of $\boxed{...}$. For example $\boxed{1, 2, 3}$.

\textbf{Revised Problem}\\
Imagine a circular arrangement of $n = 49$ hoops, each uniquely labeled with a natural number from 1 to 49. Rik numbers each hoop such that each number is used exactly once. He begins his journey at the hoop labeled 1. Following a specific movement rule, whenever he reaches a hoop numbered $k$, he proceeds to the hoop located $k$ positions ahead in a clockwise manner, skipping over any hoops in between. The journey concludes upon reaching a hoop that has already been visited. Determine a labeling of the hoops that results in a journey visiting 48 different hoops.

Present your solution as a series of numbers separated by commas within $\boxed{...}$. For example, $\boxed{1, 2, 3}$.

\subsubsection{Variation}
\textbf{Actual Problem}\\
There are $n = 33$ hoops in a circle. Rik numbers all hoops with a natural number between $1$ and $n$ such that each number occurs exactly once. After that, he makes a walk from hoop to hoop. He starts in hoop with number 1 and follows the following rule: If he arrives in hoop $k$, he moves to the hoop that is $k$ hoops away in the clockwise direction and skips the intermediate hoops. The walk ends when Rik arrives in a hoop he has already visited. The length of the walk is the number of hoops he has visited along the way.

Give a numbering of the $33$ hoops such that the length of the walk is $32$.

Output the answer as a comma separated list inside of $\boxed{...}$. For example $\boxed{1, 2, 3}$.

\textbf{Revised Problem}\\
Consider a circular arrangement of $n = 33$ hoops, each labeled distinctly with numbers from $1$ to $n$. Rik labels each hoop so that each number appears exactly once. Beginning at the hoop numbered 1, Rik proceeds by moving 'k' hoops clockwise from his current position if he's on hoop 'k', skipping all intervening hoops. His walk terminates upon first revisiting a previously visited hoop. Determine an arrangement of numbers on the hoops such that this walk visits precisely 32 unique hoops.

Present your solution as a list of numbers separated by commas within $\boxed{...}$. For instance, $\boxed{1, 2, 3}$.

\subsubsection{Variation}
\textbf{Actual Problem}\\
There are $n = 28$ hoops in a circle. Rik numbers all hoops with a natural number between $1$ and $n$ such that each number occurs exactly once. After that, he makes a walk from hoop to hoop. He starts in hoop with number 1 and follows the following rule: If he arrives in hoop $k$, he moves to the hoop that is $k$ hoops away in the clockwise direction and skips the intermediate hoops. The walk ends when Rik arrives in a hoop he has already visited. The length of the walk is the number of hoops he has visited along the way.

Give a numbering of the $28$ hoops such that the length of the walk is $28$.

Output the answer as a comma separated list inside of $\boxed{...}$. For example $\boxed{1, 2, 3}$.

\textbf{Revised Problem}\\
Imagine a circle of $n = 28$ hoops, each labeled with a unique natural number from $1$ to $n$. Rik assigns these numbers such that no number repeats. He begins walking from the hoop marked with number 1. The rule for his movement is: upon reaching a hoop numbered $k$, Rik jumps $k$ hoops forward in a clockwise direction, ignoring the hoops in between. The walk concludes when Rik revisits a hoop for the first time. Determine a sequence for numbering the hoops so the walk encompasses all 28 hoops before terminating.

Present the sequence as a list of numbers separated by commas within $\boxed{...}$. For instance, $\boxed{1, 2, 3}$.

\section{putnam}
\subsection{putnam-2009-b6}
\subsubsection{Variation}
\textbf{Actual Problem}\\
Given a positive integer $n = 7$, find a sequence of integers $a_0, a_1, \ldots, a_{10}$ with $a_0 = 0$ and $a_{10} = n$ such that
each term after $a_0$ is either an earlier term plus $2^k$ for some non-negative integer $k$, or of the form $b$ mod $c$ for some earlier positive terms $b$ and $c$.
Here $b$ mod $c$ denotes the remainder when $b$ is divided by $c$, so $0 \leq (b \mod c) < c$.

Output the answer as a comma separated list inside of $\boxed{...}$. For example $\boxed{1, 2, 3}$.

\textbf{Revised Problem}\\
Given a positive integer $n = 7$, determine a sequence of integers $a_0, a_1, \ldots, a_{10}$ where $a_0 = 0$ and $a_{10} = n$. Each subsequent term after $a_0$ must either be the sum of a previous term and $2^k$ where $k$ is a non-negative integer, or it must be the result of $b$ modulo $c$, where $b$ and $c$ are prior positive terms in the sequence. Note that $b$ mod $c$ represents the remainder obtained when $b$ is divided by $c$, thus $0 \leq (b \mod c) < c$.

Present the solution as a series of numbers separated by commas, enclosed within $\boxed{...}$. For instance, $\boxed{1, 2, 3}$.

\subsubsection{Variation}
\textbf{Actual Problem}\\
Given a positive integer $n = 16$, find a sequence of integers $a_0, a_1, \ldots, a_{32}$ with $a_0 = 0$ and $a_{32} = n$ such that
each term after $a_0$ is either an earlier term plus $2^k$ for some non-negative integer $k$, or of the form $b$ mod $c$ for some earlier positive terms $b$ and $c$.
Here $b$ mod $c$ denotes the remainder when $b$ is divided by $c$, so $0 \leq (b \mod c) < c$.

Output the answer as a comma separated list inside of $\boxed{...}$. For example $\boxed{1, 2, 3}$.

\textbf{Revised Problem}\\
Given a positive integer $n = 16$, determine a sequence of integers $a_0, a_1, \ldots, a_{32}$ where $a_0 = 0$ and $a_{32} = n$. The sequence must be constructed such that each term after $a_0$ can be obtained either by adding $2^k$ (where $k$ is a non-negative integer) to an earlier term or by taking the result of $b$ mod $c$ from some preceding positive terms $b$ and $c$. Here, $b$ mod $c$ represents the remainder when $b$ is divided by $c$, satisfying $0 \leq (b \mod c) < c$.

Present your solution as a series of numbers separated by commas within a box: for instance, $\boxed{1, 2, 3}$.

\subsubsection{Variation}
\textbf{Actual Problem}\\
Given a positive integer $n = 5$, find a sequence of integers $a_0, a_1, \ldots, a_{16}$ with $a_0 = 0$ and $a_{16} = n$ such that
each term after $a_0$ is either an earlier term plus $2^k$ for some non-negative integer $k$, or of the form $b$ mod $c$ for some earlier positive terms $b$ and $c$.
Here $b$ mod $c$ denotes the remainder when $b$ is divided by $c$, so $0 \leq (b \mod c) < c$.

Output the answer as a comma separated list inside of $\boxed{...}$. For example $\boxed{1, 2, 3}$.

\textbf{Revised Problem}\\
Starting with the positive integer $n = 5$, determine a sequence of integers $a_0, a_1, \ldots, a_{16}$ such that $a_0 = 0$ and $a_{16} = n$ where:
each subsequent term after $a_0$ is obtained either by adding $2^k$ (where $k$ is a non-negative integer) to a prior term, or by computing $b$ mod $c$, where $b$ and $c$ are previous positive terms in the sequence. Here, $b$ mod $c$ represents the remainder when $b$ is divided by $c$, ensuring $0 \leq (b \mod c) < c$.

Present the result in a comma-separated format enclosed within $\boxed{...}$. For instance, $\boxed{1, 2, 3}$.

\subsubsection{Variation}
\textbf{Actual Problem}\\
Given a positive integer $n = 5$, find a sequence of integers $a_0, a_1, \ldots, a_{11}$ with $a_0 = 0$ and $a_{11} = n$ such that
each term after $a_0$ is either an earlier term plus $2^k$ for some non-negative integer $k$, or of the form $b$ mod $c$ for some earlier positive terms $b$ and $c$.
Here $b$ mod $c$ denotes the remainder when $b$ is divided by $c$, so $0 \leq (b \mod c) < c$.

Output the answer as a comma separated list inside of $\boxed{...}$. For example $\boxed{1, 2, 3}$.

\textbf{Revised Problem}\\
Consider a positive integer $n = 5$. Your task is to create a sequence of integers $a_0, a_1, \ldots, a_{11}$ where $a_0 = 0$ and $a_{11} = n$. Each number in the sequence after $a_0$ must be obtained by either adding $2^k$ (where $k$ is a non-negative integer) to any earlier number in the sequence, or by taking the remainder of dividing one earlier positive number by another. For the operation $b$ mod $c$, it results in the remainder of $b$ divided by $c$, satisfying $0 \leq (b \mod c) < c$.

Present your solution as a sequence in the format $\boxed{...}$, with the numbers separated by commas. For instance, $\boxed{1, 2, 3}$.

\subsection{putnam-2015-a2}
\subsubsection{Variation}
\textbf{Actual Problem}\\
Let $a_0=1$, $a_1=2$, and $a_n=4a_{n-1}-a_{n-2}$ for $n\geq 2$. Find an odd prime factor of $a_{2015}$.

Output the answer as a comma separated list inside of $\boxed{...}$. For example $\boxed{1, 2, 3}$.

\textbf{Revised Problem}\\
Consider a sequence defined by $b_0=1$, $b_1=2$, and for $n\geq 2$, $b_n=4b_{n-1}-b_{n-2}$. Determine an odd prime number that divides $b_{2015}$.

Present your answer as a list separated by commas within $\boxed{...}$. For instance, $\boxed{1, 2, 3}$.

\subsubsection{Variation}
\textbf{Actual Problem}\\
Let $a_0=1$, $a_1=2$, and $a_n=4a_{n-1}-a_{n-2}$ for $n\geq 2$. Find an odd prime factor of $a_{11685}$.

Output the answer as a comma separated list inside of $\boxed{...}$. For example $\boxed{1, 2, 3}$.

\textbf{Revised Problem}\\
Consider the sequence defined by $b_0=1$, $b_1=2$, and for $n\geq 2$, $b_n=4b_{n-1}-b_{n-2}$. Determine an odd prime number that divides $b_{11685}$.

Present your answer as a list of numbers separated by commas enclosed within $\boxed{...}$. For instance, $\boxed{1, 2, 3}$.

\subsubsection{Variation}
\textbf{Actual Problem}\\
Let $a_0=1$, $a_1=2$, and $a_n=4a_{n-1}-a_{n-2}$ for $n\geq 2$. Find an odd prime factor of $a_{4095}$.

Output the answer as a comma separated list inside of $\boxed{...}$. For example $\boxed{1, 2, 3}$.

\textbf{Revised Problem}\\
Consider the sequence defined by $b_0=1$, $b_1=2$, and for $n\geq 2$, $b_n=4b_{n-1}-b_{n-2}$. Determine an odd prime number that divides $b_{4095}$.

Present your answer as a list of numbers within $\boxed{...}$, separated by commas. For example, use the format $\boxed{1, 2, 3}$.

\subsubsection{Variation}
\textbf{Actual Problem}\\
Let $a_0=1$, $a_1=2$, and $a_n=4a_{n-1}-a_{n-2}$ for $n\geq 2$. Find an odd prime factor of $a_{285}$.

Output the answer as a comma separated list inside of $\boxed{...}$. For example $\boxed{1, 2, 3}$.

\textbf{Revised Problem}\\
Consider the sequence defined by $b_0 = 1$, $b_1 = 2$, and for $n \geq 2$, $b_n = 4b_{n-1} - b_{n-2}$. Determine an odd prime factor of $b_{285}$.

Present the solution as a list separated by commas, enclosed within $\boxed{...}$. For example, use the format $\boxed{1, 2, 3}$.

\subsection{putnam-2022-b4}
\subsubsection{Variation}
\textbf{Actual Problem}\\
Given $n = 21$, find a sequence of distinct real numbers $x_1, \ldots, x_n$ such that each of the sets$$\{x_1, x_2, x_3\}, \{x_2, x_3, x_4\},\ldots,\{x_{n-2}, x_{n-1}, x_n\}, \{x_{n-1}, x_n, x_1\},\text{ and } \{x_n, x_1, x_2\}$$ forms a 3-term arithmetic progression when arranged in increasing order.

Output the answer as a comma separated list inside of $\boxed{...}$. For example $\boxed{1, 2, 3}$.

\textbf{Revised Problem}\\
For $n = 21$, determine a sequence of unique real numbers $y_1, y_2, \ldots, y_n$ such that each of the sets $$\{y_1, y_2, y_3\}, \{y_2, y_3, y_4\},\ldots,\{y_{n-2}, y_{n-1}, y_n\}, \{y_{n-1}, y_n, y_1\}, \text{ and } \{y_n, y_1, y_2\}$$ becomes a 3-term arithmetic progression once arranged in ascending order.

Present your solution as a list with elements separated by commas and enclosed within $\boxed{...}$. For instance, $\boxed{4, 5, 6}$.

\subsubsection{Variation}
\textbf{Actual Problem}\\
Given $n = 27$, find a sequence of distinct real numbers $x_1, \ldots, x_n$ such that each of the sets$$\{x_1, x_2, x_3\}, \{x_2, x_3, x_4\},\ldots,\{x_{n-2}, x_{n-1}, x_n\}, \{x_{n-1}, x_n, x_1\},\text{ and } \{x_n, x_1, x_2\}$$ forms a 3-term arithmetic progression when arranged in increasing order.

Output the answer as a comma separated list inside of $\boxed{...}$. For example $\boxed{1, 2, 3}$.

\textbf{Revised Problem}\\
For $n = 27$, determine a sequence of distinct real numbers $y_1, y_2, \ldots, y_n$ such that each of the groups $$\{y_1, y_2, y_3\}, \{y_2, y_3, y_4\}, \ldots, \{y_{n-2}, y_{n-1}, y_n\}, \{y_{n-1}, y_n, y_1\},\text{ and } \{y_n, y_1, y_2\}$$ forms a 3-term arithmetic progression when sorted in increasing order.

Present your solution as a comma-separated list enclosed within $\boxed{...}$. For example, $\boxed{1, 2, 3}$.

\subsubsection{Variation}
\textbf{Actual Problem}\\
Given $n = 15$, find a sequence of distinct real numbers $x_1, \ldots, x_n$ such that each of the sets$$\{x_1, x_2, x_3\}, \{x_2, x_3, x_4\},\ldots,\{x_{n-2}, x_{n-1}, x_n\}, \{x_{n-1}, x_n, x_1\},\text{ and } \{x_n, x_1, x_2\}$$ forms a 3-term arithmetic progression when arranged in increasing order.

Output the answer as a comma separated list inside of $\boxed{...}$. For example $\boxed{1, 2, 3}$.

\textbf{Revised Problem}\\
For $n = 15$, determine a sequence of distinct real numbers $y_1, y_2, \ldots, y_n$ such that each of the collections $$\{y_1, y_2, y_3\}, \{y_2, y_3, y_4\}, \ldots, \{y_{n-2}, y_{n-1}, y_n\}, \{y_{n-1}, y_n, y_1\}, \text{ and } \{y_n, y_1, y_2\}$$ forms a 3-term arithmetic progression when arranged in ascending order.

Express your solution as a comma-separated list enclosed in $\boxed{...}$. For example, $\boxed{1, 2, 3}$.

\subsubsection{Variation}
\textbf{Actual Problem}\\
Given $n = 9$, find a sequence of distinct real numbers $x_1, \ldots, x_n$ such that each of the sets$$\{x_1, x_2, x_3\}, \{x_2, x_3, x_4\},\ldots,\{x_{n-2}, x_{n-1}, x_n\}, \{x_{n-1}, x_n, x_1\},\text{ and } \{x_n, x_1, x_2\}$$ forms a 3-term arithmetic progression when arranged in increasing order.

Output the answer as a comma separated list inside of $\boxed{...}$. For example $\boxed{1, 2, 3}$.

\textbf{Revised Problem}\\
Let $n = 9$. Determine a sequence of distinct real numbers $x_1, x_2, \ldots, x_n$ such that the following collections of numbers:
$$\{x_1, x_2, x_3\}, \{x_2, x_3, x_4\}, \ldots, \{x_{n-2}, x_{n-1}, x_n\}, \{x_{n-1}, x_n, x_1\}, \text{ and } \{x_n, x_1, x_2\}$$
constitute 3-term arithmetic progressions when each set is sorted in ascending order.

Present the solution as a sequence of numbers separated by commas, enclosed within $\boxed{...}$. For instance, $\boxed{1, 2, 3}$.

\subsection{putnam-2023-b2}
\subsubsection{Variation}
\textbf{Actual Problem}\\
For each positive integer $n$, let $k(n)$ be the number of ones in the binary representation of $2023 \cdot n$. 
Given $m = 1$, find $m$ integers $n_1, \dots, n_m$ such that $k(n_i) = 3$ for all $i$.
Note that one can show that $k(n)$ is at least $3$ for all $n$.

Output the answer as a comma separated list inside of $\boxed{...}$. For example $\boxed{1, 2, 3}$.

\textbf{Revised Problem}\\
For each positive integer $n$, define $k(n)$ as the count of ones in the binary form of $2023 \times n$. 
When $m = 1$, identify $m$ integers $n_1, \dots, n_m$ such that for each $i$, $k(n_i) = 3$ holds true.
It is known that for every $n$, $k(n)$ is at least $3$.

Present the solution as a comma-separated list enclosed within $\boxed{...}$. An example format is $\boxed{1, 2, 3}$.

\subsubsection{Variation}
\textbf{Actual Problem}\\
For each positive integer $n$, let $k(n)$ be the number of ones in the binary representation of $2023 \cdot n$. 
Given $m = 15$, find $m$ integers $n_1, \dots, n_m$ such that $k(n_i) = 3$ for all $i$.
Note that one can show that $k(n)$ is at least $3$ for all $n$.

Output the answer as a comma separated list inside of $\boxed{...}$. For example $\boxed{1, 2, 3}$.

\textbf{Revised Problem}\\
For every positive integer \( n \), define \( k(n) \) as the count of 1s present in the binary form of \( 2023 \times n \). Determine 15 integers \( n_1, n_2, \ldots, n_{15} \) such that \( k(n_i) = 3 \) for each \( i \). It is known that \( k(n) \) is always at least 3 for any \( n \).

Present your solution as a list of numbers separated by commas within a box, like so: \(\boxed{1, 2, 3}\).

\subsubsection{Variation}
\textbf{Actual Problem}\\
For each positive integer $n$, let $k(n)$ be the number of ones in the binary representation of $2023 \cdot n$. 
Given $m = 7$, find $m$ integers $n_1, \dots, n_m$ such that $k(n_i) = 3$ for all $i$.
Note that one can show that $k(n)$ is at least $3$ for all $n$.

Output the answer as a comma separated list inside of $\boxed{...}$. For example $\boxed{1, 2, 3}$.

\textbf{Revised Problem}\\
For every positive integer \( n \), define \( k(n) \) as the count of 1's in the binary format of \( 2023 \times n \). Let \( m = 7 \). Identify \( m \) integers \( n_1, n_2, \ldots, n_m \) such that \( k(n_i) = 3 \) for each \( i \) from 1 to \( m \).
It is known that \( k(n) \) is always at least 3 for every \( n \).

Present your solution as a list of numbers separated by commas, enclosed in a box. For example, \(\boxed{1, 2, 3}\).

\subsubsection{Variation}
\textbf{Actual Problem}\\
For each positive integer $n$, let $k(n)$ be the number of ones in the binary representation of $2023 \cdot n$. 
Given $m = 4$, find $m$ integers $n_1, \dots, n_m$ such that $k(n_i) = 3$ for all $i$.
Note that one can show that $k(n)$ is at least $3$ for all $n$.

Output the answer as a comma separated list inside of $\boxed{...}$. For example $\boxed{1, 2, 3}$.

\textbf{Revised Problem}\\
For a positive integer \( n \), define \( k(n) \) as the count of '1's in the binary form of \( 2023 \times n \). Given \( m = 4 \), identify \( m \) integers \( n_1, \ldots, n_m \) such that for each \( i \), \( k(n_i) = 3 \). Note that it can be shown that \( k(n) \) is always at least 3 for any \( n \).

Provide the solution as a list separated by commas within the format \(\boxed{...}\). For instance, \(\boxed{1, 2, 3}\).

\section{serbian}
\subsection{serbian-2013-mo-4}
\subsubsection{Variation}
\textbf{Actual Problem}\\
For some even $n \in \mathbb{N}$, we want to divide the set ${1, 2, \ldots, 3n}$ into $n$ disjoint triples of the form $\{a,b,c\}$ in which $b-a$ and $c-b$ are different numbers from the set $\{n-1,n,n+1\}$.

For $n = 16$, find one such division.

Output a comma-separated list of triples inside of \boxed. For example: \boxed{(1,2,3), (4,5,6)}.

\textbf{Revised Problem}\\
For an even natural number \( n \), we aim to partition the set \(\{1, 2, \ldots, 3n\}\) into \( n \) distinct triples \(\{a, b, c\}\) such that the differences \( b-a \) and \( c-b \) are unique values chosen from \(\{n-1, n, n+1\}\).

Determine one such partition for \( n = 16 \).

Provide your answer as a list of triples, separated by commas, and enclosed within \boxed. For instance: \boxed{(1,2,3), (4,5,6)}.

\subsubsection{Variation}
\textbf{Actual Problem}\\
For some even $n \in \mathbb{N}$, we want to divide the set ${1, 2, \ldots, 3n}$ into $n$ disjoint triples of the form $\{a,b,c\}$ in which $b-a$ and $c-b$ are different numbers from the set $\{n-1,n,n+1\}$.

For $n = 12$, find one such division.

Output a comma-separated list of triples inside of \boxed. For example: \boxed{(1,2,3), (4,5,6)}.

\textbf{Revised Problem}\\
Given an even integer $n \in \mathbb{N}$, partition the set $\{1, 2, \ldots, 3n\}$ into $n$ separate groups, each containing three elements $\{a, b, c\}$. The differences $b-a$ and $c-b$ must be distinct and belong to the set $\{n-1, n, n+1\}$.

Determine such a partition for $n = 12$.

Provide your answer as a series of triples, separated by commas, enclosed within \boxed. For example: \boxed{(1,2,3), (4,5,6)}.

\subsubsection{Variation}
\textbf{Actual Problem}\\
For some even $n \in \mathbb{N}$, we want to divide the set ${1, 2, \ldots, 3n}$ into $n$ disjoint triples of the form $\{a,b,c\}$ in which $b-a$ and $c-b$ are different numbers from the set $\{n-1,n,n+1\}$.

For $n = 10$, find one such division.

Output a comma-separated list of triples inside of \boxed. For example: \boxed{(1,2,3), (4,5,6)}.

\textbf{Revised Problem}\\
Consider an even integer $n \in \mathbb{N}$. We aim to partition the set $\{1, 2, \ldots, 3n\}$ into $n$ distinct triplets $\{a, b, c\}$. For each triplet, the differences $b-a$ and $c-b$ must be distinct elements from the set $\{n-1, n, n+1\}$.

Determine one valid partition for $n = 10$.

Present the solution as a list of triples separated by commas and enclosed in \boxed. For instance: \boxed{(1,2,3), (4,5,6)}.

\subsubsection{Variation}
\textbf{Actual Problem}\\
For some even $n \in \mathbb{N}$, we want to divide the set ${1, 2, \ldots, 3n}$ into $n$ disjoint triples of the form $\{a,b,c\}$ in which $b-a$ and $c-b$ are different numbers from the set $\{n-1,n,n+1\}$.

For $n = 18$, find one such division.

Output a comma-separated list of triples inside of \boxed. For example: \boxed{(1,2,3), (4,5,6)}.

\textbf{Revised Problem}\\
Consider an even number $n \in \mathbb{N}$. The challenge is to partition the set $\{1, 2, \ldots, 3n\}$ into $n$ distinct groups, each containing three elements $\{a, b, c\}$. In each group, the differences $b-a$ and $c-b$ must be distinct values from the set $\{n-1, n, n+1\}$.

Determine one such partition when $n = 18$.

Present your solution as a list of triples separated by commas, enclosed within \boxed. For instance: \boxed{(1,2,3), (4,5,6)}.

\subsection{serbian-2016-reg-g1-4}
\subsubsection{Variation}
\textbf{Actual Problem}\\
Show that it is possible to fill an $N \times N$ table, for even $N=12$, with zeros and ones such that for each $i \in\{1,2, \ldots, n\}$, the absolute value of the difference between the number of ones in the $i$-th row and the $i$-th column is equal to $1$.

Output the board where cells are denoted with "1" or "0". Output the answer between \verb|\begin{array}{...}| and \verb|\end{array}| inside of $\boxed{...}$. For example, $\boxed{\begin{array}{ccc}1 & 0 & 0 \\ 0 & 0 & 1 \\ 1 & 1 & 1\end{array}}$.

\textbf{Revised Problem}\\
Prove that a $12 \times 12$ grid can be filled with zeros and ones in such a way that, for each $i \in \{1, 2, \ldots, 12\}$, the absolute difference between the number of ones in the $i$-th row and the $i$-th column is exactly 1.

Present the grid using "1" or "0" for each cell. Display your solution enclosed within \verb|\begin{array}{...}| and \verb|\end{array}|, all encapsulated in $\boxed{...}$. For instance, $\boxed{\begin{array}{ccc}1 & 0 & 0 \\ 0 & 0 & 1 \\ 1 & 1 & 1\end{array}}$.

\subsubsection{Variation}
\textbf{Actual Problem}\\
Show that it is possible to fill an $N \times N$ table, for even $N=16$, with zeros and ones such that for each $i \in\{1,2, \ldots, n\}$, the absolute value of the difference between the number of ones in the $i$-th row and the $i$-th column is equal to $1$.

Output the board where cells are denoted with "1" or "0". Output the answer between \verb|\begin{array}{...}| and \verb|\end{array}| inside of $\boxed{...}$. For example, $\boxed{\begin{array}{ccc}1 & 0 & 0 \\ 0 & 0 & 1 \\ 1 & 1 & 1\end{array}}$.

\textbf{Revised Problem}\\
Demonstrate that a $16 \times 16$ grid can be populated with zeros and ones so that for each index $i$ from 1 to 16, the absolute difference between the count of ones in the $i$-th row and the $i$-th column is precisely 1.

Present the grid where each cell contains either "1" or "0". Display your solution between \verb|\begin{array}{...}| and \verb|\end{array}| enclosed in $\boxed{...}$. For instance, $\boxed{\begin{array}{ccc}1 & 0 & 0 \\ 0 & 0 & 1 \\ 1 & 1 & 1\end{array}}$.

\subsubsection{Variation}
\textbf{Actual Problem}\\
Show that it is possible to fill an $N \times N$ table, for even $N=10$, with zeros and ones such that for each $i \in\{1,2, \ldots, n\}$, the absolute value of the difference between the number of ones in the $i$-th row and the $i$-th column is equal to $1$.

Output the board where cells are denoted with "1" or "0". Output the answer between \verb|\begin{array}{...}| and \verb|\end{array}| inside of $\boxed{...}$. For example, $\boxed{\begin{array}{ccc}1 & 0 & 0 \\ 0 & 0 & 1 \\ 1 & 1 & 1\end{array}}$.

\textbf{Revised Problem}\\
Demonstrate that you can populate a $10 \times 10$ grid, where $N = 10$ is even, with zeros and ones such that for every index $i$ in $\{1, 2, \ldots, N\}$, the absolute difference between the count of ones in the $i$-th row and the $i$-th column equals 1.

Present the grid where each cell is represented by "1" or "0". Display the solution enclosed within \verb|\begin{array}{...}| and \verb|\end{array}|, all within $\boxed{...}$. For instance, $\boxed{\begin{array}{ccc}1 & 0 & 0 \\ 0 & 0 & 1 \\ 1 & 1 & 1\end{array}}$.

\subsubsection{Variation}
\textbf{Actual Problem}\\
Show that it is possible to fill an $N \times N$ table, for even $N=14$, with zeros and ones such that for each $i \in\{1,2, \ldots, n\}$, the absolute value of the difference between the number of ones in the $i$-th row and the $i$-th column is equal to $1$.

Output the board where cells are denoted with "1" or "0". Output the answer between \verb|\begin{array}{...}| and \verb|\end{array}| inside of $\boxed{...}$. For example, $\boxed{\begin{array}{ccc}1 & 0 & 0 \\ 0 & 0 & 1 \\ 1 & 1 & 1\end{array}}$.

\textbf{Revised Problem}\\
Demonstrate that an $N \times N$ grid, where $N$ is an even number equal to 14, can be populated with zeros and ones. Ensure that for every $i$ in the set $\{1, 2, \ldots, N\}$, the absolute difference between the count of ones in the $i$-th row and the $i$-th column is exactly 1.

Display the grid using "1" and "0" to represent the cells. The solution should be enclosed within \verb|\begin{array}{...}| and \verb|\end{array}| inside of $\boxed{...}$. For example, $\boxed{\begin{array}{ccc}1 & 0 & 0 \\ 0 & 0 & 1 \\ 1 & 1 & 1\end{array}}$.

\subsection{serbian-2020-tst-4}
\subsubsection{Variation}
\textbf{Actual Problem}\\
There are infinitely many pairs of positive rational numbers $(x, y)$ such that 
$$ 
x^y \cdot y^x = y^y.
$$ 
Find $8$ distinct such pairs. 


Output a comma-separated list of pairs inside of \boxed. For example: \boxed{(\frac{3}{8}, \frac{3}{7}), (\frac{3}{5}, \frac{1}{4})}.

\textbf{Revised Problem}\\
Identify eight distinct pairs of positive rational numbers \((x, y)\) that satisfy the equation:
$$ 
x^y \cdot y^x = y^y.
$$

Provide the answer as a list of pairs in a comma-separated format enclosed within \boxed. For instance: \boxed{(\frac{3}{8}, \frac{3}{7}), (\frac{3}{5}, \frac{1}{4})}.

\subsubsection{Variation}
\textbf{Actual Problem}\\
There are infinitely many pairs of positive rational numbers $(x, y)$ such that 
$$ 
x^y \cdot y^x = y^y.
$$ 
Find $9$ distinct such pairs. 


Output a comma-separated list of pairs inside of \boxed. For example: \boxed{(\frac{3}{8}, \frac{3}{7}), (\frac{3}{5}, \frac{1}{4})}.

\textbf{Revised Problem}\\
Identify nine unique pairs of positive rational numbers \((x, y)\) that fulfill the equation 
\[
x^y \cdot y^x = y^y.
\]

Present your answer as a comma-separated sequence of pairs enclosed within \boxed. For instance: \boxed{(\frac{3}{8}, \frac{3}{7}), (\frac{3}{5}, \frac{1}{4})}.

\subsubsection{Variation}
\textbf{Actual Problem}\\
There are infinitely many pairs of positive rational numbers $(x, y)$ such that 
$$ 
x^y \cdot y^x = y^y.
$$ 
Find $5$ distinct such pairs. 


Output a comma-separated list of pairs inside of \boxed. For example: \boxed{(\frac{3}{8}, \frac{3}{7}), (\frac{3}{5}, \frac{1}{4})}.

\textbf{Revised Problem}\\
There are infinitely many pairs of positive rational numbers \((x, y)\) that satisfy the equation 
$$ 
x^y \cdot y^x = y^y.
$$ 
Identify $5$ distinct pairs that fulfill this condition.

Present your answer as a list of pairs separated by commas and enclosed in \boxed. For example: \boxed{(\frac{3}{8}, \frac{3}{7}), (\frac{3}{5}, \frac{1}{4})}.

\subsubsection{Variation}
\textbf{Actual Problem}\\
There are infinitely many pairs of positive rational numbers $(x, y)$ such that 
$$ 
x^y \cdot y^x = y^y.
$$ 
Find $3$ distinct such pairs. 


Output a comma-separated list of pairs inside of \boxed. For example: \boxed{(\frac{3}{8}, \frac{3}{7}), (\frac{3}{5}, \frac{1}{4})}.

\textbf{Revised Problem}\\
There exist infinitely many pairs of positive rational numbers \((x, y)\) that satisfy the equation:
$$ 
x^y \cdot y^x = y^y.
$$ 
Identify $3$ such pairs that are distinct from each other.

Present your answer as a list of pairs separated by commas inside a \boxed. For example: \boxed{(\frac{3}{8}, \frac{3}{7}), (\frac{3}{5}, \frac{1}{4})}.

\subsection{serbian-2022-tst-3}
\subsubsection{Variation}
\textbf{Actual Problem}\\
Let $n$ be an odd positive integer. On a circle there are $n$ identical tokens, of which some are black, and the rest white. For $k \in \{1, 2, \ldots, n-1\}$, we define the "quality" of the number $k$ as folllows: if we move every token $k$ steps in the clockwise direction, the quality of $k$ is the number of positions where the color of the token did not change. 

We can prove that for every $n$ there exists some $k \in \{1, 2, \ldots, n-1\}$ such that the quality of $k$ is at least $\frac{n-1}{2}$.

We also know that there are infinitely many odd positive integers $n$ for which there exists a configuration of white and black tokens, such that no $k$ has quality bigger than $\frac{n-1}{2}$.
Find $6$ distinct such numbers $n$, and for each, find a configuration of white and black tokens that satisfies the above condition (no $k$ has quality above $\frac{n-1}{2}$).


Output a comma-separated list of configurations. Each configuration is a list where the first element is $n$, and the following $n$ elements indicate the token coloring going clockwise (0 for black and 1 for white tokens). For example: \boxed{(3, 0, 1, 0), (4, 1, 0, 1)} is a valid output.

\textbf{Revised Problem}\\
Consider an odd positive integer $n$. Imagine $n$ identical tokens placed on a circular track, where each token is either black or white. For each $k$ in $\{1, 2, \ldots, n-1\}$, we define the "quality" of $k" in the following way: when each token is moved $k$ positions clockwise, the quality of $k$ is the count of tokens that maintain their original color positions.

It is possible to demonstrate that for any $n$, there exists a $k$ in $\{1, 2, \ldots, n-1\}$ such that the quality of $k$ is at least $\frac{n-1}{2}$. 

Additionally, there are infinitely many odd integers $n$ for which a token arrangement exists such that no $k$ results in a quality greater than $\frac{n-1}{2}$. Identify $6$ distinct such odd integers $n$, and for each, present an arrangement of black and white tokens that meets the condition (no $k$ achieves a quality above $\frac{n-1}{2}$).

Provide a comma-separated list of token arrangements. Each arrangement should be a list where the first entry is $n$, followed by $n$ entries representing the arrangement of tokens in a clockwise direction (use 0 for black and 1 for white tokens). For example, \boxed{(3, 0, 1, 0), (4, 1, 0, 1)} is an acceptable format.

\subsubsection{Variation}
\textbf{Actual Problem}\\
Let $n$ be an odd positive integer. On a circle there are $n$ identical tokens, of which some are black, and the rest white. For $k \in \{1, 2, \ldots, n-1\}$, we define the "quality" of the number $k$ as folllows: if we move every token $k$ steps in the clockwise direction, the quality of $k$ is the number of positions where the color of the token did not change. 

We can prove that for every $n$ there exists some $k \in \{1, 2, \ldots, n-1\}$ such that the quality of $k$ is at least $\frac{n-1}{2}$.

We also know that there are infinitely many odd positive integers $n$ for which there exists a configuration of white and black tokens, such that no $k$ has quality bigger than $\frac{n-1}{2}$.
Find $10$ distinct such numbers $n$, and for each, find a configuration of white and black tokens that satisfies the above condition (no $k$ has quality above $\frac{n-1}{2}$).


Output a comma-separated list of configurations. Each configuration is a list where the first element is $n$, and the following $n$ elements indicate the token coloring going clockwise (0 for black and 1 for white tokens). For example: \boxed{(3, 0, 1, 0), (4, 1, 0, 1)} is a valid output.

\textbf{Revised Problem}\\
Consider \( n \) to be an odd positive integer. On a circular arrangement, there are \( n \) identical tokens which are either black or white. For each \( k \) in the set \(\{1, 2, \ldots, n-1\}\), we define the "quality" of \( k \) as follows: if each token is moved \( k \) steps clockwise, the quality of \( k \) is the total number of positions where the token's color matches its original color.

It can be shown that for any odd integer \( n \), there is some \( k \) within \(\{1, 2, \ldots, n-1\}\) ensuring the quality of \( k \) is at least \(\frac{n-1}{2}\).

Moreover, there exist infinitely many odd numbers \( n \) where a particular arrangement of black and white tokens results in no \( k \) yielding a quality greater than \(\frac{n-1}{2}\).
Identify 10 different such odd integers \( n \), and for each, present a specific token arrangement where no \( k \) achieves a quality exceeding \(\frac{n-1}{2}\).

Provide your answer as a comma-separated list of token arrangements. Each arrangement should begin with \( n \), followed by \( n \) token colors in clockwise order, using 0 for black and 1 for white. For instance: \boxed{(3, 0, 1, 0), (4, 1, 0, 1)} is an acceptable format.

\subsubsection{Variation}
\textbf{Actual Problem}\\
Let $n$ be an odd positive integer. On a circle there are $n$ identical tokens, of which some are black, and the rest white. For $k \in \{1, 2, \ldots, n-1\}$, we define the "quality" of the number $k$ as folllows: if we move every token $k$ steps in the clockwise direction, the quality of $k$ is the number of positions where the color of the token did not change. 

We can prove that for every $n$ there exists some $k \in \{1, 2, \ldots, n-1\}$ such that the quality of $k$ is at least $\frac{n-1}{2}$.

We also know that there are infinitely many odd positive integers $n$ for which there exists a configuration of white and black tokens, such that no $k$ has quality bigger than $\frac{n-1}{2}$.
Find $5$ distinct such numbers $n$, and for each, find a configuration of white and black tokens that satisfies the above condition (no $k$ has quality above $\frac{n-1}{2}$).


Output a comma-separated list of configurations. Each configuration is a list where the first element is $n$, and the following $n$ elements indicate the token coloring going clockwise (0 for black and 1 for white tokens). For example: \boxed{(3, 0, 1, 0), (4, 1, 0, 1)} is a valid output.

\textbf{Revised Problem}\\
Consider an odd positive integer \( n \). Imagine placing \( n \) identical tokens around a circle, where some of these tokens are black and the remainder are white. For any \( k \) in the set \(\{1, 2, \ldots, n-1\}\), the "quality" of the number \( k \) is defined as follows: when each token is shifted \( k \) positions in a clockwise direction, the quality of \( k \) is the count of positions where the token's color remains unchanged.

It can be shown that for every odd integer \( n \), there exists a number \( k \) in \(\{1, 2, \ldots, n-1\}\) such that the quality of \( k \) is at least \(\frac{n-1}{2}\).

Furthermore, there exist infinitely many odd positive integers \( n \) for which it is possible to arrange the white and black tokens in such a way that no \( k \) achieves a quality greater than \(\frac{n-1}{2}\).
Identify 5 different such numbers \( n \), and for each, provide a configuration of the tokens that ensures no \( k \) has a quality exceeding \(\frac{n-1}{2}\).

Provide a comma-separated list of configurations. Each configuration should be presented as a list where the first entry is \( n \), followed by \( n \) numbers representing the colors of the tokens arranged clockwise (use 0 for black and 1 for white tokens). For instance, a valid output format could be: \boxed{(3, 0, 1, 0), (4, 1, 0, 1)}.

\subsubsection{Variation}
\textbf{Actual Problem}\\
Let $n$ be an odd positive integer. On a circle there are $n$ identical tokens, of which some are black, and the rest white. For $k \in \{1, 2, \ldots, n-1\}$, we define the "quality" of the number $k$ as folllows: if we move every token $k$ steps in the clockwise direction, the quality of $k$ is the number of positions where the color of the token did not change. 

We can prove that for every $n$ there exists some $k \in \{1, 2, \ldots, n-1\}$ such that the quality of $k$ is at least $\frac{n-1}{2}$.

We also know that there are infinitely many odd positive integers $n$ for which there exists a configuration of white and black tokens, such that no $k$ has quality bigger than $\frac{n-1}{2}$.
Find $8$ distinct such numbers $n$, and for each, find a configuration of white and black tokens that satisfies the above condition (no $k$ has quality above $\frac{n-1}{2}$).


Output a comma-separated list of configurations. Each configuration is a list where the first element is $n$, and the following $n$ elements indicate the token coloring going clockwise (0 for black and 1 for white tokens). For example: \boxed{(3, 0, 1, 0), (4, 1, 0, 1)} is a valid output.

\textbf{Revised Problem}\\
Consider an odd positive integer $n$. Imagine $n$ identical tokens arranged in a circle, some of which are black and the others white. For each $k$ in the set $\{1, 2, \ldots, n-1\}$, we define the "quality" of $k" as follows: when each token is shifted $k$ positions clockwise, the quality is the count of tokens that land on a position with the same initial color.

It is proven that for any odd $n$, there exists a $k$ within $\{1, 2, \ldots, n-1\}$ such that the quality of $k$ reaches at least $\frac{n-1}{2}$.

Additionally, it is known that numerous odd integers exist for which we can create a token configuration such that no $k$ results in a quality exceeding $\frac{n-1}{2}$. Identify $8$ different such integers $n$, and for each, provide a token configuration where no $k$ yields a quality greater than $\frac{n-1}{2}$.

Provide a list of configurations separated by commas. Each configuration should start with the integer $n$, followed by $n$ integers representing the token colors in clockwise order (use 0 for a black token and 1 for a white token). For example: \boxed{(3, 0, 1, 0), (4, 1, 0, 1)} is a correct format for the output.

\subsection{serbian-2023-tst-1}
\subsubsection{Variation}
\textbf{Actual Problem}\\
Right before the Battle of Thermopylae $n$ Spartans met. It turned out that among them, each pair that knew each other had a different number of acquaintances among the remaining Spartans.

For $n=28$, the maximum number of pairs of Spartans that know each other under these conditions is $322$. Construct an example that achieves this. We assume that acquintanceship is symmetric.


Output a comma-separated list of distinct pairs within a \boxed environment indicating (1-indexed) acquaintances. For example: \boxed{(1, 2), (3, 4)} indicates that Spartans 1 and 2, and Spartans 3 and 4 know each other. Output each pair once, i.e., if $(i, j)$ is in the list, $(j, i)$ should not be in the list.

\textbf{Revised Problem}\\
At the eve of the Battle of Thermopylae, 28 Spartans gathered. Among these warriors, any two Spartans who were acquainted had a unique count of other Spartans they both knew in common. 

Given $n=28$, the maximum possible number of such acquaintance pairs, adhering to these conditions, is 322. Your task is to construct an example of this scenario. Assume that if Spartan A knows Spartan B, then Spartan B also knows Spartan A.

Present the acquaintance connections as a comma-separated list of unique pairs inside a \boxed environment. The list should be 1-indexed. For instance: \boxed{(1, 2), (3, 4)} indicates that Spartan 1 knows Spartan 2, and Spartan 3 knows Spartan 4. Each acquaintance pair should be listed once, meaning if you include the pair $(i, j)$, you should not include $(j, i)$.

\subsubsection{Variation}
\textbf{Actual Problem}\\
Right before the Battle of Thermopylae $n$ Spartans met. It turned out that among them, each pair that knew each other had a different number of acquaintances among the remaining Spartans.

For $n=21$, the maximum number of pairs of Spartans that know each other under these conditions is $175$. Construct an example that achieves this. We assume that acquintanceship is symmetric.


Output a comma-separated list of distinct pairs within a \boxed environment indicating (1-indexed) acquaintances. For example: \boxed{(1, 2), (3, 4)} indicates that Spartans 1 and 2, and Spartans 3 and 4 know each other. Output each pair once, i.e., if $(i, j)$ is in the list, $(j, i)$ should not be in the list.

\textbf{Revised Problem}\\
Before the Battle of Thermopylae, $n$ Spartans gathered. It was discovered that for any two Spartans who were familiar with each other, each pair had a unique count of acquaintances among the other Spartans present. With $n=21$, determine how to achieve a maximum of 175 pairs of Spartans who are acquainted under these circumstances. Assume that if one Spartan knows another, the feeling is mutual.

Provide a list of distinct pairs (using 1-based indexing) that represents the acquaintances among the Spartans, formatted within a \boxed environment. Use commas to separate the pairs. For instance, \boxed{(1, 2), (3, 4)} indicates acquaintances between Spartans 1 and 2, and Spartans 3 and 4. Each pair should be listed only once, meaning if $(i, j)$ is included, $(j, i)$ should not appear.

\section{swiss}
\subsection{swiss-2018-8-selection}
\subsubsection{Variation}
\textbf{Actual Problem}\\
Determine 14 integers $n \in [2, 2^{15})$ such that for every integer $0 \leq i,j \leq n$:

$$i + j \equiv \binom{n}{i} + \binom{n}{j} \quad (mod~2).$$

Output the answer as a comma separated list inside of $\boxed{...}$. For example $\boxed{1, 2, 3}$.

\textbf{Revised Problem}\\
Identify 14 integers \( n \) within the range from 2 to \( 2^{15} - 1 \) inclusive, for which the following condition is true for all integers \( 0 \leq i, j \leq n \):

$$ i + j \equiv \binom{n}{i} + \binom{n}{j} \pmod{2}. $$

Present the solution as a list of numbers separated by commas, enclosed within a box, like this: \(\boxed{1, 2, 3}\).

\subsubsection{Variation}
\textbf{Actual Problem}\\
Determine 21 integers $n \in [2, 2^{22})$ such that for every integer $0 \leq i,j \leq n$:

$$i + j \equiv \binom{n}{i} + \binom{n}{j} \quad (mod~2).$$

Output the answer as a comma separated list inside of $\boxed{...}$. For example $\boxed{1, 2, 3}$.

\textbf{Revised Problem}\\
Identify 21 integers \( n \) within the interval \([2, 2^{22})\) such that for every pair of integers \(0 \leq i, j \leq n\), the following condition is satisfied:

\[
i + j \equiv \binom{n}{i} + \binom{n}{j} \pmod{2}.
\]

Present the solution as a list of numbers separated by commas enclosed within \(\boxed{...}\). For example, \(\boxed{1, 2, 3}\).

\subsubsection{Variation}
\textbf{Actual Problem}\\
Determine 13 integers $n \in [2, 2^{14})$ such that for every integer $0 \leq i,j \leq n$:

$$i + j \equiv \binom{n}{i} + \binom{n}{j} \quad (mod~2).$$

Output the answer as a comma separated list inside of $\boxed{...}$. For example $\boxed{1, 2, 3}$.

\textbf{Revised Problem}\\
Identify 13 integers $n$ within the range $[2, 2^{14})$ such that for all integers $0 \leq i,j \leq n$, the following congruence is satisfied:

$$i + j \equiv \binom{n}{i} + \binom{n}{j} \pmod{2}.$$

Present the solution as a list of numbers separated by commas, enclosed within $\boxed{...}$. For instance, $\boxed{1, 2, 3}$.

\subsubsection{Variation}
\textbf{Actual Problem}\\
Determine 10 integers $n \in [2, 2^{11})$ such that for every integer $0 \leq i,j \leq n$:

$$i + j \equiv \binom{n}{i} + \binom{n}{j} \quad (mod~2).$$

Output the answer as a comma separated list inside of $\boxed{...}$. For example $\boxed{1, 2, 3}$.

\textbf{Revised Problem}\\
Identify 10 integers \( n \) in the interval \([2, 2048)\) such that for each integer \( 0 \leq i, j \leq n \), the following congruence holds:

\[ 
i + j \equiv \binom{n}{i} + \binom{n}{j} \pmod{2}.
\]

Present your answer as a list of numbers separated by commas enclosed in \(\boxed{...}\). For instance, \(\boxed{1, 2, 3}\).

\subsection{swiss-2019-3}
\subsubsection{Variation}
\textbf{Actual Problem}\\
Determine 10 periodic sequences $x_1, x_2, ...$ of positive real numbers such that for all positive integers $n$

$$x_{n+2} = \frac{1}{4} \left(\frac{1}{x_{n+1}} + x_n\right).$$

Output the answer as a comma separated list of lists inside of $\boxed{...}$. Each list should contain the first period of the sequence that satisfies the property. For instance, \boxed{(1,0.2,0.5),(3,4,5)} indicates the sequences 1,0.2,0.5,1,0.2,0.5,... and 3,4,5,3,4,5,...

\textbf{Revised Problem}\\
Identify 10 periodic sequences $y_1, y_2, \ldots$ composed of positive real numbers such that for every positive integer $n$, the following condition holds:

$$y_{n+2} = \frac{1}{4} \left(\frac{1}{y_{n+1}} + y_n\right).$$

Present your answer as a comma-separated collection of lists enclosed within $\boxed{...}$. Each list should represent the initial cycle of the sequence that fulfills the given condition. For example, writing \boxed{(1,0.2,0.5),(3,4,5)} denotes the sequences 1,0.2,0.5,1,0.2,0.5,... and 3,4,5,3,4,5,...

\subsubsection{Variation}
\textbf{Actual Problem}\\
Determine 16 periodic sequences $x_1, x_2, ...$ of positive real numbers such that for all positive integers $n$

$$x_{n+2} = \frac{1}{26} \left(\frac{1}{x_{n+1}} + x_n\right).$$

Output the answer as a comma separated list of lists inside of $\boxed{...}$. Each list should contain the first period of the sequence that satisfies the property. For instance, \boxed{(1,0.2,0.5),(3,4,5)} indicates the sequences 1,0.2,0.5,1,0.2,0.5,... and 3,4,5,3,4,5,...

\textbf{Revised Problem}\\
Identify 16 periodic sequences \( y_1, y_2, \ldots \) consisting of positive real numbers such that for every positive integer \( n \),

$$y_{n+2} = \frac{1}{26} \left( \frac{1}{y_{n+1}} + y_n \right).$$

Present your answer as a series of lists enclosed within \(\boxed{...}\), where each list represents the initial cycle of the sequence that fulfills the condition. For example, \(\boxed{(1,0.2,0.5),(3,4,5)}\) corresponds to the sequences 1,0.2,0.5,1,0.2,0.5,... and 3,4,5,3,4,5,...

\subsubsection{Variation}
\textbf{Actual Problem}\\
Determine 19 periodic sequences $x_1, x_2, ...$ of positive real numbers such that for all positive integers $n$

$$x_{n+2} = \frac{1}{10} \left(\frac{1}{x_{n+1}} + x_n\right).$$

Output the answer as a comma separated list of lists inside of $\boxed{...}$. Each list should contain the first period of the sequence that satisfies the property. For instance, \boxed{(1,0.2,0.5),(3,4,5)} indicates the sequences 1,0.2,0.5,1,0.2,0.5,... and 3,4,5,3,4,5,...

\textbf{Revised Problem}\\
Identify 19 periodic sequences $y_1, y_2, \ldots$ of positive real values such that for every positive integer $k$, the relationship holds:

$$y_{k+2} = \frac{1}{10} \left(\frac{1}{y_{k+1}} + y_k\right).$$

Provide the solution as a comma-separated list of lists enclosed in $\boxed{...}$. Each list should represent the initial period of a sequence that meets the stated condition. For example, \boxed{(1,0.2,0.5),(3,4,5)} would represent the sequences repeating as 1,0.2,0.5,1,0.2,0.5,... and 3,4,5,3,4,5,...

\subsubsection{Variation}
\textbf{Actual Problem}\\
Determine 11 periodic sequences $x_1, x_2, ...$ of positive real numbers such that for all positive integers $n$

$$x_{n+2} = \frac{1}{5} \left(\frac{1}{x_{n+1}} + x_n\right).$$

Output the answer as a comma separated list of lists inside of $\boxed{...}$. Each list should contain the first period of the sequence that satisfies the property. For instance, \boxed{(1,0.2,0.5),(3,4,5)} indicates the sequences 1,0.2,0.5,1,0.2,0.5,... and 3,4,5,3,4,5,...

\textbf{Revised Problem}\\
Identify 11 periodic sequences $y_1, y_2, \ldots$ consisting of positive real numbers such that for every positive integer $m$, the following holds:

$$y_{m+2} = \frac{1}{5} \left(\frac{1}{y_{m+1}} + y_m\right).$$

Present the solution as a comma-separated list of lists enclosed within $\boxed{...}$. Each list should represent the initial period of the sequence that meets the specified condition. For example, \boxed{(1, 0.2, 0.5), (3, 4, 5)} represents the sequences 1, 0.2, 0.5, 1, 0.2, 0.5,... and 3, 4, 5, 3, 4, 5,...

\subsection{swiss-2020-1-selection}
\subsubsection{Variation}
\textbf{Actual Problem}\\
Let $n = 6$ be an integer. Consider an $n \times n$ chessboard with the usual chessboard colouring. A move consists of choosing a $1 \times 1$ square and switching the colour of all squares in its row and column (including the chosen square itself). Find a sequence of moves such that after executing them we get a monochrome chessboard.

Output the answer as a comma separated list of coordinates inside of $\boxed{...}$. The $i$-th element indicates the coordinates of the square on which the $i$-th move is executed. The squares are labled from $(0,0)$ to $(n-1,n-1)$ and $(0,0)$ is a white square. For instance, $\boxed{(1,0),(0,1)}$ would execute the first 2 moves on the squares $(1,0)$ and $(0,1)$.

\textbf{Revised Problem}\\
Let the integer $n = 6$ represent the dimensions of an $n \times n$ chessboard, which is initially colored in the standard alternating pattern. A valid move involves selecting a $1 \times 1$ square and flipping the colors of all squares in its corresponding row and column, including the one chosen. Determine a series of moves that will transform the chessboard into one where all squares share the same color.

Present your solution as a series of comma-separated coordinate pairs enclosed within $\boxed{...}$. Each pair represents the position of the square where a move is made, with the board's squares indexed from $(0,0)$ to $(n-1,n-1)$, starting with a white square at $(0,0)$. For example, $\boxed{(1,0),(0,1)}$ would imply making the initial two moves at squares $(1,0)$ and $(0,1)$.

\subsubsection{Variation}
\textbf{Actual Problem}\\
Let $n = 10$ be an integer. Consider an $n \times n$ chessboard with the usual chessboard colouring. A move consists of choosing a $1 \times 1$ square and switching the colour of all squares in its row and column (including the chosen square itself). Find a sequence of moves such that after executing them we get a monochrome chessboard.

Output the answer as a comma separated list of coordinates inside of $\boxed{...}$. The $i$-th element indicates the coordinates of the square on which the $i$-th move is executed. The squares are labled from $(0,0)$ to $(n-1,n-1)$ and $(0,0)$ is a white square. For instance, $\boxed{(1,0),(0,1)}$ would execute the first 2 moves on the squares $(1,0)$ and $(0,1)$.

\textbf{Revised Problem}\\
Let $n = 10$ represent an integer. Imagine an $n \times n$ chessboard with alternating colors typical of a standard chessboard. A move involves selecting a $1 \times 1$ square and flipping the color of every square in its corresponding row and column, including the chosen square itself. Determine a series of moves that will transform the chessboard into one where all squares are of the same color.

Present your solution as a list of coordinates separated by commas enclosed within $\boxed{...}$. The $i$-th coordinate in the list specifies the square for the $i$-th move. Squares are indexed from $(0,0)$ to $(n-1,n-1)$, where $(0,0)$ is initially a white square. For example, $\boxed{(1,0),(0,1)}$ indicates that the first move is performed on square $(1,0)$, followed by the second move on square $(0,1)$.

\subsubsection{Variation}
\textbf{Actual Problem}\\
Let $n = 4$ be an integer. Consider an $n \times n$ chessboard with the usual chessboard colouring. A move consists of choosing a $1 \times 1$ square and switching the colour of all squares in its row and column (including the chosen square itself). Find a sequence of moves such that after executing them we get a monochrome chessboard.

Output the answer as a comma separated list of coordinates inside of $\boxed{...}$. The $i$-th element indicates the coordinates of the square on which the $i$-th move is executed. The squares are labled from $(0,0)$ to $(n-1,n-1)$ and $(0,0)$ is a white square. For instance, $\boxed{(1,0),(0,1)}$ would execute the first 2 moves on the squares $(1,0)$ and $(0,1)$.

\textbf{Revised Problem}\\
Given an integer $n = 4$, consider a standard $n \times n$ chessboard with alternating colors. A valid operation consists of selecting a $1 \times 1$ square and flipping the colors of all squares in both its respective row and column (including the selected square itself). Your task is to determine a sequence of such operations that will result in the entire chessboard becoming a single color.

Provide your answer as a list of coordinates separated by commas within $\boxed{...}$. Each element in the list represents the coordinates of the square where the operation is performed. The squares are numbered from $(0,0)$ to $(n-1,n-1)$, with $(0,0)$ being a white square. For example, $\boxed{(1,0),(0,1)}$ would mean the first two operations are performed on the squares at $(1,0)$ and $(0,1)$.

\subsubsection{Variation}
\textbf{Actual Problem}\\
Let $n = 8$ be an integer. Consider an $n \times n$ chessboard with the usual chessboard colouring. A move consists of choosing a $1 \times 1$ square and switching the colour of all squares in its row and column (including the chosen square itself). Find a sequence of moves such that after executing them we get a monochrome chessboard.

Output the answer as a comma separated list of coordinates inside of $\boxed{...}$. The $i$-th element indicates the coordinates of the square on which the $i$-th move is executed. The squares are labled from $(0,0)$ to $(n-1,n-1)$ and $(0,0)$ is a white square. For instance, $\boxed{(1,0),(0,1)}$ would execute the first 2 moves on the squares $(1,0)$ and $(0,1)$.

\textbf{Revised Problem}\\
Consider an 8x8 chessboard that follows the standard alternating color pattern. A move is defined as selecting a $1 \times 1$ square, changing the color of all squares along its row and column, including the chosen square itself. Determine a sequence of moves that will transform the entire chessboard into a single color.

Present your answer as a series of coordinates in a comma-separated format enclosed within $\boxed{...}$. The $i$-th coordinate indicates the position of the square for the $i$-th move. The chessboard is labeled with coordinates ranging from $(0,0)$ to $(7,7)$, with the square at $(0,0)$ being white. For example, $\boxed{(1,0),(0,1)}$ would mean the first two moves are executed on the squares at $(1,0)$ and $(0,1)$.

\subsection{swiss-2021-r2-z1}
\subsubsection{Variation}
\textbf{Actual Problem}\\
Show that for a positive integer $n=25$, there exists a sequence of pairwise distinct positive integers $a_1, a_2, \ldots, a_{25}$, such that for all $k \in \{1,2,\ldots,25\}$: $$a_k \mid \sum_{i=1}^{n} a_i.$$

Output the sequence as a list of integers inside $\boxed{...}$. For example, $\boxed{[1,2,3,4]}$.

\textbf{Revised Problem}\\
Demonstrate that there is a sequence of 25 unique positive integers, $b_1, b_2, \ldots, b_{25}$, such that for each $m$ in the set $\{1, 2, \ldots, 25\}$, it holds that $b_m$ divides the total sum $\sum_{j=1}^{25} b_j$.

Present the sequence in the form of a list of integers within $\boxed{...}$. For instance, $\boxed{[5,6,7,8]}$.

\subsubsection{Variation}
\textbf{Actual Problem}\\
Show that for a positive integer $n=17$, there exists a sequence of pairwise distinct positive integers $a_1, a_2, \ldots, a_{17}$, such that for all $k \in \{1,2,\ldots,17\}$: $$a_k \mid \sum_{i=1}^{n} a_i.$$

Output the sequence as a list of integers inside $\boxed{...}$. For example, $\boxed{[1,2,3,4]}$.

\textbf{Revised Problem}\\
Demonstrate that for $n=17$, a series of distinct positive integers $b_1, b_2, \ldots, b_{17}$ can be found such that each $b_j$ divides the total sum $\sum_{j=1}^{17} b_j$ for every $j \in \{1, 2, \ldots, 17\}$.

Present the sequence as a collection of integers enclosed within $\boxed{...}$. For instance, $\boxed{[1, 2, 3, 4]}$.

\subsubsection{Variation}
\textbf{Actual Problem}\\
Show that for a positive integer $n=14$, there exists a sequence of pairwise distinct positive integers $a_1, a_2, \ldots, a_{14}$, such that for all $k \in \{1,2,\ldots,14\}$: $$a_k \mid \sum_{i=1}^{n} a_i.$$

Output the sequence as a list of integers inside $\boxed{...}$. For example, $\boxed{[1,2,3,4]}$.

\textbf{Revised Problem}\\
Prove that for $n=14$, there is a sequence of distinct positive integers $b_1, b_2, \ldots, b_{14}$ such that for every $j$ in the set $\{1, 2, \ldots, 14\}$, the integer $b_j$ divides the total sum $\sum_{i=1}^{14} b_i$.

Present the sequence as a list of numbers in the format $\boxed{[b_1, b_2, \ldots, b_{14}]}$, such as $\boxed{[1, 2, 3, 4]}$.

\subsubsection{Variation}
\textbf{Actual Problem}\\
Show that for a positive integer $n=20$, there exists a sequence of pairwise distinct positive integers $a_1, a_2, \ldots, a_{20}$, such that for all $k \in \{1,2,\ldots,20\}$: $$a_k \mid \sum_{i=1}^{n} a_i.$$

Output the sequence as a list of integers inside $\boxed{...}$. For example, $\boxed{[1,2,3,4]}$.

\textbf{Revised Problem}\\
Demonstrate that there is a sequence of 20 different positive integers $a_1, a_2, \ldots, a_{20}$ such that for each $k \in \{1,2,\ldots,20\}$, the integer $a_k$ divides the total sum $\sum_{i=1}^{20} a_i$.

Present the sequence as a list of integers enclosed within $\boxed{...}$. For instance, $\boxed{[1,2,3,4]}$.

\subsection{swiss-2022-1-selection}
\subsubsection{Variation}
\textbf{Actual Problem}\\
Let $n = 16$ be a positive integer. Give a finite sequence $S$ consisting of only zeros and ones, satisfying the following property: For any positive integer $1 < d < 100$, when $S$ is interpreted as a number in base $d$, the resulting number is non-zero and divisible by $n$.

Output the answer as a comma separated list inside of $\boxed{...}$. For example $\boxed{1, 2, 3}$.
Each number in the list should be either 0 or 1.

\textbf{Revised Problem}\\
Consider the positive integer $n = 16$. Construct a finite sequence $S$, composed solely of the digits zero and one, that meets the following condition: When $S$ is viewed as a number in any base $d$, where $d$ is an integer and $2 \leq d < 100$, the number is non-zero and divisible by $n$.

Present your answer as a list of numbers separated by commas within a $\boxed{...}$ structure. For instance: $\boxed{1, 2, 3}$. Ensure each entry in the list is either 0 or 1.

\subsubsection{Variation}
\textbf{Actual Problem}\\
Let $n = 17$ be a positive integer. Give a finite sequence $S$ consisting of only zeros and ones, satisfying the following property: For any positive integer $1 < d < 100$, when $S$ is interpreted as a number in base $d$, the resulting number is non-zero and divisible by $n$.

Output the answer as a comma separated list inside of $\boxed{...}$. For example $\boxed{1, 2, 3}$.
Each number in the list should be either 0 or 1.

\textbf{Revised Problem}\\
Consider a positive integer \( n = 17 \). Identify a finite sequence \( S \) composed solely of the digits zero and one, which fulfills this condition: For any integer \( d \) where \( 1 < d < 100 \), if \( S \) is interpreted as a number in base \( d \), the resulting value is not zero and is divisible by \( n \).

Present the answer as a list separated by commas enclosed in $\boxed{...}$. For instance, $\boxed{1, 2, 3}$. Every element of the list should be either the digit 0 or 1.

\subsubsection{Variation}
\textbf{Actual Problem}\\
Let $n = 12$ be a positive integer. Give a finite sequence $S$ consisting of only zeros and ones, satisfying the following property: For any positive integer $1 < d < 100$, when $S$ is interpreted as a number in base $d$, the resulting number is non-zero and divisible by $n$.

Output the answer as a comma separated list inside of $\boxed{...}$. For example $\boxed{1, 2, 3}$.
Each number in the list should be either 0 or 1.

\textbf{Revised Problem}\\
Consider $n = 12$, a positive integer. Construct a finite sequence $S$ made up entirely of the digits 0 and 1. This sequence must satisfy the condition that, when viewed as a number in any base $d$ (where $1 < d < 100$), the number formed is non-zero and divisible by $n$.

Present your solution as a list of numbers separated by commas within $\boxed{...}$. For instance, $\boxed{1, 2, 3}$ should be formatted this way. Ensure each element in the list is either 0 or 1.

\subsubsection{Variation}
\textbf{Actual Problem}\\
Let $n = 15$ be a positive integer. Give a finite sequence $S$ consisting of only zeros and ones, satisfying the following property: For any positive integer $1 < d < 100$, when $S$ is interpreted as a number in base $d$, the resulting number is non-zero and divisible by $n$.

Output the answer as a comma separated list inside of $\boxed{...}$. For example $\boxed{1, 2, 3}$.
Each number in the list should be either 0 or 1.

\textbf{Revised Problem}\\
Consider \( n = 15 \) as a positive integer. Identify a finite sequence \( S \) composed exclusively of the digits zero and one, which satisfies the condition that when \( S \) is interpreted as a number in any base \( d \) where \( 1 < d < 100 \), the resultant number is not zero and is divisible by \( n \).

Present your response as a list of numbers separated by commas, enclosed within the format $\boxed{...}$. Each number in the sequence should be 0 or 1. For instance, $\boxed{1, 2, 3}$.

\subsection{swiss-2023-5}
\subsubsection{Variation}
\textbf{Actual Problem}\\
Let $D$ be the set of real numbers excluding -1. Find a function $f: D \rightarrow D$ such that for all $x, y \in D$ satisfying $x \neq 0$ and $y \neq -x$, the equality

$$\left(f(f(x)) + y\right) f(\frac{y}{x}) + f(f(y)) = x$$

holds.

Write a valid LaTeX function in function of the variable $x$ that does not contain $f(x)$ or $D$ inside \boxed. For instance, \boxed{x^2} or \boxed{e^x}.

\textbf{Revised Problem}\\
Consider the set \( D \) which consists of all real numbers except \(-1\). Identify a function \( f: D \rightarrow D \) such that for any \( x, y \in D \), where \( x \neq 0 \) and \( y \neq -x \), the following equation is true:

$$\left(f(f(x)) + y\right) f\left(\frac{y}{x}\right) + f(f(y)) = x.$$

Provide your answer as a LaTeX expression representing the function in terms of \( x \), ensuring that it does not include \( f(x) \) or \( D \) within \boxed. For example, \boxed{x^2} or \boxed{e^x}.

\subsection{swiss-2024-11-selection}
\subsubsection{Variation}
\textbf{Actual Problem}\\
Let $m = 4, n = 4$. Nemo is given an $m \times n$ grid of unit squares with one chip on every unit square initially. They can repeatedly carry out the following operation: first, they pick any three distinct collinear unit squares and then they move one chip from each of the outer two squares onto the middle square. They may only do this operation if the outer two squares are not empty, but the middle square is allowed to be empty. 

The maximum number of moves that Nemo can make before they cannot continue anymore is 16. Determine the moves Nemo can make to last for 16 moves before they cannot continue anymore.

Output the answer as a comma separated list of lists of coordinates inside of $\boxed{...}$. Each list contains 3 coordinates, indicating the x and y coordinates of the 3 squares that Nemo should pick. The rows are labeled from $0$ to $m-1$ and the columns are labeled from $0$ to $n-1$. For example, $\boxed{((1,1),(1,2),(1,3)),((1,2),(3,4),(5,6))}$ indicates that Nemo should first perform a move with the squares $(1,1), (1,2), (1,3)$ and then with the squares $(1,2),(3,4),(5,6)$.

\textbf{Revised Problem}\\
Consider a grid with dimensions $4 \times 4$, denoted by $m = 4$ and $n = 4$. Initially, each unit square in this grid contains one chip. Nemo is allowed to perform the following action repeatedly: choose any three collinear unit squares and transfer one chip from each of the two outer squares to the middle square. This operation can only be completed if the outer squares each have at least one chip. It is permissible for the middle square to have no chips. 

The goal is to determine how Nemo can execute this operation 16 times, which is the maximum possible number of moves, before it is impossible to continue. Specify the sequence of moves that will allow Nemo to perform 16 moves.

Present the solution as a sequence of lists of coordinates within $\boxed{...}$. Each list should contain the coordinates of the three selected squares, with each coordinate given as an x and y pair. Rows are indexed from $0$ to $m-1$, and columns from $0$ to $n-1$. For example, $\boxed{((1,1),(1,2),(1,3)),((1,2),(3,4),(5,6))}$ indicates that Nemo should first select the squares at $(1,1), (1,2), (1,3)$, followed by the squares at $(1,2),(3,4),(5,6)$ for the moves.

\subsubsection{Variation}
\textbf{Actual Problem}\\
Let $m = 5, n = 5$. Nemo is given an $m \times n$ grid of unit squares with one chip on every unit square initially. They can repeatedly carry out the following operation: first, they pick any three distinct collinear unit squares and then they move one chip from each of the outer two squares onto the middle square. They may only do this operation if the outer two squares are not empty, but the middle square is allowed to be empty. 

The maximum number of moves that Nemo can make before they cannot continue anymore is 50. Determine the moves Nemo can make to last for 50 moves before they cannot continue anymore.

Output the answer as a comma separated list of lists of coordinates inside of $\boxed{...}$. Each list contains 3 coordinates, indicating the x and y coordinates of the 3 squares that Nemo should pick. The rows are labeled from $0$ to $m-1$ and the columns are labeled from $0$ to $n-1$. For example, $\boxed{((1,1),(1,2),(1,3)),((1,2),(3,4),(5,6))}$ indicates that Nemo should first perform a move with the squares $(1,1), (1,2), (1,3)$ and then with the squares $(1,2),(3,4),(5,6)$.

\textbf{Revised Problem}\\
Consider a grid measuring $5 \times 5$ where each unit square initially contains one chip. Nemo is allowed to perform an operation where they select any three distinct collinear unit squares and transfer one chip from each of the outer squares to the center one. This operation is only valid if both outer squares have at least one chip, although the center square may initially be empty.

Calculate the sequence of moves that Nemo can execute, amounting to a total of 50 moves, after which no further operations are possible.

Provide the solution as a series of lists within a box, where each list contains the coordinates of the three squares chosen for each move. The coordinates should be formatted as x and y values, with rows numbered from 0 to 4 and columns similarly from 0 to 4. For instance, outputting $\boxed{((0,0),(0,1),(0,2)),((1,0),(1,1),(1,2))}$ implies that the first move involves squares $(0,0), (0,1), (0,2)$, followed by a move with squares $(1,0),(1,1),(1,2)$.

\subsubsection{Variation}
\textbf{Actual Problem}\\
Let $m = 4, n = 6$. Nemo is given an $m \times n$ grid of unit squares with one chip on every unit square initially. They can repeatedly carry out the following operation: first, they pick any three distinct collinear unit squares and then they move one chip from each of the outer two squares onto the middle square. They may only do this operation if the outer two squares are not empty, but the middle square is allowed to be empty. 

The maximum number of moves that Nemo can make before they cannot continue anymore is 44. Determine the moves Nemo can make to last for 44 moves before they cannot continue anymore.

Output the answer as a comma separated list of lists of coordinates inside of $\boxed{...}$. Each list contains 3 coordinates, indicating the x and y coordinates of the 3 squares that Nemo should pick. The rows are labeled from $0$ to $m-1$ and the columns are labeled from $0$ to $n-1$. For example, $\boxed{((1,1),(1,2),(1,3)),((1,2),(3,4),(5,6))}$ indicates that Nemo should first perform a move with the squares $(1,1), (1,2), (1,3)$ and then with the squares $(1,2),(3,4),(5,6)$.

\textbf{Revised Problem}\\
Consider a grid consisting of 4 rows and 6 columns, where each square contains one chip at the start. Nemo can execute the following move multiple times: select any three collinear squares, and transfer one chip from each of the outer squares to the middle square. The operation can only be performed if both outer squares have chips, though the middle square can be empty. 

The challenge is to identify a sequence of moves that allows Nemo to perform exactly 44 moves before no further moves are possible. Find and specify the sequence of moves that achieves this maximum number before operations can no longer be carried out.

Present the solution as a comma-separated list of coordinate triplets enclosed within $\boxed{...}$. Each triplet represents the x and y coordinates of the three squares chosen for each move. The grid rows are numbered from 0 to 3 and the columns from 0 to 5. For instance, $\boxed{((1,1),(1,2),(1,3)),((1,2),(3,4),(5,6))}$ signifies that Nemo should first perform a move with squares at coordinates $(1,1), (1,2), (1,3)$, followed by a move with squares at coordinates $(1,2), (3,4), (5,6)$.

\subsubsection{Variation}
\textbf{Actual Problem}\\
Let $m = 4, n = 5$. Nemo is given an $m \times n$ grid of unit squares with one chip on every unit square initially. They can repeatedly carry out the following operation: first, they pick any three distinct collinear unit squares and then they move one chip from each of the outer two squares onto the middle square. They may only do this operation if the outer two squares are not empty, but the middle square is allowed to be empty. 

The maximum number of moves that Nemo can make before they cannot continue anymore is 30. Determine the moves Nemo can make to last for 30 moves before they cannot continue anymore.

Output the answer as a comma separated list of lists of coordinates inside of $\boxed{...}$. Each list contains 3 coordinates, indicating the x and y coordinates of the 3 squares that Nemo should pick. The rows are labeled from $0$ to $m-1$ and the columns are labeled from $0$ to $n-1$. For example, $\boxed{((1,1),(1,2),(1,3)),((1,2),(3,4),(5,6))}$ indicates that Nemo should first perform a move with the squares $(1,1), (1,2), (1,3)$ and then with the squares $(1,2),(3,4),(5,6)$.

\textbf{Revised Problem}\\
Given a grid of size $4 \times 5$, with each unit square initially containing one chip, Nemo can perform a specific operation multiple times. In this operation, Nemo selects any three collinear unit squares and transfers one chip from each outer square to the central square. This operation is permitted only when the outer squares have at least one chip each, though the central square can begin empty. You need to find a sequence of operations that allows Nemo to execute exactly 30 moves before no further moves are possible.

Present the solution as a comma-separated sequence of lists within $\boxed{...}$. Each list comprises three pairs of coordinates, representing the x and y coordinates of the chosen squares for each operation. The grid rows are numbered from $0$ to $3$ and columns from $0$ to $4$. For instance, $\boxed{((1,1),(1,2),(1,3)),((1,2),(3,4),(5,6))}$ indicates that Nemo should first use the squares at $(1,1), (1,2), (1,3)$, followed by the squares at $(1,2), (3,4), (5,6)$.

\subsection{swiss-2024-12-selection}
\subsubsection{Variation}
\textbf{Actual Problem}\\
Determine a function $f: \mathbb{N} \rightarrow \mathbb{N}$ such that 

$$\underbrace{f(f(\cdots f}_{b f(a)}((a+1) \cdots)) = (a+1)f(b)$$

holds for all $a,b \in \mathbb{N}$.

Write a valid LaTeX function as a function of the variable $n$ that does not contain $f(n)$, inside \boxed. For instance, \boxed{n^2} or \boxed{e^n}.

\textbf{Revised Problem}\\
Identify a function \( f: \mathbb{N} \rightarrow \mathbb{N} \) satisfying the following condition:

$$\underbrace{f(f(\cdots f}_{b f(a)}((a+1) \cdots)) = (a+1)f(b)$$

for every \( a, b \in \mathbb{N} \).

Express a valid LaTeX function solely in terms of the variable \( n \) that excludes \( f(n) \), using \boxed. Examples include \boxed{n^2} or \boxed{e^n}.

\subsection{swiss-2024-3}
\subsubsection{Variation}
\textbf{Actual Problem}\\
Suppose that $a,b,c,d$ are positive real numbers satisfying $ab^2 + ac^2 \geq 5bcd$. Determine a possible value for $a,b,c,d$ such that

$$(a^2+b^2+c^2+d^2)\left(\frac{1}{a^2}+\frac{1}{b^2}+\frac{1}{c^2}+\frac{1}{d^2}\right) = 24$$

Output the answer as a comma separated list inside of $\boxed{...}$. For example $\boxed{1, 2, 3}$.
The numbers should be in the order $a,b,c,d$.

\textbf{Revised Problem}\\
Let \(a\), \(b\), \(c\), and \(d\) be positive real numbers such that the condition \(ab^2 + ac^2 \geq 5bcd\) holds true. Find a set of values for \(a\), \(b\), \(c\), and \(d\) that satisfy the equation:

$$(a^2+b^2+c^2+d^2)\left(\frac{1}{a^2}+\frac{1}{b^2}+\frac{1}{c^2}+\frac{1}{d^2}\right) = 24.$$

Provide your answer as a list of numbers separated by commas within a boxed expression. For instance: \(\boxed{1, 2, 3, 4}\). Ensure the numbers are listed in the order \(a, b, c, d\).

\subsubsection{Variation}
\textbf{Actual Problem}\\
Suppose that $a,b,c,d$ are positive real numbers satisfying $ab^2 + ac^2 \geq 17bcd$. Determine a possible value for $a,b,c,d$ such that

$$(a^2+b^2+c^2+d^2)\left(\frac{1}{a^2}+\frac{1}{b^2}+\frac{1}{c^2}+\frac{1}{d^2}\right) = 72$$

Output the answer as a comma separated list inside of $\boxed{...}$. For example $\boxed{1, 2, 3}$.
The numbers should be in the order $a,b,c,d$.

\textbf{Revised Problem}\\
Given positive real numbers $a, b, c, d$ that satisfy the inequality $ab^2 + ac^2 \geq 17bcd$, identify a set of values for $a, b, c, d$ such that the following equation holds:

$$(a^2+b^2+c^2+d^2)\left(\frac{1}{a^2}+\frac{1}{b^2}+\frac{1}{c^2}+\frac{1}{d^2}\right) = 72.$$

Present your solution as a comma-separated list enclosed within $\boxed{...}$. For instance, $\boxed{1, 2, 3, 4}$. Ensure the numbers appear in the sequence $a, b, c, d$.

\subsubsection{Variation}
\textbf{Actual Problem}\\
Suppose that $a,b,c,d$ are positive real numbers satisfying $ab^2 + ac^2 \geq 9bcd$. Determine a possible value for $a,b,c,d$ such that

$$(a^2+b^2+c^2+d^2)\left(\frac{1}{a^2}+\frac{1}{b^2}+\frac{1}{c^2}+\frac{1}{d^2}\right) = 40$$

Output the answer as a comma separated list inside of $\boxed{...}$. For example $\boxed{1, 2, 3}$.
The numbers should be in the order $a,b,c,d$.

\textbf{Revised Problem}\\
Consider four positive real numbers $a, b, c, d$ that meet the condition $ab^2 + ac^2 \geq 9bcd$. Identify a set of values for $a$, $b$, $c$, and $d$ such that the following equation holds:

$$(a^2+b^2+c^2+d^2)\left(\frac{1}{a^2}+\frac{1}{b^2}+\frac{1}{c^2}+\frac{1}{d^2}\right) = 40$$

Present your solution as a list of numbers separated by commas within a boxed region, in the format $\boxed{...}$. The sequence should be ordered as $a,b,c,d$. For example, if the solution is 1, 2, 3, and 4, write it as $\boxed{1, 2, 3, 4}$.

\subsubsection{Variation}
\textbf{Actual Problem}\\
Suppose that $a,b,c,d$ are positive real numbers satisfying $ab^2 + ac^2 \geq 6bcd$. Determine a possible value for $a,b,c,d$ such that

$$(a^2+b^2+c^2+d^2)\left(\frac{1}{a^2}+\frac{1}{b^2}+\frac{1}{c^2}+\frac{1}{d^2}\right) = 28$$

Output the answer as a comma separated list inside of $\boxed{...}$. For example $\boxed{1, 2, 3}$.
The numbers should be in the order $a,b,c,d$.

\textbf{Revised Problem}\\
Consider positive real numbers \(a, b, c, d\) such that \(ab^2 + ac^2 \geq 6bcd\). Identify one combination of values for \(a, b, c, d\) that satisfies the following equation:

$$ (a^2+b^2+c^2+d^2)\left(\frac{1}{a^2}+\frac{1}{b^2}+\frac{1}{c^2}+\frac{1}{d^2}\right) = 28 $$

Provide the values as a comma-separated list within \(\boxed{...}\). For instance, \(\boxed{1, 2, 3}\).
The sequence should be in the order \(a,b,c,d\).

\subsection{swiss-2024-5-selection}
\subsubsection{Variation}
\textbf{Actual Problem}\\
Let $n = 10$ be an integer and let $a_1, ..., a_{10}$ and $b_1, ..., b_{10}$ be sequences of positive integers such that the $n+1$ products

\begin{align*}
    a_1 \cdot a_2 \cdot ... \cdot a_{n-1} \cdot a_{n} \\
    b_1 \cdot a_2 \cdot ... \cdot a_{n-1} \cdot a_{n} \\
    b_1 \cdot b_2 \cdot ... \cdot a_{n-1} \cdot a_{n} \\
    \vdots \\
    b_1 \cdot b_2 \cdot ... \cdot b_{n-1} \cdot a_{n} \\
    b_1 \cdot b_2 \cdot ... \cdot b_{n-1} \cdot b_{n} \\
\end{align*}

taken in this order, form a strictly increasing arithmetic progression. Find an example of the two sequences such that the common difference of this arithmetic progression is $n!$.

\textit{Remark: An arithmetic progression is a sequence of the form $a, a + r, a + 2r, ..., a + kr$ where $a, r$, and $k$ are integers and $r$ is called the common difference.}

Output the answer as a comma separated list of lists inside of $\boxed{...}$. The first list should from the sequence $a_1, ..., a_n$ and the second sequence forms $b_1, ..., b_n$.

\textbf{Revised Problem}\\
Consider \( n = 10 \) as an integer, and let there be sequences of positive integers \( a_1, a_2, \ldots, a_{10} \) and \( b_1, b_2, \ldots, b_{10} \). These sequences lead to the formation of \( n+1 \) products:

\[
\begin{align*}
    a_1 \cdot a_2 \cdot \ldots \cdot a_{10}, \\
    b_1 \cdot a_2 \cdot \ldots \cdot a_{10}, \\
    b_1 \cdot b_2 \cdot \ldots \cdot a_{10}, \\
    \vdots \\
    b_1 \cdot b_2 \cdot \ldots \cdot b_{9} \cdot a_{10}, \\
    b_1 \cdot b_2 \cdot \ldots \cdot b_{10}, \\
\end{align*}
\]

which are arranged as a strictly increasing arithmetic sequence. Identify an example of these sequences where the arithmetic progression has a common difference of \( n! \).

\textit{Note: An arithmetic progression is a sequence defined by \( a, a + r, a + 2r, \ldots, a + kr \), where \( a, r, \) and \( k \) are integers, and \( r \) is the common difference.}

Present the solution as a comma-separated list of lists inside \(\boxed{...}\). The first list should comprise the sequence \( a_1, a_2, \ldots, a_{10} \) and the second list should contain the sequence \( b_1, b_2, \ldots, b_{10} \).

\subsubsection{Variation}
\textbf{Actual Problem}\\
Let $n = 22$ be an integer and let $a_1, ..., a_{22}$ and $b_1, ..., b_{22}$ be sequences of positive integers such that the $n+1$ products

\begin{align*}
    a_1 \cdot a_2 \cdot ... \cdot a_{n-1} \cdot a_{n} \\
    b_1 \cdot a_2 \cdot ... \cdot a_{n-1} \cdot a_{n} \\
    b_1 \cdot b_2 \cdot ... \cdot a_{n-1} \cdot a_{n} \\
    \vdots \\
    b_1 \cdot b_2 \cdot ... \cdot b_{n-1} \cdot a_{n} \\
    b_1 \cdot b_2 \cdot ... \cdot b_{n-1} \cdot b_{n} \\
\end{align*}

taken in this order, form a strictly increasing arithmetic progression. Find an example of the two sequences such that the common difference of this arithmetic progression is $n!$.

\textit{Remark: An arithmetic progression is a sequence of the form $a, a + r, a + 2r, ..., a + kr$ where $a, r$, and $k$ are integers and $r$ is called the common difference.}

Output the answer as a comma separated list of lists inside of $\boxed{...}$. The first list should from the sequence $a_1, ..., a_n$ and the second sequence forms $b_1, ..., b_n$.

\textbf{Revised Problem}\\
Let $n = 22$ and consider two sequences of positive integers: $a_1, a_2, \ldots, a_{22}$ and $b_1, b_2, \ldots, b_{22}$. The $n+1$ products given by:

\[
\begin{align*}
    a_1 \cdot a_2 \cdot \ldots \cdot a_{22}, \\
    b_1 \cdot a_2 \cdot \ldots \cdot a_{22}, \\
    b_1 \cdot b_2 \cdot a_3 \cdot \ldots \cdot a_{22}, \\
    \vdots \\
    b_1 \cdot b_2 \cdot \ldots \cdot b_{21} \cdot a_{22}, \\
    b_1 \cdot b_2 \cdot \ldots \cdot b_{22}
\end{align*}
\]

are arranged to form a strictly increasing arithmetic sequence. Your task is to find such sequences where the common difference of this arithmetic progression is equal to $n!$.

\textit{Note: An arithmetic progression is a sequence where each term after the first is obtained by adding a constant, known as the common difference, to the previous term.}

Provide the answer as a comma-separated list of lists enclosed in $\boxed{...}$. The first list should represent the sequence $a_1, a_2, \ldots, a_{22}$, and the second list should represent the sequence $b_1, b_2, \ldots, b_{22}$.

\subsubsection{Variation}
\textbf{Actual Problem}\\
Let $n = 14$ be an integer and let $a_1, ..., a_{14}$ and $b_1, ..., b_{14}$ be sequences of positive integers such that the $n+1$ products

\begin{align*}
    a_1 \cdot a_2 \cdot ... \cdot a_{n-1} \cdot a_{n} \\
    b_1 \cdot a_2 \cdot ... \cdot a_{n-1} \cdot a_{n} \\
    b_1 \cdot b_2 \cdot ... \cdot a_{n-1} \cdot a_{n} \\
    \vdots \\
    b_1 \cdot b_2 \cdot ... \cdot b_{n-1} \cdot a_{n} \\
    b_1 \cdot b_2 \cdot ... \cdot b_{n-1} \cdot b_{n} \\
\end{align*}

taken in this order, form a strictly increasing arithmetic progression. Find an example of the two sequences such that the common difference of this arithmetic progression is $n!$.

\textit{Remark: An arithmetic progression is a sequence of the form $a, a + r, a + 2r, ..., a + kr$ where $a, r$, and $k$ are integers and $r$ is called the common difference.}

Output the answer as a comma separated list of lists inside of $\boxed{...}$. The first list should from the sequence $a_1, ..., a_n$ and the second sequence forms $b_1, ..., b_n$.

\textbf{Revised Problem}\\
Consider the integer $n = 14$ and two sequences of positive integers $x_1, x_2, \ldots, x_{14}$ and $y_1, y_2, \ldots, y_{14}$. These sequences generate $n+1$ products as follows:

\[
\begin{align*}
    x_1 \times x_2 \times \cdots \times x_{14}, \\
    y_1 \times x_2 \times \cdots \times x_{14}, \\
    y_1 \times y_2 \times x_3 \times \cdots \times x_{14}, \\
    \vdots \\
    y_1 \times y_2 \times \cdots \times y_{13} \times x_{14}, \\
    y_1 \times y_2 \times \cdots \times y_{14},
\end{align*}
\]

which form a strictly increasing arithmetic sequence. Your task is to determine such sequences where the common difference of this progression equals $14!$.

\textit{Note: An arithmetic sequence is expressed as $a, a + d, a + 2d, \ldots, a + kd$, where $a$, $d$, and $k$ are integers, and $d$ is the common difference.}

Submit the solution as a comma-separated list of lists enclosed in $\boxed{...}$. The first list should represent the sequence $x_1, x_2, \ldots, x_{14}$, and the second list should represent the sequence $y_1, y_2, \ldots, y_{14}$.

\subsubsection{Variation}
\textbf{Actual Problem}\\
Let $n = 11$ be an integer and let $a_1, ..., a_{11}$ and $b_1, ..., b_{11}$ be sequences of positive integers such that the $n+1$ products

\begin{align*}
    a_1 \cdot a_2 \cdot ... \cdot a_{n-1} \cdot a_{n} \\
    b_1 \cdot a_2 \cdot ... \cdot a_{n-1} \cdot a_{n} \\
    b_1 \cdot b_2 \cdot ... \cdot a_{n-1} \cdot a_{n} \\
    \vdots \\
    b_1 \cdot b_2 \cdot ... \cdot b_{n-1} \cdot a_{n} \\
    b_1 \cdot b_2 \cdot ... \cdot b_{n-1} \cdot b_{n} \\
\end{align*}

taken in this order, form a strictly increasing arithmetic progression. Find an example of the two sequences such that the common difference of this arithmetic progression is $n!$.

\textit{Remark: An arithmetic progression is a sequence of the form $a, a + r, a + 2r, ..., a + kr$ where $a, r$, and $k$ are integers and $r$ is called the common difference.}

Output the answer as a comma separated list of lists inside of $\boxed{...}$. The first list should from the sequence $a_1, ..., a_n$ and the second sequence forms $b_1, ..., b_n$.

\textbf{Revised Problem}\\
Consider \( n = 11 \) as a fixed integer, and let \( a_1, a_2, \ldots, a_{11} \) and \( b_1, b_2, \ldots, b_{11} \) be sequences of positive integers. The challenge is to construct these sequences such that the following \( n+1 \) products:

\[
\begin{align*}
    a_1 \times a_2 \times \cdots \times a_{10} \times a_{11}, \\
    b_1 \times a_2 \times \cdots \times a_{10} \times a_{11}, \\
    b_1 \times b_2 \times \cdots \times a_{10} \times a_{11}, \\
    \vdots \\
    b_1 \times b_2 \times \cdots \times b_{10} \times a_{11}, \\
    b_1 \times b_2 \times \cdots \times b_{10} \times b_{11},
\end{align*}
\]

are in a strictly ascending arithmetic sequence. Your task is to provide an example of these sequences with the condition that the common difference of the arithmetic sequence is \( n! \).

\textit{Note: An arithmetic sequence is defined by terms such as \( a, a + r, a + 2r, \ldots, a + kr \), where \( a, r, \) and \( k \) are integers, and \( r \) is known as the common difference.}

Present the solution as a comma-separated list of lists enclosed within \(\boxed{...}\). The first list should represent the sequence \( a_1, a_2, \ldots, a_n \), and the second list should represent the sequence \( b_1, b_2, \ldots, b_n \).

\subsection{swiss-2024-8-selection}
\subsubsection{Variation}
\textbf{Actual Problem}\\
Determine a function $f: \mathbb{R}_{> 0} \rightarrow \mathbb{R}_{> 0}$ such that 

$$x(f(x) + f(y)) \geq f(y)(f(f(x)) + y)$$

for all $x,y \in \mathbb{R}_{> 0}$.

Write a valid LaTeX function in function of the variable $x$ that does not contain $f(x)$ inside \boxed. For instance, \boxed{x^2} or \boxed{e^x}.

\textbf{Revised Problem}\\
Identify a function $f: \mathbb{R}_{> 0} \rightarrow \mathbb{R}_{> 0}$ such that the inequality 

$$x(f(x) + f(y)) \geq f(y)(f(f(x)) + y)$$

holds true for every $x, y \in \mathbb{R}_{> 0}$.

Express a valid LaTeX representation of the function in terms of the variable $x$, ensuring it does not include $f(x)$ inside \boxed. For example, \boxed{x^2} or \boxed{e^x}.

\section{tot}
\subsection{tot-2005-1}
\subsubsection{Variation}
\textbf{Actual Problem}\\
Find a sequence of $n = 15$ distinct positive integers $a_1, a_2, \ldots, a_n$ so that the sum
$$\frac{a_1}{a_2} + \frac{a_2}{a_3} + \ldots +\frac{a_n}{a_1}$$ is an integer.

Output the answer as a comma separated list inside of $\boxed{...}$. For example $\boxed{1, 2, 3}$.

\textbf{Revised Problem}\\
Determine a collection of $n = 15$ distinct positive integers $b_1, b_2, \ldots, b_n$ such that the expression
$$\frac{b_1}{b_2} + \frac{b_2}{b_3} + \cdots + \frac{b_n}{b_1}$$ results in an integer value.

Present your solution as a list of integers separated by commas, placed inside $\boxed{...}$. For instance, $\boxed{4, 5, 6}$.

\subsubsection{Variation}
\textbf{Actual Problem}\\
Find a sequence of $n = 21$ distinct positive integers $a_1, a_2, \ldots, a_n$ so that the sum
$$\frac{a_1}{a_2} + \frac{a_2}{a_3} + \ldots +\frac{a_n}{a_1}$$ is an integer.

Output the answer as a comma separated list inside of $\boxed{...}$. For example $\boxed{1, 2, 3}$.

\textbf{Revised Problem}\\
Determine a series of $n = 21$ distinct positive numbers $b_1, b_2, \ldots, b_n$ such that the expression
$$\frac{b_1}{b_2} + \frac{b_2}{b_3} + \ldots +\frac{b_n}{b_1}$$ results in an integer.

Present the solution as a sequence of numbers enclosed within $\boxed{...}$ and separated by commas. For instance, $\boxed{1, 2, 3}$.

\subsubsection{Variation}
\textbf{Actual Problem}\\
Find a sequence of $n = 17$ distinct positive integers $a_1, a_2, \ldots, a_n$ so that the sum
$$\frac{a_1}{a_2} + \frac{a_2}{a_3} + \ldots +\frac{a_n}{a_1}$$ is an integer.

Output the answer as a comma separated list inside of $\boxed{...}$. For example $\boxed{1, 2, 3}$.

\textbf{Revised Problem}\\
Identify a sequence of $n = 17$ unique positive integers $b_1, b_2, \ldots, b_n$ such that the expression
$$\frac{b_1}{b_2} + \frac{b_2}{b_3} + \ldots +\frac{b_n}{b_1}$$ evaluates to an integer.

Present your solution as a series of numbers separated by commas within $\boxed{...}$. For instance, $\boxed{4, 5, 6}$.

\subsubsection{Variation}
\textbf{Actual Problem}\\
Find a sequence of $n = 18$ distinct positive integers $a_1, a_2, \ldots, a_n$ so that the sum
$$\frac{a_1}{a_2} + \frac{a_2}{a_3} + \ldots +\frac{a_n}{a_1}$$ is an integer.

Output the answer as a comma separated list inside of $\boxed{...}$. For example $\boxed{1, 2, 3}$.

\textbf{Revised Problem}\\
Determine a sequence of $n = 18$ distinct positive integers $b_1, b_2, \ldots, b_n$ such that the expression
$$\frac{b_1}{b_2} + \frac{b_2}{b_3} + \cdots + \frac{b_n}{b_1}$$ is an integer.

Present the solution as a comma-separated list enclosed in a box, like so: $\boxed{1, 2, 3}$.

\subsection{tot-2018-1}
\subsubsection{Variation}
\textbf{Actual Problem}\\
Find 10 positive irreducible fractions, each with a different denominator, so that the denominator of the difference of any two (after reducing the fraction) is less than the denominator of any of the initial 10 fractions.

Output the fractions as a comma-separated list inside of \boxed, e.g. \boxed{\frac{1}{2}, \frac{1}{3}, \frac{2}{3}}.

\textbf{Revised Problem}\\
Identify 10 distinct positive irreducible fractions such that each fraction has a unique denominator, and the denominator of the simplified difference between any pair of these fractions is smaller than any of the denominators originally chosen.

Present your answer as a sequence of fractions separated by commas within a \boxed environment, for example, \boxed{\frac{1}{2}, \frac{1}{3}, \frac{2}{3}}.

\subsubsection{Variation}
\textbf{Actual Problem}\\
Find 17 positive irreducible fractions, each with a different denominator, so that the denominator of the difference of any two (after reducing the fraction) is less than the denominator of any of the initial 17 fractions.

Output the fractions as a comma-separated list inside of \boxed, e.g. \boxed{\frac{1}{2}, \frac{1}{3}, \frac{2}{3}}.

\textbf{Revised Problem}\\
Identify 17 distinct positive irreducible fractions, ensuring each has a unique denominator, such that when any two fractions are subtracted and reduced, the resulting fraction's denominator is smaller than the smallest denominator among the original 17 fractions.

Present the fractions in the format of a comma-separated list enclosed within \boxed, like this: \boxed{\frac{1}{2}, \frac{1}{3}, \frac{2}{3}}.

\subsubsection{Variation}
\textbf{Actual Problem}\\
Find 9 positive irreducible fractions, each with a different denominator, so that the denominator of the difference of any two (after reducing the fraction) is less than the denominator of any of the initial 9 fractions.

Output the fractions as a comma-separated list inside of \boxed, e.g. \boxed{\frac{1}{2}, \frac{1}{3}, \frac{2}{3}}.

\textbf{Revised Problem}\\
Identify 9 distinct positive irreducible fractions, each having a unique denominator, such that after reducing the difference between any two fractions, the resulting fraction's denominator is smaller than the denominator of any of the original 9 fractions.

Present the fractions as a list separated by commas inside a \boxed, like this: \boxed{\frac{1}{2}, \frac{1}{3}, \frac{2}{3}}.

\subsubsection{Variation}
\textbf{Actual Problem}\\
Find 6 positive irreducible fractions, each with a different denominator, so that the denominator of the difference of any two (after reducing the fraction) is less than the denominator of any of the initial 6 fractions.

Output the fractions as a comma-separated list inside of \boxed, e.g. \boxed{\frac{1}{2}, \frac{1}{3}, \frac{2}{3}}.

\textbf{Revised Problem}\\
Identify 6 distinct positive irreducible fractions, each having a unique denominator. Ensure that for every pair of fractions, when their difference is reduced to its simplest form, the denominator of this reduced fraction is smaller than any of the original 6 denominators.

Present the fractions as a list separated by commas within \boxed, for example, \boxed{\frac{1}{2}, \frac{1}{3}, \frac{2}{3}}.

\section{usamo}
\subsection{usamo-1990-3}
\subsubsection{Variation}
\textbf{Actual Problem}\\
Suppose that necklace $\, A \,$ has 14 beads and necklace $\, B \,$ has 19. 
Given an integer $n$, we are interested in a way to number each of the 33 beads with an integer from the sequence $\{ n, n+1, n+2, \dots, n+32 \}$ so that each integer is used once, and adjacent beads correspond to relatively prime integers. (Here a ``necklace'' is viewed as a circle in which each bead is adjacent to two other beads.) Find such a numbering for $n$ = 233.

Output the answer as a comma separated list inside of $\boxed{...}$. For example $\boxed{1, 2, 3}$.
Output your answer as a single list of integers, where the first 14 integers are for necklace $\, A \,$ and the last 19 integers are for necklace $\, B \,$.

\textbf{Revised Problem}\\
Consider two necklaces, one labeled \( A \) with 14 beads and another labeled \( B \) with 19 beads. If given an integer starting point \( n \), we need to assign integers from the sequence \(\{ n, n+1, n+2, \ldots, n+32 \}\) to these 33 beads such that each integer is used once and the integers on neighboring beads are coprime. Here, each necklace is structured as a circular arrangement, meaning that each bead is adjacent to exactly two other beads. Determine such a numbering when \( n = 233 \).

Present the solution as a comma-separated sequence enclosed in \(\boxed{...}\). For instance, \(\boxed{1, 2, 3}\). Show your answer as a single sequence of integers, where the first 14 values correspond to necklace \( A \) and the subsequent 19 values pertain to necklace \( B \).

\subsubsection{Variation}
\textbf{Actual Problem}\\
Suppose that necklace $\, A \,$ has 14 beads and necklace $\, B \,$ has 19. 
Given an integer $n$, we are interested in a way to number each of the 33 beads with an integer from the sequence $\{ n, n+1, n+2, \dots, n+32 \}$ so that each integer is used once, and adjacent beads correspond to relatively prime integers. (Here a ``necklace'' is viewed as a circle in which each bead is adjacent to two other beads.) Find such a numbering for $n$ = 826.

Output the answer as a comma separated list inside of $\boxed{...}$. For example $\boxed{1, 2, 3}$.
Output your answer as a single list of integers, where the first 14 integers are for necklace $\, A \,$ and the last 19 integers are for necklace $\, B \,$.

\textbf{Revised Problem}\\
Imagine you have two circular necklaces, one labeled as necklace $\, A \,$ with 14 beads and the other as necklace $\, B \,$ with 19 beads. Consider assigning each bead a unique number from the sequence $\{ n, n+1, n+2, \dots, n+32 \}$, where $n$ is a given integer. The goal is to arrange these numbers such that each bead on both necklaces has a number that is relatively prime to the numbers on the beads directly adjacent to it. For $n = 826$, determine such a sequence of numbers.

Express your answer as a sequence of numbers enclosed in $\boxed{...}$, separated by commas. For example, $\boxed{1, 2, 3}$. Present the sequence as a single list where the first 14 numbers correspond to necklace $\, A \,$ and the following 19 numbers correspond to necklace $\, B \,$.

\subsubsection{Variation}
\textbf{Actual Problem}\\
Suppose that necklace $\, A \,$ has 14 beads and necklace $\, B \,$ has 19. 
Given an integer $n$, we are interested in a way to number each of the 33 beads with an integer from the sequence $\{ n, n+1, n+2, \dots, n+32 \}$ so that each integer is used once, and adjacent beads correspond to relatively prime integers. (Here a ``necklace'' is viewed as a circle in which each bead is adjacent to two other beads.) Find such a numbering for $n$ = 632.

Output the answer as a comma separated list inside of $\boxed{...}$. For example $\boxed{1, 2, 3}$.
Output your answer as a single list of integers, where the first 14 integers are for necklace $\, A \,$ and the last 19 integers are for necklace $\, B \,$.

\textbf{Revised Problem}\\
Consider two circular necklaces: necklace $\, A \,$ with 14 beads and necklace $\, B \,$ with 19 beads. We need to assign to each of the 33 beads a unique integer from the set $\{ n, n+1, n+2, \ldots, n+32 \}$, where $n$ is an integer, such that each integer appears exactly once, and successive beads hold numbers that are coprime. (Note that in this context, a "necklace" is a loop where each bead is adjacent to exactly two others.) Determine a suitable numbering for the case when $n = 632$.

Present your solution as a list of numbers separated by commas within $\boxed{...}$. For example, $\boxed{1, 2, 3}$. The list should include all integers, with the first 14 numbers assigned to necklace $\, A \,$ and the remaining 19 numbers to necklace $\, B \,$.

\subsubsection{Variation}
\textbf{Actual Problem}\\
Suppose that necklace $\, A \,$ has 14 beads and necklace $\, B \,$ has 19. 
Given an integer $n$, we are interested in a way to number each of the 33 beads with an integer from the sequence $\{ n, n+1, n+2, \dots, n+32 \}$ so that each integer is used once, and adjacent beads correspond to relatively prime integers. (Here a ``necklace'' is viewed as a circle in which each bead is adjacent to two other beads.) Find such a numbering for $n$ = 136.

Output the answer as a comma separated list inside of $\boxed{...}$. For example $\boxed{1, 2, 3}$.
Output your answer as a single list of integers, where the first 14 integers are for necklace $\, A \,$ and the last 19 integers are for necklace $\, B \,$.

\textbf{Revised Problem}\\
Imagine you have two necklaces: one called $\, A \,$ with 14 beads and another called $\, B \,$ with 19 beads. You need to label each bead with a unique integer from the set $\{ n, n+1, n+2, \dots, n+32 \}$, where $n$ is an integer. The goal is to ensure that each pair of neighboring beads on the necklaces has numbers that are coprime. (Think of each necklace as a circle, where each bead is next to two others.) Determine such a labeling for $n = 136$.

Provide your solution as a series of integers separated by commas inside $\boxed{...}$. For example, $\boxed{1, 2, 3}$. Present your solution as a continuous sequence of integers, where the first 14 numbers correspond to necklace $\, A \,$ and the following 19 numbers correspond to necklace $\, B \,$.

\subsection{usamo-2000-4}
\subsubsection{Variation}
\textbf{Actual Problem}\\
Color $18$ squares of a $10 \times 10$ chessboard, so that there do not exist three colored squares whose centers form a right triangle with sides parallel to the edges of the board.

Output the answer between \verb|\begin{array}{...}| and \verb|\end{array}| inside of $\boxed{...}$. For example, $\boxed{\begin{array}{ccc}1 & 2 & 3 \\ 4 & 5 & 6 \\ 7 & 8 & 9\end{array}}$.
Output the chessboard where empty cells "o" are replaced with colored cells "x":

$$\begin{array}{cccccccccc}
o & o & o & o & o & o & o & o & o & o \\
o & o & o & o & o & o & o & o & o & o \\
o & o & o & o & o & o & o & o & o & o \\
o & o & o & o & o & o & o & o & o & o \\
o & o & o & o & o & o & o & o & o & o \\
o & o & o & o & o & o & o & o & o & o \\
o & o & o & o & o & o & o & o & o & o \\
o & o & o & o & o & o & o & o & o & o \\
o & o & o & o & o & o & o & o & o & o \\
o & o & o & o & o & o & o & o & o & o \\
\end{array}$$


\textbf{Revised Problem}\\
Color $18$ squares on a $10 \times 10$ chessboard such that it is impossible for the centers of any three colored squares to form a right triangle with legs parallel to the board's edges.

Present the solution within \verb|\begin{array}{...}| and \verb|\end{array}| wrapped in $\boxed{...}$. For instance, $\boxed{\begin{array}{ccc}1 & 2 & 3 \\ 4 & 5 & 6 \\ 7 & 8 & 9\end{array}}$. Depict the chessboard where colored squares are marked with "x" and uncolored squares with "o":

$$\begin{array}{cccccccccc}
o & o & o & o & o & o & o & o & o & o \\
o & o & o & o & o & o & o & o & o & o \\
o & o & o & o & o & o & o & o & o & o \\
o & o & o & o & o & o & o & o & o & o \\
o & o & o & o & o & o & o & o & o & o \\
o & o & o & o & o & o & o & o & o & o \\
o & o & o & o & o & o & o & o & o & o \\
o & o & o & o & o & o & o & o & o & o \\
o & o & o & o & o & o & o & o & o & o \\
o & o & o & o & o & o & o & o & o & o \\
\end{array}$$

\subsubsection{Variation}
\textbf{Actual Problem}\\
Color $41$ squares of a $22 \times 22$ chessboard, so that there do not exist three colored squares whose centers form a right triangle with sides parallel to the edges of the board.

Output the answer between \verb|\begin{array}{...}| and \verb|\end{array}| inside of $\boxed{...}$. For example, $\boxed{\begin{array}{ccc}1 & 2 & 3 \\ 4 & 5 & 6 \\ 7 & 8 & 9\end{array}}$.
Output the chessboard where empty cells "o" are replaced with colored cells "x":

$$\begin{array}{cccccccccccccccccccccc}
o & o & o & o & o & o & o & o & o & o & o & o & o & o & o & o & o & o & o & o & o & o \\
o & o & o & o & o & o & o & o & o & o & o & o & o & o & o & o & o & o & o & o & o & o \\
o & o & o & o & o & o & o & o & o & o & o & o & o & o & o & o & o & o & o & o & o & o \\
o & o & o & o & o & o & o & o & o & o & o & o & o & o & o & o & o & o & o & o & o & o \\
o & o & o & o & o & o & o & o & o & o & o & o & o & o & o & o & o & o & o & o & o & o \\
o & o & o & o & o & o & o & o & o & o & o & o & o & o & o & o & o & o & o & o & o & o \\
o & o & o & o & o & o & o & o & o & o & o & o & o & o & o & o & o & o & o & o & o & o \\
o & o & o & o & o & o & o & o & o & o & o & o & o & o & o & o & o & o & o & o & o & o \\
o & o & o & o & o & o & o & o & o & o & o & o & o & o & o & o & o & o & o & o & o & o \\
o & o & o & o & o & o & o & o & o & o & o & o & o & o & o & o & o & o & o & o & o & o \\
o & o & o & o & o & o & o & o & o & o & o & o & o & o & o & o & o & o & o & o & o & o \\
o & o & o & o & o & o & o & o & o & o & o & o & o & o & o & o & o & o & o & o & o & o \\
o & o & o & o & o & o & o & o & o & o & o & o & o & o & o & o & o & o & o & o & o & o \\
o & o & o & o & o & o & o & o & o & o & o & o & o & o & o & o & o & o & o & o & o & o \\
o & o & o & o & o & o & o & o & o & o & o & o & o & o & o & o & o & o & o & o & o & o \\
o & o & o & o & o & o & o & o & o & o & o & o & o & o & o & o & o & o & o & o & o & o \\
o & o & o & o & o & o & o & o & o & o & o & o & o & o & o & o & o & o & o & o & o & o \\
o & o & o & o & o & o & o & o & o & o & o & o & o & o & o & o & o & o & o & o & o & o \\
o & o & o & o & o & o & o & o & o & o & o & o & o & o & o & o & o & o & o & o & o & o \\
o & o & o & o & o & o & o & o & o & o & o & o & o & o & o & o & o & o & o & o & o & o \\
o & o & o & o & o & o & o & o & o & o & o & o & o & o & o & o & o & o & o & o & o & o \\
o & o & o & o & o & o & o & o & o & o & o & o & o & o & o & o & o & o & o & o & o & o \\
\end{array}$$


\textbf{Revised Problem}\\
Color 41 squares on a 22 by 22 chessboard such that no three of the colored squares' centers form a right triangle with sides aligned to the horizontal and vertical edges of the board.

Present your solution using \verb|\begin{array}{...}| and \verb|\end{array}| within $\boxed{...}$. For instance, $\boxed{\begin{array}{ccc}1 & 2 & 3 \\ 4 & 5 & 6 \\ 7 & 8 & 9\end{array}}$.
Display the chessboard where "o" represents uncolored cells and "x" represents colored cells:

$$\begin{array}{cccccccccccccccccccccc}
o & o & o & o & o & o & o & o & o & o & o & o & o & o & o & o & o & o & o & o & o & o \\
o & o & o & o & o & o & o & o & o & o & o & o & o & o & o & o & o & o & o & o & o & o \\
o & o & o & o & o & o & o & o & o & o & o & o & o & o & o & o & o & o & o & o & o & o \\
o & o & o & o & o & o & o & o & o & o & o & o & o & o & o & o & o & o & o & o & o & o \\
o & o & o & o & o & o & o & o & o & o & o & o & o & o & o & o & o & o & o & o & o & o \\
o & o & o & o & o & o & o & o & o & o & o & o & o & o & o & o & o & o & o & o & o & o \\
o & o & o & o & o & o & o & o & o & o & o & o & o & o & o & o & o & o & o & o & o & o \\
o & o & o & o & o & o & o & o & o & o & o & o & o & o & o & o & o & o & o & o & o & o \\
o & o & o & o & o & o & o & o & o & o & o & o & o & o & o & o & o & o & o & o & o & o \\
o & o & o & o & o & o & o & o & o & o & o & o & o & o & o & o & o & o & o & o & o & o \\
o & o & o & o & o & o & o & o & o & o & o & o & o & o & o & o & o & o & o & o & o & o \\
o & o & o & o & o & o & o & o & o & o & o & o & o & o & o & o & o & o & o & o & o & o \\
o & o & o & o & o & o & o & o & o & o & o & o & o & o & o & o & o & o & o & o & o & o \\
o & o & o & o & o & o & o & o & o & o & o & o & o & o & o & o & o & o & o & o & o & o \\
o & o & o & o & o & o & o & o & o & o & o & o & o & o & o & o & o & o & o & o & o & o \\
o & o & o & o & o & o & o & o & o & o & o & o & o & o & o & o & o & o & o & o & o & o \\
o & o & o & o & o & o & o & o & o & o & o & o & o & o & o & o & o & o & o & o & o & o \\
o & o & o & o & o & o & o & o & o & o & o & o & o & o & o & o & o & o & o & o & o & o \\
o & o & o & o & o & o & o & o & o & o & o & o & o & o & o & o & o & o & o & o & o & o \\
o & o & o & o & o & o & o & o & o & o & o & o & o & o & o & o & o & o & o & o & o & o \\
o & o & o & o & o & o & o & o & o & o & o & o & o & o & o & o & o & o & o & o & o & o \\
o & o & o & o & o & o & o & o & o & o & o & o & o & o & o & o & o & o & o & o & o & o \\
\end{array}$$

\subsubsection{Variation}
\textbf{Actual Problem}\\
Color $26$ squares of a $14 \times 14$ chessboard, so that there do not exist three colored squares whose centers form a right triangle with sides parallel to the edges of the board.

Output the answer between \verb|\begin{array}{...}| and \verb|\end{array}| inside of $\boxed{...}$. For example, $\boxed{\begin{array}{ccc}1 & 2 & 3 \\ 4 & 5 & 6 \\ 7 & 8 & 9\end{array}}$.
Output the chessboard where empty cells "o" are replaced with colored cells "x":

$$\begin{array}{cccccccccccccc}
o & o & o & o & o & o & o & o & o & o & o & o & o & o \\
o & o & o & o & o & o & o & o & o & o & o & o & o & o \\
o & o & o & o & o & o & o & o & o & o & o & o & o & o \\
o & o & o & o & o & o & o & o & o & o & o & o & o & o \\
o & o & o & o & o & o & o & o & o & o & o & o & o & o \\
o & o & o & o & o & o & o & o & o & o & o & o & o & o \\
o & o & o & o & o & o & o & o & o & o & o & o & o & o \\
o & o & o & o & o & o & o & o & o & o & o & o & o & o \\
o & o & o & o & o & o & o & o & o & o & o & o & o & o \\
o & o & o & o & o & o & o & o & o & o & o & o & o & o \\
o & o & o & o & o & o & o & o & o & o & o & o & o & o \\
o & o & o & o & o & o & o & o & o & o & o & o & o & o \\
o & o & o & o & o & o & o & o & o & o & o & o & o & o \\
o & o & o & o & o & o & o & o & o & o & o & o & o & o \\
\end{array}$$


\textbf{Revised Problem}\\
Paint $26$ squares on a $14 \times 14$ chessboard in such a way that it is impossible to find three painted squares whose centers will form a right-angled triangle with legs parallel to the sides of the chessboard.

Present your solution enclosed by \verb|\begin{array}{...}| and \verb|\end{array}| within a $\boxed{...}$ structure. For instance, $\boxed{\begin{array}{ccc}1 & 2 & 3 \\ 4 & 5 & 6 \\ 7 & 8 & 9\end{array}}$.
Display the chessboard substituting the uncolored cells "o" with painted cells "x":

$$\begin{array}{cccccccccccccc}
o & o & o & o & o & o & o & o & o & o & o & o & o & o \\
o & o & o & o & o & o & o & o & o & o & o & o & o & o \\
o & o & o & o & o & o & o & o & o & o & o & o & o & o \\
o & o & o & o & o & o & o & o & o & o & o & o & o & o \\
o & o & o & o & o & o & o & o & o & o & o & o & o & o \\
o & o & o & o & o & o & o & o & o & o & o & o & o & o \\
o & o & o & o & o & o & o & o & o & o & o & o & o & o \\
o & o & o & o & o & o & o & o & o & o & o & o & o & o \\
o & o & o & o & o & o & o & o & o & o & o & o & o & o \\
o & o & o & o & o & o & o & o & o & o & o & o & o & o \\
o & o & o & o & o & o & o & o & o & o & o & o & o & o \\
o & o & o & o & o & o & o & o & o & o & o & o & o & o \\
o & o & o & o & o & o & o & o & o & o & o & o & o & o \\
o & o & o & o & o & o & o & o & o & o & o & o & o & o \\
\end{array}$$

\subsubsection{Variation}
\textbf{Actual Problem}\\
Color $18$ squares of a $11 \times 11$ chessboard, so that there do not exist three colored squares whose centers form a right triangle with sides parallel to the edges of the board.

Output the answer between \verb|\begin{array}{...}| and \verb|\end{array}| inside of $\boxed{...}$. For example, $\boxed{\begin{array}{ccc}1 & 2 & 3 \\ 4 & 5 & 6 \\ 7 & 8 & 9\end{array}}$.
Output the chessboard where empty cells "o" are replaced with colored cells "x":

$$\begin{array}{ccccccccccc}
o & o & o & o & o & o & o & o & o & o & o \\
o & o & o & o & o & o & o & o & o & o & o \\
o & o & o & o & o & o & o & o & o & o & o \\
o & o & o & o & o & o & o & o & o & o & o \\
o & o & o & o & o & o & o & o & o & o & o \\
o & o & o & o & o & o & o & o & o & o & o \\
o & o & o & o & o & o & o & o & o & o & o \\
o & o & o & o & o & o & o & o & o & o & o \\
o & o & o & o & o & o & o & o & o & o & o \\
o & o & o & o & o & o & o & o & o & o & o \\
o & o & o & o & o & o & o & o & o & o & o \\
\end{array}$$


\textbf{Revised Problem}\\
Color 18 cells on an 11 by 11 chessboard in such a way that no group of three colored cells has their centers forming the vertices of a right-angled triangle, with the sides of the triangle aligned parallel to the grid lines of the board.

Present your answer within \verb|\begin{array}{...}| and \verb|\end{array}| wrapped in $\boxed{...}$. For instance, $\boxed{\begin{array}{ccc}1 & 2 & 3 \\ 4 & 5 & 6 \\ 7 & 8 & 9\end{array}}$.
Illustrate the chessboard by substituting empty cells "o" with colored cells "x":

$$\begin{array}{ccccccccccc}
o & o & o & o & o & o & o & o & o & o & o \\
o & o & o & o & o & o & o & o & o & o & o \\
o & o & o & o & o & o & o & o & o & o & o \\
o & o & o & o & o & o & o & o & o & o & o \\
o & o & o & o & o & o & o & o & o & o & o \\
o & o & o & o & o & o & o & o & o & o & o \\
o & o & o & o & o & o & o & o & o & o & o \\
o & o & o & o & o & o & o & o & o & o & o \\
o & o & o & o & o & o & o & o & o & o & o \\
o & o & o & o & o & o & o & o & o & o & o \\
o & o & o & o & o & o & o & o & o & o & o \\
\end{array}$$

\subsection{usamo-2001-1}
\subsubsection{Variation}
\textbf{Actual Problem}\\
Each of eight boxes contains six balls. Each ball has been colored with one of $n$ colors, such that no two balls in the same box are the same color, and no two colors occur together in more than one box.
For $n=23$, present one such coloring in the form of 6 x 8 table where cell (i, j) shows the color of the $i$-th ball in the $j$-th box.

Output the answer between \verb|\begin{array}{...}| and \verb|\end{array}| inside of $\boxed{...}$. For example, $\boxed{\begin{array}{ccc}1 & 2 & 3 \\ 4 & 5 & 6 \\ 7 & 8 & 9\end{array}}$.
In your solution, each cell should contain a number between 1 and $23$ representing the color of the ball.

$$\begin{array}{cccccccc}
o & o & o & o & o & o & o & o \\
o & o & o & o & o & o & o & o \\
o & o & o & o & o & o & o & o \\
o & o & o & o & o & o & o & o \\
o & o & o & o & o & o & o & o \\
o & o & o & o & o & o & o & o \\
\end{array}$$


\textbf{Revised Problem}\\
You have eight boxes, each containing six distinct balls. Each ball is painted in one of $n$ colors, ensuring that no two balls in any individual box share the same color. Furthermore, a pair of colors cannot appear together in more than one box. For $n=23$, illustrate such a coloring scheme using a 6 x 8 matrix where the element at row $i$, column $j$ indicates the color of the $i$-th ball in the $j$-th box.

Present the solution enclosed between \verb|\begin{array}{...}| and \verb|\end{array}| within $\boxed{...}$. For instance, $\boxed{\begin{array}{ccc}1 & 2 & 3 \\ 4 & 5 & 6 \\ 7 & 8 & 9\end{array}}$. In your matrix, each cell should contain a number ranging from 1 to $23$, representing the color of each ball.

$$\begin{array}{cccccccc}
o & o & o & o & o & o & o & o \\
o & o & o & o & o & o & o & o \\
o & o & o & o & o & o & o & o \\
o & o & o & o & o & o & o & o \\
o & o & o & o & o & o & o & o \\
o & o & o & o & o & o & o & o \\
\end{array}$$

\subsubsection{Variation}
\textbf{Actual Problem}\\
Each of eight boxes contains six balls. Each ball has been colored with one of $n$ colors, such that no two balls in the same box are the same color, and no two colors occur together in more than one box.
For $n=26$, present one such coloring in the form of 6 x 8 table where cell (i, j) shows the color of the $i$-th ball in the $j$-th box.

Output the answer between \verb|\begin{array}{...}| and \verb|\end{array}| inside of $\boxed{...}$. For example, $\boxed{\begin{array}{ccc}1 & 2 & 3 \\ 4 & 5 & 6 \\ 7 & 8 & 9\end{array}}$.
In your solution, each cell should contain a number between 1 and $26$ representing the color of the ball.

$$\begin{array}{cccccccc}
o & o & o & o & o & o & o & o \\
o & o & o & o & o & o & o & o \\
o & o & o & o & o & o & o & o \\
o & o & o & o & o & o & o & o \\
o & o & o & o & o & o & o & o \\
o & o & o & o & o & o & o & o \\
\end{array}$$


\textbf{Revised Problem}\\
Imagine you have eight boxes, each containing six balls. Every ball is painted one of \( n \) different colors. The condition is that no two balls in the same box share the same color, and any pair of colors cannot appear together in more than one box. Given that \( n = 26 \), create a 6 by 8 matrix where each entry (i, j) indicates the color of the \( i \)-th ball in the \( j \)-th box.

Present your solution enclosed within \verb|\begin{array}{...}| and \verb|\end{array}| inside \(\boxed{...}\). For instance, \(\boxed{\begin{array}{ccc}1 & 2 & 3 \\ 4 & 5 & 6 \\ 7 & 8 & 9\end{array}}\).
Within the matrix, each cell should be filled with a number from 1 to 26, signifying the color of the respective ball.

$$\begin{array}{cccccccc}
o & o & o & o & o & o & o & o \\
o & o & o & o & o & o & o & o \\
o & o & o & o & o & o & o & o \\
o & o & o & o & o & o & o & o \\
o & o & o & o & o & o & o & o \\
o & o & o & o & o & o & o & o \\
\end{array}$$

\subsubsection{Variation}
\textbf{Actual Problem}\\
Each of eight boxes contains six balls. Each ball has been colored with one of $n$ colors, such that no two balls in the same box are the same color, and no two colors occur together in more than one box.
For $n=24$, present one such coloring in the form of 6 x 8 table where cell (i, j) shows the color of the $i$-th ball in the $j$-th box.

Output the answer between \verb|\begin{array}{...}| and \verb|\end{array}| inside of $\boxed{...}$. For example, $\boxed{\begin{array}{ccc}1 & 2 & 3 \\ 4 & 5 & 6 \\ 7 & 8 & 9\end{array}}$.
In your solution, each cell should contain a number between 1 and $24$ representing the color of the ball.

$$\begin{array}{cccccccc}
o & o & o & o & o & o & o & o \\
o & o & o & o & o & o & o & o \\
o & o & o & o & o & o & o & o \\
o & o & o & o & o & o & o & o \\
o & o & o & o & o & o & o & o \\
o & o & o & o & o & o & o & o \\
\end{array}$$


\textbf{Revised Problem}\\
Eight boxes each contain six balls, where each ball is assigned a unique color from a total of $n$ colors. In every box, all six balls have different colors, and no two colors are repeated together in more than one box. For $n=24$, illustrate one possible coloring scheme using a 6 x 8 grid, where the entry in the $i$-th row and $j$-th column represents the color of the $i$-th ball in the $j$-th box.

Display the solution using \verb|\begin{array}{...}| and \verb|\end{array}| within $\boxed{...}$. For instance, $\boxed{\begin{array}{ccc}1 & 2 & 3 \\ 4 & 5 & 6 \\ 7 & 8 & 9\end{array}}$. Ensure each entry in the table is a number between 1 and $24$, indicating the color of the ball.

$$\begin{array}{cccccccc}
o & o & o & o & o & o & o & o \\
o & o & o & o & o & o & o & o \\
o & o & o & o & o & o & o & o \\
o & o & o & o & o & o & o & o \\
o & o & o & o & o & o & o & o \\
o & o & o & o & o & o & o & o \\
\end{array}$$

\subsubsection{Variation}
\textbf{Actual Problem}\\
Each of eight boxes contains six balls. Each ball has been colored with one of $n$ colors, such that no two balls in the same box are the same color, and no two colors occur together in more than one box.
For $n=27$, present one such coloring in the form of 6 x 8 table where cell (i, j) shows the color of the $i$-th ball in the $j$-th box.

Output the answer between \verb|\begin{array}{...}| and \verb|\end{array}| inside of $\boxed{...}$. For example, $\boxed{\begin{array}{ccc}1 & 2 & 3 \\ 4 & 5 & 6 \\ 7 & 8 & 9\end{array}}$.
In your solution, each cell should contain a number between 1 and $27$ representing the color of the ball.

$$\begin{array}{cccccccc}
o & o & o & o & o & o & o & o \\
o & o & o & o & o & o & o & o \\
o & o & o & o & o & o & o & o \\
o & o & o & o & o & o & o & o \\
o & o & o & o & o & o & o & o \\
o & o & o & o & o & o & o & o \\
\end{array}$$


\textbf{Revised Problem}\\
Consider eight boxes, each containing six uniquely colored balls. You are to assign one of $n$ colors to each ball, ensuring that no two balls in a single box share the same color, and that any pair of colors does not appear together in more than one box. Given $n=27$, construct such an assignment and represent it as a 6 x 8 matrix, where the entry in row $i$ and column $j$ indicates the color of the $i$-th ball in the $j$-th box.

Express the solution using \verb|\begin{array}{...}| and \verb|\end{array}| enclosed within $\boxed{...}$. For instance, $\boxed{\begin{array}{ccc}1 & 2 & 3 \\ 4 & 5 & 6 \\ 7 & 8 & 9\end{array}}$ can be used as a reference. Each cell should hold a number between 1 and $27$, representing the color assigned to that ball.

$$\begin{array}{cccccccc}
o & o & o & o & o & o & o & o \\
o & o & o & o & o & o & o & o \\
o & o & o & o & o & o & o & o \\
o & o & o & o & o & o & o & o \\
o & o & o & o & o & o & o & o \\
o & o & o & o & o & o & o & o \\
\end{array}$$

\subsection{usamo-2002-5}
\subsubsection{Variation}
\textbf{Actual Problem}\\
We are given integers $a = 22$ and $b = 15$. Find a finite sequence $n_1, n_2, \ldots, n_k$ of positive integers such that $n_1 = a$, $n_k = b$, and $n_in_{i+1}$ is divisible by $n_i + n_{i+1}$ for each $i$ ($1 \le i \le k$).

Output the answer as a comma separated list inside of $\boxed{...}$. For example $\boxed{1, 2, 3}$.

\textbf{Revised Problem}\\
Given the integers $a = 22$ and $b = 15$, determine a finite sequence of positive integers $m_1, m_2, \ldots, m_k$ such that $m_1 = a$, $m_k = b$, and for each consecutive pair $m_i, m_{i+1}$ in the sequence, the product $m_im_{i+1}$ is divisible by the sum $m_i + m_{i+1}$ for every $i$ ($1 \le i \le k$).

Provide the answer as a list of integers separated by commas within the expression $\boxed{...}$. For example, represent the sequence as $\boxed{1, 2, 3}$.

\subsubsection{Variation}
\textbf{Actual Problem}\\
We are given integers $a = 37$ and $b = 22$. Find a finite sequence $n_1, n_2, \ldots, n_k$ of positive integers such that $n_1 = a$, $n_k = b$, and $n_in_{i+1}$ is divisible by $n_i + n_{i+1}$ for each $i$ ($1 \le i \le k$).

Output the answer as a comma separated list inside of $\boxed{...}$. For example $\boxed{1, 2, 3}$.

\textbf{Revised Problem}\\
Consider the integers $a = 37$ and $b = 22$. Determine a finite sequence of positive integers $n_1, n_2, \ldots, n_k$ such that $n_1 = a$, $n_k = b$, and for each consecutive pair $n_i, n_{i+1}$ in the sequence, the product $n_in_{i+1}$ is divisible by the sum $n_i + n_{i+1}$, for all $i$ ($1 \le i \le k-1$).

Provide your answer as a list of numbers separated by commas enclosed in $\boxed{...}$. For instance, $\boxed{1, 2, 3}$.

\subsubsection{Variation}
\textbf{Actual Problem}\\
We are given integers $a = 37$ and $b = 35$. Find a finite sequence $n_1, n_2, \ldots, n_k$ of positive integers such that $n_1 = a$, $n_k = b$, and $n_in_{i+1}$ is divisible by $n_i + n_{i+1}$ for each $i$ ($1 \le i \le k$).

Output the answer as a comma separated list inside of $\boxed{...}$. For example $\boxed{1, 2, 3}$.

\textbf{Revised Problem}\\
Consider the integers $a = 37$ and $b = 35$. Determine a finite list of positive integers $m_1, m_2, \ldots, m_k$ such that $m_1 = a$, $m_k = b$, and for every consecutive pair of terms $m_i$ and $m_{i+1}$, the product $m_i m_{i+1}$ is divisible by the sum $m_i + m_{i+1}$ for all valid indices $i$ ($1 \leq i \leq k$).

Present the solution as a sequence separated by commas within $\boxed{...}$. For instance, $\boxed{1, 2, 3}$.

\subsubsection{Variation}
\textbf{Actual Problem}\\
We are given integers $a = 40$ and $b = 37$. Find a finite sequence $n_1, n_2, \ldots, n_k$ of positive integers such that $n_1 = a$, $n_k = b$, and $n_in_{i+1}$ is divisible by $n_i + n_{i+1}$ for each $i$ ($1 \le i \le k$).

Output the answer as a comma separated list inside of $\boxed{...}$. For example $\boxed{1, 2, 3}$.

\textbf{Revised Problem}\\
Consider two integers $x = 40$ and $y = 37$. Determine a finite sequence of positive integers $m_1, m_2, \ldots, m_j$ where $m_1 = x$, $m_j = y$, and for each index $i$ ($1 \le i < j$), the product $m_i m_{i+1}$ is divisible by the sum $m_i + m_{i+1}$.

Present your solution as a list of numbers separated by commas within a $\boxed{...}$. For instance, $\boxed{1, 2, 3}$.

\subsection{usamo-2005-1}
\subsubsection{Variation}
\textbf{Actual Problem}\\
Arrange all divisors greater than 1 of a composite positive integer $n = 123 \times 456 \times 789 \times 101$ in a circle so that no two adjacent divisors are relatively prime.

Assuming that $d_1, d_2, ..., d_k$ are divisors of $n$ greater than 1 in increasing order, output a list of indices $i_1, i_2, ..., i_k$ such that $d_{i_1}, d_{i_2}, ..., d_{i_k}$ is a permutation of $d_1, d_2, ..., d_k$ which satisfies the properties in the task.
Output the list $i_1, i_2, ..., i_k$ as a comma separated list inside of $\boxed{...}$. For example $\boxed{1, 4, 2, 3}$.

\textbf{Revised Problem}\\
Place all divisors, excluding 1, of the composite positive number $n = 123 \times 456 \times 789 \times 101$ around a circle such that no adjacent divisors are coprime.

Suppose $d_1, d_2, ..., d_k$ represent divisors of $n$ greater than 1, listed in ascending order. Provide a sequence of indices $i_1, i_2, ..., i_k$ so that $d_{i_1}, d_{i_2}, ..., d_{i_k}$ form a permutation of $d_1, d_2, ..., d_k$, ensuring the required adjacency condition is met. Present the sequence $i_1, i_2, ..., i_k$ in a comma-separated format enclosed in $\boxed{...}$. For instance, $\boxed{1, 4, 2, 3}$.

\subsubsection{Variation}
\textbf{Actual Problem}\\
Arrange all divisors greater than 1 of a composite positive integer $n = 90 \times 140 \times 13 \times 19$ in a circle so that no two adjacent divisors are relatively prime.

Assuming that $d_1, d_2, ..., d_k$ are divisors of $n$ greater than 1 in increasing order, output a list of indices $i_1, i_2, ..., i_k$ such that $d_{i_1}, d_{i_2}, ..., d_{i_k}$ is a permutation of $d_1, d_2, ..., d_k$ which satisfies the properties in the task.
Output the list $i_1, i_2, ..., i_k$ as a comma separated list inside of $\boxed{...}$. For example $\boxed{1, 4, 2, 3}$.

\textbf{Revised Problem}\\
Organize all divisors greater than 1 of the composite positive integer $n = 90 \times 140 \times 13 \times 19$ into a circular sequence such that any two neighboring divisors are not coprime.

Assuming the divisors of $n$ greater than 1 are denoted by $d_1, d_2, ..., d_k$ in ascending order, provide a sequence of indices $i_1, i_2, ..., i_k$ such that $d_{i_1}, d_{i_2}, ..., d_{i_k}$ forms a permutation of $d_1, d_2, ..., d_k$ that meets the conditions specified in the problem. Present the sequence $i_1, i_2, ..., i_k$ formatted as a comma-separated list within $\boxed{...}$. For example, $\boxed{1, 4, 2, 3}$.

\subsubsection{Variation}
\textbf{Actual Problem}\\
Arrange all divisors greater than 1 of a composite positive integer $n = 140 \times 140 \times 30 \times 94$ in a circle so that no two adjacent divisors are relatively prime.

Assuming that $d_1, d_2, ..., d_k$ are divisors of $n$ greater than 1 in increasing order, output a list of indices $i_1, i_2, ..., i_k$ such that $d_{i_1}, d_{i_2}, ..., d_{i_k}$ is a permutation of $d_1, d_2, ..., d_k$ which satisfies the properties in the task.
Output the list $i_1, i_2, ..., i_k$ as a comma separated list inside of $\boxed{...}$. For example $\boxed{1, 4, 2, 3}$.

\textbf{Revised Problem}\\
Arrange all divisors of the composite positive integer $n = 140 \times 140 \times 30 \times 94$ excluding 1, in a circular pattern such that any two neighboring divisors share at least one common factor greater than 1.

Given divisors $d_1, d_2, ..., d_k$ of $n$ greater than 1, listed in ascending order, determine a sequence of indices $i_1, i_2, ..., i_k$ so that $d_{i_1}, d_{i_2}, ..., d_{i_k}$ represents a rearrangement of $d_1, d_2, ..., d_k$ meeting the adjacency criteria outlined. Present the sequence $i_1, i_2, ..., i_k$ as a comma-separated list enclosed in $\boxed{...}$, like $\boxed{1, 4, 2, 3}$.

\subsubsection{Variation}
\textbf{Actual Problem}\\
Arrange all divisors greater than 1 of a composite positive integer $n = 72 \times 30 \times 52 \times 43$ in a circle so that no two adjacent divisors are relatively prime.

Assuming that $d_1, d_2, ..., d_k$ are divisors of $n$ greater than 1 in increasing order, output a list of indices $i_1, i_2, ..., i_k$ such that $d_{i_1}, d_{i_2}, ..., d_{i_k}$ is a permutation of $d_1, d_2, ..., d_k$ which satisfies the properties in the task.
Output the list $i_1, i_2, ..., i_k$ as a comma separated list inside of $\boxed{...}$. For example $\boxed{1, 4, 2, 3}$.

\textbf{Revised Problem}\\
Consider the task of placing all divisors greater than 1 of the composite number $n = 72 \times 30 \times 52 \times 43$ around a circular table. The objective is to arrange them such that no pair of neighboring divisors are coprime.

Let $d_1, d_2, ..., d_k$ be the divisors of $n$ that are more than 1, sorted in ascending order. Provide a sequence of indices $i_1, i_2, ..., i_k$ such that the sequence $d_{i_1}, d_{i_2}, ..., d_{i_k}$ is a permutation of $d_1, d_2, ..., d_k$ and meets the condition described. Present the indices $i_1, i_2, ..., i_k$ as a comma-separated list enclosed in $\boxed{...}$. For example, $\boxed{1, 4, 2, 3}$.

\subsection{usamo-2006-2}
\subsubsection{Variation}
\textbf{Actual Problem}\\
Given $k = 12$ and $N = 3924$, find a set of $2k+1$ distinct positive integers that has sum greater than $N$ but every subset of size $k$ has sum at most $N/2$.
Note that for given $k = 12$, one can prove that $N = 3924$ is the smallest $N$ for which such set exists.

Output the answer as a comma separated list inside of $\boxed{...}$. For example $\boxed{1, 2, 3}$.

\textbf{Revised Problem}\\
Given the values \( k = 12 \) and \( N = 3924 \), identify a collection of \( 2k+1 \) distinct positive integers such that their total sum is greater than \( N \), yet the sum of any subset consisting of \( k \) integers is at most \( N/2 \).
It is important to note that for the specified \( k = 12 \), the value of \( N = 3924 \) is the smallest for which such a set can be constructed.

Present your solution in the form of a comma-separated list enclosed within $\boxed{...}$. For example, $\boxed{1, 2, 3}$.

\subsubsection{Variation}
\textbf{Actual Problem}\\
Given $k = 22$ and $N = 22814$, find a set of $2k+1$ distinct positive integers that has sum greater than $N$ but every subset of size $k$ has sum at most $N/2$.
Note that for given $k = 22$, one can prove that $N = 22814$ is the smallest $N$ for which such set exists.

Output the answer as a comma separated list inside of $\boxed{...}$. For example $\boxed{1, 2, 3}$.

\textbf{Revised Problem}\\
Given that \( k = 22 \) and \( N = 22814 \), determine a collection of \( 2k+1 \) distinct positive integers whose total sum is greater than \( N \), while ensuring that the sum of every possible subset containing exactly \( k \) elements does not exceed \( N/2 \). It is important to note that for the specified \( k = 22 \), the value \( N = 22814 \) is the smallest possible for which such a collection can be found.

Present your solution in a format where the integers are listed in a comma-separated sequence, enclosed within a box. For example, \(\boxed{1, 2, 3}\).

\subsubsection{Variation}
\textbf{Actual Problem}\\
Given $k = 14$ and $N = 6118$, find a set of $2k+1$ distinct positive integers that has sum greater than $N$ but every subset of size $k$ has sum at most $N/2$.
Note that for given $k = 14$, one can prove that $N = 6118$ is the smallest $N$ for which such set exists.

Output the answer as a comma separated list inside of $\boxed{...}$. For example $\boxed{1, 2, 3}$.

\textbf{Revised Problem}\\
Given the values \( k = 14 \) and \( N = 6118 \), determine a collection of \( 2k+1 \) unique positive integers such that the total sum is more than \( N \), and every subset containing exactly \( k \) integers has a sum that does not exceed \( N/2 \). It is established for \( k = 14 \) that \( N = 6118 \) is the smallest possible value for which such a collection can be found.

Present the solution as a sequence of numbers separated by commas and enclosed within a box, like this: \(\boxed{1, 2, 3}\).

\subsubsection{Variation}
\textbf{Actual Problem}\\
Given $k = 11$ and $N = 3058$, find a set of $2k+1$ distinct positive integers that has sum greater than $N$ but every subset of size $k$ has sum at most $N/2$.
Note that for given $k = 11$, one can prove that $N = 3058$ is the smallest $N$ for which such set exists.

Output the answer as a comma separated list inside of $\boxed{...}$. For example $\boxed{1, 2, 3}$.

\textbf{Revised Problem}\\
Consider $k = 11$ and $N = 3058$. Your task is to identify a collection of $2k+1$ distinct positive numbers such that their total exceeds $N$, yet any group of $k$ numbers within this collection has a sum that does not surpass $N/2$. It is established that for this specific $k = 11$, $N = 3058$ is the smallest such number for which this condition is achievable.

Present your solution as a list of numbers separated by commas, enclosed within $\boxed{...}$. For example, format your answer as $\boxed{1, 2, 3}$.

\subsection{usamo-2006-4}
\subsubsection{Variation}
\textbf{Actual Problem}\\
Given $n = 51$, find $k\ge 2$ positive rational numbers $a_1, a_2, \ldots, a_k$ satisfying $a_1 + a_2 + \cdots + a_k = a_1\cdot a_2\cdots a_k = n$.

Output the answer as a comma separated list inside of $\boxed{...}$. For example $\boxed{1, 2, 3}$.

\textbf{Revised Problem}\\
Consider $n = 51$. Identify $k\ge 2$ positive rational values $b_1, b_2, \ldots, b_k$ such that the sum $b_1 + b_2 + \cdots + b_k$ and the product $b_1 \cdot b_2 \cdots b_k$ both equate to $n$.

Present your response as a list separated by commas, enclosed within $\boxed{...}$. For instance, $\boxed{1, 2, 3}$.

\subsubsection{Variation}
\textbf{Actual Problem}\\
Given $n = 74$, find $k\ge 2$ positive rational numbers $a_1, a_2, \ldots, a_k$ satisfying $a_1 + a_2 + \cdots + a_k = a_1\cdot a_2\cdots a_k = n$.

Output the answer as a comma separated list inside of $\boxed{...}$. For example $\boxed{1, 2, 3}$.

\textbf{Revised Problem}\\
Consider \( n = 74 \). Identify \( k \ge 2 \) positive rational numbers \( a_1, a_2, \ldots, a_k \) such that their sum \( a_1 + a_2 + \cdots + a_k \) and their product \( a_1 \cdot a_2 \cdots a_k \) both equal \( n \).

Present your solution as a list of numbers separated by commas, enclosed in a box. For instance, in the format \(\boxed{1, 2, 3}\).

\subsubsection{Variation}
\textbf{Actual Problem}\\
Given $n = 58$, find $k\ge 2$ positive rational numbers $a_1, a_2, \ldots, a_k$ satisfying $a_1 + a_2 + \cdots + a_k = a_1\cdot a_2\cdots a_k = n$.

Output the answer as a comma separated list inside of $\boxed{...}$. For example $\boxed{1, 2, 3}$.

\textbf{Revised Problem}\\
For $n = 58$, determine a set of $k\ge 2$ positive rational numbers $b_1, b_2, \ldots, b_k$ such that their sum $b_1 + b_2 + \cdots + b_k$ and their product $b_1 \cdot b_2 \cdots b_k$ both equal $n$.

Present your solution as a list of numbers separated by commas within $\boxed{...}$. For instance, $\boxed{1, 2, 3}$.

\subsubsection{Variation}
\textbf{Actual Problem}\\
Given $n = 53$, find $k\ge 2$ positive rational numbers $a_1, a_2, \ldots, a_k$ satisfying $a_1 + a_2 + \cdots + a_k = a_1\cdot a_2\cdots a_k = n$.

Output the answer as a comma separated list inside of $\boxed{...}$. For example $\boxed{1, 2, 3}$.

\textbf{Revised Problem}\\
Consider \( n = 53 \). Determine \( k \geq 2 \) positive rational numbers \( b_1, b_2, \ldots, b_k \) such that the sum \( b_1 + b_2 + \cdots + b_k = n \) and the product \( b_1 \cdot b_2 \cdots b_k = n \).

Present the solution as a comma-separated list enclosed within \(\boxed{...}\). For instance, \(\boxed{1, 2, 3}\).

\subsection{usamo-2017-1}
\subsubsection{Variation}
\textbf{Actual Problem}\\
Find 10 distinct pairs $(a,b)$ of relatively prime 9-digit positive integers $a>1$ and $b>1$, such that $a^b+b^a$ is divisible by $a+b$. Note that one can prove that there are infinitely many such pairs.

Output your answer as a comma separated list of tuples inside \boxed{...}. For example, if the answer is $(2,3)$ and $(3,5)$, output \boxed{(2,3),(3,5)}.

\textbf{Revised Problem}\\
Identify 10 unique pairs \((a,b)\) where both \(a\) and \(b\) are relatively prime 9-digit positive integers exceeding 1, such that \(a^b + b^a\) is divisible by \(a+b\). It can be demonstrated that there are an infinite number of such pairs.

Present your solution as a list of tuples separated by commas within a \boxed{...}. For instance, if the solution is \((2,3)\) and \((3,5)\), then write \boxed{(2,3),(3,5)}.

\subsubsection{Variation}
\textbf{Actual Problem}\\
Find 79 distinct pairs $(a,b)$ of relatively prime 9-digit positive integers $a>1$ and $b>1$, such that $a^b+b^a$ is divisible by $a+b$. Note that one can prove that there are infinitely many such pairs.

Output your answer as a comma separated list of tuples inside \boxed{...}. For example, if the answer is $(2,3)$ and $(3,5)$, output \boxed{(2,3),(3,5)}.

\textbf{Revised Problem}\\
Identify 79 unique pairs \((a, b)\) of relatively prime positive integers each with 9 digits, where \(a > 1\) and \(b > 1\), such that \(a^b + b^a\) is divisible by \(a + b\). It is known that there are infinitely many such pairs.

Present your solution as a list of tuples separated by commas within \boxed{...}. For instance, if the solution includes the pairs \((2,3)\) and \((3,5)\), write \boxed{(2,3),(3,5)}.

\subsubsection{Variation}
\textbf{Actual Problem}\\
Find 47 distinct pairs $(a,b)$ of relatively prime 9-digit positive integers $a>1$ and $b>1$, such that $a^b+b^a$ is divisible by $a+b$. Note that one can prove that there are infinitely many such pairs.

Output your answer as a comma separated list of tuples inside \boxed{...}. For example, if the answer is $(2,3)$ and $(3,5)$, output \boxed{(2,3),(3,5)}.

\textbf{Revised Problem}\\
Determine 47 unique pairs of 9-digit positive integers \((a, b)\) such that both \(a\) and \(b\) are greater than 1, \(a\) and \(b\) are coprime, and the expression \(a^b + b^a\) is divisible by \(a + b\). It is confirmed that there are infinitely many such pairings.

List your solution as a sequence of tuples, separated by commas, within \boxed{...}. For instance, if the solution includes the pairs \((2,3)\) and \((3,5)\), you should write \boxed{(2,3),(3,5)}.

\subsubsection{Variation}
\textbf{Actual Problem}\\
Find 37 distinct pairs $(a,b)$ of relatively prime 9-digit positive integers $a>1$ and $b>1$, such that $a^b+b^a$ is divisible by $a+b$. Note that one can prove that there are infinitely many such pairs.

Output your answer as a comma separated list of tuples inside \boxed{...}. For example, if the answer is $(2,3)$ and $(3,5)$, output \boxed{(2,3),(3,5)}.

\textbf{Revised Problem}\\
Identify 37 unique pairs $(x, y)$ of 9-digit integers greater than 1, where $x$ and $y$ are coprime, and ensure that $x^y + y^x$ is divisible by $x + y$. It is worth noting that there are infinitely many such pairs that satisfy these conditions.

Present your solution as a series of tuples separated by commas enclosed in \boxed{...}. For instance, if your answer consists of the pairs $(4,5)$ and $(6,7)$, you should write \boxed{(4,5),(6,7)}.

\section{usamts}
\subsection{usamts-1998-1-4}
\subsubsection{Variation}
\textbf{Actual Problem}\\
Arrange seven distinct points in the plane so that among any three of these seven points, the distance between two of the points is exactly 1.
Write the coordinates of all 7 points.

Output the answer as a comma separated list of 7 coordinates inside of $\boxed{...}$. For example $\boxed{(1, 1), (2, 2), (3, 3), (4, 4), (5, 5), (6, 6), (7, 7)}$.

\textbf{Revised Problem}\\
Position seven unique points on a flat plane such that for every combination of any three points among these seven, there exists a pair of points separated by a distance of 1. Determine the coordinates of each of these 7 points.

List your answer as a sequence of 7 coordinates, separated by commas and enclosed within $\boxed{...}$. An example output is $\boxed{(1, 1), (2, 2), (3, 3), (4, 4), (5, 5), (6, 6), (7, 7)}$.

\subsubsection{Variation}
\textbf{Actual Problem}\\
Arrange seven distinct points in the plane so that among any three of these seven points, the distance between two of the points is exactly 8.
Write the coordinates of all 7 points.

Output the answer as a comma separated list of 7 coordinates inside of $\boxed{...}$. For example $\boxed{(1, 1), (2, 2), (3, 3), (4, 4), (5, 5), (6, 6), (7, 7)}$.

\textbf{Revised Problem}\\
Determine the coordinates of seven distinct points positioned in a plane such that, for any chosen set of three points, two of them are separated by a distance of exactly 8 units. Specify the coordinates of these 7 points.

Provide the response as a list of 7 coordinates, separated by commas and enclosed within $\boxed{...}$. For instance, $\boxed{(1, 1), (2, 2), (3, 3), (4, 4), (5, 5), (6, 6), (7, 7)}$.

\subsubsection{Variation}
\textbf{Actual Problem}\\
Arrange seven distinct points in the plane so that among any three of these seven points, the distance between two of the points is exactly 4.
Write the coordinates of all 7 points.

Output the answer as a comma separated list of 7 coordinates inside of $\boxed{...}$. For example $\boxed{(1, 1), (2, 2), (3, 3), (4, 4), (5, 5), (6, 6), (7, 7)}$.

\textbf{Revised Problem}\\
Position seven unique points on a plane such that, for any selection of three points from these seven, there is at least one pair of points with a distance of exactly 4 between them.
Provide the coordinates for all 7 points.

Present the solution as a list of 7 coordinates enclosed within $\boxed{...}$, separated by commas. For example, $\boxed{(1, 1), (2, 2), (3, 3), (4, 4), (5, 5), (6, 6), (7, 7)}$.

\subsubsection{Variation}
\textbf{Actual Problem}\\
Arrange seven distinct points in the plane so that among any three of these seven points, the distance between two of the points is exactly 2.
Write the coordinates of all 7 points.

Output the answer as a comma separated list of 7 coordinates inside of $\boxed{...}$. For example $\boxed{(1, 1), (2, 2), (3, 3), (4, 4), (5, 5), (6, 6), (7, 7)}$.

\textbf{Revised Problem}\\
Place seven unique points on a plane such that for any selection of three points from these seven, there is a pair of points with a distance of exactly 2. Determine the coordinates for all seven points.

Present your answer formatted as a comma-separated list of 7 coordinates enclosed in $\boxed{...}$. For instance, $\boxed{(1, 1), (2, 2), (3, 3), (4, 4), (5, 5), (6, 6), (7, 7)}$.

\subsection{usamts-1998-4-1}
\subsubsection{Variation}
\textbf{Actual Problem}\\
Find a 31-digit integer N that is an integer multiple of $2^{ 31 }$ and whose digits consist only of digits 8 and 9.

Output the answer as an integer inside of $\boxed{...}$. For example $\boxed{123}$.

\textbf{Revised Problem}\\
Identify a 31-digit integer N that is divisible by $2^{31}$ and is composed exclusively of the digits 8 and 9.

Present the solution as a number enclosed in $\boxed{...}$. For instance, $\boxed{123}$.

\subsubsection{Variation}
\textbf{Actual Problem}\\
Find a 26-digit integer N that is an integer multiple of $2^{ 26 }$ and whose digits consist only of digits 8 and 9.

Output the answer as an integer inside of $\boxed{...}$. For example $\boxed{123}$.

\textbf{Revised Problem}\\
Identify a 26-digit integer, N, that is a multiple of \(2^{26}\) and is comprised solely of the digits 8 and 9.

Present the result as an integer enclosed within $\boxed{...}$. For instance, $\boxed{123}$.

\subsubsection{Variation}
\textbf{Actual Problem}\\
Find a 33-digit integer N that is an integer multiple of $2^{ 33 }$ and whose digits consist only of digits 4 and 3.

Output the answer as an integer inside of $\boxed{...}$. For example $\boxed{123}$.

\textbf{Revised Problem}\\
Determine a 33-digit number N consisting only of the digits 4 and 3, which is divisible by \( 2^{33} \).

Present your answer as an integer within $\boxed{...}$. For instance, $\boxed{123}$.

\subsubsection{Variation}
\textbf{Actual Problem}\\
Find a 27-digit integer N that is an integer multiple of $2^{ 27 }$ and whose digits consist only of digits 2 and 3.

Output the answer as an integer inside of $\boxed{...}$. For example $\boxed{123}$.

\textbf{Revised Problem}\\
Determine a 27-digit number, denoted as N, which is divisible by \(2^{27}\), with the condition that N consists solely of the digits 2 and 3.

Present the solution as an integer enclosed within \(\boxed{...}\). For example, \(\boxed{123}\).

\subsection{usamts-1999-1-2}
\subsubsection{Variation}
\textbf{Actual Problem}\\
Let N = 111...1222...2, where there are 1999 digits of 1 followed by 1999 digits of 2.
Express N as the product of four integers, each of them greater than 1.

Output the answer as a comma separated list inside of $\boxed{...}$. For example $\boxed{1, 2, 3}$.

\textbf{Revised Problem}\\
Consider a number N, which is formed by writing 1999 consecutive 1's followed by 1999 consecutive 2's. Find a way to express N as a product of four integers, each greater than 1.

Please provide the answer formatted as a list of numbers separated by commas, enclosed within $\boxed{...}$. For example, $\boxed{4, 5, 6}$.

\subsubsection{Variation}
\textbf{Actual Problem}\\
Let N = 111...1222...2, where there are 914 digits of 1 followed by 914 digits of 2.
Express N as the product of four integers, each of them greater than 1.

Output the answer as a comma separated list inside of $\boxed{...}$. For example $\boxed{1, 2, 3}$.

\textbf{Revised Problem}\\
Consider the number N = 111...1222...2, where N consists of 914 consecutive '1's, followed by 914 consecutive '2's. Decompose N into a product of four integers, each of which is greater than 1.

Present the answer as a list of numbers separated by commas within $\boxed{...}$. For example, $\boxed{1, 2, 3}$.

\subsubsection{Variation}
\textbf{Actual Problem}\\
Let N = 111...1222...2, where there are 187 digits of 1 followed by 187 digits of 2.
Express N as the product of four integers, each of them greater than 1.

Output the answer as a comma separated list inside of $\boxed{...}$. For example $\boxed{1, 2, 3}$.

\textbf{Revised Problem}\\
Consider the number \( N \) formed by writing 187 consecutive digits of 1 followed by 187 consecutive digits of 2. Your task is to decompose \( N \) into a product of four integers, each greater than 1.

Present your solution as a list of numbers separated by commas and enclosed within \(\boxed{...}\). For instance, \(\boxed{1, 2, 3}\).

\subsubsection{Variation}
\textbf{Actual Problem}\\
Let N = 111...1222...2, where there are 933 digits of 1 followed by 933 digits of 2.
Express N as the product of four integers, each of them greater than 1.

Output the answer as a comma separated list inside of $\boxed{...}$. For example $\boxed{1, 2, 3}$.

\textbf{Revised Problem}\\
Consider the number \( N = 111\ldots1222\ldots2 \), composed of 933 consecutive digits of 1 immediately followed by 933 consecutive digits of 2. Your task is to write \( N \) as a product of four integers, each larger than 1.

Present the solution as a comma-separated list enclosed within $\boxed{...}$. For instance, $\boxed{1, 2, 3}$.

\subsection{usamts-2001-3-3}
\subsubsection{Variation}
\textbf{Actual Problem}\\
Let $p(x) = x^n + a_{n-1}x^{n-1} + \cdots + a_1x + a_0$ be a monic polynomial with integer coefficients.
For $n = 21$, find coefficients $a_0, a_1, \cdots, a_{n-1}$ such that $(p(x))^2$ has only non-negative coefficients, but not all coefficients of $p(x)$ are non-negative.

Output the answer as a comma separated list inside of $\boxed{...}$. For example $\boxed{1, 2, 3}$.

\textbf{Revised Problem}\\
Consider the polynomial $p(x) = x^{21} + a_{20}x^{20} + \cdots + a_1x + a_0$, which is monic and has integer coefficients. Determine the coefficients $a_0, a_1, \ldots, a_{20}$ such that the polynomial $(p(x))^2$ results in a polynomial with all its coefficients being non-negative. It is required that not all of the coefficients of $p(x)$ are non-negative.

Present your solution as a list of numbers separated by commas within $\boxed{...}$. For instance, write $\boxed{1, 2, 3}$.

\subsubsection{Variation}
\textbf{Actual Problem}\\
Let $p(x) = x^n + a_{n-1}x^{n-1} + \cdots + a_1x + a_0$ be a monic polynomial with integer coefficients.
For $n = 70$, find coefficients $a_0, a_1, \cdots, a_{n-1}$ such that $(p(x))^2$ has only non-negative coefficients, but not all coefficients of $p(x)$ are non-negative.

Output the answer as a comma separated list inside of $\boxed{...}$. For example $\boxed{1, 2, 3}$.

\textbf{Revised Problem}\\
Consider the polynomial $p(x) = x^{70} + a_{69}x^{69} + \cdots + a_1x + a_0$, which is monic and has integer coefficients. Determine the set of coefficients $a_0, a_1, \ldots, a_{69}$ such that when $p(x)$ is squared, all the coefficients in the expanded polynomial $(p(x))^2$ are non-negative. Additionally, ensure that not every coefficient of $p(x)$ itself is non-negative.

Present your solution as a sequence of numbers separated by commas, enclosed within $\boxed{...}$. For instance, use $\boxed{1, 2, 3}$.

\subsubsection{Variation}
\textbf{Actual Problem}\\
Let $p(x) = x^n + a_{n-1}x^{n-1} + \cdots + a_1x + a_0$ be a monic polynomial with integer coefficients.
For $n = 38$, find coefficients $a_0, a_1, \cdots, a_{n-1}$ such that $(p(x))^2$ has only non-negative coefficients, but not all coefficients of $p(x)$ are non-negative.

Output the answer as a comma separated list inside of $\boxed{...}$. For example $\boxed{1, 2, 3}$.

\textbf{Revised Problem}\\
Consider the monic polynomial $p(x) = x^{38} + a_{37}x^{37} + \cdots + a_1x + a_0$ where each $a_i$ is an integer. Determine a sequence of coefficients $a_0, a_1, \ldots, a_{37}$ such that when $p(x)$ is squared, the resulting polynomial $(p(x))^2$ has coefficients that are all non-negative. Additionally, ensure that not every coefficient of $p(x)$ itself is non-negative.

Present your solution as a comma-separated list within $\boxed{...}$. For instance, $\boxed{1, 2, 3}$.

\subsubsection{Variation}
\textbf{Actual Problem}\\
Let $p(x) = x^n + a_{n-1}x^{n-1} + \cdots + a_1x + a_0$ be a monic polynomial with integer coefficients.
For $n = 28$, find coefficients $a_0, a_1, \cdots, a_{n-1}$ such that $(p(x))^2$ has only non-negative coefficients, but not all coefficients of $p(x)$ are non-negative.

Output the answer as a comma separated list inside of $\boxed{...}$. For example $\boxed{1, 2, 3}$.

\textbf{Revised Problem}\\
Consider a monic polynomial \( p(x) = x^{28} + a_{27}x^{27} + \cdots + a_1x + a_0 \) where each \( a_i \) is an integer. Your task is to identify the coefficients \( a_0, a_1, \ldots, a_{27} \) such that the polynomial \( (p(x))^2 \) contains only non-negative coefficients, yet not all coefficients of \( p(x) \) are non-negative.

Present your solution as a sequence of numbers separated by commas within a \(\boxed{...}\). For instance, \(\boxed{1, 2, 3}\).

\subsection{usamts-2001-4-4}
\subsubsection{Variation}
\textbf{Actual Problem}\\
A certain company has a faulty telephone system that sometimes transposes a pair of
adjacent digits when someone dials a three-digit extension. Hence a call to 318 would ring
at either 318, 138, or 381, while a call received at 044 would be intended for either
404 or 044. Rather than replace the system, the company is adding a computer to deduce
which dialed extensions are in error and revert those numbers to their correct form. They have
to leave out several possible extensions for this to work.
Assume that the company can only use digits 0-9. Output the list of 340 extensions that the company can assign under this plan.
Note that one can show that is possible to find such a list.


Output the answer as a comma separated list inside of $\boxed{...}$. For example $\boxed{1, 2, 3}$.

\textbf{Revised Problem}\\
A company is dealing with a problematic telephone system where occasionally, two consecutive digits in a three-digit extension are swapped when dialing. For instance, dialing the extension 318 might result in a call going to 318, 138, or 381. Similarly, if the extension 044 is dialed, it might be received as either 404 or 044. To address this without replacing the system, the company plans to integrate a computer that can identify incorrect dialed extensions and restore them to their intended form. This requires omitting some possible extensions. Given that the digits 0 through 9 are available, determine the collection of 340 extensions that can be safely allocated by the company under this strategy. It can be demonstrated that such a collection is achievable.

Present the solution as a list of numbers, separated by commas, enclosed within $\boxed{...}$. For example, $\boxed{1, 2, 3}$.

\subsubsection{Variation}
\textbf{Actual Problem}\\
A certain company has a faulty telephone system that sometimes transposes a pair of
adjacent digits when someone dials a three-digit extension. Hence a call to 318 would ring
at either 318, 138, or 381, while a call received at 044 would be intended for either
404 or 044. Rather than replace the system, the company is adding a computer to deduce
which dialed extensions are in error and revert those numbers to their correct form. They have
to leave out several possible extensions for this to work.
Assume that the company can only use digits 0-8. Output the list of 249 extensions that the company can assign under this plan.
Note that one can show that is possible to find such a list.


Output the answer as a comma separated list inside of $\boxed{...}$. For example $\boxed{1, 2, 3}$.

\textbf{Revised Problem}\\
A company experiences issues with their telephone system, where occasionally, two consecutive digits in a three-digit extension number are swapped. For instance, dialing 318 might connect to 318, 138, or 381, and dialing 044 could connect to 044 or 404. Instead of overhauling the system, the company plans to implement a computer solution that identifies incorrect extensions and corrects them. To facilitate this, certain extensions will be omitted from use.
Assume the company restricts the use of digits to 0 through 8. Identify and output the set of 249 extensions available for assignment under this plan.
It can be demonstrated that such a selection is feasible.

Display the result as a list of extensions separated by commas enclosed in $\boxed{...}$. For instance, $\boxed{1, 2, 3}$.

\subsubsection{Variation}
\textbf{Actual Problem}\\
A certain company has a faulty telephone system that sometimes transposes a pair of
adjacent digits when someone dials a three-digit extension. Hence a call to 318 would ring
at either 318, 138, or 381, while a call received at 044 would be intended for either
404 or 044. Rather than replace the system, the company is adding a computer to deduce
which dialed extensions are in error and revert those numbers to their correct form. They have
to leave out several possible extensions for this to work.
Assume that the company can only use digits 0-4. Output the list of 45 extensions that the company can assign under this plan.
Note that one can show that is possible to find such a list.


Output the answer as a comma separated list inside of $\boxed{...}$. For example $\boxed{1, 2, 3}$.

\textbf{Revised Problem}\\
A company is dealing with a telephone system issue where adjacent digits in a three-digit extension might be swapped when dialed. For instance, calling 318 could mistakenly connect to 318, 138, or 381. Similarly, dialing 044 might actually mean 404 or 044. Instead of replacing the entire system, the company plans to use a computer that detects such errors and corrects them to the intended number. To do this, they have to exclude certain extensions to ensure no confusion arises. Given that only the digits 0-4 can be used, provide the list of 45 extensions that do not cause such errors.

Present the solution as a list of numbers separated by commas enclosed within $\boxed{...}$. For instance, $\boxed{1, 2, 3}$.

\subsubsection{Variation}
\textbf{Actual Problem}\\
A certain company has a faulty telephone system that sometimes transposes a pair of
adjacent digits when someone dials a three-digit extension. Hence a call to 318 would ring
at either 318, 138, or 381, while a call received at 044 would be intended for either
404 or 044. Rather than replace the system, the company is adding a computer to deduce
which dialed extensions are in error and revert those numbers to their correct form. They have
to leave out several possible extensions for this to work.
Assume that the company can only use digits 0-2. Output the list of 11 extensions that the company can assign under this plan.
Note that one can show that is possible to find such a list.


Output the answer as a comma separated list inside of $\boxed{...}$. For example $\boxed{1, 2, 3}$.

\textbf{Revised Problem}\\
A company is dealing with a problematic telephone system that occasionally switches two adjacent digits when dialing a three-digit extension. For instance, dialing 318 might connect to 318, 138, or 381, whereas a call intended for 044 could reach either 404 or 044. Instead of overhauling the system, the company plans to incorporate a computer to identify incorrect extensions and correct them automatically. However, this requires omitting certain possible extensions from use.
Assuming the company can only utilize digits from 0 to 2, determine the list of 11 extensions that can be safely assigned under this scheme.
It is possible to demonstrate that such a list exists.

Present the solution as a comma-separated sequence enclosed within $\boxed{...}$. For example, $\boxed{1, 2, 3}$.

\subsection{usamts-2002-1-2}
\subsubsection{Variation}
\textbf{Actual Problem}\\
Find any 10 quadruples of distinct positive integers $(a, b, c, d)$ such that each of the four sums $a+b+c$, $a+b+d$, $a+c+d$, and $b+c+d$ is the square of an integer.
Additionally, for $i = 1, 2, \cdots, 10$, in your $i$-th quadruple, at least one of the numbers $a, b, c, d$ should have at least $i$ digits.

Output the answer as a comma separated list inside of $\boxed{...}$. For example $\boxed{1, 2, 3}$.
Each element of the list should be a quadruple of integers, e.g. (1, 2, 3, 4).

\textbf{Revised Problem}\\
Identify any 10 sets of four distinct positive integers $(a, b, c, d)$ such that each of the sums $a+b+c$, $a+b+d$, $a+c+d$, and $b+c+d$ equals a perfect square. Furthermore, for $i = 1, 2, \ldots, 10$, in the $i$-th set, at least one among $a, b, c, d$ must have at least $i$ digits.

Present your solution as a sequence of quadruples enclosed in $\boxed{...}$, separated by commas. For instance, $\boxed{(1, 2, 3, 4), (5, 6, 7, 8)}$. Each quadruple should be a set of four integers, formatted as (a, b, c, d).

\subsubsection{Variation}
\textbf{Actual Problem}\\
Find any 11 quadruples of distinct positive integers $(a, b, c, d)$ such that each of the four sums $a+b+c$, $a+b+d$, $a+c+d$, and $b+c+d$ is the square of an integer.
Additionally, for $i = 1, 2, \cdots, 11$, in your $i$-th quadruple, at least one of the numbers $a, b, c, d$ should have at least $i$ digits.

Output the answer as a comma separated list inside of $\boxed{...}$. For example $\boxed{1, 2, 3}$.
Each element of the list should be a quadruple of integers, e.g. (1, 2, 3, 4).

\textbf{Revised Problem}\\
Identify 11 unique sets of four distinct positive integers $(a, b, c, d)$ such that the following sums: $a+b+c$, $a+b+d$, $a+c+d$, and $b+c+d$, are all perfect squares. Furthermore, for each set $i = 1, 2, \ldots, 11$, at least one of the integers $a, b, c,$ or $d$ must have a minimum of $i$ digits.

Present your solution as a comma-separated list enclosed in $\boxed{...}$. For instance, $\boxed{1, 2, 3}$. Each entry in the list should be a group of four integers, formatted like (1, 2, 3, 4).

\subsubsection{Variation}
\textbf{Actual Problem}\\
Find any 7 quadruples of distinct positive integers $(a, b, c, d)$ such that each of the four sums $a+b+c$, $a+b+d$, $a+c+d$, and $b+c+d$ is the square of an integer.
Additionally, for $i = 1, 2, \cdots, 7$, in your $i$-th quadruple, at least one of the numbers $a, b, c, d$ should have at least $i$ digits.

Output the answer as a comma separated list inside of $\boxed{...}$. For example $\boxed{1, 2, 3}$.
Each element of the list should be a quadruple of integers, e.g. (1, 2, 3, 4).

\textbf{Revised Problem}\\
Identify any 7 sets of four distinct positive integers $(a, b, c, d)$ such that each of the sums $a+b+c$, $a+b+d$, $a+c+d$, and $b+c+d$ equals a perfect square. Furthermore, for each set $i$ where $i = 1, 2, \ldots, 7$, ensure that at least one number among $a, b, c, d$ contains no fewer than $i$ digits.

Present the solution as a comma-separated sequence contained within $\boxed{...}$. For instance, $\boxed{1, 2, 3}$. Each entry should be represented as a quadruple of integers, such as (1, 2, 3, 4).

\subsubsection{Variation}
\textbf{Actual Problem}\\
Find any 5 quadruples of distinct positive integers $(a, b, c, d)$ such that each of the four sums $a+b+c$, $a+b+d$, $a+c+d$, and $b+c+d$ is the square of an integer.
Additionally, for $i = 1, 2, \cdots, 5$, in your $i$-th quadruple, at least one of the numbers $a, b, c, d$ should have at least $i$ digits.

Output the answer as a comma separated list inside of $\boxed{...}$. For example $\boxed{1, 2, 3}$.
Each element of the list should be a quadruple of integers, e.g. (1, 2, 3, 4).

\textbf{Revised Problem}\\
Identify any 5 sets of four different positive integers $(a, b, c, d)$ such that each of the following sums: $a+b+c$, $a+b+d$, $a+c+d$, and $b+c+d$ is a perfect square. Furthermore, in your $i$-th set for $i = 1, 2, \ldots, 5$, at least one integer among $a, b, c, d$ should have a minimum of $i$ digits.

Present the answer as a comma-separated list enclosed in $\boxed{...}$. For instance, $\boxed{1, 2, 3}$. Each entry in the list should be a set of four integers, like (1, 2, 3, 4).

